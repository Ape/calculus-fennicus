\chapter{Gaussin ja Stokesin lauseet}

\kor{Gaussin lause} ja \kor{Stokesin lause} ovat usean muuttujan differentiaali- ja 
integraalilaskun keskeisiä tuloksia. Lauseet tunnetaan enemmän käyttökelpoisuutensa kuin 
matemaattisen suuruutensa vuoksi, ja niihin viitataankin usein arkisissa yhteyksissä nimillä 
'Gaussin kaava' ja 'Stokesin kaava'. Hieman yksinkertaistaen näissä kaavoissa on kyse 
integraalikaavan
\[
\int_a^b f'(x)\,dx=f(b)-f(a)
\]
yleistämisestä koskemaan useamman (käytännössä yleensä kahden tai kolmen) muuttujan 
vektoriarvoisia funktioita eli \pain{vektorikenttiä}. Gaussin lauseeseen (kaavaan) vedotaan 
hyvin usein silloin, kun erilaiset fysiikan \kor{säilymislait} halutaan kirjoittaa 
osittaisdifferentiaaliyhtälöitten muotoon. Myös Stokesin lauseella on tällaista käyttöä etenkin
sähkömagnetiikassa.

Luvussa \ref{polkuintegraalit} tarkastellaan ensin viivaintegraaleille sukua olevia
\kor{polkuintegraaleja}. Näillä on käyttöä Gaussin ja Stokesin lauseiden yhteydessä ja
yleisemminkin fysikaalisten vektorikenttien sovelluksissa. Luvussa \ref{gaussin lause}
johdetaan Gaussin lause tasossa ja avaruudessa ja esitetään lauseen yleistetty muoto.
Lähtökohtana ovat taso- ja avaruusintegraaleja koskevat \kor{Greenin kaavat}. Luvussa
\ref{Gaussin lauseen sovelluksia} tarkastellaan esimerkkien valossa Gaussin lauseen käyttöä,
kun halutaan johtaa fysiikan osittaisdifferentiaaliyhtälöitä tai fysikaalisten vektorikenttien
jatkuvuusehtoja materiaalirajapinnoilla. 

Luvussa \ref{stokesin lause} johdetaan Stokesin lause ensin tasoon rajoittuen. Yleistettäessä
tulos koskemaan avaruuden pintoja tarvitaan pinnan \kor{suunnistuvuuden} käsite. Luvussa 
\ref{pyörteetön vektorikenttä} ratkaistaan Stokesin lauseen avulla fysiikassa keskeinen 
vektorikenttiä koskeva kysymys: Millä ehdoilla pyörteetön kenttä on gradienttikenttä eli 
lausuttavissa skalaaripotentiaalin avulla? 
