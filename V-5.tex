\section{Ääriarvot. Sileys} \label{ääriarvot}
\alku

Derivaatta on mitä mainioin työkalu, kun halutaan luonnehtia tavallisia reaalifunktioita,
jotka useimmiten ovat derivoituvia 'melkein kaikkialla'. Tässä ja seuraavassa luvussa
tarkastellaan derivaatan käyttöä funktiotutkimuksessa.

\subsection*{Paikalliset ääriarvot}

Aloitetaan määritelmästä.
\begin{Def} \label{paikallinen ääriarvo}
\index{paikallinen maksimi, minimi, ääriarvo|emph}
\index{maksimi (funktion)!a@paikallinen|emph} 
\index{minimi (funktion)!a@paikallinen|emph}
\index{zyzy@ääriarvo (paikallinen)|emph}
\index{suhteellinen ääriarvo|emph}
\index{oleellinen ääriarvo(piste)|emph}
\index{epzyozy@epäoleellinen ääriarvo(piste)|emph}
Funktiolla $f:\DF_f\kohti\R$, $\DF_f\subset\R$, on pisteessä $c\in\DF_f$ \kor{paikallinen} eli
\kor{suhteellinen ääriarvo}, eli $c$ on $f$:n \kor{paikallinen ääriarvopiste} 
(ääri\-arvokohta), jos jollakin $\delta>0$ ja $\forall x\in\R$ pätee
$(c-\delta,c+\delta)\subset\DF_f$ ja
\begin{align*}
\text{joko:}\quad 0<\abs{x-c}<\delta \ &\impl \ f(x) \le f(c), \\
 \text{tai:}\quad 0<\abs{x-c}<\delta \ &\impl \ f(x) \ge f(c).
\end{align*}
Edellisessä tapauksessa on kyseessä \kor{paikallinen maksimi}, jälkimmäisessä 
\kor{paikallinen minimi}. Jos lisäksi jollakin $\delta>0$ on $f(x) \neq f(c)$ aina
kun $0<\abs{x-c}<\delta$, niin ääriarvo (ääriarvopiste) on \kor{oleellinen}, muulloin
\kor{epäoleellinen}.
\end{Def}
\begin{Exa} Funktio
\[
f(x)=\begin{cases}
     \,\sin^2\dfrac{1}{x}\,, &\text{kun}\ x \neq 0, \\[2mm] \,0, &\text{kun}\ x=0
     \end{cases}
\]
saavuttaa pienimmän arvonsa $f_{min}=0$ pisteessä $x=0$ sekä pisteissä 
\[
a_n=\frac{1}{n\pi}\,, \quad n\in\Z,\ n \neq 0.
\]
Suurimman arvonsa $f_{max}=1$ funktio saavuttaa pisteissä 
\[
b_n=\frac{1}{(n+\frac{1}{2})\pi}\,, \quad n\in\Z.
\]
Pisteet $a_n$ ja $b_n$ ovat myös Määritelmän \ref{paikallinen ääriarvo} mukaisia oleellisia
paikallisia minimi- ja maksimikohtia, sillä $f$:n määrittelyn ja $\sin$-funktion tunnettujen
ominaisuuksien perusteella on $\,0<f(x)<1$ kaikissa muissa kuin mainituissa pisteissä tai
pisteessä $x=0$. Tällöin jos esim.\ $c=b_n\,,\ n\in\Z$, niin Määritelmän
\ref{paikallinen ääriarvo} paikallisen maksimin ehto on voimassa, kun valitaan $\delta$ siten,
että $0<\delta<\abs{b_{n \pm 1}-b_n}$. Pisteessä $x=0$ on $f$:llä Määritelmän
\ref{paikallinen ääriarvo} mukainen epäoleellinen paikallinen minimi, sillä jos valitaan mikä
tahansa $\delta>0$, niin $a_n\in(0,\delta)$ kun $n>(\pi\delta)^{-1}$, jolloin välillä
$(0,\delta)$ on aina pisteitä, joissa $f(x)=f(0)$. \loppu
\end{Exa}
Seuraavan lauseen mukaan derivoituvan funktion paikallinen ääriarvokohta on välttämättä myös
derivaatan nollakohta. Tähän liittyen sanotaankin derivaatan nollakohtia funktion
\index{kriittinen piste}%
\kor{kriittisiksi pisteiksi}.
\begin{Lause} \label{ääriarvolause}
Jos $c\in\DF_f$ on $f$:n paikallinen ääriarvopiste ja $f$ on derivoituva $c$:ssä, 
niin $f'(c)=0$.
\end{Lause}
\tod Väittämän loogisesti ekvivalentti muoto on: Jos $f$ on $c$:ssä derivoituva ja 
$f'(c) \neq 0$, niin $c$ ei ole $f$:n paikallinen ääriarvopiste. Todistetaan väittämä tässä
muodossa, eli oletetaan, että $f'(c) = k \neq0$. Derivaatan määritelmän perusteella
(vrt.\ Luku \ref{derivaatta}) $f$ on tällöin määritelty välillä $(c-\delta,c+\delta)$ jollakin
$\delta>0$ ja ko. välillä pätee 
\[
f(x)=f(c)+k(x-c)+g(x),
\]
missä $\lim_{x\kohti c} g(x)/(x-c) =0$. Tällöin koska $k \neq 0$, niin Lauseen
\ref{approksimaatiolause} mukaan jollakin $\delta>0$ (vastaten valintaa $\eps=|k|/2>0$
ko.\ lauseessa) pätee myös
\[
\abs{g(x)/(x-c)} \le \tfrac{1}{2}\abs{k}\,\ 
   \impl\,\ \abs{g(x)} \le \tfrac{1}{2}\abs{k}\abs{x-c}, \quad \text{kun}\ 0<\abs{x-c}<\delta.
\] 
Näin ollen jos $k>0$, niin kolmioepäyhtälön nojalla
\[
\begin{cases}
\,f(x)-f(c)\ \ge\ k(x-c) - \frac{k}{2}\,\abs{x-c}
           \ =\ \frac{k}{2}\,(x-c)\ >\ 0 \quad \forall x\in(c,c+\delta), \\
\,f(x)-f(c)\ \le\ k(x-c) + \frac{k}{2}\,\abs{x-c} 
           \ =\ \frac{k}{2}\,(x-c)\ <\ 0 \quad \forall x\in(c-\delta,c).
\end{cases}
\]
Jos $k<0$, päätellään samalla tavoin, että $f(x)-f(c)<0$ kun $x \in (c,c+\delta)$ ja 
$f(x)-f(c)>0$ kun $x \in (c-\delta,c)$. Määritelmän \ref{paikallinen ääriarvo} mukaan 
kummassakaan tapauksessa $c$ ei ole $f$:n paikallinen ääriarvopiste. \loppu
\jatko \begin{Exa} (jatko) Esimerkin funktiolle pätee
\[
f'(x) = -\frac{2}{x^2}\,\sin\frac{1}{x}\cos\frac{1}{x}\,, \quad \text{kun}\ x \neq 0,
\]
joten $f'(a_n)=f'(b_n)=0$ Lauseen \ref{ääriarvolause} väittämän mukaisesti. Pisteesä $x=0$ ei
$f$ ole derivoituva --- tämäkin on sopusoinnussa Lauseen \ref{ääriarvolause} kanssa. \loppu
\end{Exa}
\begin{Exa} \label{kriittiset pisteet: esim} Rationaalifunktio $f(x)=x^2/(x+1)$ on derivoituva
koko määrittelyjoukossaan, joten $f$:n mahdolliset paikalliset ääriarvokohdat ovat väistämättä
kriittisiä pisteitä. Koska
\[
f'(x) \,=\, \frac{2x}{1+x}-\frac{x^2}{(1+x)^2} 
      \,=\, \frac{x^2+2x}{(x+1)^2}\,, \quad x \neq -1,
\]
niin kriittiset pisteet ovat $x=-2$ ja $x=0$. Osoittautuu, että edellisessä pisteessä $f$:llä
on paikallinen maksimi, jälkimmäisessä paikallinen minimi. (Ks.\ seuraavan luvun Esimerkki
\ref{monotonisuus: esim}.) \loppu
\end{Exa}

\subsection*{Ääriarvotehtävät}
\index{zyzy@ääriarvotehtävä|vahv}

Kysymys funktion paikallisista ääriarvokohdista nousee yleensä esiin osatehtävänä, kun halutaan
ratkaista \kor{ääriarvotehtävä}: On annettu funktio $f$ ja joukko $A\subset\DF_f$, ja haluttaa
määrittää $f$:n suurin ja/tai pienin arvo $A$:ssa. Sikäli kuin tehtävä ratkeaa, eli 
suurin/pienin arvo on löydettävissä, sanotaan ko.\ arvoa $f$:n
\index{maksimi (funktion)!b@absoluuttinen} \index{minimi (funktion)!b@absoluuttinen}
\index{absoluuttinen maksimi, minimi}%
\kor{absoluuttiseksi} maksimiksi/minimiksi $A$:ssa. Absoluuttinen maksimi ja minimi löytyvät
aina, jos joukko $A$ on \pain{äärellinen}, sillä tällöin kyse on vain valinnasta äärellisessä
joukossa. Jos $A$ on ääretön joukko, esim.\ väli, on tilanne ongelmallisempi. Rajoitutaan tässä
sovelluksissa usein esiintyvään perustehtävään, jossa $A$ on \pain{sul}j\pain{ettu} \pain{väli}
ja $f$ on ko.\ \pain{välillä} j\pain{atkuva} Määritelmän \ref{jatkuvuus välillä} mukaisesti.
Tällöin ääriarvotehtävän ratkeavuuden takaa kummankin ääriarvon osalta Weierstrassin lause
(Lause \ref{Weierstrassin peruslause}).

Jatkuvaa funktiota $f$ ja suljettua väliä $[a,b]$ koskevan ääriarvotehtävän ratkaiseminen
helpottuu käytännössä huomattavasti, jos $f$ on paitsi jatkuva myös derivoituva tai
'melkein derivoituva' välillä $(a,b)$. Tulos on muotoiltavissa Lauseen \ref{ääriarvolause}
korollaarina seuraavasti.
\begin{Kor} \label{ääriarvokorollaari} Olkoon $f$ on jatkuva välillä $[a,b]$ ja olkoon
$X\subset[a,b]$ joukko, jolle pätee
\begin{align*}
&\text{1.} \ \ a \in X\ \text{ja}\ b \in X, \\
&\text{2.} \,\ \{c\in(a,b) \mid \text{$f$ derivoituva $c$:ssä ja $f'(c)=0$}\} \subset X, \\
&\text{3.} \,\ \{c \in (a,b) \mid \text{$f$ ei derivoituva $c$:ssä}\} \subset X.
\end{align*}
Tällöin $\D\ \underset{x\in [a,b]}{\max/\min} \; f(x)=\underset{x\in X}{\max/\min} \; f(x)$.
\end{Kor}
Korollaarin \ref{ääriarvokorollaari} ehdot täyttävä joukko $X$ siis sisältää välin $[a,b]$
päätepisteet ja lisäksi kaikki välillä $(a,b)$ olevat $f$:n kriittiset pisteet sekä pisteet,
joissa $f$ ei ole derivoituva. Yleensä joukko $X$ voidaan valita niin, että se on äärellinen,
jolloin $f$:n maksimi- ja minimiarvojen haku pelkistyy korollaarin mukaisesti valinnaksi
äärellisessä joukossa $X$. --- Huomattakoon, että joukkoon $X$ (sikäli kuin äärellinen) voidaan
sen määrittelyn mukaisesti sisällyttää myös sellaisia pisteitä, joissa $f$:n derivoituvuus
on pelkästään epäilyksen alaista. Haluttaessa vain selvittää funktion maksimi- ja minimikohdat
ja -arvot välillä $[a,b]$ ei tarkempaa tutkimusta 'epäillyistä' tarvita. Avoimeksi voidaan myös
jättää kysymys, ovatko joukkoon $X$ sisällytettävät pisteet $f$:n paikallisia ääriarvokohtia
vai eivät.
\begin{Exa} \label{ääriarvoesimerkki} Määritä funktion $f(x) = \max\,\{2-x,\,6x-3x^2\}$ 
maksimi- ja minimikohdat ja -arvot välillä $[0,3]$. 
\end{Exa}
\ratk Kysessä on jatkuva ja paloittain polynomiarvoinen funktio. Koska \newline
$2-x = 6x-3x^2\ \ekv\ 3x^2-7x+2=0\ \ekv\ x=\tfrac{1}{3}\ \tai\ x=2$, niin päätellään, että
\[
f(x) = \begin{cases} 
       \,6x-3x^2,\ &\text{kun}\,\ \tfrac{1}{3} \le x \le 2, \\ 
       \,2-x,\     &\text{kun}\,\ x \le \tfrac{1}{3}\,\ \text{tai}\,\ x \ge 2. 
       \end{cases}
\]
Väleillä $(0,\tfrac{1}{3})$ ja $(2,3)$ on $f'(x)=-1 \neq 0$. Välillä $(\tfrac{1}{3},2)$ on
$f'(x)=6-6x$, joten tällä välillä on $f$:llä kriittinen piste $c=1$. Pisteissä
$x=\tfrac{1}{3}$ ja $x=2$ ei $f$ mahdollisesti (eikä todellisuudessakaan) ole derivoituva,
muissa välin $(0,3)$ pisteissä on, joten Korollaarissa \ref{ääriarvokorollaari} voidaan valita
$X=\{0,\tfrac{1}{3},1,2,3\}$. Laskemalla $f$:n arvot näissä pisteissä todetaan, että
$f_{max}=f(1)=3$, $f_{min}=f(3)=-1$. \loppu

Esimerkissä mahdollisia paikallisia ääriarvokohtia välillä $(0,3)$ ovat Lauseen 
\ref{ääriarvolause} mukaan pisteet $\tfrac{1}{3},1,2$. Funktion lähempi tutkimus näiden
pisteiden ympäristössä (algebran keinoin tai jatkossa esitettävin menetelmin) osoittaa, että
$c=\tfrac{1}{3}$ on oleellinen paikallinen minimi, $c=1$ on oleellinen paikallinen maksimi ja
$c=2$ ei ole paikallinen ääriarvokohta.

\subsection*{Toispuoliset derivaatat}
\index{derivaatta!toispuolinen|vahv}
\index{toispuolinen derivaatta|vahv}

Jos Esimerkissä \ref{ääriarvoesimerkki} tarkkaillaan niitä pisteitä, joissa $f$ ei ole 
derivoituva, nähdään että $f$:llä on näissäkin pisteissä \kor{toispuoliset} (engl. one-sided) 
derivaatat seuraavan määritelmän mielessä (vrt. toispuoliset raja-arvot Luvussa 
\ref{funktion raja-arvo}).
\begin{Def} \label{toispuoliset derivaatat} \index{derivoituvuus!a@vasemmalta, oikealta|emph}
\index{vasemmalta derivoituva|emph} \index{oikealta derivoituva|emph}
Funktio $f:\DF_f\kohti\R$, $\DF_f\subset\R$, on pisteessä $x\in\DF_f$
\kor{vasemmalta derivoituva}, jos $(x-\delta,x]\subset\DF_f$ jollakin $\delta>0$ ja $\exists$
raja-arvo
\[
\dif_-f(x)=\lim_{\Delta x\kohti 0^-} \frac{f(x+\Delta x)-f(x)}{\Delta x},
\]
ja \kor{oikealta derivoituva}, jos $[x,x+\delta)\subset\DF_f$ jollakin $\delta>0$ ja $\exists$ 
raja-arvo
\[
\dif_+f(x)=\lim_{\Delta x\kohti 0^+} \frac{f(x+\Delta x)-f(x)}{\Delta x}.
\]
\end{Def}
Määritelmien \ref{toispuoliset derivaatat} ja \ref{derivaatan määritelmä} perusteella $f$ on
pisteessä $x$ derivoituva täsmälleen kun $f$ on $x$:ssä sekä vasemmalta että oikealta
derivoituva ja $\dif_-f(x)=\dif_+f(x)$ ($=f'(x)$).

Jos $f$ on pisteessä $c$ jatkuva ja lisäksi sekä vasemmalta että oikealta derivoituva, niin 
pätee (vrt.\ johdatus derivaattaan Luvussa \ref{derivaatta})
\[
\begin{cases}
\,f(x)=f(x)+\dif_-f(c)(x-c)+g(x), \quad \text{kun}\ x \in (c-\delta,c], \\
\,f(x)=f(x)+\dif_+f(c)(x-c)+g(x), \quad \text{kun}\ x \in [c,c+\delta),
\end{cases}
\]
missä $g(x)/(x-c) \kohti 0$, kun $x \kohti c^-$ tai $x \kohti c^+$. Tästä nähdään oikeaksi
\begin{Lause} \label{ääriarvolause 2} Jos $f$ on pisteessä $c$ jatkuva ja lisäksi sekä 
vasemmalta että oikealta derivoituva, niin pätee
\begin{itemize}
\item[(a)] $\dif_-f(c)\,\dif_+f(c)<0 \ \impl \ f$:llä on pisteessä $c$ paikallinen ääriarvo,
joka on
\begin{itemize}
\item[-] oleellinen maksimi,\,  jos $\dif_-f(c)>0$ ja $\dif_+f(c)<0$,
\item[-] oleellinen minimi, \ \ jos $\dif_-f(c)<0$ ja $\dif_+f(c)>0$.
\end{itemize}
\item[(b)] $\dif_-f(c)\,\dif_+f(c)>0 \ \impl\ c$ ei ole $f$:n paikallinen ääriarvopiste.
\end{itemize}
\end{Lause}
\jatko \begin{Exa} (jatko) Esimerkissä on $\dif_-f(\tfrac{1}{3})=-1,\ \dif_+f(\tfrac{1}{3})=4$
ja $\dif_-f(2)=-6,\ \dif_+f(2)=-1$, joten Lauseen \ref{ääriarvolause 2} mukaan $f$:llä on
pisteessä  $x=\tfrac{1}{3}$ oleellinen paikallinen minimi ja $x=2$ ei ole paikallinen
ääriarvokohta. \loppu
\end{Exa}
Jos $f$ on pisteessä $c$ derivoituva ja $f'(c) \neq 0$, niin
$\dif_-f(c)\,\dif_+f(c) = [f'(c)]^2 > 0$, joten Lauseen \ref{ääriarvolause 2} väittämä (b)
sisältää myös Lauseen \ref{ääriarvolause} väittämän muodossa
\[
f'(c) \neq 0 \qimpl \text{$f$:llä ei ole paikallista ääriarvoa $c$:ssä}.
\]
Jos $\dif_-f(c)=0$ tai $\dif_+f(c)=0$ (myös kun $f'(c)=0$), voi $c$ olla paikallinen
ääriarvopiste tai ei, riippuen tapauksesta.
\begin{Exa} Funktioille
$\displaystyle{\
f(x)=\begin{cases}
\,x^2,  &x<0, \\
\,x,    &x\ge 0
     \end{cases} \quad \text{ja} \quad
g(x)=\begin{cases}
-x^2, &x<0, \\
\,x,    &x\ge 0
     \end{cases}}$

pätee $\dif_-f(0)=\dif_-g(0)=0$ ja $\dif_+f(0)=\dif_+g(0)=1$. Funktiolla $f$ on origossa
(oleellinen) paikallinen minimi, $g$:llä ei ole paikallista ääriarvoa origossa.
\begin{figure}[H]
\setlength{\unitlength}{1cm}
\begin{center}
\begin{picture}(10,5.5)(0,-2.5)
\multiput(0,0)(6,0){2}{
\put(0,0){\vector(1,0){4}} \put(3.8,-0.4){$x$}
\put(2,-2){\vector(0,1){4}} \put(2.2,1.8){$y$}}
\curve(0.586,2,1,1,1.5,0.25,2,0) \put(2,0){\line(1,1){2}}
\curve(6.586,-2,7,-1,7.5,-0.25,8,0)\drawline(8,0)(10,2)
\put(1.3,-3){$y=f(x)$} \put(7.3,-3){$y=g(x)$}
\end{picture}
\end{center}
\end{figure}
\end{Exa}

\subsection*{Sileys}
\index{sileys(aste)|vahv}

Seuraava määritelmä yhdistävää suljetulla välillä jatkuvuuden 
(Määritelmä \ref{jatkuvuus välillä}) ja derivoituvuuden käsitteitä.
\begin{Def} \label{sileys}
\index{jatkuvasti derivoituvuus|emph} \index{derivoituvuus!b@jatkuvasti derivoituvuus|emph}
Funktio $f:\DF_f\kohti\R$, $\DF_f\subset\R$ on \kor{jatkuvasti derivoituva} 
(engl.\ continuously differentiable) \kor{suljetulla välillä} $[a,b]\subset\DF_f$, jos $f$ on
derivoituva välillä $(a,b)$, oikealta derivoituva pisteessä $a$, vasemmalta derivoituva 
pisteessä $b$, ja derivaatta $f'$, määriteltynä toispuolisena välin $[a,b]$ päätepisteissä, on
välillä $[a,b]$ jatkuva funktio. Yleisemmin jos $m\in\N$, niin $f$ on välillä $[a,b]$ $m$ 
\kor{kertaa jatkuvasti derivoituva}, jos $f$ on $m$ kertaa derivoituva välillä $(a,b)$, $m$ 
kertaa oikealta derivoituva pisteessä $a$, $m$ kertaa vasemmalta derivoituva pisteessä $b$, ja
derivaatat $f^{(k)}$, määriteltynä toispuolisina välin päätepisteissä, ovat välillä $[a,b]$
jatkuvia, kun $\,k=1 \ldots m$. Edelleen jos $A\subset\DF_f$ on puoliavoin tai avoin väli,
niin sanotaan, että $f$ on $m$ kertaa jatkuvasti derivoituva $A$:ssa, jos $f$ on $m$ kertaa
jatkuvasti derivoituva jokaisella $A$:n suljetulla osavälillä. 
\end{Def}
Sovellettaessa määritelmää suljetun välin tapauksessa, kun $m \ge2$, ajatellaan derivaatat
$f^{(k)}$, $k=1\ldots m-1$ määritellyksi välin päätepisteissä palautuvasti (toispuolisina)
alkaen indeksistä $k=1$. Indeksiä $m=0$ vastaava termi '$0$ kertaa jatkuvasti derivoituva'
tulkitaan pelkäksi jatkuvuudeksi. 
\begin{Exa} \label{sileysesimerkki} Olkoon $n\in\N$ ja tarkastellaan funktiota
\[
f(x) = \begin{cases} \,0, &\text{kun } x\leq 0, \\ \,x^n, &\text{kun } x>0. \end{cases}
\]
Derivoimalla pistessä $x=0$ toispuolisesti todetaan, että $f$ on tässä pistessä (ja siis koko
$\R$:ssä) $n-1$ kertaa derivoituva ja pätee
\[
f^{(k)}(x) = \begin{cases}
             \,0,                           &\text{kun}\ x \le 0, \\[2mm]
             \,\dfrac{n!}{(n-k)!}\,x^{n-k}, &\text{kun}\ x>0, \quad k=1 \ldots n-1.
             \end{cases}
\]
Koska derivaatat $f^{(k)}(0)=0,\ k=1 \ldots n-1$ ovat $\R$:ssä jatkuvia ja koska $f^{(n-1)}$ 
ei ole derivoituva pisteessä $x=0$, niin päätellään, että $f$ on $\R$:ssä täsmälleen $n-1$
kertaa (ei $n$ kertaa) jatkuvasti derivoituva. \loppu
\end{Exa}
Jos $f$ on $m$ kertaa mutta ei $m+1$ kertaa jatkuvasti derivoituva välillä $A$, niin 
indeksiä $m$ kutsutaan usein $f$:n \kor{sileysasteeksi} (engl.\ degree of smoothness), ja 
voidaan myös sanoa, että $f$ on \kor{sileä astetta $m$} ko.\ välillä. Tällöin 'sileä astetta
nolla' tarkoittaa siis pelkkää jatkuvuutta. Esimerkissä $f$:n sileysaste ($\R$:ssä) on
$m=n-1$. Jos sileysasteella ei ole ylärajaa, ts.\ $f$ on 'äärettömän sileä', niin 
sileysasteeksi voidaan merkitä $m=\infty$. Termillä \kor{sileä} (ilman lisämääreitä) on usein
juuri tämä merkitys. Esimerkissä on $m=\infty$ esim.\ välillä $[0,\infty)$.
\begin{Exa} Polynomi on sileä (= 'äärettömän sileä') jokaisella välillä, eli $\R$:ssä, samoin
trigonometriset funktiot $\sin$, $\cos$ ja $\Arctan$. Trigonometrinen funktio $\,\tan\,$ on
sileä väleillä $((n-\tfrac{1}{2})\pi,(n+\tfrac{1}{2})\pi),\ n\in\Z$. \loppu
\end{Exa}
\begin{Exa} Potenssisarjan summana määritelty funktio on sileä välillä $(-\rho,\rho)$
(eli väleillä $[a,b] \subset (-\rho,\rho)$), missä $\rho=$ sarjan suppenemissäde, vrt.\ Luku 
\ref{derivaatta}. \loppu
\end{Exa}

Edellä Esimerkissä \ref{sileysesimerkki} määritelty funktio $f$ on
\index{paloittainen!c@sileys}
\kor{paloittain sileä} (engl.\ piecewise smooth) millä tahansa indeksillä $m\in\N$ mitaten.
Tämä tarkoittaa, että olipa $m\in\N$ mikä tahansa, niin $f$ täyttää seuraavan määritelmän ehdot.
\begin{Def} \label{paloittainen sileys} \index{jatkuvasti derivoituvuus|emph}
\index{derivoituvuus!b@jatkuvasti derivoituvuus|emph}
Funktio $f:\DF_f\kohti\R$, $\DF_f\subset\R$ on \kor{$m$ kertaa paloittain jatkuvasti
derivoituva} välillä $[a,b]$, jos $\exists$ pisteet $c_k$, $k=0 \ldots n$, $n\in\N$ ja 
funktiot $f_k$, $k=1 \ldots n$ siten, että pätee
\begin{itemize}
\item[(i)]   $a=c_0<c_1<\ldots<c_n=b$\, ja $\,(c_{k-1},c_k)\subset\DF_f, \quad k=1 \ldots n$,
\item[(ii)]  $[c_{k-1},c_k]\subset\DF_{f_k}$ ja $f_k$ on $m$ kertaa jatkuvasti derivoituva
             välillä $[c_{k-1},c_k]$, $k=1\ldots n$,
\item[(iii)] $f(x)=f_k(x)$, \,kun $x\in (c_{k-1},c_k), \quad k=1 \ldots n$.
\end{itemize}
\end{Def}
\begin{Exa} Jos $a<0$ ja $b>0$, niin Esimerkin \ref{sileysesimerkki} funktiolle Määritelmän 
\ref{paloittainen sileys} ehdot ovat voimassa jokaisella $m\in\N$, kun asetetaan $n=1$,
$c_1=0$, $f_1(x)=0$ ja $f_2(x) = x^n$. \loppu 
\end{Exa}
Jos $f$ on välillä $[a,b]$ paloittain sileä astetta $m$ millä tahansa $m\in\N$, kuten 
Esimerkissä \ref{sileysesimerkki}, niin voidaan sanoa yksinkertaisesti, että $f$ on 
\kor{paloittain sileä} ilman lisämääreitä. Myös Esimerkin \ref{ääriarvoesimerkki} funktio on 
tätä tyyppiä.
\begin{Exa}
Funktio
\[
f(x)=\begin{cases}
x^3, &x<0, \\
x^5\sqrt{x}, &x\geq 0
\end{cases}
\]
on välillä $[-1,1]$ sileä astetta $m=2$ (täsmälleen) ja paloittain sileä astetta $m=5$ 
(täsmälleen). Tässä paloittaista sileyttä rajoittaa, että $f$ on origossa vain viidesti 
derivoituva oikealta. \loppu
\end{Exa}

\Harj
\begin{enumerate}

\item
Tutki origon laatu mahdollisena paikallisena ääriarvopisteenä:
\begin{align*}
&\text{a)}\ \ f(x)=100+x^{99}-x^{98} \qquad
 \text{b)}\ \ f(x)=x^4-x^3\cos x \\
&\text{c)}\ \ f(x)=\begin{cases}
                  x^3\sin\dfrac{1}{x}\,, &\text{kun}\ x \neq 0, \\[2mm]
                  0,                     &\text{kun}\ x=0
                  \end{cases} \quad\
 \text{d)}\ \ f(x)=\begin{cases} 0.000001, &\text{kun}\ x=0, \\[2mm]
                  x\cos\dfrac{1}{x}\,,    &\text{kun}\ x \neq 0
                  \end{cases}
\end{align*}

\item
Määritä seuraavien funktioiden maksimi- ja minimiarvot annetulla välillä sekä pisteet
joissa maksimi/minimi saavutetaan:
\begin{align*}
&\text{a)}\ f(x)=(x-1)^3(x+1)^3, \quad \text{väli}\ [-2,4] \\[4mm]
&\text{b)}\ f(x)=\abs{x^5-80x+1}, \quad \text{väli}\ [-1,1] \\[2mm]
&\text{c)}\ f(x)=\frac{2-x}{5-4x+x^2}\,,\quad \text{väli}\ [-100,2] \\
&\text{d)}\ f(x)=\begin{cases}
                 8x^2-16x-14, &\text{kun}\ x \le 2, \\ 3x^2-24x+22, &\text{kun}\ x>2,
                 \end{cases} \quad \text{väli}\ [0,5] \\[1mm]
&\text{e)}\ f(x)=\min\{2x^3-2x,\,3-2x\}, \quad \text{väli}\ [-1,2] \\[5mm]
&\text{f)}\ f(x)=\abs{\sin x}-\cos x, \quad \text{väli}\ [0,2\pi] \\[5mm]
&\text{g)}\ f(x)=x+5\sin x, \quad \text{väli}\ [0,2\pi]
\end{align*}

\item
Mikä on lausekkeen $\sqrt{x}+2\sqrt{y}$ pienin arvo, kun $x,y \ge 0$ ja $x+y=5/6$\,?

\item
Suoran tien varressa, kohtisuorassa tietä vastaan, on mainostaulu, jonka leveys on $10$ metriä
ja lähin etäisyys tiellä kulkevien ajoradasta on $20$ m. Mikä on suurin kulma, jossa tiellä
liikkuja taulun näkee?

\item
Piste $P=(x,y)$ sijaitsee Cartesiuksen lehdellä $\,S: x^3+y^3=3xy$. Mikä on suurin mahdollinen
$x$:n arvo, jos $y \ge 0$\,?

\item 
Määritä \vspace{1mm}\newline
a) funktion $f(x,y)=(2x-y)(x+y-1)(x-y-1)$ maksimi- ja minimiarvot janalla, jonka päätepisteet
ovat $(-1,0)$ ja $(3,1)$, \vspace{1mm}\newline
b) funktion $f(x,y)=\abs{x^2+2y}+2x^2$ pienin arvo käyrällä $S:\ xy=1$, \vspace{1mm}\newline
c) funktion $f(x,y,z)=x+y+z(2x+y+z)(x+3y+z)(x+y+4z)$ pienin arvo suoralla $S:\ x=2y=3z$.

\item
Funktio $f$ on määritelty yksikkökiekon neljänneksessä 
$A=\{(x,y)\in\R^2 \mid x^2+y^2 \le 1\ \&\ x \ge 0\ \&\ y \ge 0\}$ seuraavasti
(napakoodinaatit!): $f(r,\phi)=(3r^2-2r)(2\phi^2-3\phi +1)$. Missä $A$:n pisteissä $f$ 
saavuttaa suurimman ja missä pienimmän arvonsa?

\item
Määritä napa- tai pallokoordinaatistoon siirtymällä seuraavien funktioiden maksimi- ja 
minimiarvot annetussa joukossa $A$. Anna myös karteesisessa koordinaatistossa pisteet, 
joissa nämä arvot saavutetaan.
\begin{align*}
&\text{a)}\ f(x,y)=x^3y^4,\ \ A=\{(x,y) \mid x^2+y^2 \le 1\} \\
&\text{b)}\ f(x,y)=x^2y^4,\ \ A=\{(x,y) \mid x^2+y^2 \le 25\} \\
&\text{c)}\ f(x,y,z)=xyz^3,\ \ A=\{(x,y,z) \mid x^2+y^2+z^2 \le 1\}
\end{align*}

\item
Mikä on funktion 
\[
f(x)=|x-3|^{111/11} + (x+1)^3|x+1| +\max\{1,x^2-2x+2\}
\]
sileysaste välillä \ a) $[-2,0]$, \ b) [0,2], \ c) [2,4]\,?

\item
Määritä seuraavien funktioiden sileysasteet välillä $[-1,1]$:
\begin{align*}
&\text{a)}\,\ x^2\abs{\sin x} \qquad
 \text{b)}\,\ \abs{x}^3(1-\cos x) \qquad
 \text{c)}\,\ \sin^2 x\abs{\tan x} \\
&\text{d)}\,\ f(x) = \begin{cases} 
                     \,1-\tfrac{1}{2}x^2, &\text{kun}\ x<0, \\
                     \,\cos x,            &\text{kun}\ x \ge 0
                     \end{cases} \quad\ \
 \text{e)}\,\ f(x) = \begin{cases}
                     \,\sin x,            &\text{kun}\ x \le 0, \\
                     \,x-\tfrac{1}{6}x^3, &\text{kun}\ x>0
                     \end{cases}
\end{align*} 

\item
Olkoon $f_1(x)=2x^3-7x^2+9x$ ja $f_2(x)=ax^2+bx+c$. Määritä kertoimet $a,b,c$ siten, että 
funktio
\[
f(x)=\begin{cases}
     \,f_1(x),  &\text{kun}\ x \le 1, \\ \,f_2(x),  &\text{kun}\ x>1
     \end{cases}
\]
on kahdesti jatkuvasti derivoituva välillä $[0,2]$. Piirrä samaan kuvaan $f$:n kuvaaja ko.\
välillä sekä katkoviivalla $f_2$:n kuvaaja välillä $[0,1]$ ja $f_1$:n kuvaaja välillä $[1,2]$.

\item (*)
a) Funktiosta $f: \R \kohti \R$ tiedetään, että $f$ on derivoituva $\R$:ssä ja että
\[
\lim_{x \kohti -\infty} f(x)=A_-\,, \quad \lim_{x \kohti\infty} f(x)=A_+\,,
\]
missä $A_-\in\R$ ja $A_+\in\R$. Näytä, että $f$ saavuttaa $\R$:ssä absoluuttisen
maksimiarvon täsmälleen kun jossakin $f$:n kriittisessä pisteessä $c$ on \newline
$f(c) \ge \max\{A_-\,,\,A_+\}$. \ b) Millä $a$:n arvoilla funktio
\[
f(x)=\frac{x^4+ax^2}{(x^2+7)^2}
\]
saa absoluuttisen maksimiarvon jollakin $x\in\R$\,?

\item (*)
Määritä funktion $f(x)=4\abs{\cos x}-3\sin x+2\cos x\,$ kriittiset pisteet, derivaatan
epäjatkuvuuskohdat, paikalliset ääriarvokohdat ja absoluutiset maksimi- ja minimiarvot
välillä $[0,2\pi]$. Hahmottele käyrän $y=f(x)$ kulku.

\item (*)
Määrittele välillä $(0,\infty)$ funktio $g$ siten, että funktion $\,f(x)=x+a\sin x\,$
pienin arvo välillä $[0,a]$ on $f_{min}=g(a)$. Hahmottele $g$:n kuvaaja.

\item (*) \label{H-V-5: jatkaminen polynomeilla}
a) Olkoon $p_1$ ja $p_2$ polynomeja astetta $\le n$. Näytä, että jos funktio
\[
f(x)=\begin{cases}
     \,p_1(x),  &\text{kun}\ x<a,  \\ \,p_2(x),  &\text{kun}\ x \ge a
     \end{cases}
\]
on $n$ kertaa jatkuvasti derivoituva välillä $[a-1,a+1]$, niin $p_1=p_2$.\vspace{1mm}\newline
b) Funktio $f$ olkoon $m$ kertaa jatkuvasti derivoituva välillä $[a,b]$. Näytä, että on
olemassa yksikäsitteiset polynomit $p_1$ ja $p_2$ astetta $\le m$ siten, että funktio
\[
g(x) = \begin{cases}
       \,p_1(x), &\text{kun}\ x<a, \\ 
       \,f(x), &\text{kun}\ x\in[a,b], \\ 
       \,p_2(x), &\text{kun}\ x>b
       \end{cases}
\]
on $m$ kertaa jatkuvasti derivoituva $\R$:ssä.

\item (*) \index{zzb@\nim!Vasikka-aitaus}
(Vasikka-aitaus) Maanviljelijä haluaa aidata navettansa viereen suorakulmion muotoisen 
aitauksen vasikoiden laidunmaaksi. Aitauksen mitat valitaan aidan kokonaispituuden ($x$ m) 
funktiona siten, että laitumen pinta-ala on mahdollisimman suuri ja navetan $40$ m:n pituinen
seinä käytetään hyväksi mahdollisimman hyvin. Näin suunnitellussa aitauksessa olkoon
$f(x)$ navetan seinän suuntaisen ja seinää vastaan kohtisuoran sivun pituuksien suhde.
Määritä $f(x)$ välillä $(0,\infty)$. Mikä on $f$:n sileysaste välillä $[0,\infty)$, kun
asetetaan $f(0)=f(0^+)$\,?

\item (*) \index{zzb@\nim!Pisin heitto?}
(Pisin heitto?) Jos ilmanvastusta ei huomioida, niin ilmaan heitetyn kappaleen lentorata
noudattaa heittoparaabelia 
\[
y=h+kx-(1+k^2)\frac{x^2}{2a}\,,
\]
missä $x$ mittaa vaakasuoraa etäisyyttä lähtöpisteestä, $y$ korkeutta maan
pinnan tasosta ja vakiot $h$, $k$ ja $a$ ovat yksittäiselle heitolle ominaisia
(ks.\ Luku \ref{parametriset käyrät}, Esimerkki \ref{heittoparaabeli}). Jos
$h$ ja $a$ kiinnitetään, niin millä $k$:n arvolla kappale lentää pisimmälle --- ja kuinka
pitkälle? --- ennen kuin törmää maahan?

\end{enumerate}
