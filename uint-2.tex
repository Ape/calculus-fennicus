\section{Tasointegraalien laskutekniikka} \label{tasointegraalien laskutekniikka}
\alku
\index{tasointegraali|vahv}

Tasointegraali on mahdollistaa palauttaa yhden muuttujan integrointitehtäviksi, jolloin 
integraali voidaan suotuisissa oloissa laskea käyttäen hyväksi integraalifunktioita, eli 
suljetussa muodossa. Tässä luvussa tarkastellaan tällaista laskutekniikkaa.

\index{x@$x$-projisoituva (joukko)}%
Tarkastellaan joukkoa $A\subset\R^2$ joka on nk. \kor{$x$-projisoituva}, eli muotoa
\[
A=\{(x,y)\in\R^2 \ | \ x\in [a,b], \ g_1(x)\leq y\leq g_2(x)\},
\]
missä oletetaan, että $g_1$ ja $g_2$ ovat välillä $[a,b]$ rajoitettuja. Olkoon 
$T=[a,b]\times [c,d]$ suorakulmio, joka sisältää $A$:n, ja olkoon $\mathcal{T}_h$ jokin $T$:n
jako.
\begin{figure}[H]
\begin{center}
\import{kuvat/}{kuvaUint-13.pstex_t}
\end{center}
\end{figure}
Olkoon $f$ määritelty $T$:ssä ja rajoitettu, ja tarkastellaan integraalin $\int_A f\, d\mu$
määrittävää Riemannin summaa. Olkoon $\mathcal{T}_h=\{T_{kl},\ k=1 \ldots m,\ l=1 \ldots n\}$
ja valitaan
\[
(\xi_{kl},\eta_{kl})=(\xi_k,\eta_l)\in T_{kl},\quad k=1 \ldots m,\ l=1 \ldots n.
\]
Tällöin summa muodon
\[
\sum_{T_{kl}\in\mathcal{T}_h} f_0(\xi_{kl},\eta_{kl})\mu(T_{kl})
= \sum_{k=1}^m\sum_{l=1}^n f_0(\xi_k,\eta_l)(x_k-x_{k-1})(y_l-y_{l-1}).
\]
Sikäli kuin funktio $y \map f_0(\xi_k,y)$ on Riemann-integroituva välillä $[c,d]$, on
sisemmällä summalla raja-arvo
\[
\sum_{l=1}^n f_0(\xi_k,\eta_l)(y_l-y_{l-1})\ 
                  \underset{h\kohti 0}{\Kohti}\ \int_c^d f_0(\xi_k,y)\,dy
                         = \int_{g_1(\xi_k)}^{g_2(\xi_k)} f(\xi_k,y)\,dy.
\]
Kun merkitään
\[
F(x)=\int_{g_1(x)}^{g_2(x)} f(x,y)\, dy,\quad x\in [a,b],
\]
niin päätellään, että sikäli kuin $F$ on edelleen Riemann-integroituva välillä $[a,b]$, niin
jaon $\mathcal{T}_h$ ollessa riittävän tiheä on likimäärin
\[
\sum_{T_{kl}\in\mathcal{T}_h} f_0(\xi_k,\eta_l)\mu(T_{kl}) 
            \approx \sum_{l=1}^n F(\xi_k)(x_k-x_{k-1}) \approx \int_a^b F(x)\,dx.
\]
Näyttää siis ilmeiseltä, että funktioiden $g_1$, $g_2$ ja $f$ ollessa riittävän säännöllisiä
pätee
\[
h\kohti 0 \ \impl \ 
            \sum_{T_{kl}\in\mathcal{T}_h} f_0(\xi_k,\eta_l)\mu(T_{kl})\kohti\int_a^b F(x)\,dx.
\]
Näin on johdettu (tai ainakin tehty uskottavaksi) laskukaava
\begin{equation} \label{tasopalautus-x}
\boxed{ \begin{aligned}
\ykehys\quad&A=\{(x,y)\in\R^2 \ | \ x\in [a,b], \ g_1(x)\leq y\leq g_2(x)\} \quad\\
            &\impl \quad \int_A f\,d\mu 
                         = \int_a^b\left[\int_{g_1(x)}^{g_2(x)} f(x,y)\,dy\right]dx. \quad
           \end{aligned} } 
\end{equation}
Tämän mukaan siis tasointegraali yli $x$-projisoituvan joukon on laskettavissa peräkkäisinä
\index{iteroitu integraali}%
yksiulotteisina, nk.\ \kor{iteroituina integraaleina}. Kyseessä on \kor{Fubinin lauseen} 
nimellä tunnetun väittämän eräs muoto. Lause muotoillaan täsmällisemmin ja todistetaan luvun 
lopussa.

\index{y@$y$-projisoituva joukko}%
Jos $A\subset\R^2$ on \kor{$y$-projisoituva}, ts. (vrt. kuvio vasemmalla)
\[
A=\{(x,y)\in\R^2 \ | \ y\in [a,b], \ g_1(y)\leq x \leq g_2(y)\}, \\[3mm]
\]
%\vspace{1mm}
\begin{multicols}{2} \raggedcolumns
\begin{figure}[H]
\begin{center}
\input{kuvat/kuvaUint-14.pstex_t}
\end{center}
\end{figure}
\begin{figure}[H]
\begin{center}
\input{kuvat/kuvaUint-15.pstex_t}
\end{center}
\end{figure}
\end{multicols}
niin integrointikaavassa \eqref{tasopalautus-x} on vaihdettava $x$ ja $y$, jolloin kaava saa
muodon
\begin{equation} \label{tasopalautus-y}
\int_A f\,d\mu=\int_a^b\left[\int_{g_1(y)}^{g_2(y)} f(x,y)\,dx\right]dy.
\end{equation}
Jos $A$ ei ole kumpaakaan projisoituvaa tyyppiä, niin kaavoja \eqref{tasopalautus-x} ja 
\eqref{tasopalautus-y} päästään yleensä käyttämään, kun $A$ ensin jaetaan sopiviin osiin 
(vrt.\ kuvio edellä) ja käytetään integraalin additiivisuutta:
\[
\int_A f\,d\mu=\sum_{i=1}^m \int_{A_i} f\,d\mu,\quad A
              =\left(\bigcup_{i=1}^m A_i\right)\cup B, \quad \mu(A_i\cap A_j)
              =\mu(B)=0,\ i\neq j.
\]

Iteraatiokaavojen \eqref{tasopalautus-x}--\eqref{tasopalautus-y} perusteella voidaan 
tasointegraalikin laskea suotuisissa oloissa integraalifunktioiden avulla. Olkoon esimerkiksi
$A$ $x$-projisoituva ja oletetaan, että on löydettävissä funktion $f(x,y)$ integraalifunktio
muuttujan $y$ suhteen, ts. funktio $F(x,y)$, jolle pätee
\[
\frac{\partial}{\partial y} F(x,y)=f(x,y),\quad (x,y)\in A.
\]
Tällöin kaavan \eqref{tasopalautus-x} mukaan
\[
\int_A f(x,y)\,d\mu = \int_a^b\left[\sijoitus{y=g_1(x)}{y=g_2(x)}F(x,y)\right]dx
                    =\int_a^b G(x)\,dx,
\]
missä
\[
G(x)=F(x,g_2(x))-F(x,g_1(x)).
\]
Jos edelleen on löydettävissä funktion $G$ integraalifunktio $H$, niin integraali saadaan
lasketuksi suljetussa muodossa:
\[
\int f\,d\mu=\sijoitus{a}{b} H(x)=H(b)-H(a).
\]
Laskun onnistuminen tällä tavoin riippuu siis paitsi funktiosta $f$, myös funktioista $g_1$ ja
$g_2$, eli $A$:n reunaviivan muodosta. Onnistuminen on taattu esimerkiksi silloin, kun $f$, 
$g_1$ ja $g_2$ ovat polynomeja, sillä tällöin myös $F$, $G$ ja $H$ ovat polynomeja.

Ajatellen iteraatiokaavoja \eqref{tasopalautus-x}--\eqref{tasopalautus-y} käytetään 
tasointegraaleille merkintätapoja
\[
\int_A f\, dxdy\quad\text{tai} \quad \iint_A f\, dxdy.
\]
Näissä siis kirjoitetaan $d\mu=dxdy$, mikä merkintä viittaa 'differentiaalisen suorakulmion' 
$\,[x,x+dx]\times[y,y+dy]$ pinta-alaan. Merkintää käytetään jatkossa.
\begin{Exa} Olkoon $A=[0,1]\times[0,1]$. Laske $\int_A \sqrt{\abs{x-y}}\,dxdy$. \end{Exa}
\ratk \ Kaavan \eqref{tasopalautus-x} mukaan
\begin{align*}
\int_A \sqrt{\abs{x-y}}\,dxdy 
&= \int_0^1 \left[ \int_0^1 \sqrt{\abs{x-y}}\,dy \right] dx \\
&= \int_0^1 \left[ \int_0^x \sqrt{x-y}\,dy + \int_x^1 \sqrt{y-x}\, dy \right] dx \\
&= \int_0^1 \left[ -\frac{2}{3}\sijoitus{y=0}{y=x} (x-y)^{3/2}
                   +\frac{2}{3}\sijoitus{y=x}{y=1} (y-x)^{3/2} \right] dx \\
&= \int_0^1 \frac{2}{3} \left[ x^{3/2}+(1-x)^{3/2} \right] dx \\
&= \sijoitus{0}{1} \frac{4}{15} \left[ x^{5/2}-(1-x)^{5/2} \right] 
 = \underline{\underline{\frac{8}{15}}}\,. \loppu
\end{align*}
\begin{Exa}
$A=\text{kolmio}$, jonka kärjet ovat $(0,0)$, $(1,0)$ ja $(0,1)$. Laske
\begin{multicols}{2} \raggedcolumns
\[
I=\int_A xy^2\,dxdy.
\]
\begin{figure}[H]
\begin{center}
\input{kuvat/kuvaUint-16.pstex_t}
\end{center}
\end{figure}
\end{multicols}
\end{Exa}
\ratk
\begin{align*}
I &= \int_0^1\left[\int_0^{1-x} xy^2\,dy\right] dx \\
  &= \int_0^1 \left[\sijoitus{y=0}{y=1-x} \frac{1}{3}xy^3\right] dx \\
  &= \int_0^1 \frac{1}{3}x(1-x)^3\,dx \\
  &= \sijoitus{0}{1} -\frac{1}{12}x(1-x)^4 + \int_0^1 \frac{1}{12}(1-x)^4\,dx
     \qquad \text{(osittaisintegrointi)} \\
  &= \int_0^1 \frac{1}{12}(1-x)^4\,dx = \sijoitus{0}{1} -\frac{1}{60}(1-x)^5
   = \underline{\underline{\frac{1}{60}}}\,. \loppu
\end{align*}
\begin{Exa} Laske $\mu(A)$, kun $A=\text{$R$-säteinen kiekko}$.
\end{Exa}
\ratk Kaavan \eqref{tasopalautus-x} mukaan
\begin{align*}
\mu(A) = \int_A d\mu 
      &= \int_{-R}^R \left[\int_{-\sqrt{R^2-x^2}}^{\sqrt{R^2-x^2}} dy\right] dx \\
      &= \int_{-R}^R \left[\sijoitus{-\sqrt{R^2-x^2}}{\sqrt{R^2-x^2}} y \right] dx
       = 2\int_{-R}^R \sqrt{R^2-x^2}\,dx.
\end{align*}
Sijoitus $x=R\sin t$, $dx=R\cos t\, dt$, $t\in [-\frac{\pi}{2},\frac{\pi}{2}]\,$ antaa 
\begin{align*}
2\int_{-R}^R \sqrt{R^2-x^2}\,dx &= 2R^2\int_{-\pi/2}^{\pi/2}\cos^2 t\,dt \\
                               &= R^2\sijoitus{-\pi/2}{\pi/2}(t+\sin t\cos t) 
                               = \pi R^2. \loppu
\end{align*}
Esimerkin 3 yleistyksenä nähdään kaavasta \eqref{tasopalautus-x}, että $x$-projisoituvan joukon
pinta-alamitta on laskettavissa kaavalla
\[
\mu(A)=\int_a^b [g_2(x)-g_1(x)]\,dx,
\]
joka on tuttu jo Luvusta \ref{pinta-ala ja kaarenpituus}. Kaavan käyttö vastaa pinta-alan
laskemista 'siivuttamalla' $A$ osiin muotoa 
$\Delta A=\{(x,y)\in A \ | \ x\in [x,x+\Delta x]\}$, jolloin on
$\mu(\Delta A) \approx [g_2(x)-g_1(x)]\Delta x$ pienillä $\Delta x$:n arvoilla ja olettaen,
että $g_1$ ja $g_2$ ovat jatkuvia pisteessä $x$. (Jatkuvuusoletus on lievennettävissä, vrt.\
Luku \ref{pinta-ala ja kaarenpituus}.)

\begin{figure}[H]
\begin{center}
\input{kuvat/kuvaUint-17.pstex_t}
\end{center}
\end{figure}

\subsection*{Epäoleelliset tasointegraalit}
\index{tasointegraali!a@epäoleellinen|vahv}
\index{epzyoi@epäoleellinen integraali|vahv}

Jos tasointegraalissa $\int_A f\,dxdy$ joko $A$ ei ole rajoitettu tai $f$ ei ole $A$:ssa
rajoitettu, voidaan tasointegraalin määrittelyä laajentaa vastaavaan tapaan kuin yksiulotteisen
Riemannin integraalin tapauksessa, vrt.\ Luku \ref{integraalin laajennuksia}. Jos määritelmän
laajennus tuottaa integraalille yksikäsitteisen (reaali)arvon, sanotaan jälleen, että näin
\index{suppeneminen!c@integraalin}%
määritelty \kor{epäoleellinen} integraali \kor{suppenee}. Valaistakoon asiaa ensin esimerkeillä.
\begin{Exa} Olkoon $A=[0,1]\times[0,1]$ ja $B=[0,\infty)\times[0,\infty)$. Laske
\[
\text{a)}\,\ \int_A \frac{y^2}{\sqrt{x}}\,dxdy, \qquad
\text{b)}\,\ \int_B \frac{x}{1+x^2+y^2}\,dxdy.
\]
\end{Exa}
\ratk a) \ Koska $f$ ei ole $A$:ssa rajoitettu (ei edes määritelty, kun $x=0$), niin kyseessä
on epäoleellinen integraali. Tämä voidaan (ainakin yrittää) laskea raja-arvona
\[
\int_A f\,dxdy=\lim_{\eps\kohti 0^+} \int_{A_\eps} f\,dxdy,
\]
missä $A_\eps=[\eps,1]\times [0,1]$. Tässä $\int_{A_\eps} f\,dxdy$ on tavanomainen Riemannin
integraali, joten iteraatiokaava \eqref{tasopalautus-x} soveltuu:
\begin{align*}
\int_{A_\eps} f\,dxdy 
&= \int_\eps^1 \left[ \int_0^1 \frac{y^2}{\sqrt{x}}\,dy \right]dx
 = \int_\eps^1 \frac{1}{\sqrt{x}}\left[ \int_0^1 y^2\,dy \right]dx \\
&= \int_\eps^1 \frac{1}{\sqrt{x}}\left[ \sijoitus{y=0}{y=1} \frac{y^3}{3} \right]dx
 = \int_\eps^1 \frac{1}{3\sqrt{x}}\,dx\,
 = \,\sijoitus{\eps}{1} \frac{2}{3}\sqrt{x}\,\kohti\,\underline{\underline{\frac{2}{3}}}\,,
                                              \quad \text{kun}\ \eps\kohti 0^+.
\end{align*}
Integraali siis suppenee. Oikea integraalin arvo olisi saatu myös luottaen suoraan kaavaan
\eqref{tasopalautus-x} eli ohittamalla raja-arvoprosessi:
\[
\int_A f\,dxdy = \int_0^1\left[\int_0^1 \frac{y^2}{\sqrt{x}}\,dy\right]dx 
               = \int_0^1 \frac{1}{\sqrt{x}}\,dx \cdot \int_0^1 y^2\,dy
               = 2 \cdot \frac{1}{3} = \frac{2}{3}\,.
\]
b) Myös tämä integraali suppenee, ja suoraan kaavaan \eqref{tasopalautus-y} perustuva lasku
\begin{align*}
\int_A \frac{x}{(1+x^2+y^2)^2}\,dxdy 
       &= \int_0^\infty\left[\int_0^\infty \frac{x}{(1+x^2+y^2)^2}\,dx\right]dy \\
       &= \int_0^\infty\left[\,\sijoitus{x=0}{x=\infty}-\frac{1}{2(1+x^2+y^2)}\right]dy \\
       &= \int_0^\infty \frac{1}{2(1+y^2)}\,dy
        = \sijoitus{y=0}{y=\infty}\frac{1}{2}\Arctan y = \frac{\pi}{4}
\end{align*}
antaa oikean tuloksen. Asian voi varmistaa laskemalla ensin
\[
\int_{T_{ab}} \frac{x}{(1+x^2+y^2)^2}\,dxdy 
               = \int_0^b\left[\int_0^a \frac{x}{(1+x^2+y^2)^2}\,dx\right]dy,
\]
missä $T_{ab}=[0,a]\times[0,b]$, ja toteamalla, että tämän integraalin arvo $\kohti\pi/4$ aina
kun $a\kohti\infty$ ja $b\kohti\infty$. \loppu

Esimerkin perusteella kaavat \eqref{tasopalautus-x}--\eqref{tasopalautus-y} näyttävät soveltuvan
suoraan myös epäoleellisiin integraaleihin. Useimmiten näin onkin käytännössä, mutta kyse on
vain pääsäännöstä, kuten seuraava esimerkki osoittaa.
\begin{Exa} Laske säännöillä \eqref{tasopalautus-x} ja \eqref{tasopalautus-y} integraali
\[
I=\int_A f(x,y)\,dxdy, \quad A=[0,\infty)\times[0,1], \quad f(x,y)=xy(2-xy)e^{-xy}.
\]
\end{Exa}
\ratk
\begin{align*}
\text{Sääntö \eqref{tasopalautus-x}}: \quad 
I &= \int_0^\infty \left[ \sijoitus{y=0}{y=1} xy^2e^{-xy}\right]dx
   =\int_0^\infty xe^{-x}=1. \\[2mm]
\text{Sääntö \eqref{tasopalautus-y}}: \quad 
I &= \int_0^1 \left[\lim_{a\kohti\infty}\sijoitus{x=0}{x=a} x^2ye^{-xy}\right]dy
   =\int_0^1 \left[\lim_{a\kohti\infty} a^2ye^{-ay}\right]dy=0. \loppu
\end{align*}
Esimerkin ristiriidan voi pelkistää seuraavaan päättelyketjuun, jossa kysymysmerkillä varustettu
päätelmä (ja vain se!) on väärä:
\begin{align*}
1 &= \int_0^\infty \left[\int_0^1 f(x,y)\,dy\right]dx \\
  &= \lim_{a\kohti\infty} \int_0^a \left[\int_0^1 f(x,y)\,dy\right]dx \\
  &= \lim_{a\kohti\infty} \int_0^1 \left[\int_0^a f(x,y)\,dx\right]dy \\
  &= \int_0^1 \lim_{a\kohti\infty} \left[\int_0^a f(x,y)\,dx\right]dy \qquad \text{(?)} \\
  &= \int_0^1 \left[\int_0^\infty f(x,y)\,dx\right]dy = 0.
\end{align*}

Esimerkissä ongelmaksi osoittautuu integroitavan funktion \pain{merkinvaihtelu}: Jos tämä
sallitaan, eivät iteraatiokaavat \eqref{tasopalautus-x}--\eqref{tasopalautus-y} välttämättä anna
ristiriidattomia tuloksia. Ongelma ratkaistaankin poistamalla merkinvaihtelun mahdollisuus
epäoleellisen integraalin määritelmässä seuraavasti: 

\index{positiivinen osa (funktion)} \index{negatiivinen osa (funktion)}%
Määritellään funktion $f(x,y)$ \kor{positiivinen osa} $f_+$ ja \kor{negatiivinen osa} $f_-$
asettamalla $f$:n määrittelyjoukossa $D_f$
\[
\begin{aligned}
f_+(x,y) &= \begin{cases} 
              \,f(x,y), &\text{kun}\ f(x,y)>0, \\ 
              \,0, &\text{muulloin}, 
            \end{cases} \\
f_-(x,y) &= \begin{cases} -f(x,y), &\text{kun}\ f(x,y)<0, \\ 
                          \,0, &\text{muulloin}. 
            \end{cases}
\end{aligned}
\]
Tällöin jos $A\subset D_f$ on rajoitettu joukko ja $f$ on Riemann-integroituva yli $A$:n, niin
samoin ovat $f_+$ ja $f_-$ (Harj.teht.\,\ref{tasointegraalit}:\ref{H-uint-1: fplus ja fmiinus}).
Koska on myös $f=f_+-f_-$, niin seuraa
\[
\int_A f\,d\mu=\int_A f_+\,d\mu-\int_A f_-\,d\mu.
\]
Jos $A$ ei ole rajoitettu tai $f$ ei ole $A$:ssa rajoitettu, niin otetaan tämä integraalin
\pain{määritelmäksi}, eli asetetaan vaatimus, että $f_+$ ja $f_-$ ovat molemmat integroituvia
yli $A$:n. Funktion merkinvaihtelun mahdollisuus on näin poistettu integraalin laajennetusta
määritelmästä.
\jatko \begin{Exa} (jatko). Tässä on
\[
f_+(x,y)=\begin{cases} 
          \,xy(2-xy)e^{-xy}, &\text{kun } 0 \le xy \le 2, \\ \,0, &\text{muulloin}, 
         \end{cases}
\]
joten kun $A=[0,\infty)\times [0,1]$, saadaan
\begin{align*}
\int_A f_+\,dxdy &= \int_0^1 \left[\int_0^\infty f_+(x,y)\,dx\right]dy \\
                 &= \int_0^1 \left[\int_0^{2/y} xy(2-xy)e^{-xy}\,dx\right]\,dy \\
                 &=\int_0^1 \left[\sijoitus{x=0}{x=2/y} x^2ye^{-xy}\right]dy\,
                \,=\int_0^1 4e^{-2}y^{-1}\,dy.
\end{align*}
Saatu integraali ei suppene, joten $f_+$ ei ole integroituva yli $A$:n, eikä määritelmän mukaan
siis myöskään $f$. \loppu
\end{Exa}
Todettakoon lopuksi, että em.\ määritelmänkään nojalla iteraatiokaavat
\eqref{tasopalautus-x}--\eqref{tasopalautus-y} eivät sovellu epäoleellisiin integraaleihin
aivan yleispätevästi, mutta poikkeukset näistä (pää)säännöistä ovat melko eksoottisia,
ks.\ Harj.teht.\,\ref{Riemann vs Fubini}.

\subsection*{Fubinin lause}

Tarkoituksena on perustella tasointegraalin iteraatiokaava \eqref{tasopalautus-x}. Tässä riittää
rajoittua erikoistapaukseen, jossa $A$ on suorakulmio, eli laskukaavaan
\[
\int_T f\,d\mu = \int_a^b\left[\int_c^d f(x,y)\,dy\right]dx, \quad T = [a,b] \times [c,d].
\]
Nimittäin jos tässä on
\[ 
T \supset A = \{(x,y) \in \R^2 \mid x \in [a,b]\ \ja\ g_1(x) \le y \le g_2(x) \}, 
\]
niin kyse on kaavasta \eqref{tasopalautus-x}, kun valitaan $f=f_0=$ $f$:n nollajatko $A$:n 
ulkopuolelle.
\begin{Lause} \label{Fubini} \index{Fubinin lause|emph} 
\vahv{(Tasointegraalin iteraatiokaava -- Fubinin\footnote[2]{Italialainen matemaatikko
\hist{Guido Fubini} eli vuosina 1897-1943. \index{Fubini, G.|av}} lause)}
Olkoon $f$ määritelty, rajoitettu ja 
Riemann-integroituva suorakulmiossa $T=[a,b]\times [c,d]$. Olkoon edelleen $f(x,y)$ muuttujan
$y$ suhteen Riemann-integroituva välillä $[c,d]$ jokaisella $x \in [a,b]$ ja muuttujan $x$ 
suhteen Riemann-integroituva välillä $[a,b]$ jokaisella $y \in [c,d]$. Tällöin funktiot
\[ 
F(x) = \int_c^d f(x,y)\,dy, \quad G(y) = \int_a^b f(x,y)\,dx 
\]
ovat Riemann-integroituvia väleillä $[a,b]$ ja $[c,d]$, ja pätee
\[ 
\int_T f(x,y)\,dxdy = \int_a^b F(x)\,dx = \int_c^d G(y)\,dy. 
\]
\end{Lause}
\tod Olkoon $\mathcal{T}_h = \{T_{kl},\ k=1\ldots m,\ l=1\ldots n\},\ 
T_{kl}=[x_{k-1},x_k]\times[y_{l-1},y_l]$, suorakulmion $T$ jako, ja olkoon
\[ 
m_{kl} = \inf_{(x,y) \in T_{kl}} f(x,y), \quad M_{kl} = \sup_{(x,y) \in T_{kl}} f(x,y). 
\]
Tällöin $m_{kl} \le f(x,y) \le M_{kl}\ \forall (x,y) \in T_{kl}$, joten integraalien 
vertailuperiaatteen nojalla
\[ 
m_{kl}(y_l-y_{l-1}) \le \int_{y_{l-1}}^{y_l} f(x,y)\,dy 
                    \le M_{kl}(y_l-y_{l-1}) \quad \forall x \in [x_{k-1},x_k]. 
\]
Summaamalla tämä yli indeksin $l$ seuraa
\[ 
m_k\,=\,\sum_{l=1}^n m_{kl}(y_l-y_{l-1}) \le F(x) 
         \le \sum_{l=1}^n M_{kl}(y_l-y_{l-1})\,=\,M_k \quad \forall x \in [x_{k-1},x_k], 
\]
mistä puolestaan seuraa
\begin{align*}
\sum_{k=1}^m\sum_{l=1}^n m_{kl}(x_k-x_{k-1})(y_l-y_{l-1})\,
                &=\,\sum_{k=1}^m m_k(x_k-x_{k-1}) \\
                &\le \underline{\int_a^b}F(x)\,dx\,\le \overline{\int_a^b}F(x)\,dx \\
                &\le\,\sum_{k=1}^m M_k(x_k-x_{k-1}) \\
                &=\,\sum_{k=1}^m\sum_{l=1}^n M_{kl}(x_k-x_{k-1})(y_l-y_{l-1}),
\end{align*}
eli
\[ 
\underline{\sigma}(f,\mathcal{T}_h) \le \underline{\int_a^b}F(x)\,dx 
                    \le \overline{\int_a^b}F(x)\,dx \le \overline{\sigma}(f,\mathcal{T}_h), 
\]
missä $\underline{\sigma}(f,\mathcal{T}_h)$ ja $\overline{\sigma}(f,\mathcal{T}_h)$ ovat
jakoon $\mathcal{T}_h$  liittyvät Riemannin ala- ja yläsummat. Ottamalla tässä vasemmalla
supremum ja oikealla infimum kaikkien jakojen $\mathcal{T}_h$ suhteen seuraa tasointegraalin
määritelmän nojalla (vrt.\ edellinen luku)
\[ 
\int_T f\,d\mu \le \underline{\int_a^b}F(x)\,dx \le \overline{\int_a^b}F(x)\,dx 
                                                \le \int_T f\,d\mu. 
\]
Tämän mukaan $F$:n ylä- ja alaintegraalit välillä $[a,b]$ ovat samat, joten $F$ on välillä
$[a,b]$ Riemann-integroituva (Lause \ref{Riemann-integroituvuus}), ja epäyhtälöketjusta seuraa
myös väitetty (ensimmäinen) laskukaava. Toinen laskukaava perustellaan vastaavasti.  
%\[ 
%\underline{\int_a^b}F(x)\,dx = \overline{\int_a^b}F(x)\,dx 
%                             = \int_a^b F(x)\,dx = \int_T f\,d\mu. \loppu 
%\]

Mainittakoon, että Fubinin lauseessa ei välttämättä tarvita oletuksia funktion $f$ 
integroituvuudesta erikseen muuttujien $x$ ja $y$ suhteen, vaan riittää, että $f$ on 
integroituva yli $T$:n. Tällöin on väittämässä funktiot $F$ ja $G$ määriteltävä joko ylä- tai
alaintegraaleina (mikä tahansa neljästä vaihtoehdosta), muuten lauseen väittämä säilyttää 
pätevyytensä, ja todistuskin pysyy olennaisesti samana (!). Jos siis oletetaan ainoastaan, että
$f$ on Riemann-integroituva yli $T$:n, niin pätee esimerkiksi laskukaava
\[ 
\int_T f\,dxdy = \int_a^b\left[\overline{\int_c^d} f(x,y)\,dy\right]dx. 
\]
Huomautettakoon lopuksi, että Lauseiden \ref{Analyysin peruslause} ja
\ref{jatkuvan funktion integroituvuus tasossa} perusteella Fubinin lauseen ehdot toteutuvat
oletetussa muodossa, jos $f$ on jatkuva suorakulmiossa $T=[a,b]\times[c,d]$.

\Harj
\begin{enumerate}

\item
Olkoon $f(x,y)=e^{x^4-2xy-y^2}$ ja $A=[-1,1]\times[0,2]$. Laske $\int_A f_{xy}(x,y)\,dxdy$.

\item
Laske annetun funktion $f(x,y)$ integraali yli annetun joukon $A$: \vspace{1mm}\newline
a) \ $xy+y^2,\,\ A:\ 0 \le x \le 1\ \ja\ 0 \le y \le 1-x$ \newline
b) \ $(x+y)e^{x+y},\,\ A:\ x,y \ge 0\ \ja\ x+y \le 1$ \newline
c) \ $\cos y,\,\ A:\ 0 \le x \le \pi\ \ja\ \abs{y} \le x$ \newline
d) \ $y^2e^{xy}\,\ A:\ 0 \le x \le 2\ \ja\ x \le y \le 2$ \newline
e) \ $xy^2,\,\ A:\ 0 \le x \le 1\ \ja\ x^2 \le y \le \sqrt{x}$ \newline
f) \ $x\cos y,\,\ A:\ 0 \le x \le 1\ \ja\ 0 \le y \le 1-x^2$ \newline
g) \ $\sqrt{a^2-y^2}\,,\,\ A=$ kolmio, jonka kärjet $(0,0)$, $(0,a)$ ja $(a,a)$ \ ($a>0$)
\newline
h) \ $xy(1+x^4)^{-1},\,\ A=$ kolmio, jonka kärjet $(0,0)$, $(1,0)$ ja $(1,1)$ 

\item
Määritä pinta-ala $\mu(A)$, kun $A$ määritellään annetuilla ehdoilla ($a>0$).
\vspace{1mm}\newline 
a) \ $ax \ge y^2,\,\ x^3 \le ay^2 \qquad\qquad\,$
b) \ $ax \ge y^2,\,\ x^2+y^2 \le a^2$ \newline
c) \ $a^2y \le x^3,\,\ y \ge 3x-2a \qquad\ $
d) \ $x^2-y^2 \le a^2,\,\ |y| \le x/2$ \newline
e) \ $x,y \ge 0,\ ax \ge y^2,\ ay \ge x^2,\ 8xy^2 \le a^3$

\item
Määritä ja luokittele funktion
\[
f(x,y)=\int_A (10-xu-yv^2)^2\,dudv, \quad A=[0,1]\times[0,1]
\]
kriittiset pisteet.

\item
Suppeneeko vai hajaantuuko integraali $\int_A(x-y)^{-1}\,dx$, kun \newline
a) $A=[0,1]\times[0,1]$, \ b) $A=[0,1]\times[-1,0]$\,?

\item
Laske seuraavat integraalit kaikilla arvoila $\alpha\in\R$, joilla integraali suppenee.
\vspace{1mm}
a) \ $\int_A x^\alpha y^\alpha\,dxdy, \quad A=[1,\infty)\times[1,\infty)$ \newline
b) \ $\int_A |x-2y|^\alpha\,dxdy, \quad A=[1,2]\times[0,1]$ \vspace{1mm}\newline
c) \ $\int_A |x-2y|^\alpha\,dxdy, \quad A=[1,2]\times[1,2]$ \vspace{1mm}\newline
d) \ $\int_A x^\alpha ye^{-xy}\,dxdy, \quad A=[0,1]\times[0,\infty)$

\item
Seuraavat integraalit ovat muotoa $\int_A f(x,y)\, dxdy$. Määritä ensin $A$ ja laske sitten
integraalin arvo valitsemalla sopiva integroimisjärjestys!
\begin{align*}
&\text{a)}\ \ \int_0^1\left[\int_x^1 e^{-y^2}\,dy\right]dx \qquad\quad
 \text{b)}\ \ \int_0^{\pi/2}\left[\int_y^{\pi/2} \frac{\sin x}{x}\,dx\right]dy \\
&\text{c)}\ \ \int_0^1\left[\int_{x^2}^x xy^{-1}e^y\,dy\right]dx \qquad
 \text{d)}\ \ \int_0^1\left[\int_x^1 \frac{y^\alpha}{x^2+y^2}\,dy\right]dx\ \ (\alpha>0)
\end{align*}

\item
Tiedetään, että $\int_\R e^{-x^2}\,dx=\sqrt{\pi}$.\footnote[2]{Ks.\ Propositio \ref{Gamma(1/2)}
jäljempänä.} Laske tällä perusteella
\begin{align*}
&\text{a)}\,\ \int_{\R^2} |x|e^{-x^2-y^2}\,dxdy \qquad\quad\ \
 \text{b)}\,\ \int_{\R^2} (x^2+y^2)e^{-x^2-y^2}\,dxdy \\
&\text{c)}\,\ \int_{\R^2} |x+y|e^{-x^2-y^2}\,dxdy \qquad
 \text{d)}\,\ \int_{\R^2} |x-y^2|e^{-|x|-y^2}\,dxdy
\end{align*}

\item (*)
a) Näytä integraaliin vertaamalla, että 2-ulotteinen sarja
\[
\sum_{i=1}^\infty\sum_{j=1}^\infty \frac{1}{(i+j)^\alpha}
\]
suppenee täsmälleen kun $\alpha>2$. \ b) Näytä, että approksimaation
\[
\sum_{i=1}^\infty\sum_{j=1}^\infty \frac{1}{(i+j)^\alpha} \,\approx\,
\sum_{i=1}^N\sum_{j=1}^N \frac{1}{(i+j)^\alpha} \quad (\alpha>2,\ N\in\N)
\]
virhe on luokkaa $\Ord{N^{2-\alpha}}$. \ c) Näytä, että tarkemmin pätee
\[
\sum_{i=1}^\infty\sum_{j=1}^\infty \frac{1}{(i+j)^\alpha} \,=\,
\sum_{i=1}^N\sum_{j=1}^N \frac{1}{(i+j)^\alpha} +
\frac{2-2^{2-\alpha}}{(\alpha-2)(\alpha-1)}\,N^{2-\alpha} + \Ord{N^{1-\alpha}}.
\]

\item (*) \label{Riemann vs Fubini} \index{zzb@\nim!Riemann vastaan Fubini}
(Riemann vastaan Fubini) Järjestetään rationaaliluvut välillä $\,[0,1]\,$ jonoksi
$\,\{ x_n,\ n = 0, 1, 2, \ldots \}\,$ ja määritellään joukossa $A = [0,1] \times [0, \infty)$
funktio $f$ seuraavasti:
\[
f(x,y) = \begin{cases} 
         \,e^{n-y}, &\text{kun}\ \ x = x_n\ \ \text{ja}\ \ y \ge n, \\ \,0, &\text{muulloin}.
         \end{cases}
\]
a) Näytä, että $f$ on Riemann-integroituva yli $A$:n (laajennettu määritelmä).
b) Näytä, että laskukaavoista \eqref{tasopalautus-x}--\eqref{tasopalautus-y} jälkimmäinen
toimii, edellinen ei, ts. 
\[
\int_A f\,dx dy\, = \int_0^\infty \left[\int_0^1 f(x, y)\,dx \right] dy\, \neq \, 
                    \int_0^1 \left[\int_0^\infty f(x, y)\,dy \right] dx.
\]

\end{enumerate}