\section{Reaalilukujen ominaisuuksia}  \label{reaalilukujen ominaisuuksia}
\alku

Tässä luvussa esitellään eräitä reaalilukujoukkoihin liittyviä matemaattisen analyysin 
peruskäsitteitä. Samalla tuodaan esiin myös reaalilukujen yleisiä ominaisuuksia,
vaihtoehtoisia määritelmiä ja luokittelutapoja.

\subsection*{Supremum ja infimum}

\begin{Def} \label{supremum} \index{supremum|emph} \index{pienin yläraja|emph} 
Luku $a \in \R$ on joukon $A \subset \R$ \kor{pienin yläraja} eli
\kor{supremum}, jos pätee
\begin{itemize}
\item[(1)] $\quad x \le a\ \ \forall x \in A$.
\item[(2)] $\quad b<a\,\ \impl\,\ \exists x \in A\ (x>b)$.
\end{itemize}
Merkitään $a = \sup A$. 
\end{Def}
\index{ylzy@yläraja (joukon)}%
Määritelmässä ehto (1) merkitsee, että $a$ on joukon $A$ \kor{yläraja}, ja ehto (2) kertoo, että
jos $b<a$, niin $b$ ei ole yläraja. Ehdot merkitsevät siis yhdessä, että $a$ on ylärajoista 
pienin mahdollinen. Jotta tällainen pienin yläraja voisi olla olemassa, on joukon $A$ oltava
\index{ylhäältä rajoitettu}%
ainakin \kor{ylhäältä rajoitettu} (ehto (1) voimassa jollakin $a \in \R$) ja ei-tyhjä 
(ehto (2) voimassa jollakin $a \in \R$). Sikäli kuin $\sup A$ on olemassa, on määritelmästä 
varsin ilmeistä, että se on yksikäsitteinen. Jos $a=\sup A$ ja $a \in A$, niin sanotaan, että 
\index{maksimi (joukon)}%
$a$ on $A$:n \kor{maksimi} (= suurin luku).

Jos Määritelmässä \ref{supremum} vaihdetaan epäyhtälöiden suunnat, tulee määritellyksi joukon 
\index{infimum} \index{suurin alaraja}%
$A$ \kor{suurin alaraja} eli \kor{infimum}, jota merkitään $\inf A$. Infimum voi olla olemassa
\index{alhaalta rajoitettu}%
vain, jos $A$ on ei-tyhjä ja \kor{alhaalta rajoitettu}, eli jos jollakin $a \in \R$ pätee 
\index{minimi (joukon)}%
$\,x \ge a\ \ \forall x \in A$. Jos $a=\inf A$ ja $a \in A$, niin $a$ on $A$:n \kor{minimi} 
(= pienin luku).
\begin{Exa} Jos $A = \{x \in \R \mid x \le 1\}$, niin Määritelmän \ref{supremum} mukaan 
$\ \sup A = 1$. Koska $1 \in A$, niin kyseessä on myös $A$:n maksimi. Koska $A$ ei ole alhaalta
rajoitettu, ei lukua $\inf A$ ole olemassa. 
\end{Exa}
\begin{Exa} \label{sup-esimerkki} Reaalilukujoukolla $A = \{\,x \in \R \mid x^2 < 2\,\}$ ja 
rationaalilukujoukolla $B = \{\,x \in \Q \mid x^2 \le 2\,\}$ on kummallakin sama pienin yläraja
ja suurin alaraja: $\ \sup A = \sup B = \sqrt{2}$ ja $\ \inf A = \inf B = -\sqrt{2}$. Maksimia
tai minimiä ei kummallakaan joukolla ole. \loppu 
\end{Exa}
Esimerkin \ref{sup-esimerkki} mukaan rajoitetun rationaalilukujoukon pienin yläraja tai suurin
alaraja ei välttämättä ole rationaalinen. Vastaava ilmiö on entuudestaan tuttu lukujonoista:
Rationaalilukujonon raja-arvo ei välttämättä ole rationaalinen. Lukujonoista tiedetään myös,
että kasvavalle ja rajoitetulle rationaalilukujonolle on aina konstruoitavissa raja-arvo 
reaalilukuna (eli äärettömänä desimaalilukuna, ks.\ Lauseen \ref{monotoninen ja rajoitettu jono}
todistuskonstruktio). Vastaavalla tavalla on jokaiselle ei-tyhjälle ja ylhäältä rajoitetulle
reaalilukujoukolle konstruoitavissa pienin yläraja reaalilukuna.
\begin{*Lause} (\vahv{Supremum--lause}) \label{supremum-lause} Jokaisella ei-tyhjällä, ylhäältä
rajoitetulla reaalilukujoukolla on supremum. 
\end{*Lause}
\tod Tarkastellaan oletukset täyttävää joukkoa $A \subset \R$. Koska $A$ on ei-tyhjä, on
olemasssa $c_1 \in \R$ siten, että $x>c_1$ jollakin $x \in A$, ts.\ $c_1$ ei ole $A$:n yläraja.
Koska $A$ on ylhäältä rajoitettu, niin $A$:lla on yläraja $c_2 = c_1 + L \in \R$. Yleisyyttä 
rajoittamatta voidaan edelleen olettaa, että $c_1$ ja $c_2$ ovat rationaalilukuja, tai jopa 
kokonaislukuja. Konstruoidaan nyt kasvava rationaalilukujono $\seq{a_n}$ ja vähenevä 
rationaalilukujono $\seq{b_n}$ puolituskonstruktiolla seuraavasti:
\begin{itemize}
\item[1.] Asetetaan $a_0=c_1$, $b_0=c_2$ ja $n=1$.
\item[2.] Asetetaan $c = (c_1 + c_2)/2$. Jos $c$ on $A$:n yläraja, niin asetetaan $c_2=c$,
          muussa tapauksessa asetetaan $c_1=c$. 
\item[3.] Asetetaan $a_n=c_1,\ b_n=c_2$.
\item[4.] Lisätään indeksin $n$ arvoa yhdellä\ ($n \leftarrow n+1$) ja siirrytään kohtaan 2.
\end{itemize}
Konstruktion mukaisesti jono $\seq{a_n}$ on kasvava ja rajoitettu ja jono $\seq{b_n}$ on
vähenevä ja rajoitettu, ja lisäksi $b_n-a_n=2^{-n}L$. Siis $\seq{a_n}$ ja $\seq{b_n}$
suppenevat kohti yhteistä raja arvoa: $a_n \kohti a$ ja $b_n \kohti a$ jollakin $a\in\R$.
Konstruktion perusteella jokainen $b_n$ on $A$:n yläraja ja mikään $a_n$ ei ole $A$:n yläraja,
joten ilmeinen kandidaatti $A$:n pienimmäksi ylärajaksi on yhteinen raja-arvo $a$. Ensinnäkin
tämä on $A$:n yläraja, sillä jos $x \in A$, niin $x \le b_n\ \forall n$ (koska jokainen $b_n$ 
oli yläraja), jolloin seuraa $\,x \le \lim_n b_n=a$ (Propositio \ref{jonotuloksia} (V1)).
Siis pätee $x \in A\ \impl\ x \le a$, eli $a$ on $A$:n yläraja.

Vielä on näytettävä, että $a$ on ylärajoista pienin, joten olkoon $b<a$. Tällöin koska 
$a_n \kohti a$ ja $a>b$, niin jollakin $n$ on $a_n>b$. Silloin $b$ ei voi olla $A$:n yläraja, 
koska edes $a_n$ ei ole. Siis mikään lukua $a$ pienempi luku ei ole $A$:n yläraja, eli 
$a=\sup A$. \loppu

Jos $A$ on ei-tyhjä, alhaalta rajoitettu reaalilukujoukko, niin on ilmeistä, että 
$B = \{-x \mid x \in A\}$ on ylhäältä rajoitettu ja että $\inf A = - \sup B$. Tästä ja 
Lauseesta \ref{supremum-lause} seuraa välittömästi, että jokaisella ei-tyhjällä, alhaalta 
rajoitetulla reaalilukujoukolla on infimum.

\subsection*{Aksiomaattiset reaaliluvut}
\index{reaaliluvut!d@aksiomaattisina lukuina|vahv}

Edellisessä luvussa esitettyjä, desimaalilukuihin tai Cauchyn jonoihin perustuvia reaaliluvun 
\index{konstruktiivinen (määrittely)}%
määritelmiä sanotaan \kor{konstruktiivisiksi}. Konstruktiivisille määritelmille on 
tunnusomaista, että mitään 'ulkopuolista maailmaa' suhteessa rationaalilukuihin ei oleteta. 
Päätetään vain kutsua tietyn tyyppisiä rationaalilukujen jonoja reaaliluvuiksi, ja määritellään
näiden lukujen väliset laskuoperaatiot sekä samastus- ja järjestysrelaatiot palauttamalla kaikki
operaatiot viime kädessä rationaalisiksi. Konstruktiivisen määrittelyn hyvä puoli on, että se 
muistuttaa reaalilukujen todellista, laskennallista 'määrittelyä' laskinten ja tietokoneiden 
avulla. Toisaalta kun reaaliluvuilla halutaan laskea symbolisesti, on reaaliluvut usein 
helpompaa mieltää suoraan abstrakteina lukuina kuin konkreettisempien lukujen jonoina 
(vrt.\ Esimerkki \ref{reaalinen geometrinen sarja} Luvussa \ref{reaaliluvut}). Abstrakti
\index{aksiomaattinen (määrittely)}%
ajattelu viedään pisimmälle reaalilukujen \kor{aksiomaattisessa} määrittelyssä, jossa
yksinkertaisesti \pain{sovitaan}, että muitakin kuin rationaalilukuja on olemassa. Menettely on
tällöin samankaltainen kuin Luvun \ref{kunta} Esimerkissä \ref{muuan kunta}, jossa luvun
$\sqrt{2}\ $  'olemassaoloon' suhtauduttiin sopimuskysymyksenä. Kuten konstruktiivisia
määritelmiä, myös aksiomaattisia lähestymistapoja reaalilukuihin on monia samanarvoisia.
Määritelmien yhteisenä lähtökohtana on oletus, että reaaliluvut muodostavat järjestetyn kunnan,
joka on rationaalilukujen kunnan laajennus. Nämä oletukset jättävät mahdollisuuden, että
$\R = \Q$, joten tarvitaan vielä aksiooma, joka erottaa joukot toisistaan. Varsin usein
käytetty menettely on tehdä joko Lauseesta \ref{supremum-lause} tai Lauseesta 
\ref{Cauchyn kriteeri} aksiooma, eli asettaa jompi kumpi seuraavista.
\begin{itemize}
\item[(A)] (\vahv{Supremum-aksiooma}) Jokaisella ylhäältä rajoitetulla reaalilukujoukolla on 
           pienin yläraja.\footnote[2]{Supremum-aksioomaan tai supremum-lauseeseen liittyen
           reaaliluvut voidaan määritellä myös ylhäältä rajoitettuina rationaalilukujen
           j\pain{oukkoina}. Tämän nk.\ \kor{Dedekindin leikkauksiin} nojaavan konstruktiivisen
           määritelmän esitti saksalainen matemaatikko \hist{Richard Dedekind} (1831-1916).
           Dedekindin määritelmä on ollut aikanaan suosittu opetuksessakin.
           \index{Dedekind, R.|av} \index{Dedekindin leikkaus|av}
           \index{reaaliluvut!da@Dedekindin leikkauksina|av}} 
\item[(B)] (\vahv{Täydellisyysaksiooma}) Jokainen reaalilukujen Cauchyn jono suppenee kohti 
           reaalilukua.
\end{itemize}
Reaalilukujen em.\ konstruktiivisessa määrittelyssä nämä ovat siis tosia väittämiä. Aksioomina
ne ovat myös keskenään vaihdannaiset: Jos oletetaan (A), niin (B) on tosi väittämä, ja
päinvastoin.

Kuten konstruktiivisessa, myös reaalilukujen aksiomaattisessa määrittelyssä lähtökohtana ovat 
siis aluksi rationaaliluvut. Perustuen aksioomaan (A) tai (B), rationaalilukujen joukko 
\kor{täydennetään} liittämällä joukkoon joko ylhäältä rajoitettujen rationaalilukujoukkojen 
pienimmät ylärajat tai rationaalisten Cauchyn jonojen raja-arvot, jotka siis oletetun aksiooman
mukaan ovat olemassa reaalilukuina. Koska täydentäminen on taka-ajatuksena myös
\index{tzy@täydennyskonstruktio}%
konstruktiivisessa määrittelyssä, niin tätä on tapana kutsua \kor{täydennyskonstruktioksi}.
Rationaaliluvut voi siis täydentää reaaliluvuiksi joko konstruktiivisesti 'rakentamalla' tai
aksiomaattisesti 'olettamalla'.

\subsection*{$\R$ on ylinumeroituva}
\index{ylinumeroituva joukko|vahv}%

Jos $a,b \in \R$ ja $a<b$, niin on helposti nähtävissä, että lukujen $a,b$ väliin voidaan aina
sijoittaa rationaaliluku, ts.\ löytyy $x \in \Q$ jolle pätee $a<x<b$. Samantien lukuja löytyy
ääretön (numeroituva) määrä, eikä konstruktiossa itse asiassa edes tarvita koko joukkoa $\Q$,
vaan esim.\ äärelliset desimaaliluvut (tai äärelliset binaariluvut) riittävät
(Harj.teht.\,\ref{H-I-11: rationaaliluvut välillä}). Tuloksen perusteella sanotaan, että $\Q$
\index{tihzy@tiheä (osajoukko)}%
(tai mainitun tyyppinen $\Q$:n osajoukko) on $\R$:ssä \kor{tiheä} (engl.\ dense). Tästä 
tuloksesta huolimatta on $\R$ joukkoa $\Q$ mahtavampi, sillä pätee (vrt. Luku \ref{jono})
\begin{Lause} \label{R on ylinumeroituva} $\R$ on ylinumeroituva joukko. 
\end{Lause}
\tod Olkoon \seq{x^{(n)}} jono reaalilukuja, joille pätee $0 < x^{(n)} < 1\ \ \forall n$. 
Kirjoitetaan jonon termit äärettöminä desimaalilukuina\,:
\begin{align*}
x^{(1)}\ &= 0.d_1^{(1)}d_2^{(1)}d_3^{(1)} \ldots \\
x^{(2)}\ &= 0.d_1^{(2)}d_2^{(2)}d_3^{(2)} \ldots \\
x^{(3)}\ &= 0.d_1^{(3)}d_2^{(3)}d_3^{(3)} \ldots \\
          &\ \vdots
\end{align*}
Määritellään sitten ääretön desimaaliluku $y = 0.d_1d_2 \ldots\ $ valitsemalla 
\[
d_n \in \{1,\ldots 8\},\ \ d_n \neq d_n^{(n)}, \quad n = 1,2, \ldots
\] 
Tällöin $0<y<1$ ja $y$ ei ole äärellinen desimaaliluku, joten $y$ ei samastu mihinkään muuhun,
merkkijonona vähänkään erilaiseen desimaalilukuun. Konstruktiosta seuraa tällöin, että 
$y \neq x^{(n)}\ \forall n$. Päätellään, että mikään lukujoukon 
$A = \{\,x \in \R \mid 0<x<1\,\}$ numerointiyritys ei kata kaikkia joukon $A$ lukuja, joten 
edes $A$ ei ole numeroituva. \loppu

Olkoon $a,b \in \R$, $a<b$ ja $B = \{\,x \in \R \mid a<x<b\,\}$. Koska
\[
B = \{\,x = a + (b-a)t \mid t \in \R,\ 0<t<1\,\},
\]
niin Lauseen \ref{R on ylinumeroituva} todistuksesta voidaan päätellä, että $B$ on 
ylinumeroituva. Toisin sanoen: kahden eri suuren reaaliluvun väliin mahtuu aina ylinumeroituva
määrä reaalilukuja.

\subsection*{Avoin ja suljettu väli}

\index{vzy@väli}%
Joukko $A\subset\R$ on \kor{väli} (engl.\ interval), jos $A$:ssa on enemmän kuin yksi alkio ja
pätee $\,x,y \in A\ \ja\ x<z<y\ \impl\ z \in A$. Väli voi olla joko
\index{zyzy@äärellinen, ääretön väli}%
\kor{äärellinen}, eli joukkona ylhäältä ja alhaalta rajoitettu, tai \kor{ääretön}
(ylhäältä ja/tai alhaalta rajoittamaton). Äärellisen välin päätyypit ovat
\index{avoin väli} \index{suljettu väli}%
\kor{avoin} ja \kor{suljettu} väli, joiden merkinnät ja määritelmät ovat:
\begin{align*}
\text{Avoin väli:}    \quad &(a,b)\     =\ \{\,x \in \R \mid a<x<b\,\}. \\
\text{Suljettu väli:} \quad &[\,a,b\,]\ =\ \{\,x \in \R \mid a \le x \le b\,\}.
\end{align*}
\index{pzyzy@päätepiste (välin)}%
Tässä $a=\inf A$ ja ja $b=\sup A$ ovat välin \kor{päätepisteet} ($a,b\in\R,\ a<b$). Avoimen
välin vaihtoehtoinen merkintätapa on $]a,b[$. Väli voi myös olla muotoa $(a,b]$ tai $[a,b)$ 
(vaihtoehtoiset merkinnät $]a,b]$ ja $[a,b[\,$), jolloin sanotaan, että väli on
\index{puoliavoin väli}%
\kor{puoliavoin} --- määritelmät ovat ilmeiset. Muita kuin välin päätepisteitä sanotaan välin
\index{siszy@sisäpiste}%
\kor{sisäpisteiksi}. Avoimen välin kaikki pisteet ovat siis sisäpisteitä. Suljettu väli on
aina äärellinen. Muun tyyppiset välit voivat olla myös äärettömiä, jolloin päätepisteen
puuttuminen merkitään symbolilla $\pm\infty$\,:
\begin{align*}
(a,\infty)\  &=\ \{\,x \in \R \mid x>a\,\}, \\ 
(-\infty,b]\ &=\ \{\,x \in \R \mid x \le b\,\}, \\
(-\infty,\infty)\ &=\ \R.
\intertext{Mainittakoon tässä yhteydessä myös yleisesti käytetyt merkinnät}
(0,\infty)\       &=\ \R_+ \quad \text{(positiiviset reaaliluvut, 'R plus')}\,, \\
(-\infty,0)\      &=\ \R_- \quad \text{(negatiiviset reaaliluvut, 'R miinus')}\,.
\end{align*}

\pain{Sul}j\pain{ettu}j\pain{en} välien $[a_n, b_n],\ n =1,2, \ldots$ muodostamaa j\pain{onoa}
sanotaan \kor{sisäkkäiseksi} (engl.\ nested = 'pesitetty'), jos 
$[a_n, b_n] \supset [a_{n+1}, b_{n+1}]\ \ \forall n$. Tällaiseen jonoon liittyen reaaliluvuilla 
on seuraava hauska ominaisuus:
\begin{itemize}
\item[(C)] Jos $\{\,[a_n, b_n],\ n = 1,2, \ldots\,\}$ on jono sisäkkäisiä suljettuja välejä ja
           pätee $b_n - a_n \kohti 0$, niin on olemassa yksikäsitteinen reaaliluku $x$ siten, 
           että $x \in [a_n, b_n]\ \ \forall n \in \N$.
\end{itemize}
Reaalilukujen konstruktiivisen määrittelyn perusteella (C) on tosi väittämä, sillä $x$ määräytyy
jonojen $\seq{a_n}$ (kasvava ja rajoitettu) ja $\seq{b_n}$ (vähenevä ja rajoitettu) yhteisenä 
raja-arvona. Reaalilukujen aksiomaattisessa määrittelyssä tämä tulos voidaan myös ottaa 
perusaksioomaksi (olettaen $a_n,b_n\in\Q$), jolloin edellä mainituista aksioomista (A),\,(B) 
tulee tosia väittämiä.

Jos $x \in \R$ ja $\delta\in\R,\ \delta > 0$, niin $x$:n 
\index{ympzy@($\delta$-)ympäristö}%
$\delta$-\kor{ympäristö} määritellään \pain{avoimena} välinä
\[
U_{\delta}(x)\ =\ (x-\delta,x+\delta).
\]
Termit 'päätepiste' ja 'ympäristö' viittaavat ajatukseen, että reaaliluku on myös miellettävissä
'pisteeksi' jossakin 'paikassa'. Tämä ajatus ei ole peräisin algebrasta vaan \kor{geometriasta},
toisesta matemaattisen ajattelun suuresta päähaarasta. Geometrian perusteita käsitellään 
lähemmin Luvussa II.

\subsection*{Algebralliset ja transkendenttiset luvut}
\index{algebrallinen luku|vahv} \index{transkendenttinen luku|vahv}%

Määritelmänsä perusteella reaaliluvut jakautuvat luonnostaan rationaalisiin ja ei-rationaalisiin
\index{irrationaalinen luku}%
eli \kor{irrationaalisiin} lukuihin. Toinen, myös yleisesti käytetty luokittelutapa on jakaa 
reaaliluvut \kor{algebrallisiin} ja ei-algebrallisiin eli \kor{transkendenttisiin} lukuihin. 
Lukua $x \in \R$ sanotaan algebralliseksi, jos se toteuttaa yhtälön muotoa
\[
\sum_{k=0}^n a_k x^k\ =\ 0,
\]
missä $n \in \N$ ja $a_k \in \Z,\ k = 0,1,\ldots,n$. Luku on siis algebrallinen, jos se on 
jonkin \pain{kokonaislukukertoimisen} p\pain{ol}y\pain{nomin} \pain{nollakohta}.
\begin{Exa} Juuriluvut ja niiden erikoistapauksena rationaaliluvut toteuttavat yhtälön muotoa 
$p x^m - q = 0$, missä $m \in \N$ ja $p,q \in \Z$, joten tällaiset luvut ovat algebrallisia. 
\loppu \end{Exa}
Jos algebrallisten lukujen joukkoa merkitään symbolilla $\A$, niin esimerkin perusteella on siis
\[ 
\Q\ \subset\ \A\ \subset\ \R. 
\]
Kuten $\Q$, myös $\A$ on osoitettavissa numeroituvaksi joukoksi, joten reaalilukujen 'enemmistö'
on transkendenttisia.\footnote[2]{Matematiikan lajia, joka tutkii lukujen, kuten reaalilukujen
tai luonnollisten lukujen, ominaisuuksia mm.\ erilaisten luokittelujen näkökulmasta, sanotaan 
\kor{lukuteoriaksi}. Lukuteorian tunnettuja tuloksia on esimerkiksi, että Neperin luku $e$ on 
transkendenttinen (\hist{Charles Hermite}, 1873), samoin luku $\pi$ 
(\hist{Ferdinand von Lindemann}, 1882). Molempien lukujen irrationaalisuus osoitettiin jo 
1700-luvulla. Sen sijaan niinkään yksinkertaisia väittämiä kuin
\begin{itemize}
\item[P1:] Luku $\,e + \pi\,$ ei ole rationaaliluku.
\item[P2:] Luku $\,e \pi\,$ ei ole rationaaliluku.
\end{itemize}
ei ole vielä tätä kirjoitettaessa (2015) pystytty osoittamaan tosiksi (saati epätosiksi). 
--- Lukuteoria on  tunnettu monista helposti muotoiltavissa olevista, mutta usein vaikeasti 
ratkaistavista matemaattisista pähkinöistään.}

\pagebreak

\Harj
\begin{enumerate}

\item
Määritä seuraavien joukkojen supremum, infimum, maksimi ja minimi, sikäli kuin olemassa\,: 
\newline
a) \ $\{x\in\R \mid (x+1)(x-2)(x+4)>0\}\quad\ $ 
b) \ $\{x\in\R \mid \abs{x}+\abs{x+2}<5\}$ \newline
c) \ $\{x\in\R \mid \abs{x}\abs{x+2}<5\}\qquad\qquad\qquad\ \ $ 
d) \ $\{n/(n+1) \mid n\in\N\}$ \newline 
e) \ $\{(2-3n)/(5+n) \mid n\in\N\}\qquad\qquad\quad\ $
f) \ $\{n^3 2^{-n} \mid n\in\N\}$

\item \label{H-I-11: sup ja inf}
a) Millaisille ei-tyhjille joukoille $A\subset\R$ pätee $\,\inf A = \sup A$\,? \newline
c) Olkoon $A\subset\R$ ylhäältä rajoitettu joukko ja $a=\sup A$. Näytä, että jokaisella
$\eps>0$ on olemassa $x \in A$ siten, että $x > a-\eps$. \newline
c) Näytä, että jos $A,B\subset\R$ ovat molemmat ylhäältä (alhaalta) rajoitettuja ja
$A \subset B$, niin $\sup A \le \sup B$ ($\inf A \ge \inf B$).

\item
Olkoon \ a) $A=\{x\in\R \mid x^3<200\}$, \ b) $A=\{x\in\R \mid x^3<-400\}$. Seuraa Lauseen
\ref{supremum-lause} todistuskonstruktiota algoritmina indeksiin $n=3$ asti valitsemalla 
$c_0$:n ja $c_1$:n alkuarvot siten, että algoritmin tuottamille lukujonoille pätee: \ 
a) $\seq{a_n}=a$, \ b) $\seq{b_n}=a$, missä $\,a=\sup A\,$ äärettömänä binaarilukuna.

\item \label{H-I-11: rationaaliluvut välillä}
Olkoon $a,b\in\R$ ja $a<b$. Näytä, että on olemassa äärettömän monta eri suurta äärellistä 
desimaalilukua $x$, joille pätee $a<x<b$.

\item
Määritä kaikki välit, joilla on ominaisuus: Välin päätepisteet --- sikäli kuin niitä on --- 
sisältyvät joukkoon $\{-1,2\}$.

\item
Anna seuraaville joukoille yksinkertaisempi määritelmä väleinä, eli muodossa $(a,b)$, $[a,b]$,
$(a,b]$ tai $[a,b)$.
\begin{align*}
&\text{a)}\ \ \bigcup_{n=1}^\infty \left(\frac{1}{3n}\,,\ \frac{n}{n+1}\right) \qquad
 \text{b)}\ \ \bigcap_{n=1}^\infty \left(\frac{1}{3n}\,,\ \frac{n+1}{n}\right) \\
&\text{c)}\ \ \bigcap_{n=1}^\infty \left[\,\frac{1}{3n}\,,\ \frac{n}{n+1}\,\right] \qquad
 \text{d)}\ \ \bigcup_{n=1}^\infty \left[\,\frac{1}{3n}\,,\ \frac{n+1}{n}\,\right]
\end{align*}

\item
Näytä: \ a) Jos $x$ on irrationaalinen, niin samoin on $y=(3+x)/(x-2)$. \
b) Jos $x \neq 0$ on algebrallinen luku, niin samoin on $x^{-1}$. \ c) Jos $x>0$ on
algebrallinen luku, niin samoin on $\sqrt[m]{x}\,\ \forall m\in\N,\ m \ge 2$. \
d) Jokainen rationaalikertoimisen polynomin nollakohta on algebrallinen luku.

\item (*)
Olkoon $\{\,[a_n,b_n],\ n=1,2,\ldots\,\}$ jono suljettuja välejä. Näytä, että 
$A=\bigcap_{n=1}^\infty [a_n,b_n]$ on joko tyhjä joukko, sisältää täsmälleen yhden alkion, tai
on suljettu väli.

\end{enumerate}