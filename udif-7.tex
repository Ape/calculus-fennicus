\section[Käänteisfunktiolause. Implisiittifunktiolause. \\ Kontraktiokuvauslause]
{Käänteisfunktiolause. Implisiittifunktiolause. \\ Kontraktiokuvauslause}
\label{käänteisfunktiolause} 
\sectionmark{Käänteisfunktiolause}
\alku

Tässä luvussa tarkastellaan peruskysymystä epälineaarisen yhtälöryhmän ratkeavuudesta, kun 
yhtälöitä ja tuntemattomia on yhtä monta. Olkoon $\mf(\mx)=(f_1(\mx),\ldots,f_n(\mx))$, \
$\mx\in\DF_\mf\subset\R^n$ ja tarkastellaan yhtälöryhmää
\begin{equation} \label{epälin yryhmä}
\mf(\mx)=\my, \tag{$\star$}
\end{equation}
missä $\my\in\R^n$. Sikäli kuin yhtälöryhmällä on yksikäsitteinen ratkaisu $\mx\in\DF_\mf$
jollakin $\my$, voidaan ratkaisu muodollisesti kirjoittaa
\[
\mx = \mf^{-1}(\my),
\]
\index{kzyzy@käänteisfunktio} \index{funktio B!i@käänteisfunktio}%
missä $\mf^{-1}(\my)=(g_1(\my),\ldots,g_n(\my))$ on $\mf$:n \kor{käänteisfunktio}. Jos
yhtälöryhmällä on yksikäsitteinen jokaisella $\my \in B$ ($B\subset\R^n$) ja
$A=\mf^{-1}(B)=\{\mx\in\DF_\mf\ | \ \mf(\mx) \in B\}$, niin $\mf: A \kohti B$ on bijektio.
\begin{Exa} \label{udif-7: esim 1} Olkoon $\mf(x,y)=(x+y,x^2-y^2),\ \mx=(x,y)\in\R^2$. Kun
merkitään $\my=(u,v)$, niin yhtälöryhmä \eqref{epälin yryhmä} on
\[
\begin{cases} \,x+y =u, \\ \,x^2-y^2 =v. \end{cases}
\]
Jälkimmäisen yhtälön muodosta $u(x-y)=v$ nähdään, että jos $u \neq 0$, niin yhtälöryhmällä on
yksikäsitteinen ratkaisu $x=\frac{1}{2}(u^2+v)/u$, $y=\frac{1}{2}(u^2-v)/u$. Suoralla
$S:\,x+y=0$ on $\mf(x,y)=(0,0)$, joten tällä suoralla $\mf$ ei ole 1-1. Päätellään, että jos
$A=\{(x,y)\in\R^2\ | \ x+y \neq 0\}\,$ ja $\,B=\{(u,v)\in\R^2\ |\ u \neq 0\}$, niin
$\mf:\,A \kohti B$ on bijektio ja
\[
\inv{\mf}(\my) \,=\, \left(\frac{u^2+v}{2u}\,,\,\frac{u^2-v}{2u}\right), \quad 
                     \my = (u,v) \in B. \loppu
\]
\end{Exa}

\subsection*{Paikallinen käänteisfunktio. Käänteisfunktiolause}

Esimerkissä \ref{udif-7: esim 1} yhtälöryhmä \eqref{epälin yryhmä} ratkesi täydellisesti.
Tavallisemmin epälineaarisen yhtälöryhmän täydellinen ratkaiseminen on vaikeaa ellei
mahdotonta, eikä myöskään mitään lineaaristen yhtälöryhmien teoriaan
verrattavaa yleisempää ratkeavuusteoriaa ole. Sen sijaan voidaan suhteellisen yleisin ehdoin
selvittää kysymys yhtälöryhmän p\pain{aikallisesta} ratkeavuudesta seuraavan (sovelluksissa
tyypillisen, vrt.\ edellinen luku) ongelman asettelun mukaisesti: Jos $\ma \in \DF_\mf$ ja
$\mb=\mf(\ma)$, niin onko (ja millä ehdoilla) yhtälöryhmällä \eqref{epälin yryhmä}
yksikäsitteinen ratkaisu \pain{lähellä} \pain{$\ma$:ta}, kun $\my$ on samoin \pain{lähellä}
\pain{$\mb$:tä}\,?
\begin{Def} \label{paikallinen kääntyvyys} \index{paikallinen käänteisfunktio|emph}
\index{kzyzy@käänteisfunktio!a@paikallinen|emph}
Funktio $\mf:\DF_f\kohti\R^n$, $\DF_\mf\subset\R^n$, on \kor{paikallisesti kääntyvä} 
(engl.\ locally invertible) pisteessä $\ma\in D_\mf$, jos $\exists\delta>0$ siten, että $\mf$
on määritelty ympäristössä $U_\delta(\ma)=\{\mx\in\R^n \ | \ \abs{\mx-\ma}<\delta\}$ ja on
tähän ympäristöön rajoitettuna kääntäen yksikäsitteinen kuvaus. Käänteisfunktiota
\[
\mx=\inv{\mf}(\my) \ \ekv \ \my=\mf(\mx) \ \ja \ \mx\in U_\delta(\ma)
\]
sanotaan $\mf$:n \kor{paikalliseksi} (lokaaliksi) \kor{käänteisfunktioksi}.
\end{Def}
\jatko \begin{Exa} (jatko) Ratkaisemalla yhtälöryhmä $\mf(\mx)=\my$ todettiin esimerkin
funktio $\mf$ kääntyväksi rajoitettuna joukkoon $A=\{(x,y)\in\R^2\ | \ x+y \neq 0\}$.
Myös Määritelmän \ref{paikallinen kääntyvyys} mukaisesti $\mf$ on paikallisesti kääntyvä
jokaisessa pisteessä $\ma = (a,b) \in A$, sillä $A$ on avoin joukko, jolloin
$\forall (a,b) \in A\ \exists \delta>0$ siten, että $U_\delta(a,b) \subset A$, jolloin $\mf$ on
tähän ympäristöön rajoitettuna 1--1. (Tarkemmin pätee tässä: $U_\delta(a,b) \subset A$
täsmälleen kun $\delta \le d$, missä $d=|a+b|/\sqrt{2}$, on pisteen $P=(a,b)$ etäisyys suorasta
$S: x+y=0$.) \loppu
\end{Exa}

Yhden reaalimuuttujan funktion tapauksessa paikalliselle kääntyvyydelle voidaan helposti antaa
riittävät ehdot derivaatan avulla. Oletetaan:
\begin{itemize}
\item[(i)]   $f$ on derivoituva välillä $(a-\delta,a+\delta)$ jollakin $\delta>0$.
\item[(ii)]  $f'(x)$ on jatkuva pisteesä $x=a$.
\item[(iii)] $f'(a) \neq 0$.
\end{itemize}
Näiden oletuksien perusteella on joko $f'(x)>0$ tai $f'(x)<0$ välillä $[a-\rho,a+\rho]$
jollakin $\rho\in(0,\delta)$ (Propositio \ref{jatkuvan funktion jäykkyys}), jolloin $f$ on
ko.\ välillä aidosti monotoninen (Lause \ref{monotonisuuskriteeri}) ja siis 1-1. Ehdot
(i)--(iii) ovat siis riittävät $f$:n paikalliselle kääntyvyydelle pisteessä $a$. Koska $f$ on
näillä ehdoilla sekä aidosti monotoninen että jatkuva välillä $[a-\rho,a+\rho]$, niin
päätellään myös (Lause \ref{ensimmäinen väliarvolause}): Yhtälöllä $f(x)=y$ on yksikäsitteinen
ratkaisu välillä $[a-\rho,a+\rho]$ aina kun $|y-f(a)|\le\eps$, missä
$\eps \,=\, \min\{\,|f(a)-f(a-\rho)|,\,|f(a)-f(a+\rho)|\,\} \,>\, 0$. Edelleen ehdoilla
(i)---(iii) pätee myös (Lause \ref{käänteisfunktion derivoituvuus}, $\delta=\rho$): $f$:n
paikallinen käänteisfunktio $\inv{f}$ on derivoituva pisteessä $b=f(a)$ ja
$\dif\inv{f}(b)=1/f'(a)$.
\begin{Exa} Funktio $f(x)=x^2$ täyttää oletukset (i)--(iii) pisteissä $a \neq 0$, joten näissä
pisteissä $f$ on paikallisesti kääntyvä. --- Tämä on selvää muutenkin, sillä $f$ tiedetään
kääntyväksi väleillä $(-\infty,0]$ ja $[0,\infty)$. Välillä $(-\delta,\delta)$ ei $f$ ole 1-1
millään $\delta>0$, joten $f$ ei ole paikallisesti kääntyvä pisteessä $a=0$. Funktio
$f(x)=x^3$ on sen sijaan 1-1 välillä $(-\infty,\infty)$, joten $f$ on paikallisesti kääntyvä
jokaisessa pisteessä, myös pisteessä $a=0$, jossa $f'(a)=0$. \loppu
\end{Exa}
Esimerkistä nähdään, että tapaus $f'(a)=0$ on $f$:n paikallisen kääntyvyyden kannalta avoin
tapaus, eli $f$ voi olla paikallisesti kääntyvä pisteessä $a$ tai ei.

\vspace{3mm}

Edellä mainituilla väittämillä, perustuen oletuksiin (i)--(iii), on myös useamman muuttujan
funktioita koskevat vastineet, jotka seuraavassa kootaan yhdeksi lauseeksi.
Tämän \kor{Käänteisfunktiolauseen} todistus nojaa keskeisesti toiseen huomattavaan lauseeseen,
\kor{Kontraktiokuvauslauseeseen}, joka esitetään ja todistetaan jäljempänä.
Käänteiskuvauslauseen todistus esitetään vasta luvun lopussa.
\begin{*Lause} \label{käänteiskuvauslause} \index{Kzyzy@Käänteisfunktiolause|emph}
\vahv{(Käänteisfunktiolause)} Oletetaan, että funktio $\,\mf:\DF_\mf\kohti\R^n$,
$\DF_\mf\subset\R^n$ täyttää ehdot: (i) $\exists\delta>0$ siten, että $f$ on differentioituva
pisteen $\ma$ ympäristössä $U_\delta(\ma)\subset\DF_\mf$. (ii) Osittaisderivaatat
$\partial f_i(\mx)/\partial x_j,\ i,j=1 \ldots n\,$ ovat jatkuvia pisteessä $\mx=\ma$.
(iii) Jacobin matriisi $\mJ\mf(\mx)=(\partial f_i(\mx)/\partial x_j)$ on säännöllinen
pisteessä $\mx=\ma$.
Tällöin on olemassa $\rho\in(0,\delta)$ ja $\eps>0$ siten, että $\mf$ on 1-1 joukossa
$K=\{\mx\in\R^n \ | \ |\mx-\ma| \le \rho\}$ ja yhtälöryhmällä $\,\mf(\mx)=\my\,$ on ratkaisu
$\mx \in K$ aina kun $|\my-\mb|\le\eps$, missä $\mb=f(\ma)$. Lisäksi $\mf$:n paikallinen
käänteisfunktio $\inv{\mf}$ on differentioituva pisteessä $\mb$ ja
$\mJ\,\inv{\mf}(\mb)=\inv{[\mJ\,\mf(\ma)]}$.
\end{*Lause}
Kuviossa on joukot $\,B=\{\my\in\R^n\ | \ |\my-\mb|\le\eps\}\,$ ja $\,\inv{\mf}(B)$ rajattu
yhtenäisellä viivalla ja joukot $K$ ja $\mf(K)$ pisteviivalla. Lauseen väittämän mukaisesti
paikallinen käänteisfunktio $\mf^{-1}$ on määritelty $B$:ssä ja $\mf^{-1}(B) \subset K$.
\begin{figure}[H]
\setlength{\unitlength}{1cm}
\begin{center}
\begin{picture}(14,6)(0,0.5)
\curvedashes[2mm]{0,1,2}
\put(3,3){\bigcircle{6}}
\curvedashes{}
\put(12,3){\circle{3}}
\curvesymbol{$\scriptscriptstyle{\bullet}$}
\put(3,3){\bigcircle[-2]{4}}
\curve(3.9,4.2,7.5,5,11.06,4.18)
\curve(4.08,4.65,7.5,5.5,11.4,4.82)
\put(7.5,5.7){$\mf$} \put(7.5,4.6){$\mf$}
\put(2.9,2.9){$\bullet$} \put(11.9,2.9){$\bullet$}
\put(3,3){\vector(3,1){2.8}} \put(12,3){\vector(1,2){0.68}} \put(3,3){\vector(3,-2){1.65}}
\put(2.9,2.5){$\ma$} \put(11.8,2.5){$\mb$} \put(12.2,4.0){$\eps$} 
\put(5.4,3.9){$\delta$} \put(4.3,2.3){$\rho$}
\Thicklines
\put(11.22,4.89){\vector(2,-1){0.2}}
\put(10.87,4.27){\vector(2,-1){0.2}}
\thinlines
\renewcommand{\yscale}{1.3}
\put(12,3){\bigcircle[-2]{3}}
\renewcommand{\yscale}{1.9}
\renewcommand{\yscalex}{0.5}
\put(3,3){\bigcircle{2}}
\end{picture}
\end{center}
\end{figure}
\jatko\jatko \begin{Exa} (jatko) Esimerkissä $\mf$:n paikallinen kääntyvyys voidaan selvittää
myös ratkaisematta yhtälöryhmää \eqref{epälin yryhmä}\,: Lauseen \ref{käänteiskuvauslause}
mukaan $\mf$ on paikallisesti kääntyvä pisteessä $(a,b)$ aina kun $a+b \ne 0$, sillä
\[
\mJ\mf(a,b) = \begin{rmatrix} 1&1\\2a&-2b \end{rmatrix} 
              \impl\quad \det\mJ(a,b)=-2(a+b) \neq 0,\ \ \text{kun}\ a+b \neq 0. \loppu
\]
\end{Exa} \seur
\begin{Exa} (Vrt.\ Esimerkki \ref{yryhmä-esim} edellisessä luvussa.) \, Käänteisfunktiolauseen
mukaan yhtälöryhmällä
\[
\mf(x,y)=\my \ \ekv \ \begin{cases}
\ x^3-2xy^5-x= u \\
-x^3y+2xy^2+y^5= v
\end{cases}
\]
on pisteen $\,\ma=(2,1)$ lähiympäristössä yksikäsitteinen ratkaisu, kun $|\my-\mb|\le\eps$,
missä $\mb=(2,3)$ ja $\eps>0$ on riittävän pieni. Pisteen $\mx=\ma$ lähiympäristössä on
likimain
\[
\mf(\mx) \,\approx\, \begin{bmatrix} 2\\3 \end{bmatrix}
 + \begin{rmatrix} 9&-20\\-10&5\end{rmatrix} \begin{bmatrix} x-2\\y-1 \end{bmatrix},
\]
joten ehdot $\mf(\mx)=\my$ ja $\my \in B:\,|\my-\mb|\le\eps\,\ekv\,(u-2)^2+(v-3)^2 \le \eps^2$
vastaavat pisteen $\mx=\ma$ lähellä likimain ehtoa
\begin{align*}
[9(x-2)-20(y-1)]^2+[-10(x-2)+5(y-1)]^2    &\le \eps^2 \\
\qekv 181(x-2)^2-460(x-2)(y-1)+425(y-1)^2 &\le \eps^2.
\end{align*}
Tämän mukaan joukkoa $\mf^{-1}(B)$ rajaa likimain toisen asteen käyrä --- Kyseessä on
\index{ellipsi}%
(vino) \kor{ellipsi}, jonka keskipiste on $\ma=(2,1)$, vrt.\ kuvio edellä. \loppu
\end{Exa}

\subsection*{Implisiittifunktiolause}

Käänteisfunktiolauseen hieman yleisempi muoto on \kor{Implisiittifunktiolause}, jossa 
tarkastellaan yhtälöryhmää muotoa
\[
\mF(\mx,\my)=\mo,
\]
missä $\mx\in\R^n$, $\my\in\R^p$ ja $\mF:D_\mF\kohti\R^p$, $D_\mF\subset\R^{n+p}$. Olkoon 
annettu piste $(\mx,\my)=(\ma,\mb)$, jossa $\mF(\ma,\mb)=\mo$. Kysytään: Voidaanko $\my$ 
ratkaista yhtälöryhmästä muodossa $\my=\mg(\mx)$ pisteen $(\ma,\mb)$ ympäristössä? 
--- Riittävä ehto ratkeavuudelle saadaan jälleen tutkimalla $\mF$:n Jacobin matriisia
\[
\mJ\,\mF(\mx,\my)=\begin{bmatrix}
\dfrac{\partial F_1}{\partial x_1} & \ldots & \dfrac{\partial F_1}{\partial x_n} & 
\dfrac{\partial F_1}{\partial y_1} & \ldots & \dfrac{\partial F_1}{\partial y_p} \\ \vdots \\
\dfrac{\partial F_p}{\partial x_1} & \ldots & \dfrac{\partial F_p}{\partial x_n} & 
\dfrac{\partial F_p}{\partial y_1} & \ldots & \dfrac{\partial F_p}{\partial y_p}
\end{bmatrix}.
\]
Kun kirjoitetaan tämä muotoon 
\[
\mJ\,\mF(\mx,\my)=[\mJ_\mx\mF(\mx,\my), \mJ_\my\mF(\mx,\my)],
\]
missä $\mJ_\mx\mF$ sisältää $\mJ\,\mF$:n ensimmäiset $n$ saraketta, niin ratkeavuuden kannalta
kriittinen on matriisi $\mJ_\my\mF(\ma,\mb)$ (kokoa $p \times p$). Jos tämä on säännöllinen,
niin ratkaiseminen yleensä onnistuu. Tarkemmin muotoiltuna tämä väittämä on mukaelma Lauseesta
\ref{käänteiskuvauslause}, ja myös todistus (jota ei esitetä) noudattaa tämän lauseen
todistuksen ajatuskulkua, ks.\ luvun loppu 
\begin{*Lause} \label{implisiittifunktiolause} \index{Implisiittifunktiolause|emph} 
\vahv{(Implisiittifunktiolause)} \ Oletetaan, että funktio $\,\mF(\mx,\my)$, \linebreak 
missä $\mx\in\R^n$, $\my\in\R^p$ ja $\mF:\DF_\mF\kohti\R^p$, $\DF_\mF\subset\R^{n+p}$, täyttää
ehdot: \ (i) $\exists\delta>0\,$ siten, että $\mF$ on differentioituva pisteen $(\ma,\mb)$
ympäristössä $U_\delta(\ma,\mb) \subset D_\mF$. \ \linebreak (ii) Osittaisderivaatat
$[\mJ_\mx\mF(\mx,\my)]_{ij}=\partial F_i(\mx,\my)/\partial x_j,\ i=1 \ldots p,\ j=1 \ldots n\,$
ja $\,[\mJ_\my\mF(\mx,\my)]_{ij}=\partial F_i(\mx,\my)/\partial y_j$, $i,j=1 \ldots p\ $ ovat
jatkuvia pisteessä $\,(\mx,\my)=(\ma,\mb)$. (iii) Matriisi $\mJ_\my\mF(\ma,\mb)$ on
säännöllinen. Tällöin on olemassa $\rho\in(0,\delta)$ ja $\eps>0$ siten, että yhtälöryhmä
$\mF(\mx,\my)=\mo$ ratkeaa ehdolla $|\my-\mb|\le\rho\,$ yksi-käsitteisesti muotoon
$\my=\mg(\mx)$ aina kun $|\mx-\ma|\le\eps$. Lisäksi $\mg$ on differentioituva pisteessä $\ma$
ja $\mJ\mg(\ma) = -[\mJ_\my\mF(\ma,\mb)]^{-1}\mJ_\mx\mF(\ma,\mb)$.
\end{*Lause}
Lauseen \ref{implisiittifunktiolause} erikoistapaaus on Lause \ref{käänteiskuvauslause},
kun $\mF(\mx,\my)=\mx-\mf(\my)$ $(p=n)$. Jos $\mg$:n differentioituvuus oletetaan, niin Jacobin
matriisin $\mJ\mg(\ma)$ laskusäännön voi johtaa implisiittisellä osittaisderivoinnilla 
(Harj.teht.\,\ref{H-udif-7: implisiittifunktion derivoimissääntö}).
\begin{Exa}
Missä tasokäyrän $S:\ F(x,y)=x^2-2xy-y^2-1=0\,$ pisteissä voidaan käyrän yhtälöstä ratkaista
paikallisesti a) $y$ $x$:n avulla, \, b) $x$ $y$:n avulla?
\end{Exa}
\ratk \ a) \, Ratkeavuusehto on $F_y(x,y)=-2x-2y\neq 0\ \ekv\ x+y\neq 0$. Koska 
$x+y=0 \ \impl \ F(x,y)=2x^2-1$, niin poikkeuspisteitä ovat ainoastaan 
$(1/\sqrt{2},-1/\sqrt{2})$ ja $(-1/\sqrt{2},1/\sqrt{2})$. Muissa pisteissä ratkaiseminen
onnistuu.

b) \, Ratkeavuusehto on $F_x(x,y)=2x-2y\neq 0\ \ekv\ y\neq x$. Tämä ehto toteutuu
kaikissa käyrän pisteissä (koska $y=x \ \impl \ F(x,y)=-2x^2-1<0$), joten myös
ratkaiseminen onnistuu kaikissa pisteissä. \loppu

\begin{Exa} Yhtälöryhmä
\[
\begin{cases} \,2x^2+xy-yz=0 \\ \,2xz-2y^2+z^2=0 \end{cases}
\]
ratkaistaan pisteen $(x,y,z)=(1,2,2)$ ympäristössä muotoon $y=y(x),\ z=z(x)$. Laske
$y'(1)$ ja $z'(1)$.
\end{Exa}
\ratk \ Pisteessä $(x,y,z)=(1,2,2)$ on
\[
\mJ_{y,z}\mF(x,y,z) = \begin{bmatrix} 
                      \partial_y F_1 & \partial_z F_1 \\ \partial_y F_2 & \partial_z F_2 
                      \end{bmatrix} 
                   = \begin{bmatrix} x-z & -y \\ -4y & 2x+2z \end{bmatrix}
                   = \begin{rmatrix} -1&-2\\-8&6 \end{rmatrix}.
\]
Tämä on säännöllinen matriisi, joten ratkaiseminen onnistuu väitetyllä tavalla. Koska 
$\partial_x F_1=4x+y=6$ ja $\partial_x F_2=2z=4$, kun $(x,y,z)=(1,2,2)$, niin saadaan
\[
\mJ\mg(1)=\begin{bmatrix} y'(1)\\z'(1) \end{bmatrix}
         = -\begin{rmatrix} -1&-2\\-8&6 \end{rmatrix}^{-1} \begin{bmatrix} 6\\4 \end{bmatrix}
         = \begin{bmatrix} 2\\2 \end{bmatrix}. 
\]
Samaan tulokseen tullaan, kun derivoidaan implisiittisesti yhtälöt
\[
2x^2+xy(x)-y(x)z(x)=0, \quad 2xz(x)-2[y(x)]^2+[z(x)]^2=0,
\]
asetetaan $(x,y,z)=(1,2,2)$ ja ratkaistaan derivaatat $y'(1)$ ja $z'(1)$ syntyvästä
(lineaarisesta) yhtälöryhmästä. \loppu

\subsection*{*Kontraktiokuvauslause}

\index{kiintopisteiteraatio|(}%
Yhtälöryhmän $\mf(\mx)=\my$ paikallisessa ratkeavuusteoriassa lähtökohta on sama kuin
ratkaisua käytännössä etsittäessä, eli kirjoitetaan yhtälöryhmä muotoon $\mx=\mF(\mx)$,
jolloin kyse on siitä, onko $\mF$:llä y\pain{ksikäsitteinen}
\index{kiintopiste}%
\kor{kiintopiste} $\ma$:n lähellä. Kuten arvata saattaa, $\mF$ pyritään valitsemaan niin, että
kiintopiste löytyy kiintopisteiteraatiolla. Iteraation suppenemiselle, ja samalla kiintopisteen
yksikäsitteisyydelle annetun pisteen $\ma$ lähiympäristössä, antaa takeet
\begin{*Lause} \label{kontraktiokuvauslause} \index{Kontraktiokuvauslause|emph}
(\vahv{Kontraktiokuvauslause}\footnote[2]{Kontraktiokuvauslause tunnetaan myös nimellä 
\kor{Banachin kiintopistelause}, syystä että lause on pätevä euklidisia avaruuksia $\R^n$ 
yleisemmissä \kor{Banach-avaruuksissa}. Hyvin monet matemaattisten yhtälöiden ratkavuutta 
koskevat väittämät nojaavat Kontraktiokuvauslauseeseen sen yleisemmissä muodoissa. Tällainen
on esimerkiksi differentiaaliyhtälöiden ratkeavuutta koskeva Picardin-Lindelöfin lause
(Luku \ref{Picard-Lindelöfin lause}). \index{Banachin kiintopistelause|av}})
Olkoon $\mF:\DF_\mF\kohti\R^n,\ \DF_\mF\subset\R^n$ ja 
$K=\{\,\mx\in\R^n \mid \abs{\mx-\ma} \le \rho\,\}\subset\DF_\mF$ jollakin $\ma\in\R^n$ ja 
$\rho>0$. Tällöin jos \index{kontraktio(kuvaus)}%
\begin{itemize}
\item[(i)] $\mF$ on joukkoon $K$ rajoitettuna \kor{kontraktio(kuvaus)}, eli
\[
\abs{\mF(\mx)-\mF(\my)} \le L\abs{\mx-\my}, \quad \mx,\my\in K,\quad \text{missä}\,\ L<1,
\]
\item[(ii)] $\mF(K)\subset K$,
\end{itemize}
niin
\begin{itemize}
\item[(1)] $\mF$:llä on $K$:ssa täsmälleen yksi kiintopiste $\mc$.
\item[(2)] Kiintopisteiteraatio $\mx_{k+1}=\mF(\mx_k)$, \ $k=0,1,\ldots$ \ 
           \index{suppeneminen!b@kiintopisteiteraation|emph}%
           suppenee kohti $\mc$:tä jokaisella $\mx_0\in K$. \newline
\end{itemize}
\end{*Lause}
\tod Ensinnäkin oletuksesta $(ii)$ seuraa, että kiintopisteiteraatiolle $\mx_{k+1}=\mF(\mx_k)$
pätee
\[
\mx_0\in K \ \impl \ \mx_k\in K \quad \forall k,
\]
joten oletuksen (i) perusteella
\begin{align*}
\abs{\mx_{k+1}-\mx_k} &= \abs{\mF(\mx_k)-\mF(\mx_{k-1})} \\
&\le L\abs{\mx_k-\mx_{k-1}} \\
& \ \ \vdots \\
&\leq L^k \abs{\mx_1-\mx_0}.
\end{align*}
Tästä seuraa, että jos $l>k$, niin
\begin{align*}
\abs{\mx_l-\mx_k} &\le \abs{\mx_l-\mx_{l-1}}+\ldots + \abs{\mx_{k+1}-\mx_k} \\
&\le \sum_{i=k}^{l-1} L^i\abs{\mx_1-\mx_0} \\
&\le \abs{\mx_1-\mx_0}\sum_{i=k}^{\infty} L^i \\
&=L^k(1-L)^{-1}\abs{\mx_1-\mx_0}.
\end{align*}
Siis yleisemmin
\[
\abs{\mx_k-\mx_l}<L^{\min \{k,l\}} (1-L)^{-1}\abs{\mx_1-\mx_0}.
\]
\index{Cauchyn!ea@jono $\R^n$:ssä}%
Tämän mukaan $\{\mx_k\}$ on \kor{$\R^n$:n Cauchyn jono}, ts.\
\[
|\mx_k-\mx_l| \kohti 0, \quad \text{kun}\,\ \min\{k,l\} \kohti \infty.
\]
Kun tässä merkitään $\mx_k = ((\mx_k)_i,\ i=1 \ldots n)$, niin seuraa
\[
|(\mx_k)_i-(\mx_l)_i| \le |\mx_k-\mx_l| \kohti 0, \quad 
               \text{kun}\,\ \min\{k,l\} \kohti \infty, \quad i=1 \ldots n.
\]
Siis reaalilukujono $\{(\mx_k)_i,\ k=0,1,\ldots\}$ on Cauchyn jono jokaisella $i$. Cauchyn
kriteerin (Lause \ref{Cauchyn kriteeri}) mukaan on tällöin olemassa reaaliluvut
$c_i,\ i=1 \ldots n$ siten, että $\lim_k (\mx_k)_i = c_i,\,\ i=1 \ldots n$, jolloin on myös
$\lim_k|\mc_k-\mc|=0$, $\mc=(c_i)$ (vrt.\ Luku \ref{usean muuttujan jatkuvuus}). On siis
päätelty, että jono $\seq{\mx_k}$ (ja yleisemminkin jokainen $\R^n$:n Cauchyn jono) suppenee:  
\[
\mx_k\kohti \mc\in \R^n.
\]
Mutta koska $K$ on määritelmänsä perusteella suljettu (itse asiassa kompakti, ks.\
Määritelmä \ref{kompakti joukko - Rn}), niin $\mc\in K$. Tällöin koska $\mF$ on oletuksen (i)
perusteella $K$:ssa jatkuva, niin seuraa $\mF(x_k)\kohti\mF(\mc)\,$, jolloin iteraatiokaavan 
$\mx_{k+1}=\mF(\mx_k)$ perusteella on $\mc=\mF(\mc)$, eli $\mc$ on kiintopiste. On siis 
osoitettu, että ainakin yksi kiintopiste $\mc\in K$ on olemassa, ja että kiintopisteiteraatio
suppenee jokaisella $\mx_0\in K$ kohti jotakin $K$:ssa olevaa kiintopistettä. Lopuksi
näytetään, että kiintopisteitä on $K$:ssa vain yksi: Olkoot $\mc_1$ ja $\mc_2\,$ $\mF$:n
kiintopisteitä $K$:ssa, eli olkoon $\mc_1=\mF(\mc_1)$, $\mc_2=\mF(\mc_2)$ ja
$\mc_1\,,\mc_2\in K$. Oletuksesta (i) seuraa silloin
\[
\abs{\mc_1-\mc_2} =   \abs{\mF(\mc_1)-\mF(\mc_2)} 
                  \le L\abs{\mc_1-\mc_2} \qekv (1-L)\abs{\mc_1-\mc_2} \le 0.
\]
Koska tässä on $1-L>0$ (oletus (i)), niin on oltava $\abs{\mc_1-\mc_2}=0\ \ekv\ \mc_1=\mc_2$. 
Kontraktiokuvauslause on näin todistettu. \loppu
\index{kiintopisteiteraatio|)}

\subsection*{*Käänteisfunktiolauseen todistus}
\index{Kzyzy@Käänteisfunktiolause|vahv}

Lähdetään differentiaalikehitelmästä
\begin{equation} \label{K-lause a}
\mf(\mx)=\mb+\mJ\mf(\ma)(\mx-\ma)+\mr(\mx), \quad \mb=\mf(\ma), \quad \mx\in K, \tag{a}
\end{equation}
missä $K=\{\mx\in\R^n \ | \ |\mx-\ma|\le\rho\}$, $\,\rho\in(0,\delta)$. Tämän perusteella on
\begin{equation} \label{K-lause b}
\mf(\mx_1)-\mf(\mx_2)=\mJ\mf(\ma)(\mx_1-\mx_2)+\mr(\mx_1)-\mr(\mx_2), \quad
                                                  \mx_1,\mx_2 \in K. \tag{b}
\end{equation}
Kun tässä sovelletaan vasemmalla puolella $\R^n$ väliarvolausetta
(Lause \ref{väliarvolause-Rn})  funktioihin $f_i=(\mf)_i$, niin seuraa
\begin{equation} \label{K-lause c}
f_i(\mx_1)-f_i(\mx_2) = [\nabla f_i(\boldsymbol{\xi}_i)]^T(\mx_1-\mx_2),
                                                     \quad i=1 \ldots n, \tag{c}
\end{equation}
missä $\boldsymbol{\xi}_i\in K$ (tarkemmin $\boldsymbol{\xi}_i=t_i(\mx_1-\mx_2),\ t_i\in(0,1)$).
Jatkuvuusoletuksien perusteella on tässä edelleen
\begin{equation} \label{K-lause d}
\nabla f_i(\boldsymbol{\xi}_i) = \nabla f_i(\ma)+\md_i, \quad |\md_i| \kohti 0 \,\ 
                                 \text{kun}\ \rho \kohti 0, \quad i = 1 \ldots n. \tag{d}
\end{equation}
(Tässä sivuutetaan hieman teknisiä yksityiskohtia.) Yhdistämällä \eqref{K-lause b},
\eqref{K-lause c} ja \eqref{K-lause d} nähdään, että
\[
\mr(\mx_1)-\mr(\mx_2)\,          =\,[\md_1 \ldots \md_n]^T(\mx_1-\mx_2),% \notag \\
\]
joten Cauchyn--Schwarzin epäyhtälön perusteella
\begin{equation} \label{K-lause e}
|\mr(\mx_1-\mr(\mx_2)| \,=\, \Bigl(\sum_{i=1}^n\scp{\md_i}{\mx_1-\mx_2}^2\Bigr)^{1/2}\,
                       \,\le\,L|\mx_1-\mx_2|, \tag{e}
\end{equation}
missä
\[
L=\bigl(\sum_{i=1}^n|\md_i|^2\bigr)^{1/2} \kohti 0, \quad \text{kun}\ \rho \kohti 0.
\]

Koska $\mJ\mf(\ma)$ on säännöllinen, niin yhtälö \eqref{K-lause b} on toisin kirjoitettuna
\begin{equation} \label{K-lause f}
\mx_1-\mx_2 = [\mJ\mf(\ma)]^{-1}[\mf(\mx_1)-\mf(\mx_2)]
             -[\mJ\mf(\ma)]^{-1}[\mr(\mx_1)-\mr(\mx_2)]. \tag{f}
\end{equation}
Kun tässä sovelletaan kolmioepäyhtälöä ja arviota \eqref{K-lause e}, niin seuraa
\begin{align} \label{K-lause g}
|\mx_1-\mx_2| 
          &\le |[\mJ\mf(\ma)]^{-1}[\mf(\mx_1)-\mf(\mx_2)]|
              +|[\mJ\mf(\ma)]^{-1}[\mr(\mx_1)-\mr(\mx_2)]|  \notag \\
          &\le C|\mf(\mx_1)-\mf(\mx_2)| + CL|\mx_1-\mx_2|, \quad \mx_1,\mx_2 \in K, \tag{g}
\end{align}
missä $\,C=\norm{[\mJ\mf(\ma)]^{-1}}$ (ks.\ 
Harj.teht.\,\ref{matriisialgebra}:\ref{H-m-1: matriisin normi}). Koska $\,C\,$ on vain
matriisista $[\mJ\mf(\ma)]^{-1}$ riippuva vakio ja $L \kohti 0$ kun $\rho \kohti 0$, niin
voidaan valita $\rho\in(0,\delta)$ siten, että pätee
\begin{equation} \label{K-lause h}
CL\le\frac{1}{2}\,, \tag{h}
\end{equation}
jolloin epäyhtälöistä \eqref{K-lause g} ja \eqref{K-lause h} seuraa
\begin{equation} \label{K-lause i}
|\mx_1-\mx_2| \,\le\, 2C|\mf(\mx_1)-\mf(\mx_2)| \quad \forall\ \mx_1,\mx_2 \in K. \tag{i}
\end{equation}
Tämän mukaan $\,\forall\,\mx_1,\mx_2 \in K$ pätee $\mf(\mx_1)=\mf(\mx_2)\,\impl\,\mx_1=\mx_2$,
joten $\mf$ on $K$:ssa 1-1, kun $\rho\in(0,\delta)$ on valittu niin, että ehto
\eqref{K-lause h} toteutuu. Käänteisfunktiolauseen 1.\ osaväittämä on näin todistettu.

Käänteisfunktiolauseen toisen väittämän todistamiseksi osoitetaan, että kun $\rho$ on valittu
em.\ tavalla, niin $\exists\,\eps>0$ siten, että Kontraktiokuvauslauseen ehdot
toteutuvat, kun $|\my-\mb|\le\eps$ ja ratkaistava yhtälöryhmä kirjoitetaan muotoon
\[
\mf(\mx)=\my\,\ \ekv\,\ \mx=\mF(\mx), \quad \mF(\mx)=\mx-[\mJ\mf(\ma)]^{-1}[\,\mf(\mx)-\my\,],
                                      \quad \mx \in K.
\]
--- Huomattakoon, että kiintopisteiteraatio $\mx_{k+1}=\mF(\mx_k)$ on tällöin sama kuin
edellisessä luvussa tarkasteltu yksinkertaistettu Newtonin iteraatio sovellettuna yhtälöryhmään
$\mf(\mx)-\my=\mo$ (alkuarvauksella $\mx_0=\ma$).

Deifferentiaalikehitelmän \eqref{K-lause a} perusteella $\mF$:n lauseke saadaan muotoon
\[
\mF(\mx) = \ma + [\mJ\mf(\ma)]^{-1}(\my-\mb) - [\mJ\mf(\ma)]^{-1}\mr(\mx), \quad \mx \in K.
\]
Kun käytetään arviota \eqref{K-lause e} tässä lausekkeessa, niin seuraa
\begin{align} \label{K-lause j}
\abs{\mF(\mx_1)-\mF(\mx_2)}\ 
    &=\ \left|[\mJ\mf(\ma)]^{-1}[\,\mr(\mx_1)-\mr(\mx_2)\,]\right| \notag \\[1mm]
    &\le\ CL\abs{\mx_1-\mx_2}, \quad \mx_1,\mx_2\in K \tag{j}
\end{align}
ja koska $\mr(\ma)=\mo$, niin seuraa myös
\begin{align} \label{K-lause k}
\abs{\mF(\mx)-\ma}\ 
    &=\ \left|[\mJ\mf(\ma)]^{-1}(\my-\mb)
         -[\mJ\mf(\ma)]^{-1}[\,\mr(\mx)-\mr(\ma)\,]\right| \notag\\[1mm]
    &\le\ C|\my-\mb|+CL|\mx-\ma|, \quad \mx\in K \tag{k}
\end{align}
($C$ kuten edellä). Valitaan nyt $\eps$ siten, että toteutuu
\begin{equation} \label{K-lause l}
0<\eps\le\frac{\rho}{2C}\,. \tag{l}
\end{equation}
Tällöin jos $\abs{\my-\mb}\le\eps$, niin epäyhtälöistä \eqref{K-lause j}, \eqref{K-lause k},
\eqref{K-lause h} ja \eqref{K-lause l} seuraa
\[
\begin{cases} 
\,\abs{\mF(\mx_1)-\mF(\mx_2)}\ 
  \le\ \frac{1}{2}\abs{\mx_1-\mx_2}, \quad \mx_1\,,\mx_2 \in K, \\
\,\abs{\mF(\mx)-\ma}\ \le \frac{\rho}{2}+\frac{\rho}{2} = \rho, \quad \mx \in K.
\end{cases}
\]
Näistä ensimmäisen ehdon mukaan $\mF$ on kontraktiokuvaus $K$:ssa ja toisen ehdon mukaan
$\mF(K) \subset K$, joten Kontraktiokuvauslauseen ehdot ovat täytetyt. Tämän lauseen mukaan
yhtälöryhmällä $\mf(\mx)=\my\,\ekv\,\mx=\mF(\mx)$ on $K$:ssa yksikäsitteinen ratkaisu, kun
$\rho$ ja $\eps$ on valittu em.\ tavalla ja $|\my-\mb|\le\eps$. Käänteisfunktiolauseen
toinen väittämä on näin todistettu.

Viimeisen osaväittämän todistamiseksi valitaan yhtälössä \eqref{K-lause f} 
$\,\mx_1=\mx=\mf^{-1}(\my)$, missä $\,0<|\my-\mb|\le\eps\,$ (jolloin $\mx\neq\ma$) ja
$\,\mx_2=\ma$, jolloin yhtälö saa muodon
\begin{equation} \label{K-lause m}
\mf^{-1}(\my)-\ma = [\mJ\mf(\ma)]^{-1}(\my-\mb)-[\mJ\mf(\ma)]^{-1}\mr(\mx). \tag{m}
\end{equation}
Epäyhtälön \eqref{K-lause i} mukaan on (samoilla $\mx$ ja $\my$)
\[
|\mx-\ma| \le 2C|\my-\mb|.
\]
Tämän sekä arvion $|[\mJ\mf(\ma)]^{-1}\mr(\mx)| \le C|\mr(\mx)|$ perusteella
\[
\frac{|[\mJ\mf(\ma)]^{-1}\mr(\mx)|}{|\my-\mb|} \le\,
\frac{C|\mr(\mx)|}{|\mx-\ma|}\,\cdot\,\frac{|\mx-\ma|}{|\my-\mb|}
\,\le\, 2C^2\,\frac{|\mr(\mx)|}{|\mx-\ma|}\,.
\]
Koska tässä $|\my-\mb| \kohti 0\,\impl\,|\mx-\ma| \kohti 0$ (edellisen epäyhtälön mukaan) ja
koska $|\mr(\mx)|/|\mx-\ma| \kohti 0$, kun $|\mx-\ma| \kohti 0$, niin
$|[\mJ\mf(\ma)]^{-1}\mr(\mx)|=\ord{|\my-\mb|}$. Yhtälö \eqref{K-lause m} voidaan siis
kirjoittaa
\[
\mf^{-1}(\my)=\mf^{-1}(\mb)+[\mJ\mf(\ma)]^{-1}(\my-\mb)+\mr(\my), \quad 
                                              |\mr(\my)|=\ord{|\my-\mb|}.
\]
Näin ollen $\mf^{-1}$ on differentioituva pisteessä $\mb$ ja 
$\mJ\mf^{-1}(\mb)=[\mJ\mf(\ma)]^{-1}$. \loppu

\Harj
\begin{enumerate}

\item
Tarkastellaan funktiota $\mf(x,y)=(2x^2-y^2,-x^2+2y^2)$. a) Määritä $\mf$:n arvojoukko
$\mf(\R^2)$. b) Millaisille joukoille $A\subset\R^2$ pätee: $f:\,A \kohti f(A)$ on bijektio?
c) Näytä, että $\mf$ on paikallisesti kääntyvä pisteessä $(a,b)$, jos $ab \neq 0$. Onko tämä
ehto myös välttämätön paikalliselle kääntyvyydelle? d) Määritä $\mf$:n paikallisten
käänteisfunktioiden lausekkeet pisteissä $(1,1)$, $(-1,1)$, $(1,-1)$ ja $(-1,-1)$.

\item
Olkoon $\mf(x,y)=(x-y+e^{x+2y},x+y+e^{x-2y})$ ja olkoon $\mf^{-1}=\mf$:n paikallinen 
käänteisfunktio origon ympäristössä. Edelleen olkoon \vspace{1mm}\newline
a) \ $K_a(x_0,y_0)=\{(x,y)\in\R^2 \mid \abs{x-x_0} \le a\ \ja\ \abs{y-y_0} \le a\}$, \newline
b) \ $K_a(x_0,y_0)=\{(x,y)\in\R^2 \mid (x-x_0)^2+(y-y_0)^2 \le a^2\}$, \vspace{1mm}\newline
missä $a>0$ on pieni luku. Approksimoimalla $\mf$ affiinikuvauksella määritä likimäärin joukot
$\mf\bigl(K_a(0,0)\bigr)$ ja $\mf^{-1}\bigl(K_a(1,1)\bigr)$. Piirrä kuviot!

\item
Näytä, että seuraavat funktiot ovat bijektioita kuvauksina $\mf:\,\R^2\kohti\R^2$. Määritä
käänteisfunktion lauseke, mikäli mahdollista. Määritä myös pisteet, joissa $\mJ\mf$ on 
singulaarinen. \vspace{1mm}\newline
a) \ $\mf(x,y)=(x^3+y^3,x^3-y^3) \quad$ 
b) \ $\mf(x,y)=((x+y)^3,(x-y)^3)$ \newline
c) \ $\mf(x,y)=(x-y,x+y^3) \qquad\ $
d) \ $\mf(x,y)=(e^x-y,x+x^3+y+y^3)$

\item
Olkoon $\mf=(e^x\cos y,e^x\sin y)$. \ a) Näytä, että $\mf$ on kaikkialla paikallisesti 
kääntyvä. \ b) Määritä $\mf(A),\ A=\R\times[0,2\pi]$. Onko $\,\mf:\,A\kohti\mf(A)$ 1-1\,? \
\linebreak c) Olkoon $A=\{(x,y)\in\R^2 \ | \ x^2+y^2<R^2\}$. Mikä on suurin $R$:n arvo, jolla
$\mf:\,A\kohti\mf(A)$ on bijektio?

\item
Missä pisteissä $(x,y)$ toteutuvat Käänteisfunktiolauseen oletukset funktiolle
$\mf(x,y)=(x^2+2y,y^2-2x)$\,? Määritä Jacobin matriisi $\mJ\mf^{-1}(u,v)$, $(u,v)=\mf(x,y)$
näissä pisteissä.

\item
Tasossa liikkuvan robotin käsivarsi koostuu origoon kiinnitetystä olkavarresta (janasta)
$OP$ ja pisteeseen $P$ kiinnitetystä kyynärvarresta $PQ$. Kummankin varren pituus $=1$.
Kättä ($Q$) ohjataan säätämällä olkavarren ja $x$-akselin välistä napakulmaa $\varphi$ ja
kyynärvarren ja olkavarren välistä kulmaa $\theta$. Kulmat ovat vaihdeltavissa väleillä 
$\varphi\in[0,\pi]$ ja $\theta\in[-5\pi/6,5\pi/6]$ ($\theta>0$ vastapäivään).
\vspace{1mm}\newline
a) \ Määritä robotin käden sijainti funktiona $\mf(\varphi,\theta)$. \newline
b) \ Mihin tason pisteisiin käsi ulottuu, eli mikä on $\mf$:n arvojoukko? 
     Kuvio! \newline
c) \ Totea, että $\mf$ ei ole 1--1. Missä pisteissä 
     $(\varphi,\theta)\in(0,\pi)\times(-5\pi/6,5\pi/6)$ toteutuvat
     Käänteisfunktiolauseen ehdot? \newline
d) \ Robotin käsi halutaan siirtää pisteestä $(1.4,0.2)$ pisteeseen $(1.42,0.22)$.
     Arvioi differentiaalin avulla, millaisilla kulman muutoksilla $\Delta\varphi$
     ja $\Delta\theta$ tämä liike saadaan aikaan, kun käsivarsi on asennossa, jossa
     $\theta<0$.

\item \label{H-udif-7: implisiittifunktion derivoimissääntö}
Johda implisiittifunktion $\my=\mg(\mx)$ (Lause \ref{implisiittifunktiolause}) Jacobin
matriisin laskusääntö implisiittisellä osittaisderivoinnilla yhtälöryhmästä
\[
\mF(\mx,\mg(\mx))=\mo \qekv F_i\bigl(\mx,g_1(\mx),\ldots,g_p(\mx)\bigr)=0,\quad i = 1 \ldots p.
\]

\item
Oletetaan, että yhtälöryhmä $F(x,y,u,v)=0,\ G(x,y,u,v)=0$ toteutuu, kun $(x,y,u,v)=(a,b,c,d)$,
ja että ko.\ pisteen lähellä yhtälöryhmä ratkeaa sekä muotoon $(u,v)=\mg(x,y)$ että muotoon
$(x,y)=\mh(u,v)$. Johda Implisiittifunktiolauseen viimeisestä osaväittämästä (ko.\ lauseen
oletuksin) tulos $\mJ\mg(a,b)\mJ\mh(c,d)=\mI$. Millä muulla tavalla tämä on pääteltävissä?

\item
Olkoon $f$ säännöllinen funktio tyyppiä $f:\,\R^2\kohti\R$. Jos yhtälöryhmä
%pisteessä $(x,y,z)=(a,b,c)$ on
\[
\begin{cases} \, f(f(x,y),z)=0 \\ \,f(x,f(y,z))=0 \end{cases}
\]
toteutuu, kun $(x,y,z)=(a,b,c)$, niin millaisella $f$:n osittaisderivaattoja $f_x$ ja $f_y$
koskevalla ehdolla voidaan taata, että pisteen $(a,b,c)$ ympäristössä yhtälöryhmä ratkeaa
muotoon $x=x(z),\ y=y(z)$\,? 

\item
Olkoon $a,b,c>0$. Näytä, että yhtälöryhmä
\[
\begin{cases}
\,\dfrac{x^2}{a^2}+\dfrac{y^2}{b^2}+\dfrac{z^2}{c^2}=1 \\[2mm]
\,\dfrac{(2x-a)^2}{a^2}+\dfrac{4y^2}{b^2}=1
\end{cases}
\]
määrittelee pisteen $P=(a,b,c)$ ympäristössä parametrisen käyrän \newline 
$S:\,x=f(z)\,\ja\,y=g(z)$. Mikä on $S$:n tangenttivektori pisteessä $P$\,?

\item
Näytä, että yhtälö tai yhtälöryhmä määrittelee annetun pisteen $P$ ympäristössä 
implisiittifunktion annettua muotoa sekä ratkaise lisätehtävä: \vspace{1mm}\newline
a) \ $x^6+2y^4-3x^2y=0,\,\ P=(1,1),\,\ y=f(x)$. \ Käyrän $y=f(x)$ tangentin yhtälö pisteessä 
$P$\,? \vspace{1mm} \newline
b) \ $\sin(yz)+y^2 e^z=x,\,\ P=(e\pi^2,\pi,1),\,\ z=f(x,y)$. \ Arvioi $f(27,3)$ \newline
differentiaalin avulla. \vspace{1mm}\newline
c) \ $ye^{xz}+\sin(x-y+z)=0,\,\ P=(0,0,0),\,\ z=f(x,y)$. \ Pinnan $z=f(x,y)$ tangenttitason
yhtälö origossa? \vspace{1mm}\newline
d) \ $x=z+y\sin z,\,\ P=(0,0,0),\,\ z=f(x,y)$. \ Laske $f_{xy}(0,0)$. \vspace{1mm}\newline
e) \ $x-y-z+1=0\,\ja\,x^2+y^2-2z=0,\,\ P=(1,1,1),\,\ (x,y)=\mf(z)$. \ 
Geometrinen tulkinta? \vspace{1mm}\newline
f) \ $2x^2+3y^2+z^2=47\,\ja\,x^2+2y^2-z=0,\,\ P=(-2,1,6),\,\ (y,z)=(f(x),g(x))$. Laske
$f'(-2)$ ja $g'(-2)$. \vspace{1mm}\newline
g) \ $y^5+xy+z^2=4\,\ja\,e^{xz}=y^2,\,\ (3,1,0),\,\ (x,y)=(f(z),g(z))$. \ Laske $f'(0)$ ja 
$g'(0)$. \vspace{1mm}\newline
h) \ $2x^2+3y^2+z^2=47\,\ja\,x^2+2y^2-z=0,\,\ P=(-2,1,6),\,\ (y,z)=(f(x),g(x))$. Laske
$f'(-2)$ ja $g'(-2)$.

\item
Näytä, että seuraavat yhtälöryhmät määrittelevät annettua muotoa olevan implisiittifunktion
$\mg$ annetun pisteen lähellä, ja laske Jacobin matriisi $\mJ\mg$ ko.\ pisteessä.
\begin{align*}
&\text{a)}\ \ \begin{cases} \,x-u^3-v^3=0 \\ \,y-uv+v^2=0, \end{cases} \quad
              (u,v)=\mg(x,y), \quad (x,y,u,v)=(2,0,1,1) \\
&\text{b)}\ \ \begin{cases} \,xe^y-u^3z+\sin v=0 \\ \,x^3u^2+yv-uz\cos v=0, \end{cases} \quad 
              \begin{aligned} &(u,v)=\mg(x,y,z), \\ &(x,y,z,u,v)=(1,0,1,1,0) \end{aligned} \\
&\text{c)}\ \ \begin{cases} 
              \,xy^2+zu+v^2=3 \\ \,x^3z+2y-uv=2 \\ \,xu+yv-xyz=1,
              \end{cases} \quad 
              \begin{aligned} &(x,y,z)=\mg(u,v), \\ &(x,y,z,u,v)=(1,1,1,1,1) \end{aligned}
\end{align*}

\item (*)
Näytä, että funktio $\mf(x,y)=(2x-xy,4y-xy)$ on paikallisesti kääntyvä pisteessä $(a,b)$
täsmälleen kun $a+2b \neq 4$.

\item (*)
Oletetaan, että yhtälö $F(x,y,z)=0$ määrittelee (Implisiittifunktiolauseen oletuksin)
implisiittifunktiot $x=x(y,z)$, $y=y(z,x)$ ja $z=z(x,y)$. Näytä, että pätee:
\[
\pder{x}{y}\cdot\pder{y}{z}\cdot\pder{z}{x}=-1.
\]
Tarkista säännön pätevyys, kun $F(x,y,z)=x^3y^2z-1$. Muotoile ja todista vastaava yleisempi
väittämä koskien yhtälön $F(x_1,\ldots,x_n)=0$ määrittelemiä implisiittifunktioita 
($n$ kpl, $n \ge 2$).

\end{enumerate}