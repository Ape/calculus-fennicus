\section{Pyörteetön vektorikenttä} \label{pyörteetön vektorikenttä}
\alku
\index{pyzz@pyörteetön vektorikenttä|vahv}
\index{vektorikenttä!b@pyörteetön|vahv}

Jos $\vec F$ on fysikaalinen virtaus-, sähkö-, magneetti- ym. kenttä, sanotaan kenttää
$\vec \omega=\nabla\times\vec F$ kentän pyörrekentäksi (vrt.\ Luku 
\ref{divergenssi ja roottori}). Stokesin lause ilmaisee yhteyden kentän $\vec F$ ja pyörrekentän
$\vec\omega$ välillä:
\[
\oint_{\partial A} \vec F\cdot d\vec r=\int_A \vec\omega\cdot d\vec a.
\]
Tällä yhteydellä on paljon käyttöä esimerkiksi sähkömagneetikassa, jossa pyörrekentät ovat 
näkyvässä roolissa jo perusyhtälöissä (Maxwellin yhtälöissä, vrt.\ Luku 
\ref{divergenssi ja roottori}). Jos $\vec F$ on pyörteetön kenttä, eli jos $\vec\omega=\vec 0$,
antaa Stokesin lause tuloksen
\begin{equation} \label{Kpot-1}
\oint_{\partial A} \vec F\cdot d\vec r=0.
\end{equation}
Luvun \ref{polkuintegraalit} termein tämä voidaan lukea: Pyörteettömän vektorikentän tekemä työ
suljettua polkua pitkin = 0. Sanotaan, että kenttä $\vec F$ on tällöin
\index{konservatiivinen vektorikenttä} \index{vektorikenttä!c@konservatiivinen}%
\kor{konservatiivinen} (energian säilyttävä).

Jos $\vec F$ on gradienttikenttä, ts. $\vec F=-\nabla u$, niin $\vec F$ on pyörteetön. Tulos 
\eqref{Kpot-1} on tällöin voimassa --- ja tiedettiin ilman Stokesin lausettakin, sillä 
gradienttikentän tekemä työ = 0, jos polun alku- ja loppupisteet ovat samat, vrt.\ Luku 
\ref{polkuintegraalit}. --- Mutta entä jos kysytäänkin toisin päin: Millainen on riittävän 
säännöllinen, mutta muuten mahdollisimman yleinen vektorikenttä $\vec F$, joka on joko (a)
konservatiivinen tai (b) pyörteetön? Seuraavassa keskitytään vastaamaan näihin kysymyksiin. 
Tarkastellaan aluksi tason vektorikenttiä.

\index{yhtenzy@yhtenäinen (joukko)}%
Olkoon $A\subset\R^2$ \pain{avoin} joukko. Sanotaan, että tällainen joukko on \kor{yhtenäinen}
(engl.\ connected), jos mielivaltaiset kaksi $A$:n pistettä ovat yhdistettävissä jatkuvalla, 
suoristuvalla parametrisella käyrällä, jonka päätepisteet ovat mainitut pisteet ja joka on 
kokonaisuudessaan $A$:ssa. Jatkossa sanotaan avointa ja yhtenäistä joukkoa
\index{alue}%
\kor{alueeksi} (engl.\ domain)\footnote[2]{Alue määritellään hieman yleisemmin joukkona
$B = A \cup S$, missä $A$ on avoin ja yhtenäinen ja $S \subset \partial A$. Jos
$S = \emptyset$, niin $B=A$ on \kor{avoin alue}. Jos $S=\partial A$, niin $B=\overline{A}$ on
\kor{suljettu alue}. \index{avoin alue|av} \index{suljettu alue|av}}. 
\begin{Lause} \label{Lpot-1} \index{gradienttikenttä|emph}
Jos tason vektorikenttä $\vec F$ on jatkuva ja konservatiivinen alueessa $A \subset \R^2$, niin
$\vec F$ on gradienttikenttä, ts.\ $\vec F=-\nabla u$, missä $u$ on $A$:ssa jatkuvasti
derivoituva.
\end{Lause}
\tod Olkoon $\vec r_0\vastaa (x_0,y_0)\in A$ kiinteä ja $\vec r\vastaa (x,y)\in A$ muuttuva 
$A$:n piste. Olkoon edelleen $p$ yksinkertainen, suoristuva parametrinen käyrä (polku) 
$\vec r=\vec r\,(t)$, $t\in [a,b]$, jolle $\vec r\,(a)=\vec r_0$ ja
$\vec r_b=\vec r\vastaa (x,y)$. Määritellään
\[
u(p,x,y)=-\int_p \vec F\cdot d\vec r.
\]
Jos $u$ ei riipu polun $p$ valinnasta, ts. $u(p,x,y)=u(x,y)$, niin jatkamalla polkua pisteestä
$\vec r$ seuraa jatkuvuusoletuksesta, että
\[
u(\vec r+\Delta\vec r\,)=-\vec F(\vec r\,)\cdot\Delta\vec r+o(|\Delta\vec r\,|).
\]
Tämän mukaan $u$ on differentioituva ja $\nabla u=-\vec F$. Riittää siis osoittaa, että
$u(p,x,y)$ riippuu vain polun päätepisteistä. Olkoon $p_1$ ja $p_2$ kaksi polkua joilla on sama
alkupiste $(x_0,y_0)$ ja sama loppupiste $(x,y)$. Jos vastaavat geometriset käyrät ovat $S_1$
ja $S_2$, niin
\[
S_1\cup S_2=\bigcup_i S_i,\quad S_i\subset A,
\]
missä $S_i$:t ovat suljettuja käyriä (vrt.\ kuvio).
\vspace{3mm}
\begin{figure}[H]
\begin{center}
\import{kuvat/}{kuvapot-18.pstex_t}
\end{center}
\end{figure}
Tällöin 
\[
\int_{p_1} \vec F\cdot d\vec r-\int_{p_2} \vec F\cdot d\vec r
                         = \sum_i \oint_{S_i} \vec F\cdot d\vec r.
\]
Koska $\vec F$ on konservatiivinen, on $\oint_{S_i} \vec F\cdot d\vec r=0$, joten
\[
\int_{p_1} \vec F\cdot d\vec r=\int_{p_2} \vec F\cdot d\vec r.
\]
Siis $u(p,x,y)=u(x,y)$ on polun valinnasta riippumaton. \loppu

\index{yhdesti yhtenäinen alue}%
Sanotaan, että alue $A\subset\R^2$ on \kor{yhdesti yhtenäinen} (engl.\ simply connected), jos
pätee $B \subset A$ aina kun $B$ on alue, jolle $\partial B\subset A$. Kun kirjoitetaan
$S=\partial B$, niin ehdon hieman havainnollisempi muotoilu on: $A\subset\R^2$ on yhdesti
yhtenäinen, jos jokainen suljettu käyrä $S \subset A$ voidaan kutistaa pisteeksi niin, että
käyrä pysyy kutistuessaan koko ajan $A$:ssa. Yhdesti yhteinäinen joukko on 'järvi ilman saaria'.
\begin{figure}[H]
\begin{center}
\import{kuvat/}{kuvapot-19.pstex_t}
\end{center}
\end{figure}
Seuraava Stokesin lauseen seurannaistulos on vektorikenttien teorian keskeisimpiä. Tuloksella
on paljon käyttöä fysiikassa.
\begin{Lause} \label{Lpot-2} \index{gradienttikenttä|emph}
Jos tason vektorikenttä $\vec F$ on jatkuvasti derivoituva ja pyörteetön (rot\,$\vec F=0$)
yhdesti yhtenäisessä alueessa $A$, niin $\vec F$ on gradienttikenttä.
\end{Lause}
\tod Kentän potentiaali määritetään polkuintegraalina kuten edellä, jolloin riittää jälleen
osoittaa, että integraalin arvo riippuu vain polun päätepisteistä. Tämän todistamiseksi
valitaan jälleen kaksi polkua, joilla on samat päätepisteet, jolloin (vrt.\ todistus edellä)
\[
\int_{p_1} \vec F\cdot d\vec r-\int_{p_2} \vec F\cdot d\vec r
                               = \sum_i \oint_{S_i} \vec F\cdot d\vec r.
\]
Nytkin on oikea puoli = 0, sillä oletetun yhdesti yhtenäisyyden perusteella käyrien $S_i$ 
sisäänsä sulkemat alueet $A_i$ sisältyvät joukkoon $A$, jolloin Stokesin tasokaavan
(edellinen luvun kaava \eqref{Stokesin tasokaava}) ja $\vec F$:n pyörteettömyyden
perusteella
\[
\oint_{S_i} \vec F\cdot d\vec r=\int_{A_i} \text{rot}\,\vec F\,dxdy=0. \loppu
\]
\vspace{0.2mm}
\begin{multicols}{2} \raggedcolumns
Lauseiden \ref{Lpot-1}--\ref{Lpot-2} todistuksien perusteella gradienttikentän potentiaali
saadaan lasketuksi työintegraalina:
\[
u(\vec r\,)=-\int_{p:\,\vec r_0\,\kohti\,\vec r} \vec F\cdot d\vec r.
\]
\setlength{\unitlength}{1cm}
\begin{center}
\begin{picture}(4,1.5)(0,0.5)
\put(-0.1,-0.1){$\bullet$}
\put(3.9,0.9){$\bullet$}
\spline(0,0)(1,1)(2,1)(3,0)(4,1)
\put(1,0.86){\vector(3,2){0.1}}
\put(0.2,-0.1){$\vec r_0$} \put(3.9,1.2){$\vec r$}
\put(1.2,0.5){$p$}
\end{picture}
\end{center}
\end{multicols}
Laskukaavassa voidaan polku $p$ voidaan valita vapaasti, kunhan se pysyy alueessa $A$
(jossa $\vec F$ on määritelty). Jos $A$:n geometria ei aseta rajoituksia, voidaan polku valita
esimerkiksi seuraamaan koordinaattiakselien suuntaisia suoria, jolloin potentiaalin
laskukaavaksi tulee
\begin{align*}
\vec F(x,y) &= F_1(x,y)\vec i + F_2(x,y)\vec j\,: \\
     u(x,y) &= -\int_{x_0}^x F_1(t,y_0)\,dt-\int_{y_0}^y F_2(x,t)\,dt.
\end{align*}
\begin{Exa}
Onko vektorikenttä
\[
\vec F(x,y)=(x^3-y^3+\cos x)\vec i-(3xy^2+e^{-y})\vec j
\]
gradienttikenttä?
\end{Exa}
\ratk Koska
\[
\text{rot}\, \vec F=-\partial_x(3xy^2+e^{-y})-\partial_y(x^3-y^3+\cos x)=0,
\]
niin $\vec F$ on gradienttikenttä $\R^2$:ssa (Lause \ref{Lpot-2}). Valinnalla $(x_0,y_0)=(0,0)$ 
potentiaaliksi saadaan
\begin{align*}
u(x,y) &= -\int_0^x (t^3+\cos t)\,dt+\int_0^y (3xt^2+e^{-t})\,dt \\
&= -\frac{1}{4}x^4-\sin x+xy^3-e^{-y}+2.
\end{align*}
Yleinen potentiaalifunktio on
\[
u(x,y)=-\frac{1}{4}x^4+xy^3-\sin x-e^{-y}+C\quad (C=\text{vakio}). \loppu
\]
\begin{Exa}
Polaarikoordinaatistossa on määritelty vektorikenttä $\vec F=\frac{1}{r}\,\vec e_\varphi$.
Tutki, onko $\vec F$  gradienttikenttä alueessa $A$, kun \vspace{1mm}\newline
a) $A=\{\,(x,y)\in\R^2 \; | \; x>0 \; \ja \; y>0\,\}$, \newline
b) $A=\{\,(x,y)\in\R^2 \; | \; (x,y)\neq(0,0)\,\}$.
\end{Exa}
\ratk Polaarikoordinaatistossa on (vrt.\ Luku \ref{divergenssi ja roottori})
\[
\vec F=F_r\vec e_r+F_\varphi \vec e_\varphi\,: \quad 
      \text{rot}\,\vec F=\frac{1}{r}\,\partial_r(rF_\varphi)-\frac{1}{r}\,\partial_\varphi F_r.
\]
Tässä on $F_r=0$ ja $F_\varphi=1/r$, joten $\vec F$ on pyörteetön jokaisessa alueessa, joka ei
sisällä origoa.

a) Tässä $A$ on yhdesti yhtenäinen, joten $\vec F$ on gradienttikenttä. Valitsemalla $r_0>0$
voidaan potentiaali $u(x,y)=v(r,\varphi)$ laskea polkuintegraalina
\begin{align*}
v(r,\varphi)
&= -\int_{r_0}^r F_r(t,0)\,dt-\int_0^\varphi F_\varphi(r,t)\,r\,dt \,=\, -\varphi \\ 
&\impl\quad u(x,y) \,=\, -\underline{\underline{\Arcsin\left(\frac{y}{\sqrt{x^2+y^2}}\right)}}.
\end{align*}

b) Tässä $A$ ei ole yhdesti yhtenäinen, joten tutkitaan erikseen, onko $\vec F$ 
konservatiivinen. Valitaan polku $p:\vec r=\vec r(\varphi)$, $0\leq\varphi\leq 2\pi$ siten,
että $\vec r(\varphi)$ kiertää $r_0$-säteisen ympyräviivan:
\[
\vec r(\varphi)=r_0\,\vec e_r,\quad r_0>0,\ 0\leq\varphi\leq 2\pi.
\]
Tällöin polun alku- ja loppupisteet ovat samat, $\vec r(\varphi)\vastaa (x,y)\in A$ jokaisella
$\varphi$ ja
\[
\int_p \vec F\cdot d\vec r=\int_0^{2\pi} \frac{1}{r_0}\cdot r_0\,d\varphi=2\pi\neq 0,
\]
Siis $\vec F$ ei ole $A$:ssa konservatiivinen eikä näin muodoin myöskään gradienttikenttä
$A$:ssa. \loppu

\subsection*{Avaruuden pyörteetön vektorikenttä}

Vertaamalla Lauseen \ref{Lpot-1} todistusta Stokesin lauseen kolmiulotteiseen versioon 
nähdään, että todistus (ja siis myös Lauseen \ref{Lpot-1} väittämä) on sellaisenaan pätevä
myös avaruuden vektorikentälle. Sikäli kuin alueen geometria ei aseta rajoituksia, voidaan
kolmiulotteisen, konservatiivisen vektorikentän 
$\vec F = F_1(x,y,z)\vec i + F_2(x,y,z)\vec j + F_3(x,y,z)\vec k$ potentiaali laskea esimerkiksi
polkuintegraalina
\[
u(x,y,z) = -\int_{x_0}^x F_1(t,y_0,z_0)\,dt -\int_{y_0}^y F_2(x,t,z_0)\,dt
                                            -\int_{z_0}^z F_3(x,y,t)\,dt.
\]
Myös Lause \ref{Lpot-2} yleistyy kolmeen ulottuvuuteen, kunhan yhdesti yhtenäisyys
määritellään sopivasti. Tarkastellaan yksinkertaisia, suljettuja, suoristuvia käyriä
$S\subset A$, joille on löydettävissä Stokesin lauseen ehdot täyttävä pinta $B\subset\R^3$ 
siten, että $\partial B=S$. Jos jokaiselle tällaiselle käyrälle on pinta $B$ vielä määrättävissä
siten, että $B\subset A$, niin aluetta $A$ sanotaan yhdesti yhtenäiseksi. Esimerkiksi
\[
A=\{\,(x,y,z)\in\R^3 \; | \; a^2<x^2+y^2+z^2<R^2\,\} \quad (0 \le a < R)
\]
on yhdesti yhtenäinen (!). Sen sijaan
\[
A=\{\,(x,y,z)\in\R^3 \; | \; x^2+y^2+z^2<R^2 \; \ja \; x^2+y^2>a^2\,\} \quad (0 \le a < R)
\]
ei ole yhdesti yhtenäinen, sillä jos
\[
S=\{\,(x,y,z)\in\R^3 \; | \; x^2+y^2=b^2 \; \ja \; z=0\,\},
\]
niin $S\subset A$, kun $a<b<R$, mutta jos $B\subset\R^3$ on pinta, niin ehdot $S=\partial B$ ja
$B\subset A$ eivät voi yhtä aikaa toteutua.
\begin{Exa}
Pallokoordinaatistossa on määritelty vektorikenttä
\[
\vec F(r,\theta,\varphi)
            =\frac{1}{r^3}(2\cos\theta\,\vec e_r+\sin\theta\,\vec e_\theta),\quad r\neq 0.
\]
Määrää kentän potentiaali, jos sellainen on.
\end{Exa}
\ratk Tässä on $\vec F=F_r\vec e_r+F_\theta\vec e_\theta+F_\varphi\vec e_\varphi$, missä
\[
F_r=2r^{-3}\cos\theta,\quad F_\theta=r^{-3}\sin\theta,\quad F_\varphi=0,
\]
joten (ks.\ Luku \ref{div ja rot käyräviivaisissa})
\[
\nabla\times\vec F = 0\,\vec e_r + 0\,\vec e_\theta
  + \frac{1}{r}[\partial_r(rF_\theta)-\partial_\theta F_r]\vec e_\varphi = \vec 0\quad (r\neq 0).
\]
Koska origon poissulkeminen ei aiheuta yhtenäisyysongelmia kolmessa dimensiossa, niin $\vec F$
on gradienttikenttä. Potentiaali voidaan laskea ottamalla lähtöpisteeksi esimerkisi
$\vec r_0=(r_0,0,0)$, $r_0>0$, ja etenemällä koordinaattiviivojen suuntaisia polkuja:
\begin{align*}
u(r,\theta,\varphi) &= -\int_{r_0}^r F_r(t,0,0)\,dt-\int_0^\theta F_\theta(r,t,0)r\,dt
                                 -\int_0^\varphi F_\varphi(r,\theta,t)r\sin\theta\,dt \\
                    &= -\int_{r_0}^r 2t^{-3}\,dt-\int_0^\theta r^{-2}\sin t\,dt \\
                    &= \sijoitus{r_0}{r}t^{-2}+r^{-2}\sijoitus{0}{\theta} \cos t \\
                    &= r^{-2}\cos\theta+\text{vakio} \; (=-r_0^{-2}).
\end{align*}
Potentiaaliksi kelpaa siis $u(r,\theta,\varphi)=\underline{\underline{r^{-2}\cos\theta}}$. 
\loppu

\Harj
\begin{enumerate}

\item
Totea vektorikenttä $\vec F(x,y)=(y+2x)\vec i+x\vec j$ pyörteettömäksi. Laske tätä havaintoa
hyödyksi käyttäen polkuintegraali $\int_p \vec F \cdot d\vec r$, missä $p$ kulkee pisteestä
$(1,0)$ origon kautta pisteeseen $(1,2)$ pitkin ympyrän kaarta.

\item
Määritä funktio $f$ siten, että $f(0)=1$ ja tason vektorikenttä 
\[
\vec F(x,y)=f(y)[2xy\vec i+x^2(y+1)\vec j\,]
\] 
on pyörteetön. Mikä tällöin on kentän potentiaali?

\item
Osoita seuraavat vektorikentät koko avaruudessa pyörteettömiksi ja määritä kenttien 
potentiaalit.
\vspace{1mm}\newline
a) \ $z\vec i+z\cos y\vec j+(x+\sin y)\vec k \qquad$
b) \ $e^x\sin y\vec j+e^x\cos y\vec j+z\vec k$ \newline
c) \ $2xyz\vec i+x^2z\vec j+x^2y\vec k \qquad$
d) \ $-2xe^{-y}\vec i+(x^2e^{-y}+\sin z)\vec j+y\cos z\vec k$ \newline
e) \ $(x^2+y^2+z^2)(x\vec i+y\vec j+z\vec k)=r^2\vec r$

\item 
Vektorikentästä $\vec F$ tiedetään, että se on koko avaruudessa pyörteetön ja muotoa 
$\vec F(x,y,z)=xy^2\vec i+(\cos z+x^2y)\vec j+yf(z)\vec k$. Määrää kentän 
skalaaripotentiaali ja tämän avulla polkuintegraali $\int_p \vec F\cdot d\vec r$, missä
$p$:n alkupiste on $(0,0,0)$ ja loppupiste $(2,-1,3)$. 

\item (*) \index{zzb@\nim!Hzy@Häiriö kentässä}
(Häiriö kentässä) Tason vektorikenttään $\vec U_0=u_0\vec i$ ($u_0=$ vakio) tuodaan este
\[
A=\{(x,y)\in\R^2 \mid x^2+y^2 \le R^2\},
\] 
jolloin kenttä häiriintyy muotoon $\vec U=u(x,y)\vec i + v(x,y)\vec j$. Häiriintyneestä
kentästä $\vec U$ tiedetään, että se on sekä lähteetön että pyörteetön esteen ulkopuolella,
ts.\ $\text{div}\,\vec U = \text{rot}\,\vec U=0$ muualla kuin $A$:ssa. Lisäksi tiedetään, että
$\vec U \cdot \vec n = 0\,$ reunalla $\partial A$ (eli kun $x^2+y^2=R^2$, $\vec n=\,$
$\partial A$:n normaali) ja että $\vec U(x,y)\kohti\vec U_0$, kun $x^2+y^2\kohti\infty$.
Ongelmana on löytää nämä ehdot täyttävä kenttä $\vec U$. Koska $A$:n ulkopuolinen alue ei
ole yhdesti yhtenäinen, ei $\vec U$ välttämättä ole gradienttikenttä (vaikka onkin 
pyörteetön).

a) Näytä, että problemalla on ratkaisu, jonka napakoordinaattimuoto on
\[
\vec U=-\nabla f(r)\cos\varphi + g(r)\vec e_\varphi
\] 
eli määritä funktiot $f(r),g(r)$ mahdollisimman yleisessä muodossa niin, että annetut ehdot
toteutuvat. (Muita ratkaisuja ei ongelmalla ole). \newline
b) Totea, että em.\ ratkaisusta tulee yksikäsitteinen, kun asetetaan lisäehto muotoa
$v(R,0)=v_0$ (tässä karteesiset koordinaatit).

\end{enumerate}