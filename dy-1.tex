\section{Differentiaaliyhtälöiden peruskäsitteet} \label{DY-käsitteet}
\alku

\kor{Differentiaaliyhtälön} (jatkossa lyhennys DY), tai tarkemmin 
\kor{tavallisen differentiaaliyhtälön} (engl.\ ordinary differential equation eli ODE) yleinen
muoto on \index{differentiaaliyhtälö!d@yleinen}
\[
F(x,y,y',\ldots,y^{(n)})=0,
\]
missä $x$ on reaalimuuttuja, $y=y(x)$ on tuntematon reaaliarvoinen funktio, ja $F$ jokin
(tunnettu) $n+2$ reaalimuuttujan lauseke (kyseessä on funktio tyyppiä $F:\R^{n+2}\kohti\R$, 
vrt.\ Luku \ref{kahden ja kolmen muuttujan funktiot}). Kun yhtälö kirjoitetaan tarkemmin
muodossa
\[
F(x,y(x),y'(x),\ldots,y^{(n)}(x))=0,
\]
niin nähdään selvemmin, että yhtälössä on vain yksi vapaa muuttuja ($x$). Nimitys 'tavallinen'
DY tulee juuri tästä ominaisuudesta. ('Epätavallisista' differentiaaliyhtälöistä ei
toistaiseksi puhuta.)

Ym.\ differentiaaliyhtälössä indeksi $n\in\N$ on yhtälön
\index{kertaluku!c@differentiaaliyhtälön}%
\kor{kertaluku} (engl.\ order). Jos 
yhtälö kirjoitetaan muotoon
\[
y^{(n)}=f(x,y,\ldots,y^{(n-1)}),
\]
niin tällaista muotoa sanotaan differentiaaliyhtälön
\index{differentiaaliyhtälön!a@normaalimuoto} \index{normaalimuoto!a@DY:n, DY-systeemin}%
\kor{normaalimuodoksi}. Ensimmäisen
kertaluvun differentiaaliyhtälön normaalimuoto on siis
\[
y'=f(x,y).
\]
Normaalimuoto on matemaattisen teorian kannalta sikäli edullinen, että siihen perustuen voidaan
differentiaaliyhtälön ratkeavuusehdot asettaa paljon helpommin kuin yleisemmässä muodossa.

\subsection*{Yksittäisratkaisu.  Yleinen ratkaisu}
\index{differentiaaliyhtälön!b@yksittäisratkaisu, yleinen ratk.|vahv}
\index{yksittäisratkaisu (DY:n)|vahv}
\index{yleinen ratkaisu (DY:n)|vahv}

Jos differentiaaliyhtälön kertaluku on $n$, niin sen ratkaisu, tarkemmin 
\kor{yksittäisratkaisu}, on jokainen funktio $y(x)$, joka on jollakin \pain{avoimella}
\pain{välillä} $(a,b)$ (voi olla $\,a=-\infty\,$ ja/tai $\,b=\infty\,$) \pain{$n$} \pain{kertaa} \pain{derivoituva} ja 
toteuttaa yhtälön ko.\ välillä. Differentiaaliyhtälön \kor{yleisellä ratkaisulla} tarkoitetaan
sellaista \pain{funktio}j\pain{oukkoa}, joka sisältää kaikki (tai 'melkein kaikki', ks.\
huomatukset jäljempänä) ratkaisut. Jos yhtälön kertaluku on $n$, niin yleinen ratkaisu on
pääsääntöisesti muotoa
\[
y=Y(x,C_1,\ldots,C_n),
\]
missä $C_1,\ldots,C_n$ ovat vapaasti (tai esim.\ joiltakin väleiltä vapaasti) valittavia 
\kor{vakioita}, ja $Y$ on jokin $n+1$:n muuttujan funktio. Sikäli kuin vakiot ovat täysin
vapaasti valittavissa, yleinen ratkaisu on siis funktiojoukko
\[
\mathcal{Y}=\{\,y(x)=Y(x,C_1,\ldots,C_n) \ | \ C_1,\ldots,C_n\in\R\,\}.
\]
Differentiaaliyhtälön 'ratkaisemisella' tarkoitetaan yleensä yleisen ratkaisun määrittämistä. 
\begin{Exa}
Differentiaaliyhtälöiden
\begin{itemize}
\item[a)] $y'=e^{2x}$, $\quad\ \text{b)}\,\ y'=2y$
\end{itemize}
yleiset ratkaisut $\R$:ssä (välillä $(-\infty,\infty)$) ovat
\begin{itemize}
\item[a)] $y(x)=\frac{1}{2}\,e^{2x}+C\,\ (C\in\R)$,
          $\quad\ \text{b)}\,\ y(x)=Ce^{2x}\,\ (C\in\R)$. \loppu
\end{itemize}
%Tässä $C\in\R$ on vapaasti valittavissa ja ratkaisut ovat voimassa koko $\R$:ssä.
\end{Exa}

\begin{Exa} Ratkaise: \ a) \ $y'''=0$, \ \ b) \ $y'''+2y''=0$.
\end{Exa}
\ratk a) Integroimalla saadaan
\begin{align*} y'''(x) = 0 &\qimpl y''(x) = C_1 \\[2mm]
                           &\qimpl y' = C_1 x + C_2 \\
                           &\qimpl y(x) = \frac{1}{2}C_1 x^2 + C_2 x + C_3.
\end{align*}
Tässä voidaan $C_1$:n tilalle yhtä hyvin kirjoittaa $2C_1$, jolloin yleiselle ratkaisulle 
saadaan luontevampi muoto
\[ 
y(x) = C_1 x^2 + C_2 x + C_3 \quad (C_1,C_2,C_3 \in \R). 
\]

b) Ratkaisu voidaan tässä keksiä kirjoittamalla ensin $y''(x)=u(x)$, jolloin yhtälö 
yksinkertaistuu muotoon $\,u'+2u=0$. Tämän yleinen ratkaisu on $\,u(x)=C_1e^{-2x}$. Koska 
$u=y''$, niin integroimalla seuraa
\begin{align*}
y''(x) = C_1 e^{-2x} &\qimpl y'(x)=\int C_1 e^{-2x}\,dx=-\frac{1}{2}C_1e^{-2x}+C_2 \\
                     &\qimpl y(x)=\int y'(x)\,dx=\frac{1}{4}C_1e^{-2x}+C_2x+C_3.
\end{align*}
Kirjoittamalla $C_1$:n tilalle $4C_1$ saadaan yleiselle ratkaisulle muoto
\[
y(x)=C_1e^{-2x}+C_2x+C_3 \quad (C_1,C_2,C_3 \in \R). \loppu
\]

\subsection*{Erikoisratkaisut}
\index{differentiaaliyhtälön!c@erikoisratkaisu|vahv}
\index{erikoisratkaisu (DY:n)|vahv}

Jos differentiaaliyhtälöllä on muitakin ratkaisuja kuin yleiseen ratkaisulausekkeeseen 
$y=Y(x,C_1,\ldots,C_n)$ sisältyvät, niin tällaisia 'yllätysratkaisuja' sanotaan 
\kor{erikoisratkaisuiksi}. Milloin ratkaisu on 'erikoinen' ja milloin ei, voi riippua
yleisen ratkaisun esitysmuodosta.
\begin{Exa} \label{erikoinen dy}
Differentiaaliyhtälöiden
\begin{itemize}
\item[a)] $y'=y^2,\quad \text{b)}\,\ (y')^2=y$
\end{itemize}
yleiset ratkaisut ovat
\begin{itemize}
\item[a)] $y(x)=\dfrac{1}{C-x}\,,\quad \text{b)}\,\ y(x)=\dfrac{1}{4}(x-C)^2$,
\end{itemize}
kuten saatetaan helposti tarkistaa ($C\in\R$). Yhtälöillä on myös ilmeinen ratkaisu,
jota ei saada yleisen ratkaisun lausekkeesta, nimittäin
\[
y(x)=0.
\]
Tätä on siis pidettävä erikoisratkaisuna. Tapauksessa a) voi kuitenkin yleisen ratkaisun
esittää myös muodossa
\begin{itemize}
\item[a)] $y(x)=\dfrac{C}{1-Cx}$
\end{itemize}
(aiemmassa ratkaisussa kirjoitettu $C$:n tilalle $1/C$), jolloin $y(x)=0$ sisältyy tähän
($C=0$). Tapauksessa b) vieläkin 'erikoisempi' ratkaisu on
\[
y(x)=\begin{cases}
\,0,                  &\text{ kun } x\leq C, \\
\,\frac{1}{4}(x-C)^2, &\text{ kun } x>C.
\end{cases} \quad\loppu
\]
\end{Exa}
%\begin{Exa}
%Differentiaaliyhtälön
%\[
%y'=\sqrt[3]{y^3+1}
%\]
%yleinen ratkaisu koostuu funktioista, jotka eivät ole alkeisfunktiota. Yksi 
%alkeisfunktioratkaisu on kuitenkin keksittävissä: $y(x)=-1$. Tätä on syytä epäillä
%erikoisratkaisuksi. \loppu
%\end{Exa}

\subsection*{Käyräparven differentiaaliyhtälö}
\index{differentiaaliyhtälö!e@käyräparven|vahv}
\index{kzyyrzy@käyräparvi|vahv}

Differentiaaliyhtälön yleinen ratkaisu voidaan tulkita geometrisesti \kor{käyräparveksi} 
(eli käyrien joukoksi). Jos tunnetaan käyräparvi, niin sen differentiaaliyhtälö on 
johdettavissa derivoimalla. Nimittäin jos käyräparven funktiot ovat muotoa 
$y=Y(x,C_1,\ldots,C_n)$, niin derivoimalla $n$ kertaa saadaan $n+1$:n yhtälön ryhmä, josta 
vakiot $C_1,\ldots,C_n$ ovat (ainakin periaatteessa) eliminoitavissa. Tällöin saadaan ko.\ 
käyräparvelle differentiaaliyhtälö kertalukua $n$.
\begin{Exa}
Minkä differentiaaliyhtälön yleinen ratkaisu on 
\[ 
\text{a)}\ \ y=(C+x)e^x, \quad\ \text{b)}\ \ y=\frac{C_1}{x+C_2}\ ? 
\]
\end{Exa}
\ratk a) \ Derivoimalla kerran saadaan yhtälöryhmä
\[
\left\{
\begin{aligned}
y &=Ce^x+xe^x \\
y'&=Ce^x+xe^x+e^x
\end{aligned}
\right.
\]
Vähennyslaskulla saadaan differentiaaliyhtälöksi
\[
y'-y=e^x.
\]
b) \ Derivoidaan kahdesti:
\[
\begin{cases}
\,y\  =C_1(x+C_2)^{-1} \\
\,y'\,=-C_1(x+C_2)^{-2} \\
\,y'' =2C_1(x+C_2)^{-3}
\end{cases}
\]
Eliminoimalla $C_1$ ja $C_2$ saadaan differentiaaliyhtälöksi
\[
yy''=2(y')^2. \loppu
\]

\subsection*{Kohtisuorat leikkaajat}
\index{differentiaaliyhtälö!f@kohtisuorien leikkaajien|vahv}
\index{kzyyrzy@käyräparvi|vahv}
\index{kohtisuora leikkaus!b@käyräparvien|vahv}

Jos yksiparametrisen käyräparven $y=Y(x,C)$ differentiaaliyhtälö on 
\[ 
y'=f(x,y), 
\] 
niin ratkaisemalla differentiaaliyhtälö
\[ 
y'=-\frac{1}{f(x,y)} 
\]
löydetään käyräparven \kor{kohtisuorat leikkaajat}
(vrt.\ Esimerkki \ref{derivaatta geometriassa}:\ref{kohtisuora leikkaus}).
\begin{Exa}
Ympyräparven $\,x^2+y^2=C^2\,$ differentiaaliyhtälöksi saadaan implisiittisesti derivoimalla
\[
x+yy'=0.
\]
Kohtisuorien leikkaajien differentiaaliyhtälön
\[
x-y/y'=0
\]
yleinen ratkaisu on $y=Cx$, kuten saattoi (geometrisesti) arvata. \loppu
\end{Exa}

\subsection*{Alku- ja reunaehdot}
\index{alkuehto (DY:n)|vahv}
\index{reunaehto|vahv}

Koska differentiaaliyhtälön ratkaisu sisältää määräämättömiä vakioita, tarvitaan 
sovellustilanteissa lisäehtoja, jotta ratkaisu olisi yksikäsitteinen. Lisäehdot on pääteltävä 
sovellustilanteesta, eli ne kuuluvat matemaattiseen malliin samoin kuin itse
differentiaaliyhtälökin.

Jos differentiaaliyhtälön kertaluku on $n$, niin ratkaisussa on yleensä $n$ määrämätöntä 
vakiota, jolloin tarvitaan $n$ lisäehtoa. Yksinkertaisin tapa asettaa
lisäehdot on kiinnittää jossakin pisteessä $x_0$ derivaattojen $y^{(k)}(x_0)$ arvot, kun
$k=0\ldots n-1$. Näin saadaan
\index{differentiaaliyhtälön!d@alkuarvotehtävä} \index{alkuarvotehtävä}%
\kor{alkuarvotehtävä} (engl.\ initial value problem)
\[
\left\{
\begin{aligned}
&F(x,y',\ldots,y^{(n)}) = 0,\quad x\in (a,b), \\
&y(x_0)\qquad = A_0, \\
&\quad\vdots \qquad\quad\,\ \vdots \; \; \; \vdots \\
&y^{(n-1)}(x_0) = A_{n-1}.
\end{aligned}
\right.
\]
Tässä voi olla $x_0\in(a,b)$ tai myös $x_0=a$ tai $x_0=b$. Jos $x_0$ on välin päätepiste, niin 
alkuehdot on tulkittava derivaattojen $y^{(k)}(x)$ \pain{tois}p\pain{uolisina} 
\pain{ra}j\pain{a-arvoina} (tai toispuolisina derivaattoina) kun $x\kohti a^+$ ($x_0=a$) tai 
$x\kohti b^-$ ($x_0=b$). Edellytys on tällöin, että differentiaaliyhtälön ratkaisuille nämä ovat
olemassa.
\begin{Exa}
Jos kappale (massa $=m$) on hetkellä $t=0$ levossa pisteessä $x_0$, ja kappaleeseen vaikuttaa 
voima $f(t)$ kun $t>0$, niin kappaleen sijainti $x(t)$ hetkellä $t$ saadaan selville 
ratkaisemalla alkuarvoprobleema
\[
\left\{ \begin{aligned}
&mx''(t) = f(t),\quad t>0, \\
&x(0)\, = x_0, \\
&x'(0)  = 0.
\end{aligned} \right.
\]
Alkuehdot voi asettaa täsmällisemmin muodossa $x(0^+)=x_0,\ D_+ x(0)=0$. \loppu
\end{Exa}
Lisäehtoja voidaan myös asettaa useammassa pisteessä. Alkuarvotehtävän ohella tyypillisin on
\index{differentiaaliyhtälön!e@reuna-arvotehtävä} \index{reuna-arvotehtävä}%
\kor{kahden pisteen reuna-arvotehtävä} (engl. two-point boundary value problem), jossa ehdot
asetetaan tarkasteltavan välin päätepisteissä. Jos $n=2$, niin kahden pisteen reuna-arvotehtävän
normaalimuoto on
\[ \left\{ \begin{aligned}
&y'' =f(x,y,y'),\quad x\in(a,b), \\
&y(a)=A, \ y(b)=B.
\end{aligned} \right. \]
Reunaehtojen asettelussa on tässä oletettava, että ratkaisu on oikealta jatkuva $a$:ssa ja
vasemmalta jatkuva $b$:ssä. Koska ratkaisu on joka tapauksesa (kahdestikin) derivoituvana
jatkuva välillä $(a,b)$, niin lisäoletukset tarkoittavat samaa kuin jatkuvuus välillä $[a,b]$.
\begin{Exa}
Ratkaise kahden pisteen reuna-arvotehtävä
\[
\left\{ \begin{aligned}
&y'''+2y''=0,\quad x\in (0,1), \\
&y(0)=1, \ y'(0)=-1, \ y(1)=0.
\end{aligned} \right.
\]
\end{Exa}
\ratk Yleinen ratkaisu on (ks.\ Esimerkki 2)
\[
y(x)=C_1e^{-2x}+C_2x+C_3,
\]
joten saadaan yhtälöryhmä
\[ \left\{ \begin{array}{rrrrrrrrr}  y(0)&=&        C_1& &   &+&C_3&=& 1 \\
                                    y'(0)&=&    -2\,C_1&+&C_2& &   &=&-1 \\
                                     y(1)&=&e^{-2}\,C_1&+&C_2&+&C_3&=& 0
\end{array} \right. 
   \qimpl \left\{ \begin{aligned} C_1&=\ 0 \\ C_2&=-1 \\ C_3&=\ 1 \end{aligned} \right. \] 
Ratkaisu on siis $\,y(x)=-x+1$. \loppu 

\subsection*{Differentiaaliyhtälösysteemit}
\index{differentiaaliyhtälö!g@--systeemi|vahv}

\kor{Differentiaaliyhtälösysteemillä} tarkoitetaan useamman differentiaaliyhtälön muodostamaa
yhtälöryhmää. 
\index{differentiaaliyhtälön!a@normaalimuoto} \index{normaalimuoto!a@DY:n, DY-systeemin}%
\kor{Normaalimuotoinen} tavallinen differentiaaliyhtälösysteemi on jollakin
$n\in\N,\ n \ge 2$ muotoa
\[
 \left\{ \begin{aligned} 
         y'_1 &= f_1(x,y_1, \ldots, y_n), \\
         y'_2 &= f_2(x,y_1, \ldots, y_n), \\
              &\vdots \\
         y'_n &= f_n(x,y_1, \ldots, y_n).
         \end{aligned} \right.
\]
Tässä $x$ on riippumaton muuttuja ja funktiot $y_i(x),\ i=1 \ldots n,$ ovat tuntemattomia.
Ratkaisu (yksittäinen tai yleinen) on ko.\ systeemin jollakin avoimella välillä toteuttavien
funktioiden $y_i$ muodostama (järjestetty) joukko
\[
\my(x)=(y_1(x), \ldots y_n(x)).
\]
Tämä on itse asiassa funktioiden muodostama \pain{vektori} eli vektoriarvoinen funktio.
Tapauksissa $n=2,3$ ratkaisu on haluttaessa tulkittavissa tason tai avaruuden parametriseksi 
käyräksi (parametrina tässä $x$, vrt.\ Luku \ref{parametriset käyrät}).
\index{alkuarvotehtävä}%
\kor{Alkuarvotehtävässä} vaaditaan, että ratkaisu toteuttaa differentiaaliyhtälöiden lisäksi
$n$ lisäehtoa muotoa
\[ 
y_i(x_0) = A_i, \quad i = 1 \ldots n.
\]

Normaalimuotoinen korkeamman kertaluvun differentiaaliyhtälö
\[
y^{(n)}=f(x,y',\ldots,y^{(n-1)})
\]
voidaan aina kirjoittaa normaalimuotoiseksi differentiaaliyhtälösysteemiksi. Nimittäin kun 
kirjoitetaan
\[
y_1=y,\ y_2=y',\ \ldots,\ y_n=y^{(n-1)},
\]
niin nämä yhtälöt yhdessä differentiaaliyhtälön kanssa muodostavat systeemin
\[
 \left\{ \begin{aligned} 
         y'_1 \quad  &= y_2, \\
                     &\vdots \\
         y'_{n-1}\,  &= y_n, \\
         y'_n \quad  &= f(x,y_1, \ldots y_n).
         \end{aligned} \right.
\]
Tämä on em.\ normaalimuotoa.
\begin{Exa} Differentiaaliyhtälön $\,y'''=x^2y^3+x(y')^2+y''\,$ systeemimuoto on
\[
 \left\{ \begin{aligned} 
         y'_1 &= y_2, \\
         y'_2 &= y_3, \\
         y'_3 &= x^2y_1^3+xy_2^2+y_3.
         \end{aligned} \right. \loppu
\]
\end{Exa}

Korkeamman kertaluvun differentiaaliyhtälön kirjoittaminen systeemimuotoon auttaa sekä 
teoreettisissa tarkasteluissa että numeerisissa ratkaisumenetelmissä (ks.\ Luvut 
\ref{vakikertoimiset ja Eulerin DYt}, \ref{DYn numeeriset menetelmät},
\ref{Picard-Lindelöfin lause}). Joskus systeemimuoto on edullinen myös klassisissa
ratkaisumenetelmissä (ks.\ Luku \ref{toisen kertaluvun dy}).

\subsection*{Ratkaisujen säännöllisyys}
\index{differentiaaliyhtälön!f@ratkaisun säännöllisyys|vahv}

Alkuarvotehtävää voi pitää implisiittisenä funktion $y(x)$ määritelmänä, jolloin implisiittisen
derivoinnin avulla on mahdollista laskea ratkaisufunktion korkeampia derivaattoja 
alkuarvopisteessä $x_0$. Tällä tavoin voidaan usein myös selvittää, kuinka säännöllinen ratkaisu
on sellaisella välillä, jolla se on ($n$ kertaa derivoituvana) olemassa.
\begin{Exa} \label{Airyn DY} Olkoon $a>0$ ja tarkastellaan alkuarvotehtävää
\[
\begin{cases} \,y'=x+y^2, \quad x\in(-a,a), \\ \,y(0)=0. \end{cases}
\]

Jos oletetaan tehtävä ratkeavaksi (kysymystä tarkastellaan myöhemmin Luvussa 
\ref{Picard-Lindelöfin lause}), niin differentiaaliyhtälöstä voi päätellä, että ratkaisu on 
ko.\ välillä mielivaltaisen monta kertaa derivoituva (sileä). Nimittäin ensinnäkin, koska
ratkaisu $y(x)$ on välillä $(-a,a)$ derivoituva (perusoletus), niin differentiaaliyhtälön
mukaan se on myös kahdesti derivoituva:
\[ 
y''(x) = \frac{d}{dx}[x+(y(x))^2] 
       = 1 + 2y(x)y'(x) = 1+2y(x)[x+(y(x))^2], \quad x \in (-a,a). 
\]
Tämän mukaan $y''$ on edelleen derivoituva välillä $(-a,a)$, eli $y$ on kolmesti derivoituva,
jne. Päätellään, että $y(x)$ on mielivaltaisen monta kertaa derivoituva välillä $(-a,a)$.
Ratkaisufunktion derivaatat alkuarvopisteessä $x=0$ ovat myös suoraan laskettavissa, sillä
koska $y(0)=0$ (alkuehto), niin differentiaaliyhtälön mukaan on oltava $y'(0)=0$, jolloin em.\
lauseke $y''(x)$:lle antaa $y''(0)=1$. Jatkamalla implisiittistä derivointia nähdään, että 
ratkaisufunktion \pain{kaikki} derivaatat pisteessä $x=0$ määräytyvät yksikäsitteisesti (!).
\loppu
\end{Exa}

\subsection*{Ratkaiseminen kvadratuureilla}
\index{differentiaaliyhtälön!g@ratkaiseminen kvadratuureilla|vahv}

Jatkossa tarkastellaan lähinnä sellaisia diferentiaaliyhtälöiden erkoistapauksia, joille on
mahdollista laskea 'tarkka' ratkaisu palauttamalla tehtävä tunnettujen funktioiden
integraalifunktioiden etsimiseksi. Sanotaan tällöin, että differentiaaliyhtälö ratkeaa 
\index{kvadratuuri}%
\kor{kvadratuureilla} eli 'integroimisilla'. (Kvadratuuri tarkoittaa sananmukaisesti
'neliöimistä', vrt.\ alaviite Luvussa \ref{numeerinen integrointi}.) Ratkeaminen kvadratuureilla
\pain{ei} edellytä, että integraalifunktiot ovat alkeisfunktioita.
\begin{Exa}
Alkuarvotehtävä
\[
\left\{ \begin{aligned}
&\,y''' = \sin x/x=1-\frac{x^2}{6}+\frac{x^4}{120}-\ldots,\quad x>0, \\
&\,y(0) =1,\ y'(0)=y''(0)=0
\end{aligned} \right.
\]
ratkeaa kolmella peräkkeisellä kvadratuurilla:
\begin{align*}
y''(x) &= y''(x)-y''(0) =\int_0^x y'''(t)dt =\int_0^x (\sin t / t)dt \\
       &= x-\frac{x^3}{18}+\frac{x^5}{600}-\ldots \\[1mm]
y'(x)  &= y'(x)-y'(0)=\int_0^x y''(t)dt \\
       &= \frac{x^2}{2}-\frac{x^4}{72}+\frac{x^6}{3600}-\ldots
\end{align*}
\begin{align*}
y(x)   &= y(0)+[y(x)-y(0)] = y(0) + \int_0^x y'(t)dt \\
       &= 1+\frac{x^3}{6}-\frac{x^5}{360}+\frac{x^7}{25200}-\ldots \quad\loppu
\end{align*}
\end{Exa}
\jatko\jatko \begin{Exa} (jatko) Esimerkin differentiaaliyhtälö $\,y'=x+y^2\,$ ei ratkea
kvadratuureilla.
\index{Riccatin differentiaaliyhtälö} \index{differentiaaliyhtälö!q@Riccatin}%
(Yhtälö on \kor{Riccatin} tyyppiä,
ks.\ Harj.teht.\,\ref{lineaarinen 1. kertaluvun DY}:\ref{H-dy-4: Riccatin DY}.) \loppu
\end{Exa}
Viimeisessä esimerkissä ei tarkoiteta, ettei ratkaisuja ole, vaan ainoastaan, että
ratkaiseminen ei palaudu 'integroimisiin'. Ratkaisu on tällöin määrättävä muilla keinoin.
Esim.\ jos alkuarvo $y(x_0)$ on tunnettu, voidaan käyttää alkuarvotehtävien numeerisia
ratkaisumenetelmiä (ks.\ Luku \ref{DYn numeeriset menetelmät}). Myös Taylorin polynomit 
antavat likimääräistä tietoa ratkaisun kulusta (vrt.\ Harj.teht.\,\ref{H-dy-1: DY ja Taylor}).

\Harj
\begin{enumerate}

\item
Tarkista, että $y=2x+Ce^{x}$ on differentiaaliyhtälön $y'=y+2(1-x)$ yleinen ratkaisu. Piirrä
pisteiden $(0,1)$ ja $(0,-1)$ kautta kulkevat ratkaisukäyrät.

\item
Ratkaise kvadratuureilla (yleinen ratkaisu tai alkuarvotehtävän ratkaisu):
\begin{align*}
&\text{a)}\ \ y''=\sin x \qquad
 \text{b)}\ \ y'''=24x+\cos x \qquad
 \text{c)}\ \ y''=\ln x \\[1mm]
&\text{d)}\ \ y''=\frac{1}{x^2+1}\,,\,\ x\in\R,\,\ y(0)=1,\ y'(0)=0 \\
&\text{e)}\ \ y''=\frac{1}{x^2-2x}\,,\,\ x\in(0,2),\,\ y(1)=y'(1)=0
\end{align*}

\item Määritä seuraavien differentiaaliyhtälöiden yleiset ratkaisut palauttamalla yhtälöt
ensimmäiseen kertalukuun (sijoitus $u(x)=y^{(k)}(x)$).
\[
\text{a)}\ \ y''-y'=0 \qquad
\text{b)}\ \ 2y'''+3y''=0 \qquad
\text{c)}\ \ y^{(5)}-5y^{(4)} = 0
\]

\item
Määritä seuraavien käyräparvien differentiaaliyhtälöt
\begin{align*}
&\text{a)}\ \ y=\sin(x+C) \qquad
 \text{b)}\ \ y=C_1+C_2\ln\abs{x} \qquad
 \text{c)}\ \ y=C_1+\frac{1}{x+C_2} \\
&\text{d)}\ \ y=(1+C_1)\ln\abs{x+C_2}-C_1 x+C_2 \qquad
 \text{e)}\ \ x=C_1e^y+C_2e^{-y}+3 \\[2mm]
&\text{f)}\ \ \text{$x$-kaselia sivuavat ympyrät} \qquad
 \text{g)}\ \ \text{suoraa $\,y=x\,$ sivuavat ympyrät}
\end{align*}

\item 
Käyrän $y=F(x)$ liukuessa pitkin $y$-akselia muodostuu käyräparvi. Määritä ko.\ parven
kohtisuorat leikkaajat, kun a) $F(x)=e^x$, b) $F(x)=\ln\abs{x}$.

\item
Käyrä $y=u(x)$ leikkaa kohtisuorasti differentiaaliyhtälön $y'=x+y^2$ ratkaisukäyrät.
Minkä differentialiyhtälön ratkaisu $u$ on?

\item
Esitä normaalimuotoisena differentiaaliyhtälösysteeminä:
\begin{align*}
&\text{a)}\ \ yy''+xy'=0 \qquad
 \text{b)}\ \ y''=(x+y')^2+y''' \qquad
 \text{c)}\ \ y^{(4)}=\frac{y'y''}{1+x+y'''} \\
&\text{d)}\ \ \begin{cases} u'=(u+v)^2, \\ v''=x+uv' \end{cases} \quad
 \text{e)}\ \ \begin{cases} u''=uv, \\ v''=-xuv \end{cases} \quad
 \text{f)}\ \ \begin{cases} u^{(4)}=u''v''+v, \\ v''=x+u'''+2v' \end{cases}
\end{align*}

\item
Alkuarvotehtävällä
\[
\begin{cases} \,y'=xy+\sin y, \quad x\in\R, \\ \,y(x_0)=y_0 \end{cases}
\]
on yksikäsitteinen ratkaisu jokaisella $(x_0,y_0)\in\Rkaksi$. \ a) Päättele, että ratkaisu
on $\R$:ssä mielivaltaisen monta kertaa derivoituva (sileä). \ b) Näytä, että ratkaisukäyrä
joko sivuaa $x$-akselia tai ei kosketa sitä lainkaan.

\item
Alkuarvotehtävä $xy'=x+y,\ y(1)=1$ määrittelee pisteen $P=(1,1)$ kauttaa kulkevan käyrän $S$.
Määritä $S$:n kaarevuuskeskiö pisteessä $P$.

\item \label{H-dy-1: DY ja Taylor}
Määritä alkuarvotehtävän ratkaisufunktion $y(x)$ Taylorin polynomi $T_n(x,0)$ yleisellä tai 
annetulla $n$:n arvolla: \newline
a) \ $y''=y,\ y(0)=y'(0)=1$ \newline
b) \ $y''=-y,\ y(0)=1,\ y'(0)=0$ \newline
c) \ $y'=x+y^2,\ y(0)=1;\ n=3$ \newline
d) \ $y'=xy+\sin y,\ y(0)=1;\ n=3$ \newline
e) \ $yy''+y'+y=0,\ y(0)=1,\ y'(0)=0;\ n=4$ \newline
f) \ $y''=yy'-x^2,\ y(0)=y'(0)=1;\ n=4$

\item (*) \index{zzb@\nim!Koirakäyrä}
(Koirakäyrä) Rekan perävaunu on kiinnitetty vetoautoon akselitapilla, joka on origossa.
Vetoauton nokka osoittaa postiivisen $y$-akselin suuntaan. Perävaunun keskiviiva on
$x$-akselilla, ja perävaunun akselin (siis sen jolla pyörät ovat) keskipiste on pisteessä
$(a,0),\, a > 0$. Vetoauton liikkuessa pitkin positiivista $y$-akselia piirtää perävaunun
akselikeskiö erään käyrän $y=y(x)$ välillä $0<x\le a$. Johda tälle ''koirakäyrälle''
differentiaaliyhtälö
\[
y'=-\frac{\sqrt{a^2-x^2}}{x},
\]
ja määritä käyrä $y=y(x)$ tämän ratkaisuna. Huomioi myös alkuehto.

\item (*)
Seuraavat käyräparvet on annettu joko parametrimuodossa tai napakoordinaattien avulla.
Johda käyräparvien ja niiden kohtisuorien leikkaajien differentiaaliyhtälöt normaalimuodossa
$y'=f(x,y)$.
\begin{align*}
&\text{a)}\ \ \begin{cases} \,x=t+\cos t+C, \\ \,y=1+\sin t \end{cases} \qquad
 \text{b)}\ \ \begin{cases} \,x=e^t+t+C, \\ \,y=2e^t-t+2C \end{cases} \\[2mm]
&\text{c)}\ \ r=C\cos\varphi \qquad
 \text{d)}\ \ r=C\varphi \qquad
 \text{e)}\ \ r=Ce^\varphi
\end{align*}

\item (*) \index{verhokäyrä}
Käyrää $S:\,y=y_0(x)$ sanotaan yksiparametrisen käyräparven $y=Y(x,C)$ \kor{verhokäyräksi}
(engl.\ envelope),
jos $S$ sivuaa jokaista parven käyrää (eli $S$:llä ja jokaisella parven käyrällä on yhteinen
piste ja siinä yhteinen tangentti). \vspace{1mm}\newline
a) Näytä, että jos käyräparven differentiaaliyhtälö on $F(x,y,y')=0$, niin myös $y=y_0(x)$ on
tämän differentiaaliyhtälön (erikois)ratkaisu. \newline
b) Suoraparvella $y=Cx+2C^2,\ C\in\R$ on verhokäyränä eräs toisen asteen polynomikäyrä
(paraabeli). Määritä tämä ja varmista piirtämällä kuvio! 

\item (*) \index{zzb@\nim!Sotaharjoitus 3}
(Sotaharjoitus 3) Tykinkuulan lentorata on parametrinen käyrä $\vec r=\vec r(t)$ ($t=$ aika).
Lentoradan pisteessä $(x,y,z) \vastaa \vec r\,$ kuulaan vaikuttavat voimat ovat
\begin{align*}
\vec G &= -mg\vec k, \\
\vec T &= T_1(x,y,z,t)\vec i+T_2(x,y,z,t)\vec j, \\
\vec F &= -k\abs{\vec v}\vec v,
\end{align*}
missä $\vec G$ on painovoima ($m=$ massa, $g=$ maan vetovoiman kiihtyvyys), $\vec T$ edustaa
tuulta ja $\vec F$ vauhdin neliöön verrannollista ilmanvastusvoimaa ($\vec v=\dvr,\ k=$ vakio).
Esitä ammuksen liikeyhtälö $\,m\vec r\,''=\vec G+\vec T+\vec F$ normaalimuotoisena 
differentiaaliyhtälösysteeminä kokoa $n=6$ kirjoittamalla
\[
(x,y,z,x',y',z') = (y_1, \ldots , y_6).
\]

\end{enumerate}