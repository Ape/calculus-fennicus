\chapter{Differentiaaliyhtälöt}
\index{differentiaaliyhtälö|vahv}

''Luonnonlait on kirjoitettu matematiikan kielellä'', lausui fyysikko ja tähtitieteilijä
\index{Galilei, G.}%
\hist{Galileo Galilei} (1564-1642). Galilein yhä ajankohtainen lausuma tuo erityisesti mieleen
\kor{differentiaaliyhtälöt}, sillä melkeinpä kaikki luonnonlait ovat sellaisia. Paitsi
luonnonlaeissa, differentiaaliyhtälöitä tavataan nykyisin mitä erilaisimmissa (ihmisen
luomissa) fysiikan, biologian, talousieteen ym.\ matemaattisissa malleissa. --- Tällaisten
'kantaäitinä' voi pitää aiemmin Luvuissa \ref{exp(x) ja ln(x)} ja 
\ref{eksponenttifunktio fysiikassa} tarkasteltua eksponentiaalisen kasvun tai vaimenemisen
mallia.

Tässä luvussa käydään ensin läpi differentiaaliyhtälöihin liittyvät peruskäsitteet
(Luku \ref{DY-käsitteet}), minkä jälkeen tarkastellaan differentiaaliyhtälöiden klassisia
ratkaisumenetelmiä ja näihin liittyviä sovellusesimerkkejä
(Luvut \ref{separoituva DY}--\ref{2. kertaluvun lineaarinen DY}). Differentiaaliyhtälöiden
perinteinen ratkaisemistekniikka on Luvuissa \ref{integraalifunktio}--\ref{osamurtokehitelmät}
läpikäytyyn integroimistekniikkaan pitkälti perustuva (ja tähän tekniikkaan verrattavissa oleva)
matematiikan taitolaji. Joillekin differentiaaliyhtälöiden erikoistyypeille ratkaisut ovat 
löydettävissä pelkällä 'sivistyneellä arvauksella', muissa tapauksissa ratkaiseminen pyritään
palauttamaan 'integroimisiin' eli \kor{kvadratuureihin}.

Differentiaaliyhtälöiden matemaattisessa paljoudessa perinteisin menetelmin ratkeavia voi pitää
harvinaisuuksina, mutta sovelluksissa tällaiset erikoistapaukset ovat kuitenkin melko yleisiä.
Luvuissa \ref{separoituva DY}--\ref{2. kertaluvun lineaarinen DY} käydään läpi näistä 
erikoistapauksista tavallisimmat. Luvussa \ref{DYn numeeriset menetelmät} tarkastellaan
perinteisiä ratkaisumenetelmiä yleispätevämpiä \kor{numeerisia} menetelmiä, joilla
differentiaaliyhtälöiden ja myös useamman yhtälön muodostamien 
\kor{differentiaaliyhtälösysteemien} ratkaisuja voidaan laskea likimäärin. Likimääräisen
ratkaisemisen ideoihin perustuu myös differentiaaliyhtälöiden teorian päälause,
\kor{Picardin--Lindelöfin lause}, joka esitellään Luvussa \ref{Picard-Lindelöfin lause}.
