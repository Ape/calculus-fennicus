\section{Kompleksiluvut ja niillä laskeminen} 
\label{kompleksiluvuilla laskeminen}
\alku
\sectionmark{Kompleksiluvut}
\index{laskuoperaatiot!e@kompleksilukujen|vahv}

\kor{Kompleksilukujen} määrittelyn lähtökohdaksi otetaan matematiikassa usein kunta 
$(\Rkaksi,+,\cdot)$, jossa laskutoimitukset on määritelty edellisen luvun säännöillä 
\eqref{yhteenlasku} ja \eqref{kertolasku}. Suoraviivaisin määritelmä on yksinkertaisesti sopia,
että $(\Rkaksi,+,\cdot)$ itse on kompleksilukujen kunta, ts. kompleksiluvut ovat $\Rkaksi$:n 
lukupareja, joiden väliset laskutoimitukset on määritelty mainitulla tavalla. Käytännön 
laskurutiineissa käytetään kuitenkin yleensä havainnollisempia kompleksilukujen esitystapoja. 
Näitä ovat edellä kuvattu polaarimuoto (osoitinmuoto) tai sitäkin yleisempi,
komponenttimuodosta johdettu kolmas esitystapa, joka seuraavassa otetaan kompleksilukujen
käytännöllisen määrittelyn lähtökohdaksi.

Aloitetaan merkitsemällä $\Rkaksi$:n '$x$-akselia' symbolilla $\Rkaksi_0$:
\[
\Rkaksi_0 = \{(x,0) \ | \ x \in \R\}.
\]
Havaitaan, että $\Rkaksi_0$ sisältää $(\Rkaksi,+,\cdot)$:n nolla-alkion $(0,0)$, samoin 
ykkösalkion $(1,0)$. Lisäksi jos $(x,0)$, $(y,0)$ ovat mitkä tahansa kaksi $\Rkaksi_0$:n 
alkiota, niin edellisen luvun laskusääntöjen perusteella näiden summa, erotus, tulo ja
osamäärä (jos $y \neq 0$) ovat myös $\Rkaksi_0$:ssa. Nämä tosiasiat yhdessä merkitsevät, että 
$(\Rkaksi_0,+,\cdot)$ on itsekin kunta ja siis $(\Rkaksi,+,\cdot)$:n \pain{alikunta} 
(vrt.\ Luku \ref{kunta}). Toisaalta havaitaan, että $\Rkaksi_0$:n ja  $\R$:n välinen ilmeinen
vastaavuus
\[
(x,0) \in \Rkaksi_0 \ \leftrightarrow \ x \in \R
\]  
ulottuu samanlaisena laskutoimituksiin: Jos $\Rkaksi_0$:ssa suoritetaan laskutoimitus kunnan 
$(\Rkaksi,+,\cdot)$ säännöillä, niin tätä vastaa $(\R,+,\cdot)$:n laskutoimitus normaalissa 
reaalilukujen kunnassa. Esimerkiksi kertolasku
\[
(x_1,0) \cdot (x_2,0) = (x_1x_2,0)
\]
(sovellettu edellisen luvun sääntöä \eqref{kertolasku}) vastaa kertolaskua $\R$:ssä, kun 
käytetään em.\ vastaavuusperiaatetta:
\[
\begin{array}{rccccc}
\Rkaksi_0: &(x_1,0) &\cdot &(x_2,0) &= &(x_1x_2,0) \\
&\updownarrow & &\updownarrow & &\updownarrow\\ 
\R: &x_1 &\cdot &x_2 &= &x_1x_2
\end{array}
\]
Sama pätee yhteelaskulle, joten kunnilla $(\Rkaksi_0,+,\cdot)$ ja $(\R,+,\cdot)$ ei käytännön 
laskennan kannalta ole mitään eroa. Tehdäänkin tämän perusteella kunnassa $(\Rkaksi,+,\cdot)$ 
samastussopimus
\[ (x,0) = x \quad \forall x \in \R. \]
Tällä sopimuksella reaalilukujen kunnasta $(\R,+,\cdot)$ tulee kunnan $(\Rkaksi,+,\cdot)$ 
alikunta. Merkitään vielä $(0,1)=i$, 
jolloin pätee
\begin{align*} 
(x,y) = (x,0) + (0,y) &= (x,0) + (y,0)(0,1) \\
                      &= x + yi = x+iy. 
\end{align*}
Kunnan $(\Rkaksi,+,\cdot)$ kertolaskun määritelmän ja tehdyn samastussopimuksen mukaan on
\[ 
i^2 = i \cdot i = (-1,0) = -1. 
\]
\begin{Def} \vahv{(Kompleksiluvut)} \label{kompleksilukujen määritelmä}
\index{kompleksiluvut|emph} \index{imaginaariluku $i$|emph}
Kompleksilukujen joukko $\C$ sisältää reaaliluvut ja lisäksi nk.\ \kor{imaginaariluvun} $i$,
joka toteuttaa
\[
i^2=i \cdot i = -1 \in \R. 
\]
Kompleksiluvut ovat kunta $(\C,+,\cdot)$, missä jokaisella $z\in\C$ on yksikäsitteinen
esitysmuoto
\[
z=x+iy, \quad x,y \in \R.
\]
Erityisesti on $x+0i=x\in\R$ ja $0+1i=i$.
\end{Def}
Määritelmän mukaisesti $(\C,+,\cdot)$ syntyy reaalilukujen kunnan $(\R,+,\cdot)$ laajennuksena,
kun lukujoukkoon $\R$ lisätään imaginaariluku $i$ ja huomioidaan kunnan perusaksiooma (K0)
--- vrt.\ Luvun \ref{kunta} Esimerkki \ref{muuan kunta}, jossa kuntaa $(\Q,+,\cdot)$ 
laajennettiin vastaavalla tavalla lisäämällä luku $a=\sqrt{2}$. Luvulle $i$ asetettu ehto 
$i^2=-1$ ei toteudu millekään $i\in\R$, joten $i$ on 'aidosti imaginaarinen' luku. 
Osoitinkunnassa $(\Pkunta, + , \cdot)$ laskusääntöä $i^2=-1$ vastaa tulos
\[
\pti \cdot \pti = 1 \vkulma{\pi} = -(1 \vkulma{0}) = - \pointer{1}.
\]

Määritelmän \ref{kompleksilukujen määritelmä} perusteella kompleksiluvuilla voi operoida 
jokseenkin normaalisti, eli reaalilukujen algebrasta tutulla tavalla. Normaalista poikkeaa
vain laskusääntö $i^2=-1$. Laskennassa kompleksiluvun esitysmuotoja voi vaihdella vapaasti 
edellisen luvun muunnossääntöjen puitteissa. Tavallisimmin käytetään joko Määritelmän 
\index{perusmuoto!bb@kompleksiluvun}%
\ref{kompleksilukujen määritelmä} mukaista kompleksiluvun \kor{perusmuotoa}, tai sitten
polaarimuotoa. Saman luvun eri esitysmuotojen välillä käytetään jatkossa joko
vastaavuusmerkintää '$\vastaa$' tai, sikäli kuin luontevaa, yksinkertaisesti samastusta '$=$'.
\begin{Exa} Saata kompleksiluku $(2+3i)\cdot(5-2i)$ perusmuotoon $x+iy$. \end{Exa}
\ratk Määritelmän \ref{kompleksilukujen määritelmä} perusteella
\[
(2+3i)\cdot(5-2i) = 2\cdot 5 - 2\cdot 2i + 5\cdot 3i - 3\cdot 2\cdot i^2 = 16+11i. \loppu
\]
\begin{Exa} Olkoon $z=a+ib \neq 0$. Laske $z^{-1}$ perusmuodossa. \end{Exa}
\ratk Jos $z^{-1}=x+iy$, niin on oltava
\begin{align*}
&(a+ib)(x+iy)=1 \\
\ekv \ &(ax-by)+i(bx+ay)=1=1+0i \\
\ekv \ &\left\{ \begin{array}{ll}
ax-by=1 \\
bx+ay=0
\end{array} \right. \\
\ekv \ & x=\frac{a}{a^2+b^2}\,, \quad y=-\frac{b}{a^2+b^2}\,.
\end{align*}
Siis
\[
z^{-1}=\frac{1}{a^2+b^2}(a-ib) = \frac{a}{a^2+b^2} - \frac{b}{a^2+b^2}\,i. \loppu
\]
\begin{Def} \label{kompleksilukujen terminologiaa}
\index{reaaliosa|emph} \index{imaginaariosa|emph} \index{itseisarvo|emph}
\index{vaihekulma|emph} \index{napakulma|emph} \index{liittoluku|emph} \index{konjugaatti|emph}
\index{moduuli (kompleksiluvun)|emph} \index{argumentti (kompleksiluvun)|emph}
Kompleksiluvun
\[
z=x+iy \in \C \ \vastaa \ (x,y) \in \Rkaksi \ \vastaa \ r \vkulma{\varphi} \in \Pkunta
\]
\begin{itemize}
\item[-] \kor{reaaliosa} on $\ \re z = x = r \cos{\varphi}$.
\item[-] \kor{imaginaariosa} on $\ \im z = y = r \sin{\varphi}$.
\item[-] \kor{itseisarvo} (moduuli) on $\ \abs{z}=\sqrt{x^2+y^2}=r$.
\item[-] \kor{vaihekulma} (argumentti, napakulma) on $\ \arg z = \varphi$.
\item[-] \kor{liittoluku} eli \kor{konjugaatti} on 
              $\overline{z}=x-iy \ \vastaa \ (x,-y) \ \vastaa \ r \vkulma{-\varphi}\,$.
\end{itemize}
\end{Def}
Kompleksiluvuilla laskettaessa seuraavia Määritelmään \ref{kompleksilukujen terminologiaa} 
liittyviä kaavoja tarvitaan usein:
\[ \boxed{ \begin{aligned}
\quad\ykehys &\text{(1)} \qquad \overline{z_1+z_2} = \overline{z_1} + \overline{z_2} \\
             &\text{(2)} \qquad \overline{z_1z_2} = \overline{z_1}\,\,\overline{z_2} \\
             &\text{(3)} \qquad z \overline{z} = \abs{z}^2 \\
             &\text{(4)} \qquad \abs{z_1z_2} = \abs{z_1} \abs{z_2} \\
             &\text{(5)} \qquad \abs{z^{-1}} = \abs{z}^{-1} \\
             &\text{(6)} \qquad \abs{\overline{z}} = \abs{z} \\
             &\text{(7)} \qquad z+\overline{z} = 2\,\text{Re} \, z, \quad z-\overline{z} 
                                               = 2i\,\text{Im} \, z \qquad\\
             &\text{(8)} \qquad \overline{\overline{z}} = z \akehys
           \end{aligned} } \]
Nämä voi helposti perustella määritelmistä (kaavat (2)--(5) suorimmin polaarimuodosta). 

\jatko \begin{Exa} (jatko) Esimerkissä laskettiin kompleksiluvun $z \neq 0$ käänteisluku 
$z^{-1}$. Koska $z \neq 0\,\ekv\,\overline{z} \neq 0$, niin kaavaa (3) käyttäen voidaan
päätellä myös seuraavasti:
\begin{align*}
           & zz^{-1} = 1 \\
\ekv \quad &\overline{z}(zz^{-1}) = \overline{z} \\
\ekv \quad &(\overline{z}z)z^{-1} = \overline{z} \\
\ekv \quad &\abs{z}^2z^{-1} = \overline{z} \\
\ekv \quad &z^{-1} = \abs{z}^{-2} \overline{z}.
\end{align*}
Tämä vastaa normaalia lavennusmenettelyä
\[
z^{-1} = \frac{1}{a+ib} = \frac{(a - ib)}{(a+ib)(a-ib)} 
= \frac{1}{a^2+b^2}(a-ib)\,.
\]
Vielä nopeampi on kuitenkin osoitinlasku:
\[
z = r \vkulma \varphi\ \impl\ z^{-1} = r^{-1} \vkulma{-\varphi} 
  = r^{-2}(r \vkulma{-\varphi}) = \abs{z}^{-2} \overline{z}. \loppu
\]
\end{Exa}
\begin{Exa} Saata $z=(1+i)^7$ perusmuotoon $x+iy$. \end{Exa}
\ratk Tässäkin polaariesitys on tehokkain: Koska
\[
1+i = r \vkulma{\varphi}\,,\quad \text{missä}\ \quad r
    = \sqrt{1+1} = \sqrt{2}, \quad \varphi = \frac{\pi}{4}\,, \\
\]
niin
\begin{align*}
(1+i)^7 \ &=\ (\sqrt{2})^7 \vkulma{7\pi/4} \\
          &=\ 8 \sqrt{2}\,\vkulma{-\pi/4} \\
          &=\ 8 \sqrt{2}\,(\cos\tfrac{\pi}{4} - i\,\sin\tfrac{\pi}{4}) \\[1mm]                
          &=\ 8(1-i). \loppu
\end{align*}
Kompleksialgebran kaavoista maininnan arvoinen on vielä
\index{de Moivren kaava}%
\kor{de Moivren kaava}
\[
\boxed{\kehys\quad \text{(9)} \qquad (\cos \varphi + i \sin \varphi)^n 
                                        = \cos n \varphi + i \sin n \varphi. \quad}
\]
Tämä seuraa välittömästi, kun osoitinlaskennan tulos
\[
(1 \vkulma{\varphi})^n = 1 \vkulma{n \varphi}
\]
esitetään kompleksilukujen perusmuodossa.
\begin{Exa} de Moivren kaavan ja binomikaavan mukaan
\begin{align*}
\cos 3 \varphi + i \sin 3 \varphi\ &=\ (\cos \varphi + i \sin \varphi)^3 \\
&=\ \cos^3 \varphi + 3i \cos^2 \varphi \sin \varphi + 3 i^2 \cos \varphi \sin^2 \varphi 
                                                    +i^3 \sin^3 \varphi \\
&=\ (\cos^3 \varphi - 3\cos \varphi \sin^2 \varphi) + i(3 \cos^2 \varphi \sin \varphi 
                                                    - \sin^3 \varphi) \\[3mm]
\impl \ &\left\{
\begin{aligned}
\cos 3 \varphi\ &=\ \cos^3 \varphi - 3 \cos \varphi \sin^2 \varphi, \\
\sin 3 \varphi\ &=\ 3 \cos^2 \varphi \sin \varphi - \sin^3 \varphi
\end{aligned} \right.
\end{align*}
(vrt. Luvun \ref{trigonometriset funktiot} Esimerkki \ref{sin kolme alpha}). \loppu
\end{Exa}

\subsection*{Kolmioepäyhtälö $\C$:ssä}
\index{kolmioepäyhtälö!e@kompleksilukujen|vahv}

Kompleksilukujen yhteenlasku on samanlainen operaatio kuin tason vektorien yhteenlasku, ja
myös kompleksiluvun itseisarvo vastaa vektorin itseisarvoa. Tästä syystä kompleksiluvuille
pätee myös samaa muotoa oleva kolmioepäyhtälö kuin vektoreille:
\[
\boxed{\kehys\quad \abs{\abs{z_1}-\abs{z_2}} \le \abs{z_1 + z_2} \leq \abs{z_1} + \abs{z_2}, 
                                                                 \quad z_1,z_2 \in \C. \quad}
\]

Kolmioepäyhtälö on hyvin keskeinen työkalu kompleksilukuihin perustuvassa matemaattisessa
analyysissä eli \kor{kompleksianalyysissä}. Tyypillisenä esimerkkinä kolmioepäyhtälön käytöstä 
tarkasteltakoon väittämää, joka koskee yleistä
\index{kompleksimuuttujan!a@polynomi} \index{polynomi (kompleksimuuttujan)}%
\kor{kompleksimuuttujan polynomia} muotoa
\[
p(z) = c_0 + c_1z + \cdots + c_nz^n = \sum_{k=0}^{n} c_k z^k, \quad c_n \neq 0. 
\]
Tässä siis $z \in \C$ on \kor{kompleksimuuttuja} ja myös luvut $c_k$, eli polynomin
\index{kerroin (polynomin)} \index{aste (polynomin)}% 
\kor{kertoimet}, ovat kompleksilukuja. Luku $n\in\N\cup\{0\}$ on polynomin \kor{aste}.

Kun kompleksimuuttujan polynomiin sovelletaan kolmioepäyhtälöä sekä em.\ kaavoja (4),\,(5), 
tullaan väittämään, jonka mukaan polynomin itseisarvo $\abs{p(z)}$ kasvaa riittävän suurilla 
$\abs{z}$:n arvoilla kvalitatiivisesti samalla tavoin kuin polynomin korkeimman asteisen
termin itseisarvo, eli riittävän suurilla $\abs{z}$:n arvoilla
\[ 
\abs{p(z)} \sim \abs{c_n z^n} = \abs{c_n}\abs{z}^n 
\]
(tässä on käytetty kaavaa (4)). Väittämä voidaan muotoilla täsmällisemmin esim.\ seuraavasti
(ks.\ myös Harj.teht.\,\ref{H-III-2: polynomin kasvu}):
\begin{Prop} \label{polynomin kasvu} \vahv{(Polynomin kasvu)} Jos 
$p(z) = \sum_{k=0}^{n} c_k z^k$, missä  $c_k \in \C$, $n \in \N$ ja $c_n \neq 0$, niin on
olemassa $R \in \R_+\,$ siten, että pätee
\[
\frac{1}{2}\abs{c_n}\abs{z}^n\,\le\,\abs{p(z)}\,\le\,\frac{3}{2}\abs{c_n}\abs{z}^n,\quad 
                                        \text{kun} \ z \in \C\ \text{ja}\ \abs{z} \ge R.
\]
\end{Prop}
\tod Koska $c_n \neq 0$, voidaan kirjoittaa
\[
p(z) = c_n z^n(1 + b_{n-1}z^{-1} + \cdots + b_0z^{-n}),
\]
missä $b_k = c_k / c_n$. Käyttämällä kolmioepäyhtälön ensimmäistä ja toista osaa muodoissa
\[
\abs{z_1+z_2} \ge \abs{z_1}-\abs{z_2}, \quad -\abs{z_1 + z_2} \geq - \abs{z_1} - \abs{z_2}
\]
sekä kaavoja (4),\,(5), päätellään
\begin{align*}
\abs{p(z)} &=   \abs{c_n}\abs{z}^n \abs{1 + b_{n-1}z^{-1} + \cdots + b_0 z^{-n}} \\
           &\ge   \abs{c_n}\abs{z}^n (1 - \abs{b_{n-1}z^{-1} + \cdots + b_0 z^{-n}}) \\
           &\ge \abs{c_n}\abs{z}^n (1 - \abs{b_{n-1}z^{-1}} - \cdots - \abs{b_0 z^{-n}}) \\
           &= \abs{c_n}\abs{z}^n \left(1 - \frac{\abs{b_{n-1}}}{\abs{z}} - \cdots 
                                                     - \frac{\abs{b_0}}{\abs{z}^{n}}\right).
\end{align*}
Kun
\[
\abs{z} \geq \max \{1, \ 2(\abs{b_{n-1}} + \cdots + \abs{b_0}) \} = R,
\]
niin pätee
\[
\frac{\abs{b_{n-1}}}{\abs{z}} + \cdots + \frac{\abs{b_0}}{\abs{z}^n} \leq
\frac{\abs{b_{n-1}} + \cdots + \abs{b_0}}{\abs{z}} \leq \frac{1}{2}\,,
\]
joten seuraa väittämän ensimmäinen osa:
\[
\abs{p(z)} \ge \frac{1}{2} \abs{c_n}\abs{z}^n, \quad \text{kun} \ \abs{z} \geq R.
\]
Jälkimmäinen osa seuraa vastaavasti soveltamalla kolmioepäyhtälön jälkimmäistä osaa. \loppu
\begin{Exa} Jos $p(z) = z^5 -4i\,z^3 + 10z + (30+40i)$, niin em.\ todistuksen logiikkaa 
seuraamalla nähdään, että
\begin{align*}
\abs{p(z)} &\ge \abs{z}^5
                \left(1 - \frac{\abs{5i}}{\abs{z}^2} - \frac{10}{\abs{z}^4} 
                                                     - \frac{\abs{30+40i}}{\abs{z}^5}\right) \\
           &=   \abs{z}^5\left(1 - \frac{5}{\abs{z}^2} - \frac{10}{\abs{z}^4} 
                                                       - \frac{50}{\abs{z}^5}\right) \\
           &\ge \abs{z}^5\left(1 - \frac{5}{\abs{z}} - \frac{10}{\abs{z}} 
                                                     - \frac{50}{\abs{z}}\right) \\
           &=   \abs{z}^5\left(1 - \frac{65}{\abs{z}}\right), \quad \text{kun}\ \abs{z} \ge 1.
\end{align*}
Vastaavasti päätellään, että $\abs{p(z)}\le\abs{z}^5(1+65\abs{z}^{-1})$, kun $\abs{z} \ge 1$,
joten seuraa
\[
\frac{1}{2}\abs{z}^5\,\le\,\abs{p(z)}\,\le\,\frac{3}{2}\abs{z}^5, \quad 
                                            \text{kun}\ \abs{z} \ge 130 = R. \loppu
\]
\end{Exa}

\Harj
\begin{enumerate}

\item
Muunna seuraavat kompleksiluvut perusmuotoon $x+iy$ annetusta osoitinmuodosta tai 
osoitinmuotoon annetusta perusmuodosta. Käytä tarvittaessa likiarvoja.
\[
\text{a)}\,\ \sqrt{2}\vkulma{-\tfrac{5\pi}{4}} \quad\
\text{b)}\,\ \sqrt{3}-2i \quad\
\text{c)}\,\ 2\vkulma{700\aste} \quad\
\text{d)}\,\ -3-4i 
\]

\item
Saata seuraavat kompleksiluvut perusmuotoon $x+iy$ ja polaarimuotoon: \newline
a) \ $(1+i)(1-i)^5 \quad$ 
b) \ $(1-i)(1+i\sqrt{3})^{-1} \quad$
c) \ $(\sqrt{2}+1+i)^8$

\item
Laske itseisarvo ja vaihekulma käyttäen polaarimuotoa: \newline
a) \ $(1+i)^6 \quad$ 
b) \ $(1-i\sqrt{3})(1-i)^2 \quad$
c) \ $(2-2i)(\sqrt{3}+i)^{-2}$

\item
Olkoon $z=\tfrac{1}{2}(1+i\sqrt{3})$. Millä $n\in\Z$ pätee\, a) $z^n=z$, \ b) $z^n=-z$\,?

\item
Kompleksilukujen itseisarvoille pätee $\abs{z_1 z_2}^2=\abs{z_1}^2\abs{z_2}^2$. Tarkista kaavan
pätevyys suoraan kompleksilukujen perusmuodosta, eli kirjoittamalla \newline
$z_1=x_1+iy_1\,, \ z_2=x_2+iy_2\,$.

\item
Johda de Moivren kaavan avulla: \newline 
a) $\sin 5x$:lle lauseke $\sin x$:n polynomina, \newline
b) $\sin^5x$:lle lauseke muodossa $\,a\sin x+b\sin 3x+c\sin 5x,\ a,b,c\in\R$.

\item
Jos kompleksilukujen kunnassa on määritelty järjestysrelaatio, niin ensimmäisen
järjestysaksiooman (J1) mukaan on oltava joko $i>0$, $i=0$ tai $i<0$. Päättele, että
järjestysrelaatiota ei voi määritellä.

\item
Määritä jokin $R$ siten, että pätee
\[
\abs{z}^5\,\le\,\abs{2z^5+1000z^4-10^4(3+4i)}\,\le\,3\abs{z}^5, \quad 
                                                                    \text{kun} \abs{z} \ge R.
\]

\item (*)
Todista, että kaikilla $z_1,z_2\in\C$ pätee $(1+\abs{z_1}^2)(1+\abs{z_2}^2)\ge\abs{1+z_1z_2}^2$.

\item (*) \label{H-III-2: polynomin kasvu}
Näytä, että Propositon \ref{polynomin kasvu} oletuksin on olemassa $R\in\R_+$ siten, että
\[
0.999\abs{c_n}\abs{z}^n\,\le\,\abs{p(z)}\,\le\,1.001\abs{c_n}\abs{z}^n,\quad 
                                        \text{kun} \ z \in \C\ \text{ja}\ \abs{z} \ge R.
\]
Millaisesta vielä yleisemmästä väittämästä tämä on erikoistapaus?

\end{enumerate}