\section{Potenssisarjan derivointi} \label{potsarjan derivaatta}
\alku

Edellisen luvun tarkastelujen perusteella voidaan todeta pääsääntönä, että tavanomaiset
(algebran keinoin määritellyt) funktiot ovat yleensä derivoituvia määrittelyjoukossaan,
lukuunottamatta mahdollisia erillisiä pisteitä, joita voi olla äärellinen tai numeroituva
määrä. Tässä luvussa näytetään, että pääsäännöstä eivät tee poikkeusta myöskään funktiot,
jotka on määritelty potenssisarjojen summina. Tällaisten funktioiden myötä derivoituvien
funktioiden joukko laajeneekin merkittävästi, kuten tullaan näkemään myöhemmissä luvuissa.

Olkoon reaalifunktio $f$ määritelty muodossa
\[
f(x) = \sum_{k=0}^\infty a_k x^k.
\]
Oletetaan, että tässä potenssisarjan suppenemissäde on joko $\rho\in\R_+$ tai $\rho=\infty$.
Tällöin suurin avoin väli, joka sisältyy $f$:n määrittelyjoukkoon, on $(-\rho,\rho)$
(vrt.\ Luku \ref{potenssisarja}). Jatkossa näytetään, että $f$ on derivoituva koko tällä
välillä ja että derivaatta voidaan laskea yksinkertaisesti derivoimalla sarja termeittäin,
ts.\ pätee
\[
f'(x) = \sum_{k=1}^\infty k a_k x^{k-1}, \quad x \in (-\rho,\rho).
\]
Derivoinnin tuloksena saadun sarjan suppenemissäde on myös $\rho$ (Lause
\ref{potenssisarjan skaalaus}), joten $f'$ on edelleen derivoituva välillä $(-\rho,\rho)$,
ja derivoinnin tulos on siis
\[
f''(x) = \sum_{k=2}^\infty k(k-1)\,a_k x^{k-2}, \quad x \in (-\rho,\rho).
\]
Myös tämän sarjan suppenemissäde $=\rho$ (Korollaari \ref{potenssisarjan yleinen skaalaus}),
joten $f''$ on välillä $(-\rho,\rho)$ edelleen derivoituva, jne. Päätellään, että
potenssisarjan summana määritelty funktio on avoimella suppenemisvälillään itse asiassa
mielivaltaisen monta kertaa derivoituva funktio (!). Päätelmä persutui siis potenssisarjojen
suppenemisteoriaan ja seuraavaan väittämään, joka on todistettavissa suoraan derivaatan
määritelmästä lähtien eli aiemmista derivoimissäännöistä riippumattomana tuloksena.
\begin{Lause} \label{potenssisarja on derivoituva} \vahv{(Potenssisarjan derivaatta)}\, Jos 
potenssisarjan $\sum_{k=0}^\infty a_k x^k$ suppenemissäde on $\rho>0$, niin sarjan summana 
määritelty funktio $f(x)$ on derivoituva välillä $(-\rho,\rho)$ ja ko.\ välillä pätee 
$f'(x)=\sum_{k=1}^\infty k a_k x^{k-1}$. 
\end{Lause}
\tod Olkoon $x\in (-\rho,\rho)$ ja valitaan $\Delta x \neq 0$, siten että
\[
\abs{x}+\abs{\Delta x} \le \rho_0<\rho.
\]
Binomikaavan mukaan
\begin{align*}
(x+\Delta x)^k\ &=\ \sum_{l=0}^k \binom{k}{l} (\Delta x)^l x^{k-l} \\
                &=\ x^k + kx^{k-1}\Delta x + \sum_{l=2}^k \binom{k}{l} (\Delta x)^l x^{k-l}.
\end{align*}
Kirjoitetaan tässä viimeinen termi summeerausindeksin vaihdolla muotoon
\begin{align*}
\sum_{l=2}^k \binom{k}{l} (\Delta x)^l x^{k-l}\ 
              &= \sum_{i=0}^{k-2} \binom{k}{i+2} (\Delta x)^{i+2} x^{k-2-i} \\
              &=\ (\Delta x)^2 \sum_{i=0}^{k-2} \binom{k}{i+2} (\Delta x)^i x^{k-2-i},
\end{align*}
ja edelleen
\[
\sum_{i=0}^{k-2} \binom{k}{i+2} (\Delta x)^i\,x^{k-2-i} 
                = \sum_{i=0}^{k-2} c_i\,\binom{k-2}{i} (\Delta x)^i\,x^{k-2-i},
\]
missä
\[
c_i = \binom{k}{i+2} \binom{k-2}{i}^{-1} 
    = \frac{k!}{(k-i-2)!\,(i+2)!} \cdot \frac{(k-i-2)!\,i!}{(k-2)!} 
    = \frac{k(k-1)}{(i+1)(i+2)}\,.
\]
Koska
\[
c_i \le \frac{1}{2}\,k(k-1) < \frac{1}{2}\,k^2, \quad i = 0 \ldots k-2,
\]
ja oli $\abs{x}+\abs{\Delta x} \le \rho_0$, niin saadaan jokaisella $k \ge 2$ arvio
\begin{align*}
\left|\frac{(x+\Delta x)^k - x^k}{\Delta x} - kx^{k-1}\right|\ 
 &=\ \abs{\Delta x} \left|\sum_{i=0}^{k-2} c_i\,\binom{k-2}{i} (\Delta x)^i\,x^{k-2-i}\right| \\
 &\le\ \abs{\Delta x} \sum_{i=0}^{k-2} c_i\,\binom{k-2}{i} \abs{\Delta x}^i\,\abs{x}^{k-2-i} \\
 &\le\ \frac{1}{2}k^2\abs{\Delta x} 
               \sum_{i=0}^{k-2} \binom{k-2}{i} \abs{\Delta x}^i\,\abs{x}^{k-2-i} \\
 &= \frac{1}{2}k^2\abs{\Delta x}(\abs{x}+\abs{\Delta x})^{k-2} \\
 &\le \frac{1}{2}k^2\abs{\Delta x}\rho_0^{k-2}.
\end{align*}
Näin ollen
\begin{align*}
\left| \frac{f(x+\Delta x)-f(x)}{\Delta x} - \sum_{k=1}^\infty ka_kx^{k-1} \right|\ 
       &=\ \left|\sum_{k=2}^\infty a_k
                 \left[\frac{(x+\Delta x)^k - x^k}{\Delta x} - kx^{k-1}\right]\right| \\
       &\le\ \sum_{k=2}^\infty \abs{a_k}
             \left|\frac{(x+\Delta x)^k - x^k}{\Delta x} - kx^{k-1}\right| \\
       &\le\ \abs{\Delta x}\sum_{k=2}^\infty \frac{1}{2}k^2\abs{a_k}\rho_0^{k-2}.
\end{align*}
Tässä oikealla oleva sarja suppenee Lauseen \ref{potenssisarjan skaalaus} perusteella, koska
$\rho_0<\rho$, joten
\[
\left|\frac{f(x+\Delta x)-f(x)}{\Delta x} - \sum_{k=1}^\infty ka_kx^{k-1}\right| 
       \le\ C\,\abs{\Delta x}, \quad\ C = \sum_{k=2}^\infty \frac{1}{2}k^2\abs{a_k}\rho_0^{k-2}.
\]
Saatu arvio on pätevä, kun $x \in (-\rho,\rho)$ ja $\abs{\Delta x} < \delta$, missä esim.\ 
\[ 
\delta = \frac{1}{2}\,(\rho-\abs{x}) > 0. 
\]
Siis $f$ on jokaisessa pisteessä $x \in (-\rho,\rho)$ derivoituva ja
\[
f'(x) = \lim_{\Delta x \kohti 0}\,\frac{f(x+\Delta x)-f(x)}{\Delta x}\ 
      =\ \sum_{k=1}^\infty ka_kx^{k-1}. \loppu
\]
\begin{Exa} Laske sarjan $\,\sum_{k=1}^\infty k q^k\,$ summa, kun $\abs{q}<1$. \end{Exa}
\ratk Dervivomalla funktio
\[ 
g(x) = \sum_{k=0}^\infty x^k = \frac{1}{1-x}\,, \quad x \in (-1,1) 
\]
toisaalta rationaalifunktiona ja toisaalta Lauseen \ref{potenssisarja on derivoituva} 
perusteella seuraa
\[ 
g'(x) \,=\, \frac{1}{(1-x)^2} \,=\, \sum_{k=1}^\infty kx^{k-1} 
                              \,=\, x^{-1}\sum_{k=1}^\infty kx^k, \quad x \in (-1,1). 
\]
Kun $x=q$, niin pätee siis
\[
\frac{1}{(1-q)^2} \,=\, q^{-1}\sum_{k=1}^\infty kq^k 
                  \qimpl \sum_{k=1}^\infty kq^k \,=\, \frac{q}{(1-q)^2} \loppu
\]
\begin{Exa} Määrittele potenssisarjana välillä $(-1,1)$ funktio, jolle pätee
\[
F(0)=0, \quad F'(x) = \frac{1}{1+x}, \quad x \in (-1,1).
\]
\ratk Jos $F(x)=\sum_{k=0}^\infty a_k x^k,\ x \in (-1,1)$, ja sarjan suppenemissäde on
$\rho \ge 1$, niin Lauseen \ref{potenssisarja on derivoituva} mukaan
\[
F'(x) = \sum_{k=1}^\infty ka_k x^{k-1} 
      = \sum_{l=0}^\infty (l+1)\,a_{l+1}\,x^l 
      = \sum_{k=0}^\infty (k+1)\,a_{k+1}\,x^k, \quad x \in (-1,1).
\]
Koska 
\[
F'(x) = \frac{1}{1+x} = \sum_{k=0}^\infty (-1)^k x^k, \quad x \in (-1,1),
\]
niin ehto $F'(x)=1/(1+x),\ x \in (-1,1)$ toteutuu, kun valitaan 
\[
(k+1)\,a_{k+1}=(-1)^k,\ k=0,1,\ldots \qekv a_k = \frac{(-1)^{k+1}}{k},\ k=1,2,\ldots
\]
Ehto $F(0)=0$ toteutuu valinnalla $a_0=0$, joten
\[
F(x) = \sum_{k=1}^\infty (-1)^{k+1}\,\frac{x^k}{k} = x - \frac{x^2}{2} + \frac{x^3}{3} - \ldots
\]
Sarjan suppenemissäde on $\rho=1$ (Lause \ref{potenssisarjan skaalaus}), joten $F$ täyttää
asetetut ehdot. \loppu
\end{Exa}
\begin{Exa} Jos $a \in \R$, niin funktio
\[
y(x) = \sum_{k=0}^\infty \frac{a^k}{k!}\,x^k = \sum_{k=0}^\infty \frac{(ax)^k}{k!}
\]
on määritelty kaikkialla suppenevana potenssisarjana ($\rho=\infty$). Derivoimalla sarja
termeittäin nähdään, että
\[
y'(x) = a \sum_{k=1}^\infty \frac{(ax)^{k-1}}{(k-1)!} 
      = a \sum_{l=0}^\infty \frac{(ax)^l}{l!} = ay(x), \quad x\in\R.
\]
Funktio $y(x)$ on siis (ainakin eräs) ratkaisu ongelmaan
\[ 
\begin{cases} y'=ay, \quad y=y(x),\ x\in\R \\ y(0)=1 \end{cases} \loppu
\]
\end{Exa}

\Harj
\begin{enumerate}

\item
Laske $f''(x)$ potenssisarjana, kun
\[
f(x)=\sum_{k=1}^\infty \frac{x^k}{k!(k-1)!}\,.
\]

\item \label{H-V-4: cos ja sin potenssisarjoina}
Määritellään funktiot
\[
u(x)=\sum_{k=0}^\infty (-1)^k \frac{x^{2k}}{(2k)!}, \quad
v(x)=\sum_{k=0}^\infty (-1)^k \frac{x^{2k+1}}{(2k+1)!}, \quad x\in\R.
\]
Näytä, että $u'=-v$ ja $v'=u$.
 
\item
Seuraavat funktiot ovat rationaalifunktioita välillä $(-1,1)$. Laske funktioiden lausekkeet
potenssisarjaa $\sum_{k=0}^\infty x^k$ derivoimalla.
\begin{align*}
&\text{a)}\ f(x)=\sum_{k=2}^\infty kx^k \qquad 
 \text{b)}\ f(x)=\sum_{k=3}^\infty (-1)^k kx^k \qquad
 \text{c)}\ f(x)=\sum_{k=1}^\infty k^2 x^k \\
&\text{d)}\ f(x)=\sum_{k=0}^\infty (k+2)^2 x^k \quad\ \
 \text{e)}\ f(x)=\sum_{k=1}^\infty k^3 x^k \quad\ \
 \text{f)}\ f(x)=\sum_{k=0}^\infty (k+1)^3 x^k 
\end{align*}

\item (*)
a) Näytä, että jos potenssisarja $\sum_k a_k x^k$ suppenee välillä $(-\rho,\rho)$ ja
funktio $g$ on derivoituva välillä $(a,b)\subset(-\rho,\rho)$, niin funktion
\[
f(x) = \sum_{k=0}^\infty x^k g(x), \quad x\in(a,b)
\]
derivaatta välillä $(a,b)$ on laskettavissa derivoimalla sarja
termeittäin.\vspace{1mm}\newline
b) Määritä $f'(x)$ ja $f''(x)$ välillä $(0,\infty)$, kun
\[
f(x) = \sum_{k=1}^\infty \frac{1}{k!}\,x^{k+\tfrac{1}{2}}, \quad x \ge 0.
\]
\end{enumerate}
