\section{Differentiaali ja muutosnopeus} \label{differentiaali}
\sectionmark{Differentiaali}
\alku

Jos $f$ on derivoituva pisteessä $x$, niin linearisoivan approksimaatioperiaatteen mukaisesti
on likimain
\[
\Delta f(x)=f(x+\Delta x)-f(x)\approx f'(x)\Delta x,
\]
kun $\abs{\Delta x}$ on pieni. Lauseketta $f'(x)\Delta x$ sanotaan $f$:n
\index{differentiaali}%
\kor{differentiaaliksi} pisteessä $x$ ja merkitään
\[
df(x,\Delta x) = f'(x)\Delta x.
\]
Differentiaalin avulla voi siis arvioida likimäärin funktion arvon muutoksen $\Delta f$, joka
vastaa muuttujan pientä muutosta $\Delta x$. Jos approksimaation virhettä merkitään
\[
r(x,\Delta x) = f(x+\Delta x)-f(x)-f'(x)\Delta x,
\]
niin derivaatan määritelmän mukaisesti pätee
\[
\lim_{\Delta x \kohti 0} \frac{r(x,\Delta x)}{\Delta x} = 0.
\]
\begin{Exa}
Neliön muotoista mökkiä, joka on ulkomitoiltaan $6\text{ m}\times 6\text{ m}$, kaupataan 
$36\text{ m}^2$:n mökkinä. Arvioi mökin todellinen lattiapinta-ala differentiaalin avulla
olettaen seinän paksuudeksi $20\text{ cm}$.
\end{Exa}
\ratk Jos $f(x)=x^2$, niin lattiapinta-ala on differentiaalin avulla arvioiden
\[
A=f(x-\Delta x) \approx f(x)-f'(x)\Delta x = x^2-2x\Delta x.
\]
Arvoilla $x=6\text{ m}$, $\Delta x=0.4\text{ m}$ (huom!) saadaan
\[
A\approx 31.2\text{ m}^2.
\]
Approksimaation virhe on tässä tapauksessa tarkasti
\[
r(x,\Delta x) = (\Delta x)^2 = 0.16\text{ m}^2,
\]
eli 'todellinen' lattiapinta-ala on $A\approx 31.4\text{ m}^2$. \loppu

Jäljempänä Luvussa \ref{taylorin lause} näytetään, että differentiaaliin perustuvan
approksimaation virhe on yleisesti suuruusluokkaa $\,\sim (\Delta x)^2\,$ silloin kun
funktio on riittävän säännöllinen (kuten esimerkissä).
\begin{Exa}
Ideaalikaasun adiabaattisissa (äkillisissä) paineen ja tiheyden vaihteluissa pätee tilanyhtälö
\[
p\rho^{-\gamma}=K=\text{vakio},
\]
missä $\gamma$ on kaasulle ominainen vakio ($\gamma>1$). Jos tiheys muuttuu 2\%, niin paljonko
muuttuu paine?
\end{Exa}
\ratk Koska $\,p=K\rho^\gamma=f(\rho)$, niin
\begin{align*}
\Delta p &\approx f'(\rho)\Delta\rho =\gamma K\rho^{\gamma-1}\Delta\rho 
                                    =\gamma p\inv{\rho}\Delta\rho \\
         &\impl \ \frac{\Delta p}{p} \approx \gamma\frac{\Delta\rho}{\rho}
                                     =\underline{\underline{2\gamma\,\%}}. \loppu
\end{align*}

\subsection*{Muutosnopeus} 
\index{muutosnopeus|vahv}

Derivaatan tavallisin tulkinta fysiikan ym.\ sovelluksissa on
\[
\boxed{\kehys\quad \text{Derivaatta}=\text{(hetkellinen) muutosnopeus}. \quad}
\]
Jos $m$ on jokin fysikaalinen suure, joka muuttuu ajan $(t)$ (mittayksikkö s) mukana, eli 
$m=m(t)$, niin hetkellinen muutosnopeus on
\[
\lim_{\Delta t\kohti 0} \frac{m(t+\Delta t)-m(t)}{\Delta t} 
                 = \lim_{\Delta t \kohti 0} \frac{\Delta m}{\Delta t} = m'(t).
\]
Jos $m$:n yksikkö on M, niin muutosnopeuden yksikkö on M/s.
\begin{Exa}
Jos $A=\text{kansantalous}$ (yksikkö E), niin talouskasvu on $A$:n hetkellinen muutosnopeus
(yksikkö E/s). Kun sanotaan 'talouskasvu kiihtyy' tai 'talouskasvu hidastuu', tarkoitetaan 
$A'(t)$:n muutosnopeutta, eli toista derivaattaa $A''(t)$ (yksikkö E/s$^2$). \loppu
\end{Exa}
Jos kappale liikkuu siten, että sen sijainti hetkellä $t$ on $x(t)$ (yksiulotteinen liike),
niin tunnetusti
\[
x'(t)=\text{\pain{no}p\pain{eus}},\quad x''(t)=\text{\pain{kiiht}y\pain{v}yy\pain{s}}.
\]
\begin{Exa}
Kaksimetrinen mies kävelee neljän metrin korkeudella olevan katulampun ali hetkellä $t=0$. 
Määritä miehen varjon kärjen paikka, nopeus ja kiihtyvyys hetkellä $t\geq 0$, kun mies kävelee
vakionopeudella $v_0=2$ m/s. (Ol.\ katu vaakasuora).
\end{Exa}
\ratk Varjon kärjen paikka määräytyy ehdosta (ks.\ kuvio alla)
\[
\frac{x(t)-v_0t}{2}=\frac{x(t)}{4},
\]
%\begin{multicols}{2} \raggedcolumns
joten
\begin{align*}
x(t)&=2v_0t, \\
x'(t)&=2v_0=4\,\text{m}/\text{s}, \\
x''(t)&=0. \quad\loppu
\end{align*}
\begin{figure}[H]
\setlength{\unitlength}{1cm}
\begin{center}
\begin{picture}(7,4)(0,0)
\put(1,0){\vector(1,0){6}} \put(5.3,-0.5){$x(t)$}
\linethickness{0.05cm}
\put(0,0){\line(0,1){3.7}}
\curve(0,3.7,0.5,4,1,3.8)
\curve(1,3.8,0.9,3.75,0.8,3.6)
\put(0.8,3.6){\line(1,0){0.4}}
\curve(1,3.8,1.1,3.75,1.2,3.6)
\thinlines
\dashline{0.2}(1,0)(1,3.6)
\drawline(1,3.6)(5.6,0)
\drawline(3,0)(3,2.0) \put(2.9,1.83){$\bullet$}
\drawline(3,0.7)(2.7,0)
\path(3,1.8)(3.2,1.2)(3.3,1.3)
\path(3,1.8)(2.8,1.3)(2.9,1.2)
\multiput(1,-0.5)(2,0){2}{\line(0,-1){0.2}}
\put(1.5,-0.6){\vector(-1,0){0.5}} \put(2.5,-0.6){\vector(1,0){0.5}} \put(1.8,-0.7){$v_0t$}
\end{picture}
\end{center}
\end{figure}
\index{zza@\sov!Moottori}%
\begin{Exa}: \vahv{Moottori}. Männänvarren, jonka pituus = 2 (mittayksikkö = $10$ cm),
kampiakseliin kiinnitetty pää liikkuu pitkin yksikköympyrää siten, että napakulma hetkellä $t$
on $\varphi(t)=at$ ($a=$ vakio), ja toinen pää (johon mäntä on kiinnitetty) liukuu pitkin
$x$-akselia. a) Arvioi differentiaalin avulla, paljonko mäntä liikkuu kampiakselin pyöriessä
kulmasta $\varphi=30^\circ$ kulmaan $\varphi+\Delta\varphi=31^\circ$. b) Mikä on männän nopeuden 
(vauhdin) maksimiarvo, kun kampiakselin pyörimisnopeus on $3600$ kierrosta/min ? 
\end{Exa}
\begin{figure}[H]
\setlength{\unitlength}{1cm}
%\begin{center}
\begin{picture}(10,4)(0,0)
\put(4,2){\bigcircle{4}}
\put(4,2){\line(1,0){5.156}}
%\dashline{0.2}(4,2)(8.472,2)
\put(3.9,1.89){$\bullet$} \put(5.302,3.302){$\bullet$} \put(9.054,1.9){$\bullet$}
\put(4,2){\arc{0.7}{-0.785}{0}} \put(4.5,2.2){$\varphi(t)$}
\put(4.3,2.8){$1$} \put(7,3){$2$}
\put(9.956,2){\vector(1,0){2}} \put(12.2,1.9){$x$}
\put(9.156,1.8){\line(0,-1){0.4}} \put(8.8,1){$x(\varphi)$}
\thicklines
%\put(4,2){\line(1,2){0.894}} \put(4.894,3.789){\line(2,-1){3.578}}
\path(4,2)(5.414,3.414)(9.156,2)
\multiput(9.156,1.8)(0.8,0){2}{\line(0,1){0.4}}
\multiput(9.156,1.8)(0,0.4){2}{\line(1,0){0.8}}
\end{picture}
\end{figure}
\ratk a) Kampiakselin kiertokulman ollessa $\varphi$ on männän paikka
\[
x(\varphi) = \cos\varphi + \sqrt{4-\sin^2\varphi} 
           = \cos\varphi + \sqrt{\rule{0mm}{4mm}3+\cos^2\varphi}.
\]
Kulman muuttuessa $\Delta\varphi$:n verran on
\[
\Delta x(\varphi) \approx x'(\varphi)\Delta\varphi
= -\left(\sin\varphi + \frac{\cos\varphi\sin\varphi}{\sqrt{3+\cos^2\varphi}}\right)\Delta\varphi.
\]
Arvoilla $\varphi=30^\circ,\ \Delta\varphi=1^\circ \vastaa \pi/180\,$ saadaan
\[
\abs{\Delta x} 
   \approx \frac{1}{2}\left(1+\frac{1}{\sqrt{5}}\right)\frac{\pi}{180}\cdot 10\,\text{cm} 
   \approx  \underline{\underline{1.3\ \text{mm}}}.
\]

b) Kun kirjoitetaan $x(t)=x(\varphi(t))=x(at)$, niin
\begin{align*}
x'(t)  &= a\left(-\sin at - \frac{\cos at\sin at}{\sqrt{3+\cos^2 at}}\right), \\
x''(t) &= a^2\left(-\cos at + \frac{\sin^2 at-\cos^2 at}{\sqrt{3+\cos^2 at}}
                            - \frac{\cos^2 at\,\sin^2 at}{(3+\cos^2 at)^{3/2}}\right).
\end{align*}
Nopeuden $x'(t)$ ääriarvokohdissa on oltava
\[
x''(t)=0\ \ \ekv\ \ \cos at\left(3+\cos^2 at\right)^{3/2} = 3-6\cos^2 at-\cos^4 at.
\]
Neliöimällä puolittain ja merkitsemällä $u=\cos^2 at$ tämä sievenee polynomiyhtälöksi
\[
u^3+u^2-21u+3=0.
\]
Välillä $[0,1]$ tällä on yksikäsitteinen ratkaisu
\[
u = 0.143987..\ \impl\ \cos\varphi(t) = \sqrt{0.143987..} = 0.379456..\ 
                \impl\ \varphi(t) \approx 68\aste.
\]
Tätä vastaa vauhdin maksimiarvo
\[
\abs{x'(t)} = a\sqrt{1-u}\left(1+\sqrt{\frac{u}{3+u}}\right) \approx 1.12\,a.
\]
Annetulla pyörimisnopeudella on $a=60 \cdot 2\pi/$s, joten numeroarvoksi saadaan
\[
v_{max} \approx 1.12 \cdot 120\pi\,\text{s}^{-1} \cdot 10\,\text{cm} 
                          \approx \underline{\underline{42\,\text{m/s}}}. \loppu
\]

\subsection*{Differentiaali ja differentiaaliyhtälöt}
\index{differentiaaliyhtälö!a@eksponenttifunktion|vahv}

Differentiaaleihin perustuva ajattelu on hyvin tavallista silloin, kun luonnonilmiötä 
(tai muuta ilmiötä) kuvaava matemaattinen malli on differentiaaliyhtälö, ja halutaan 
j\pain{ohtaa} tämä yhtälö. Seuraavissa esimerkeissä päädytään eksponenttifunktion
differentiaaliyhtälöön $y'=ay$ fysikaalisesta (tai muusta) lainalaisuudesta muotoa
\[ 
\Delta y(x)\ =\ y(x+\Delta x)-y(x)\ \approx\ ay(x)\Delta x, \quad 
                                       \text{kun}\ \abs{\Delta x}\ \text{pieni}. 
\]
Kun tulkitaan tämän tarkoittavan, että
\[ 
\Delta y(x) = ay(x)\Delta x + r(x,\Delta x), 
\]
missä $\lim_{\Delta x \kohti 0} r(x,\Delta x)/\Delta x = 0$, niin jakamalla $\Delta x$:llä ja 
antamalla $\Delta x \kohti 0$ saadaan 'äärettömän pienille' muutoksille pätevä laki
\[ 
\lim_{\Delta x \kohti 0} \frac{\Delta y}{\Delta x} = \frac{dy}{dx} = ay(x). 
\]
\index{zza@\sov!Radioaktiivinen hajoaminen}%
\begin{Exa}: \vahv{Radioaktiivinen hajoaminen}. Radioaktiivisessa aineessa on yksittäisen
ytimen todennäköisyys hajota (ja niinmuodoin 'kadota') aikavälillä $[t,t+\Delta t]$
verrannollinen aikavälin pituuteen $\Delta t$, kun $\Delta t$ on pieni. Jos radioaktiivisten
(toistaiseksi hajoamattomien) ytimien lukumäärä hetkellä $t$ on $A(t)$, niin saadaan
likimääräiseksi hajoamislaiksi
\[
\frac{A(t+\Delta t)-A(t)}{A(t)}\ =\ \frac{\Delta A}{A(t)}\ \approx\ -a\Delta t,
\]
missä $a$ on ytimille ominainen vakio (dimensiottomana positiivinen, mittayksikkö 1/s). Rajalla
$\Delta t\kohti 0$ saadaan tarkka hajoamislaki
(vrt.\ Esimerkki \ref{eksponenttifunktio fysiikassa}:\,\ref{radioaktiivisuus})
\[
A'(t)=-aA(t) \qimpl A(t) = A(0)\,e^{-at}, \quad t \ge 0. \loppu
\]
\end{Exa}
\index{zza@\sov!Szy@Säteilyn vaimeneminen}%
\begin{Exa}: \vahv{Säteilyn vaimeneminen}. Säteilyn kulkiessa homogeenisessa väliaineessa on
yksityisen hiukkasen todennäköisyys törmätä väliaineen atomiytimeen verrannollinen kuljettuun
matkaan $\Delta x$, kun $\abs{\Delta x}$ on pieni. Jos säteilyn intensiteetti on $I(x)$
(hiukkasta/m$^2$/s), niin säteily vaimenee tällöin törmäysten johdosta määrällä
\[
I(x+\Delta x)-I(x)\ =\ \Delta I\ \approx\ -aI(x)\Delta x,
\]
missä $a$ on aineelle ominainen vakio (dimensiottomana positiivinen, mittayksikkö 1/m). Jos 
tässä oletetaan, että approksimaation virhe on $r(x,\Delta x)$, missä 
$\lim_{\Delta x \kohti 0} r(x,\Delta x)/\Delta x = 0$, niin jakamalla $\Delta x$:llä ja 
antamalla $\Delta x \kohti 0$ saadaan tarkka säteilyn vaimenemislaki
(vrt.\ Esimerkki \ref{eksponenttifunktio fysiikassa}:\,\ref{säteilyvaimennus})
\[
I'(x)=-aI(x) \qimpl I(x) = I(0)\,e^{-ax}, \quad x \ge 0. \loppu 
\]
\end{Exa}
\index{zza@\sov!Koronkorko}%
\begin{Exa}: \vahv{Koronkorko}. Pääoman kasvaessa jatkuvaa (koron)korkoa noudattaa pääoman
määrä $A(t)$ lakia
\[ 
A(t+\Delta t) - A(t)\ =\ \Delta A\ \approx\ aA(t)\Delta t \quad (\Delta t\ \text{pieni}), 
\]
missä $a = \tfrac{p}{100T}$, korkoprosentin ollessa $p$ ajassa $T$ (esim.\ $\,T = 1$ vuosi).
Rajalla $\Delta t \kohti 0$ päädytään pääoman kasvulakiin
\[ 
A'(t) = aA(t) \qimpl A(t) = A(0)\,e^{at}, \quad t \ge 0. \loppu 
\] 
\end{Exa}
\index{zza@\sov!Ilmanpaine}%
\begin{Exa}: \vahv{Ilmanpaine}.\ Määritä ilmanpaine korkeudella $x$ maan pinnasta. Oletetaan 
vakiolämpötila. \end{Exa}
\ratk Jos ilman tiheys korkeudella $x$ on $\rho(x)$, niin voimatasapainon perusteella saadaan
paineen muutoksen ja tiheyden välille yhteys
\[ 
\Delta p = p(x+\Delta x) - p(x) = - g\rho(x) \Delta x + r(x,\Delta x), 
\]
missä $g \approx 10\ \text{m/s}^2$ on maan vetovoiman kiihtyvyys ja 
$r(x,\Delta x)/\Delta x \kohti 0$, kun $\Delta x \kohti 0$. 
\begin{figure}[H]
\setlength{\unitlength}{1cm}
\begin{center}
\begin{picture}(6,4)
\put(0,2){\line(1,0){4} $ \quad x+\Delta x$}
\put(0,1){\line(1,0){4} $\quad x$}
\put(2,0){\vector(0,1){1} $\ p(x)$}
\put(2,3){\vector(0,-1){1} $\ p(x+\Delta x)$}
\put(1.75,1.4){$\rho(x)$}
\end{picture}
%\caption{Paineen differentiaalinen muutos}
\end{center}
\end{figure}
Kaasun tilanyhtälö vakiolämpötilassa on
\[ 
p/\rho = K = \text{vakio}. 
\]
Sijoittamalla tästä $\rho(x) = p(x)/K$ tasapainoyhtälöön ja antamalla $\Delta x \kohti 0$ 
päädytään paineen vähenemislakiin
\[ 
p' = - p/a \qimpl p(x) = p(0) e^{-x/a}, 
\]
missä
\[ 
a\ =\ \dfrac{K}{g}\ =\ \dfrac{p(0)}{\rho(0)g}\ 
      \approx\ \dfrac{10^5\ \text{N/m}^2}{1\ \text{kg/m}^3 \cdot 10\ \text{m/s}^2}\ 
   =\ 10\ \text{km}. 
\]
Paineen puoliarvokorkeus on $\,h_{1/2} = (\ln 2)a \approx 6.9$ km. \loppu

\Harj
\begin{enumerate}

\item
Laske $df(x,\Delta x)=f'(x)\Delta x$ ja $f(x+\Delta x)-f(x)$ kuudella desimaalilla, kun
$\,x=1$, $\Delta x=0.02\,$ ja \ a) \ $f(x)=x^2+200x+700,\quad$ b) \ $f(x)=x^{-2}-x^{-1},$ 
\newline
c) \ $f(x)=e^x,\quad$ d) $f(x)=\ln(1+x),\quad$ e) \ $f(x)=\tan\tfrac{\pi x}{2},\quad$
f) \ $f(x)=x^x$.

\item
Arvioi differentiaalin avulla, paljonko kuution a) tilavuus kasvaa, kun särmä pitenee $p\%$, \ 
b) pinta-ala pienenee, kun tilavuus pienenee $p\%$.

\item
Lentokone lentää suoraan maassa olevan katsojan yli $10$ km:n korkeudella. Lentokoneen
näkyessä vaakatasoon nähden kulmassa $60\aste$ havaitaan, että kulma muuttuu $4.3\aste$
viidessä sekunnissa. Arvioi koneen nopeus käyttäen differentiaalia.

\item
Hiihtäjä, jonka paino on $80$ kg, laskee mäkeä, jonka kaltevuus on vakio. Hiihtäjän nopeus $v$
nousee arvoon $50$ km/h, jolloin painovoima, kitka ja ilmanvastus saavuttavat tasapainon.
Painovoima ja kitkavoima ovat verrannollisia hiihtäjän massaan $m$ ja ilmanvastusvoima on
$F_v=\text{vakio}\times Av^\gamma$, missä $A$ on hiihtäjän efektiivinen poikkipinta-ala ja 
$\gamma \ge 1$ on vakio. \newline
Oletetaan lisäksi, että $m=\text{vakio}\times L^{2.4}$ ja $A=\text{vakio}\times L^{1.2}$, missä 
$L$ on hiihtäjän vatsanympärys. Arvioi tapauksissa a) $\gamma =1$, b) $\gamma=2$, kuinka paljon
prosentteina hiihtäjän laskuvauhti kasvaa tai pienenee hänen lihottuaan $2$ kg. Käytä 
differentiaalia! (Muut olosuhteet, kuten keli, oletetaan vakioksi.)

\item
a) Valmistettaessa $x$ kpl jääkaappeja on tuotantokustannus
\[
K(x)=8000+200x+\max\{0,\,200x-0.5x^2\}.
\]
Arvioi differentiaalin avulla yhden jääkaapin nk.\ marginaalinen tuotantokustannus, eli luku
$K(x+1)-K(x)$, kun $x=200$. \vspace{1mm}\newline
b) Erään valtion verotuksessa on marginaalinen (eli pienen lisätulon) veroprosentti vuositulon
$x$ (yksikkö $10^5$ euroa) funktiona
\[
p(x)=20+10\,\min\{x+3x^2,\,4\}.
\]
Millä vuositulon arvolla marginaalinen veroprosentti $=50$\,? Entä millä vuositulon arvolla
vero on puolet vuositulosta?

\item
a) Kuution pinta-ala kasvaa $50\ \text{cm}^2/\text{s}$. Mikä on tilavuuden kasvunopeus, kun
särmän pituus $=20$ cm? \vspace{1mm}\newline
b) Vesisäiliön tilavuus on $400$ l. Säiliöstä lasketaan vettä niin, että veden määrä
(yksikkö l) säiliössä hetkellä $t$ (yksikkö min) on
\[
V(t)=(20-t)^2, \quad 0 \le t \le 20.
\]
Mikä on veden hetkellinen virtausnopeus säiliöstä silloin, kun säiliö on vettä puolillaan?

\item
Piste $P$ liikkuu pitkin $x$-akselia negatiiviseen suuntaan vakionopeudella $v_1$ ja piste $Q$
liikkuu $y$-akselia pitkin negatiiviseen suuntaan vakionopeudella $v_2$. Hetkellä $t=0$ on
$P=(a,0)$ ja $Q=(0,b)$, missä $a,b>0$. Jos $s(t)$ janan $PQ$ pituus hetkellä $t$, niin millä 
hetkellä on $s'(t)=0$, ja mikä on $s$:n pienin arvo?

\item \index{zzb@\nim!IKEA}
(IKEA) Kaupan pysäköintialueelle tulee aikavälillä $[0,T]$ $Q_{in}(t)$ autoa/h ja alueelta
lähtee $Q_{out}(t)$ autoa/h (hetkellisiä arvoja). Johda differentiaaliyhtälö
(= autojen säilymislaki!) pysäköintialueella olevien autojen määrälle $m(t)$ hetkellä $t$, ja
ratkaise $m(t)$ aikavälillä $\,[0,T]\,$, kun $\,T=6$h, pysäköinti\-alueella on $120$ autoa
hetkellä $t=0$ ja
\[
Q_{in}(t)=200(1-t/T), \quad Q_{out}(t)=270(t/T)^2.
\]
Jos halutaan vain selvittää, millä ajan hetkellä pysäköintialueella on eniten autoja, niin
miten tämä saadaan yksinkertaisimmin selville?

\item (*) \index{zzb@\nim!Lusi}
(Lusi) Moottoritie kapenee kaksikaistaiseksi tieksi L:ssä. Moottoritietä pitkin tulee L:ään
$Q(t)$ autoa/h (hetkellinen arvo), ja L:stä alkavalle kapeammalle tielle mahtuu ajamaan
enintään $Q_0=1800$ autoa/h. Olkoon $m(t)=$ L:ään ruuhkautuneiden autojen määrä hetkellä
$t$ (aikayksikkö =h). Muodosta ruuhkautumisen matemaattinen malli ja ratkaise sen avulla
$m(t)$ aikavälillä $t\in[0,20]$, kun tiedetään, että $m(0)=0$ ja
\[
Q(t)=540+\max\,\{0,\,1440t-180t^2\}.
\]

\item (*) \index{zzb@\nim!Puimakone} 
(Puimakone) Hihnapyörän keskipiste on origossa ja säde on $R=0.3$ (yksikkö m). Hihnapyörää
pyörittää myötäpäivään hihna, joka koskettaa pyörää napakulmissa $\varphi\in[0,\pi]$. Hihnaa
liikutetaan toisella samansäteisellä, moottorikäyttöisellä hihnapyörällä, joka sijaitsee
negatiivisella $y$-akselilla.
\begin{multicols}{2} \raggedcolumns
Olkoon $F(\varphi)$ hihnaa jännittävä voima napakulmassa $\varphi,\ \varphi\in[0,\pi]$ ja
olkoon $\mu$ hihnan ja hihnapyörän välinen kitkakerroin. Tarkastelemalla välillä 
$[\varphi-\Delta\varphi/2,\varphi+\Delta\varphi/2]$ olevan hihnan palan voimatasapainoa johda 
differentiaaliyhtälö
\[
F'(\varphi) = -\mu F(\varphi), \quad \varphi\in(0,\pi)
\]
ja laske hihnaa jännittävät voimat $F_1$ ja $F_2$ hihnapyörien välisissä hihnan osissa
(ks.\ kuva), kun $\mu=0.2$ ja hihnapyörää vääntävä momentti on $M=(F_1-F_2)R=1200$ Nm. 
\begin{figure}[H]
\setlength{\unitlength}{1cm}
\begin{picture}(5,6)(0,0)
\thicklines
\put(3,3){\bigcircle{3}}
\put(4.5,3){\line(0,-1){3}} \put(1.5,3){\line(0,-1){3}}
\put(3,1.5){\vector(-1,0){0.2}}
\put(4.5,0.1){\vector(0,-1){0.1}} \put(1.5,0.1){\vector(0,-1){0.1}}
\put(3,3){\line(1,1){1.061}} \put(3,3){\vector(1,0){2.5}} \put(3,3){\vector(0,1){2.5}}
\thinlines
\put(3.3,3.8){$R$}
\put(3,3){\arc{0.7}{-0.785}{0}} \put(3.6,3.2){$\varphi$}
\put(5.7,2.9){$x$} \put(3.2,5.5){$y$}
\put(1.65,0){$F_2$} \put(4.65,0){$F_1$}
\end{picture}
\end{figure}
\end{multicols}

\end{enumerate}