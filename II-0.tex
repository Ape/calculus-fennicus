\chapter{Vektorit ja analyyttinen geometria}

Ihmisten luoman matemaattisen kulttuurin peruspilareita on matemaattinen malli nimelt� 
\kor{euklidinen taso}. Termi viittaa kreikkalaisen 
\index{Eukleides}%
\hist{Eukleideen} n. 300 eKr kirjoittamaan teokseen Stoikheia eli 'Alkeet'. Eukleideen
pyrkimyksen� oli asettaa selke�t aksiomaattiset perusteet silloiselle matematiikalle,
erityisesti \kor{geometrian} nimell� tunnetulle matematiikan suuntaukselle. Geometria on ---
toisin kuin edell� tarkasteltu lukujen algebra --- n�k�aistimuksiin hyvin voimakkaasti vetoava
matematiikan laji. Algebra ja geometria edustavat nykyisess�kin matematiikassa kahta
matemaattisen ajattelun suurta p��haaraa.\footnote[2]{Algebran ja geometrian v�linen rajanveto
matematiikassa ei ole aina helppoa. Ongelmaa kuvastaa esim.\ moderni matematiikan laji nimelt�
\kor{algebrallinen geometria}.}

Eukleideen teos on kaikkien aikojen menestynein matematiikan oppikirja, jonka vaikutus
varhaisena aksiomaattisen ajattelun esikuvana on ulottunut matematiikan ulkopuolellekin.
K�yt�nn�n kannalta 'tyhj�st�' teos ei tietenk��n syntynyt, vaan sit� edelsi itse asiassa
vuosituhantinen laskentamenetelmien perinne sek� Egyptiss� ett� Babyloniassa. Esimerkiksi
Egyptiss� geometrisia menetelmi� olivat kehitt�neet (jos historioitsijoihin on uskominen) sek�
maanmittarit ett� papit.

T�ss� luvussa l�hdet��n euklidisen geometrian perusteista tasossa ja avaruudessa ja paneudutaan
tarkastelemaan geometrian pohjalta syntyv�� \kor{vektorin} k�sitett� ja vektoreilla laskemista 
eli \kor{vektorialgebraa}. Vektoreiden avulla geometriset ongelmat on mahdollista 
'algebralisoida' eli muotoilla ja ratkaista vektorialgebran ongelmina. T�llaista --- 
historiallisesti melko my�h�syntyist� --- l�hestymistapaa geometriaan kutsutaan 
\kor{analyyttiseksi} geometriaksi.