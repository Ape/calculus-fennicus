\section{Suppenevien lukujonojen ominaisuuksia} \label{suppenevat lukujonot}
\alku
\sectionmark{Suppenevat lukujonot}

Tässä luvussa tarkastellaan suppenevien lukujonojen ominaisuuksia yleiseltä kannalta 
--- 'luku' (mat.) ymmärretään samoin kuin edellisessä luvussa.
\begin{Def} \label{rajoitettu jono} \index{lukujono!e@rajoitettu|emph}
\index{rajoitettu!a@lukujono|emph}
Lukujono $\{a_n\}$ on \kor{rajoitettu} (engl.\ bounded),  jos on olemassa luku $C$ siten, että
pätee $\abs{a_n} \le C\ \ \forall n$. 
\end{Def}
\begin{Lause} \label{suppeneva jono on rajoitettu} Suppeneva lukujono on rajoitettu. 
\end{Lause}
\tod Olkoon jono yleistä muotoa $\{a_n\}_{n=m}^{\infty},\ m \in \Z$. Jos pätee $a_n \kohti a$,
niin suppenemisen määritelmän mukaan $\exists N \in \N$ siten, että
\[
\abs{a_n - a} < 1, \quad \text{kun}\ n > N
\]
(Määritelmässä \ref{jonon raja} valittu $\eps = 1$). Tällöin on kolmioepäyhtälön mukaan
\begin{align*}
\abs{a_n}\ &=\ \abs{\,(a_n - a) + a\,} \\
           &\le\ \abs{a_n - a} + \abs{a}\ <\ 1 + \abs{a}, \quad \text{kun}\ n > N,
\end{align*}
joten
\[
\abs{a_n}\ \le\ \max\{\,\abs{a_m}, \ldots, \abs{a_N}, \abs{a}+1\,\}\ 
           =\ C \quad \forall n \ge m. \loppu
\]

Seuraavat yleiset laskusäännöt ovat huomattavan käyttökelposia sekä raja-arvoja
laskettaessa että teoreettisemmissa tarkasteluissa.
\begin{Lause} \label{raja-arvojen yhdistelysäännöt} (\vahv{Raja-arvojen yhdistelysäännöt})
\index{lukujonon raja-arvo!64@raja-arvojen yhdistely|emph}
Olkoon $\seq{a_n}$ ja $\seq{b_n}$ lukujonoja ja $a,b,\lambda$ lukuja. Tällöin jos 
$\lim_n a_n = a$ ja $\lim_n b_n = b$, niin pätee:
\begin{itemize}
\item[(1)] $\quad \lim_n (a_n + b_n)\ =\ a+b$.
\item[(2)] $\quad \lim_n (\lambda a_n)\ =\ \lambda a$.
\item[(3)] $\quad \lim_n (a_n b_n)\ =\ ab$.
\item[(4)] $\quad \lim_n (a_n / b_n)\ =\ a/b \quad\,\ $ lisäehdoilla: \
                  $b_n \neq 0\ \forall n$ ja $b \neq 0$.
\end{itemize} \end{Lause}
Todistetaan ainoastaan väittämät (3) ja (4)
(muut ovat helpompia: Harj.teht.\,\ref{H-I-7: todistuksia}).

\pain{Väittämä} (3) \ Kirjoitetaan ensin
\begin{align*}
a_n b_n - ab\ &=\ (a_n b_n - a b_n) + (a b_n - ab) \\
              &=\ b_n(a_n - a) + a(b_n - b).
\end{align*} 
Koska $\{b_n\}$ on suppeneva jono, niin Lauseen \ref{suppeneva jono on rajoitettu} mukaan on 
jollakin $C$
\[
\abs{b_n} \le C\ \ \forall n,
\]
jolloin kolmioepäyhtälön perusteella
\[
\abs{a_n b_n - ab}\ \le\ C \abs{a_n - a} + \abs{a} \abs{b_n - b}.
\]
Jatkossa oletetaan, että $C > 0$ ja $\abs{a} > 0$ (jos $C=0$ tai $a=0$, niin päättely
yksinkertaistuu). Valitaan mielivaltainen $\eps > 0$, jolloin
\[
\eps_1 = \frac{\eps}{2C} > 0 \quad \text{ja} \quad \eps_2 = \frac{\eps}{2\abs{a}} > 0.
\]
Koska lukujonot $\{a_n\}$ ja $\{b_n\}$ suppenevat, on tällöin löydettävissä indeksit 
\mbox{$N_1 \in \N$} ja $N_2 \in \N$, siten että 
pätee
\begin{align*}
\abs{a_n - a}  &< \eps_1\,, \quad \text{kun}\ n > N_1, \\
\abs{b_n\,- b} &< \eps_2\,, \quad \text{kun}\ n > N_2.
\end{align*}
Tässä molemmat epäyhtälöt ovat voimassa indeksistä $N = \max\{N_1,N_2\}\in\N$ eteenpäin,
joten pätee
\[
\abs{a_n b_n - ab}\ <\ C \eps_1 + \abs{a} \eps_2\ 
                    =\ \frac{\eps}{2} + \frac{\eps}{2}\ =\ \eps, \quad \text{kun}\ n > N. 
\]
Koska tässä $\eps > 0$ oli mielivaltainen ja $N\in\N$ jokaisella $\eps>0$, niin lukujonon
raja-arvon määritelmän mukaan $a_n b_n \kohti ab$. \loppu

\pain{Väittämä} (4) \ Tässä aloitetaan kirjoittamalla
\begin{align*}
\frac{a_n}{b_n} - \frac{a}{b}\ &=\ \frac{a_n b - a b_n}{b_n b}\
                                =\ \frac{(a_n b - ab) + (ab - ab_n)}{b_n b} \\
                               &=\ \frac{a_n - a}{b_n} - \frac{a(b_n - b)}{b_n b}\,.
\end{align*}
Koska $b_n \kohti b \neq 0$, niin $\exists N_0 \in \N$ siten, että
\[
\abs{b_n - b} < \tfrac{1}{2} \abs{b}, \quad \text{kun}\ n > N_0
\]
(raja-arvon määritelmässä asetettu $\eps = \frac{1}{2} \abs{b},\ N=N_0$). Tällöin on
kolmioepäyhtälön nojalla
\[
\abs{b_n}\ \ge\ \abs{b} - \abs{b_n - b}\ >\ \tfrac{1}{2} \abs{b} \quad \forall n >N_0,
\]
joten
\begin{align*}
\abs{\,\frac{a_n}{b_n} - \frac{a}{b}\,}\ 
         &\le\ \frac{\abs{a_n - a}}{\abs{b_n}} + \abs{\,\frac{a}{b_n b}\,}\,\abs{b_n - b} \\
         &<\ \frac{2}{\abs{b}}\,\abs{a_n - a} + \frac{2 \abs{a}}{b^2}\,\abs{b_n - b} \quad 
                                                                         \forall n > N_0.
\end{align*}
Tässä on oikea puoli $< \eps\ (\eps > 0)$, kun
\begin{align*}
\frac{2}{\abs{b}}\,\abs{a_n - a}\ <\ \frac{\eps}{2} \quad 
        &\text{ja} \quad \frac{2 \abs{a}}{b^2}\,\abs{b_n - b}\ < \frac{\eps}{2}\,, \\
\intertext{eli kun}
\abs{a_n - a} < \frac{\abs{b}}{4}\,\eps = \eps_1 \quad 
        &\text{ja} \quad \abs{b_n - b} < \frac{b^2}{4 \abs{a}}\,\eps = \eps_2\,,
\end{align*}
olettaen että tässä $a \neq 0$. (Jos $a=0$, päättely yksinkertaistuu.) Koska $a_n \kohti a$ ja
$b_n \kohti b$ ja koska $\eps_1, \eps_2 > 0$, niin ensimmäinen ehto toteutuu jostakin indeksistä
$N_1$ ja toinen jostakin indeksistä $N_2$ eteenpäin, jolloin
\[
\abs{\,\frac{a_n}{b_n}-\frac{a}{b}\,}\ <\ \eps, \quad \text{kun}\ n>N = \max\{N_0,N_1,N_2\}\in\N.
\]
Tässä oli $\eps > 0$ mielivaltainen, joten määritelmän mukaan $a_n/b_n \kohti a/b$. \loppu

\begin{Exa} Laske $\ \lim_n a_n\,$, kun
\[
a_n = \frac{n^2+2n+2}{3n^2+n}\,.
\] \end{Exa}

\ratk Kirjoittamalla $a_n$ muotoon
\[
a_n = \frac{1 + 2 \cdot \tfrac{1}{n} + 2 \cdot \tfrac{1}{n} \cdot \tfrac{1}{n}}{3 + \tfrac{1}{n}}
\]
voidaan suoraan soveltaa Lauseen \ref{raja-arvojen yhdistelysäännöt} sääntöjä (1)--(4)\,:
\begin{align*}
\lim_n a_n\ &=\ \frac{\lim_n(1 + 2 \cdot \tfrac{1}{n} 
              + 2 \cdot \tfrac{1}{n} \cdot \tfrac{1}{n})}{\lim_n(3 + \tfrac{1}{n})} \\
            &=\ \frac{\lim_n(1) + 2\,\lim_n(\tfrac{1}{n}) 
              + 2\,\lim_n(\tfrac{1}{n}) \cdot \lim_n(\tfrac{1}{n})}{3 + \lim_n(\tfrac{1}{n})} \\
            &=\ \frac{1 + 2 \cdot 0 + 2 \cdot 0 \cdot 0}{3 + 0} = \frac{1}{3}\,.
\end{align*}
Tässä tarvittiin siis ainoastaan yhtä suoraan määritelmästä todistettavaa raja-arvotulosta
$\ \lim_n (\tfrac{1}{n}) = 0\ $ (vrt.\ Esimerkki \ref{jonoja}d edellisessä luvussa). \loppu

\begin{Exa} Yleinen geometrinen sarja on lukujono 
\[
\sum_{k=0}^{\infty} aq^k\ =\ \{\,\sum_{k=0}^n aq^k,\ \ n=0,1, \ldots\,\}. 
\]
Kun perusmuotoisen sarjan summa tiedetään (edellisen luvun Esimerkki \ref{geometrinen sarja}),
niin Lauseen \ref{raja-arvojen yhdistelysäännöt} säännön (2) mukaan yleisen geometrisen
sarjan summa on
\[
\sum_{k=0}^{\infty} aq^k\ =\ a\,\sum_{k=0}^{\infty} q^k\ = \frac{a}{1-q}, \quad |q|<1.
\]
Jos $\abs{q} \ge 1$, niin yleinen geometrinen sarja suppenee vain kun $a=0$. \loppu 
\end{Exa}
\begin{Exa} Edellisen luvun Esimerkissä \ref{sqrt 2} tarkasteltiin palautuvaa lukujonoa
\[
a_0 = 2, \quad a_{n+1} = \frac{a_n}{2} + \frac{1}{a_n}\,, \quad n = 0,1,\ldots
\] 
Jos oletetaan, että $a_n>0\ \forall n$ ja $a_n \kohti a>0$ (vrt.\ edellisen luvun tarkastelut),
niin Lauseen \ref{raja-arvojen yhdistelysäännöt} mukaan on oltava
\[
a = \lim_n a_{n+1} = \lim_n \left(\frac{a_n}{2} + \frac{1}{a_n}\right) 
                   = \frac{a}{2} + \frac{1}{a}\,.
\]
Siis
\[
a = \frac{a}{2} + \frac{1}{a} \qimpl a^2 = 2.
\]
Koska $a>0$, niin seuraa $a=\sqrt{2}$.  \loppu
\end{Exa}
\begin{Exa} Jos palautuva lukujono
\[
a_{n+1} = qa_n - 1, \quad n=0,1,\ldots
\]
suppenee (jollakin $a_0$) kohti raja-arvoa $a$, niin tästä oletuksesta seuraa:
\begin{align*}
q=1     &\qimpl a=a-1  \,\ \qimpl 0=1. \\
q\neq 1 &\qimpl a=qa-1 \qimpl a=1/(q-1).
\end{align*}
Siis päätellään, että jono \pain{ei} suppene (millään $a_0$), jos $q=1$, ja että muissa
tapauksissa ainoa mahdollinen raja-arvo on $a=1/(q-1)$. Jonoa tarkemmin tutkimalla selviää, että 
\[
a_n = a_0 q^n - \sum_{k=0}^{n-1} q^k, \quad n=1,2,\ldots,
\]
joten jono suppenee alkuarvosta $a_0$ riippumatta täsmälleen kun $\abs{q}<1$, ja tällöin siis
$a_n \kohti 1/(q-1)$. --- Huomattakoon, että päättely tapauksessa $q=1$ on itse asiassa epäsuora
todistus väittämälle: $\seq{a_n}$ ei suppene. (Vrt.\ Luku \ref{logiikka}.) \loppu
\end{Exa}

Seuraavassa vielä kolme väittämää, jotka perustuvat suoraan lukujonon raja-arvon 
määritelmään. Väittämistä keskimmäinen tunnetaan '\kor{voileipälauseena}'.
\begin{Lause} \label{jonotuloksia} \index{lukujonon raja-arvo!68@'voileipälause'|emph}
Lukujonoille pätee
\begin{itemize}
\item[V1.] Jos $\,a_n \le b\ \forall n\ (a_n \ge b\ \forall n)$ ja $a_n \kohti a$,
           niin $a \le b\ (a \ge b)$.
\item[V2.] Jos $a_n \kohti c\ \ja\ b_n \kohti c\ \ja\ a_n \le c_n \le b_n\ \forall n$,
           niin $c_n \kohti c$. 
\item[V3.] Jos $a_n \kohti 0$ ja jono $\seq{b_n}$ on rajoitettu, niin $a_n b_n \kohti 0$.
\end{itemize}
\end{Lause}
\tod \ V1. \ Oletetaan, että $a_n \le b\ \forall n$, ja tehdään vastaoletus: 
$\lim_n a_n = a > b$. Merkitään $\eps = a-b$, jolloin vastaoletuksen perusteella on $\eps > 0$.
Koska $a_n \le b\ \forall n$ (oletus) ja $b = a-\eps,\ \eps > 0$ (vastaoletuksen seuraamus),
niin seuraa $a_n \le a - \eps\ \forall n$ ja siis $\abs{a_n - a} \ge \eps > 0\ \forall n$. 
Suppenevuuden määritelmästä seuraa tällöin, että $a_n \not\kohti a$. Oletuksen mukaan oli 
kuitenkin $a_n \kohti a$, joten on päädytty loogiseen ristiriitaan: $a_n \kohti a$ ja 
$a_n \not\kohti a$. Vastaoletus $a>b$ on näin ollen väärä, eli on oltava $a \le b$. Väittämän 
ensimmäinen osa on siis todistettu. Toinen osa seuraa, kun jo todistettua väittämää
sovelletaan jonoon $\seq{-a_n}$.

V2. \ Koska $a_n \le c_n \le b_n\ \forall n$, niin
\begin{align*}
a_n-c\,\le\,c_n-c\,\le\,b_n-c\ \ \ja\ \ c-b_n\,&\le\,c-c_n\,\le\,c-a_n \\
                   \impl \quad \abs{c-c_n}\,&\le\,\max\{\abs{a_n-c},\abs{b_n-c}\} \,\ \forall n.
\end{align*}
Koska $a_n \kohti c$ ja $b_n \kohti c$, niin jokaisella $\eps>0$ on olemassa indeksit 
$N_1,N_2\in\N$ siten, että $\abs{a_n-c}<\eps$ kun $n>N_1$ ja $\abs{b_n-c}<\eps$ kun $n>N_2$.
Tällöin
\[
\abs{c-c_n} \le \max\{\abs{a_n-c},\abs{b_n-c}\} < \eps, \quad \text{kun}\ n>N=\max\{N_1,N_2\}.
\]
Määritelmän mukaan $c_n \kohti c$.

V3. \ Harjoitustehtävä (Tehtävä \ref{H-I-7: jonotulos 3}a). \loppu

\begin{Exa} Väittämän (vrt.\ Lauseen \ref{jonotuloksia} väittämämä V1) 
\[
a_n < b\,\ \forall n\,\ \ja\,\ a_n \kohti a \qimpl a<b
\]
osoittaa vääräksi vastaesimerkki $\,a_n=1-1/n,\ n\in\N$, $\,b=1$. \loppu
\end{Exa}
\begin{Exa} Kunnassa $(\J,+,\cdot,<)$ (ks.\ Luku \ref{kunta}) määritelty palautuva lukujono
\[
a_0=0, \quad a_{n+1}=\sqrt{a_n + 1}, \quad n=0,1,\ldots
\]
ilmoitetaan suppenevaksi. Mikä on raja-arvo?
\end{Exa}
\ratk Palautuskaavasta seuraa, että $a_n \ge 0\ \forall n$ ja että
\[
a_{n+1}^2 = a_n + 1, \quad n=0,1,\ldots
\]
Kun merkitään $\,\lim_n a_n=a$, niin Lauseen \ref{raja-arvojen yhdistelysäännöt} perusteella
seuraa
\[
\lim_n a_{n+1}^2 = a^2 = \lim_n(a_n + 1) = a+1.
\]
Siis $a^2-a-1=0\ \impl\ a=\tfrac{1}{2}(1\pm\sqrt{5})$. Koska $a_n \ge 0\ \forall n$, niin
on oltava $a \ge 0$ (Lause \ref{jonotuloksia}, väittämä V1). Siis 
$\,\lim_n a_n =\tfrac{1}{2}(1+\sqrt{5})$. \loppu

\subsection*{Jaksolliset desimaaliluvut}
\index{jaksollinen desimaaliluku|vahv}
\index{zyzy@ääretön desimaaliluku!69@jaksollinen desimaaliluku|vahv}%

Luvussa \ref{desimaaliluvut} todettiin, että jokaiseen rationaalilukuun $x \in \Q$ voidaan
liittää ääretön, jaksollinen desimaaliluku $\x = \{x_n\} \in \DD_p$ kriteerillä
\[
\abs{x - x_n}\ <\ 10^{-n} \quad \forall n.
\]
Tämä merkitsee (vrt.\ Propositio \ref{jonon raja 2}), että
\[
\lim_n x_n = x.
\]
Jatkossa on luontevaa hieman väljentää Luvussa \ref{desimaaliluvut} sovittua samastusrelaatiota
rationaalilukujen ja äärettömien desimaalilukujen välillä. Asetetaan

\begin{Def} \label{samastus QD} Jos $x\in\Q\,$ ja $\x = \{x_n\} \in \DD$, niin
\[
x=\x \qekv \lim_n x_n = x.
\] 
\end{Def}

Määritelmän \ref{samastus QD} kriteerillä siis jokainen rationaaliluku samastuu edelleen ainakin
yhteen jaksolliseen desimaalilukuun, nimittäin jakokulma-algoritmin antamaan. Entä toisinpäin: 
Jos lähtökohtana on jaksollinen desimaaliluku, niin onko tämä aina samastettavissa johonkin 
rationaalilukuun Määritelmän \ref{samastus QD} mukaisesti? Asian tutkimiseksi oletetaan, että
desimaaliluku $\x$ on muotoa
\[
\x\ =\ x_0. d_1 \ldots d_k d_{k+1} \ldots d_{k+m} d_{k+1} \ldots d_{k+m} \ldots,
\]
eli jaksollisuus alkaa (viimeistään) desimaalista n:o $k+1$ ja jakson pituus on $m$.
Tarkastellaan jonoa
$\{x_k, x_{k+m}, x_{k+2m}, \ldots\,\} = \{x_{k+lm},\ l=0,1,\ldots\,\} = \seq{y_l}$. Tälle pätee
\[
y_{l+1}-y_l\ =\ \sum_{i=k+1}^{k+m} d_i \cdot 10^{-lm-i}\ 
                         =\ 10^{-lm}\,(y_1 - y_0), \quad l = 0,1 \ldots
\]
joten
\begin{align*}
y_{l+1}\ &=\ y_0\ +\ (y_1-y_0)\ +\ \ldots\ +\ (y_{l+1}-y_l) \\
         &=\ y_0\ +\ (y_1 -y_0)\,\sum_{j=0}^l 10^{-jm}, \quad l = 0,1 \ldots
\end{align*}
Näin ollen jono $\seq{y_l}$ suppenee geometrisena sarjana:
\[
\lim_{l \kohti \infty} y_l\ =\ y_0\ +\ \frac{y_1-y_0}{1 - 10^{-m}}\
                            =\ x_k\ +\ \frac{x_{k+m} - x_k}{1 - 10^{-m}}\ =\ x \in \Q.
\]
Tästä ja jonon $\{x_n\}$ monotonisuudesta (vrt.\ Luku \ref{desimaaliluvut}) seuraa, että myös
koko jono $\{x_n\}$ suppenee. Nimittäin jos esim.\ $\,x > 0$, niin jono $\seq{x_n}$ on kasvava,
jolloin jokaisella $n>k$ pätee
\[
x_{k+lm}=y_l\ \le\ x_n\ \le\ y_{l+1}=x_{k+(l+1)m},
\]
missä $l\in\N\cup\{0\}$ on valittu ($n$:stä riippuen) siten, että
\[
k + lm\ <\ n\ \le\ k + (l+1)m.
\]
Kun tässä $n\kohti\infty$, niin $l\kohti\infty$, jolloin $y_l \kohti x$ ja $y_{l+1} \kohti x$
ja näin ollen $x_n \kohti x$ (vrt.\ Lause \ref{jonotuloksia}, väittämä V2).
Samastussopimuksen \ref{samastus QD} mukaisesti on siis todistettu (ks.\ myös Lause 
\ref{raja-arvon yksikäsitteisyys})
\begin{Lause} \label{samastuslause QD} Jokainen jaksollinen desimaaliluku samastuu
Määritelmän \ref{samastus QD} mukaisesti yksikäsitteiseen rationaalilukuun. 
\end{Lause}
\begin{Exa}
\[
5.5027027027\ldots\ =\ 5.5 + 10^{-1}\cdot\frac{0.027}{1-10^{-3}}\ 
                    =\ \frac{55}{10}+\frac{27}{9990}\ =\ \frac{1018}{185}\,. \loppu
\]
\end{Exa}
\begin{Exa} \label{samastuksia} Mitkä desimaaliluvut samastuvat lukuihin $0$, $10$ ja $ 3/125$ ? 
\end{Exa}
\ratk Vaihtoehtoja on kaikissa tapauksissa kaksi:
\begin{align*}
0\     &=\ +0.0000000 \ldots\ =\ -0.0000000 \ldots \\
10\    &=\ +10.000000 \ldots\ =\ +9.9999999 \ldots \\
3/125\ &=\ +0.0080000 \ldots\ =\ +0.0079999 \ldots \qquad \loppu
\end{align*}

\subsection*{Desimaalilukujen samastus}

Esimerkin \ref{samastuksia} mukaan on mahdollista, että kaksi merkkijonoina erilaista
(eli ei-identtistä) desimaalilukua $\x,\y\in\DD$ samastuu samaan rationaalilukuun $x$. Tällöin
on luonnollista samastaa $\x$ ja $\y$ myös keskenään, eli kirjoittaa $\x=\y$. Jos $\x=\seq{x_n}$
ja $\y=\seq{y_n}$, niin Määritelmän \ref{samastus QD} mukaan mainitussa tilanteessa on
$\,\lim_n x_n = \lim_n y_n = x$, jolloin Lauseen \ref{raja-arvojen yhdistelysäännöt} mukaan on
$\lim_n (x_n-y_n) = 0$. Otetaan tämä raja-arvoehto yleiseksi samastuskriteeriksi verrattaessa 
äärettömiä desimaalilukuja suoraan keskenään:
\begin{Def} \label{samastus DD} \index{samastus '$=$'!e@desimaalilukujen|emph} Kaksi
desimaalilukua $\x = \seq{x_n} \in \DD$ ja $\y = \seq{y_n} \in \DD$ samastetaan kriteerillä
\[
\x = \y \qekv \lim_n\,(x_n - y_n) = 0.
\] 
\end{Def}

Kahden 'eri näköisen' desimaaliluvun samastuminen on siis mahdollista ainakin, jos molemmat
samastuvat äärelliseen desimaalilukuun. --- Tämä osoittautuu ainoaksi erikoistapaukseksi:
\begin{Lause} \label{samastuslause DD} Jos desimaaliluvut $\x=\seq{x_n}\in\DD$ ja
$\y=\seq{y_n}\in\DD$ samastuvat keskenään Määritelmän \ref{samastus DD} mukaisesti, niin joko
$\x$ ja $\y$ ovat merkkijonoina identtiset tai $\x=x=\y$, missä $x$ on äärellinen desimaaliluku. 
\end{Lause}
\tod Jos $\seq{x_n}=\seq{0}$, niin myös $\seq{y_n}=\seq{0}$, sillä muuten
$\abs{x_n-y_n} = \abs{y_n} \not\kohti 0$, koska jono $\seq{\abs{y_n}}$ on kasvava. Tällöin on
siis $\x,\y=\pm 0.000 \ldots\,$ ja $\x=0=\y$. Jatkossa oletetetaan, että 
$\seq{x_n} \neq \seq{0}$ ja $\seq{y_n} \neq \seq{0}$. Tällöin on $\x$:n ja $\y$:n etumerkkien
oltava samat, muuten $\abs{x_n-y_n}=\abs{x_n}+\abs{y_n} \not\kohti 0$. Koska edelleen $\x$ ja
$\y$ voidaan kumpikin skaalata tekijällä $\pm 10^m,\ m \in \Z$ samastuskriteerin muuttumatta,
niin voidaan olettaa, että $\,0 \le x_n < 1\,$ ja $\,0 \le y_n <1\,$ jokaisella $n$, jolloin
$\x$ ja $\y$ ovat jollakin $k\in\N$ muotoa
\begin{align*}
\x\ &=\ 0.d_1 \ldots d_{k-1} d_k \ldots\ldots d_m \ldots\ldots, \\
\y\ &=\ 0.d_1 \ldots d_{k-1} \tilde{d}_k \ldots\ldots \tilde{d}_m \ldots\ldots,
\end{align*}
missä $\tilde{d}_k \neq d_k$. Tällöin on $\,x_n=y_n\,$ kun $n<k$ ja 
$\,x_k-y_k=(d_k-\tilde{d}_k)\,10^{-k}$. Yleisyyttä rajoittamatta voidaan edelleen olettaa, että 
$\tilde{d}_k < d_k$ ($d_k \neq 0$). Tällöin nähdään, että erotus $\,x_n-y_n$ saa pienimmän
mahdollisen arvonsa jokaisella $n>k$, kun $d_n=0,\,\tilde{d}_n=9\ \forall n>k$, eli kun
$\x$ ja $\y$ ovat muotoa
\begin{align*}
\x\ &=\ 0.d_1 \ldots d_{k-1} d_k 0 0 0 \ldots \\
\y\ &=\ 0.d_1 \ldots d_{k-1} \tilde{d}_k 9 9 9 \ldots
\end{align*} 
Tässä tapauksessa on (vrt.\ Propositio \ref{desim})
\begin{align*} \label{I-7: välitulos}
x_n-y_n &= \left(x_k + 0 \cdot 10^{-k-1} + \ldots + 0 \cdot 10^{-n}\right)
          -\left(y_k + 9 \cdot 10^{-k-1} + \ldots + 9 \cdot 10^{-n}\right) \\
        &= (x_k-y_k) - \left(10^{-k}-10^{-n}\right) \\
        &= (d_k-\tilde{d}_k-1)\,10^{-k} + 10^{-n}, \quad n>k.
\end{align*}
Siis erotuksen $x_n-y_n$ pienin mahdollinen arvo on $10^{-n}$ kun $n \ge k$, ja tämä
saavutetaan, kun $d_k=\tilde{d}_k+1$ ja $d_n=0$ ja $\tilde{d}_n=9$ jokaisella $n>k$. Tässä
erikoistapauksessa on siis $x_n=x_k\ \forall n \ge k$ ja $\lim_n y_n=x_k$, jolloin 
$\,\x=x_k=\y\,$ Määritelmän \ref{samastus QD} mukaisesti.

Muissa kuin em.\ erikoistapauksessa on oltava joko $d_k-\tilde{d}_k \ge 2$ (tapaus 1),
$d_m \ge 1$ jollakin $m>k$ (tapaus 2) tai $\tilde{d}_m \le 8$ jollakin $m>k$ (tapaus 3). 
Tapauksessa 1 seuraa em.\ tuloksesta, että $x_n-y_n > 10^{-k}=\eps>0\ \forall n>k$, jolloin 
$x_n-y_n \not\kohti 0$. Tapauksissa 2 ja 3 on oltava $x_m-y_m \ge 2 \cdot 10^{-m}$
(koska erotuksen pienin arvo oli $10^{-m}$). Tällöin koska jono $\seq{x_n}$ on kasvava
ja koska $y_n < y_m + 10^{-m}\ \forall n>m$ (Propositio \ref{desim}), niin seuraa, että
$x_n-y_n > 10^{-m}=\eps>0$, kun $n>m$. Siis myös tässä tapauksessa $x_n-y_n \not\kohti 0$.
On päätelty, että jos $\x$ ja $\y$ eivät ole merkkijonoina identtiset, niin oletus
$\lim_n(x_n-y_n)=0$ toteutuu vain edellä kuvatussa erikoistapauksessa, jossa $\x$ ja $\y$ 
samastuvat äärelliseen desimaalilukuun. \loppu

Yhteenvetona rationaalilukujen ja äärettömien desimaalilukujen samastuksesta voidaan todeta:
\begin{Kor} \label{rat ja desim samastus} Jokainen rationaaliluku $\,x\,$ samastuu vähintään
yhteen ja enintään kahteen äärettömään desimaalilukuun, jotka ovat jaksollisia.
Samastusvaihtoehtoja on kaksi täsmälleen kun $x$ on äärellinen desimaaliluku. \end{Kor}
\begin{Exa} Mihin äärettömiin binaarilukuihin samastuu kymmenjärjestelmän luku \ 
a) $x = 3/4$, \ b) $y=4/3$\,? 
\end{Exa}
\ratk a) Koska $\,3/4 = 2^{-2}(1 \cdot 2 + 1) = 1 \cdot 2^{-1} + 1 \cdot 2^{-2}$, niin
\[
x\ =\ 0.11000000 \ldots \ =\ 0.10111111 \ldots
\]
Vaihtoehtoja on kaksi, koska $x$ on äärellinen binaariluku.
 
b) Koska
\begin{align*}
\frac{4}{3}\ =\ \frac{1}{1-\frac{1}{4}}\
             =\ 1+\frac{1}{4}+\frac{1}{4^2}+\ldots\ 
             =\ 1 \cdot 2^0 + 0 \cdot 2^{-1} +1 \cdot 2^{-2} + 0 \cdot 2^{-3} + \ldots\,,
\end{align*}
niin $\,y=1.010101\ldots\,$ Tässä on vain yksi vaihtoehto. \loppu

\Harj
\begin{enumerate}

\item \label{H-I-7: todistuksia}
Todista Lauseen \ref{raja-arvojen yhdistelysäännöt} väittämät (1) ja (2).

\item
Määritä raja-arvojen yhdistelysääntöjä hyväksi käyttäen
\begin{align*}
&\text{a)}\ \ \lim_n \frac{2n^3-100n^2-5000n}{(n+200)^3} \qquad\quad
 \text{b)}\ \ \lim_n \frac{1}{n^3}\left[(n+3)^4-(n-2)^4\right] \\
&\text{c)}\ \ \lim_n \left(\frac{1+2+ \ldots +n}{n+2}-\frac{n}{2}\right) \qquad 
 \text{d)}\ \ \lim_n \left[\frac{(n+2)!-n!}{(n+1)!+(n-1)!}-n\right]
\end{align*}

\item 
Laske lukujonojen (sikäli kuin \kor{luku}jonoja)
\[
a_n=\frac{x^n}{1+x^n}\,, \quad b_n=\frac{x^n}{1+x^{2n}}\,, \quad n=1,2, \ldots\,, 
\]
raja-arvot kaikilla $x$:n rationaalisilla arvoilla.

\item \label{H-I-7: sqrt-kunta}
Määritä seuraavien palautuvien lukujonojen mahdolliset raja-arvot olettaen, että
jonot ovat suppenevia:
\begin{align*}
&\text{a)}\ \ a_{n+1}=-\frac{1}{3}\,a_n+\frac{2n^2}{n^2+1} \qquad\qquad\
 \text{b)}\ \ a_{n+1} = \frac{na_n-8n+3}{2n+5} \\
&\text{c)}\ \ a_{n+2} = \frac{2n}{3n+4} - a_{n+1} - \frac{1}{4}\,a_n \qquad
 \text{d)}\ \ a_{n+1} = \frac{1}{2+a_n} \\[2mm]
&\text{e)}\ \ a_{n+1} = \sqrt{a_n+2} \qquad\qquad\qquad\quad\,\ \
 \text{f)}\ \ a_{n+1} = \sqrt[3]{a_n+6} \\[2mm]
&\text{g)}\ \ a_{n+1} = \frac{3}{4-a_n^2} \qquad\qquad\qquad\qquad\
 \text{h)}\ \ a_{n+1} = \frac{1}{\sqrt{4-a_n^2}}
\end{align*}

\item 
Olkoon $a_n+a \neq 0\ \forall n$ ja $\,\lim_n (a_n+a)^{-1} = b \neq 0$. Näytä, että jono 
$\seq{a_n}$ suppenee ja määritä $\lim_n a_n$.

\item \label{H-I-7: jonotulos 3}
a) Todista Lauseen \ref{jonotuloksia} väittämä V3. \vspace{1mm}\newline
b) Olkoon $\seq{a_n}$ rajoitettu jono. Todista, että 
   $\displaystyle{\,\lim_n \frac{2n+a_n}{n}=2}$.

\item 
a) Mitkä seuraavista luvuista ovat eri suuria ja mitkä samoja\,? \newline
$1.285714285714285714..\,$, $\frac{11}{9}\,$, $\frac{1287}{1250}\,$, $\frac{9}{7}\,$, 
$1.029599999..\,$, $1.02949999..\,$, \newline
$1.029600000..\,$, $a=$ yksi kokonaista kaksi yhdeksäsosaa. \vspace{2mm}\newline
b) Mitkä seuraavista desimaaliluvuista ovat kaksittain verrattaessa varmasti eri suuret ja mitkä
mahdollisesti samat\,? \newline
$\x=2.41789..,\ \ \y=2.41799..,\ \ \breve{a}=2.41788..,\ \ \breve{b}=2.41798.., \ \ 
 \breve{c}=2.4179..$

\item
Määritä rationaaliluku $x$ siten, että $x=\x$, kun \newline
a) \ $\x=-0.46127127127 \ldots\,$, \ b) \ $\x=2015.201520152015 \ldots$

\item
Eräässä lukujärjestelmässä, jossa kirjoitetaan $1+1=2$, muodostavat äärettömien desimalilukujen
vastineet joukon $\mathbb L$. Tässä joukossa on luvulla $\x=0.222\ldots \in \mathbb L$ toinenkin
esitysmuoto. Mikä on ko.\ lukujärjestelmän kantaluku, ja mihin rationaalilukuun $\x$ samastuu?

\item 
Olkoon $x,y\in\Q$ ja $\x,\y\in\DD$. Lähtien Määritelmistä \ref{samastus QD} ja \ref{samastus DD}
ja käyttäen ainoastaan raja-arvojen yhdistelysääntöjä 
(Lause \ref{raja-arvojen yhdistelysäännöt}) näytä, että pätee \
a) \ $\x=x\ \ja\ x=\y\ \,\impl\,\ \x=\y$, \ \ b) \  $\x=\y\ \ja\ \y=y \,\impl\,\ \x=y$.

\item (*)
Todista: \ 
$\displaystyle{\lim_n a_n = a\,\ \impl\,\ \lim_n \frac{a_1+a_2+ \ldots + a_n}{n} = a}$.

\item (*)
Määritellään palautuva lukujono
\[
a_1\in\Q, \quad a_{n+1}=\frac{a_n}{1+10^{-n}a_n}\,, \quad n=1,2, \ldots
\]
Näytä, että $\seq{a_n}$ suppenee ja
\[
\lim_n a_n = \frac{9a_1}{a_1+9}\,, 
\]
paitsi eräillä poikkeuksillisilla $a_1$:n arvoilla --- millä\,? \newline
\kor{Vihje}: Tutki jonoa $\seq{b_n}=\seq{a_n^{-1}}$\,.

\item (*)
\ a) Määritä palautuvan lukujonon
\[
a_{n+1}=(2-a_n)^2, \quad n=0,1, \ldots
\]
mahdolliset raja-arvot $\,a=\lim_n a_n$.\, b) Tutkimalla jonoa $\seq{b_n}=\seq{a_n-a}$ päättele,
että $\,\lim_n a_n = a$ vain kun $a_n=a$ jollakin $n$.\, c) Päättele, että jos $a_0\in\Q$, niin
$\seq{a_n}$ suppenee täsmälleen, kun $a_0$ on jokin luvuista $0,1,2,3,4$.

\end{enumerate}