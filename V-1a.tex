% Proposition \ref{jatkuvuuspropositio} todistus
\tod Jos $x_n \in (a,c)\ \forall n$ ja $x_n \kohti c$, niin $f(x_n)=f_1(x_n)\ \forall n$, 
jolloin $f_1$:n oletetun jatkuvuuden perusteella $f(x_n) = f_1(x_n) \kohti f_1(c)$. Ehto 
$f(c)=f_1(c)$ on siis välttämätön $f$:n jatkuvuudelle $c$:ssä. Valitsemalla 
$x_n \in (c,b),\ x_n \kohti c$, seuraa vastaavasti, että ehto $f(c)=f_2(c)$ on samoin 
välttämätön, jolloin on todistettu väittämän osa \fbox{$\Rightarrow$}. 

Osan \fbox{$\Leftarrow$} todistamiseksi olkoon $\seq{x_n}$ mikä tahansa jono, jolle pätee 
$x_n \in (a,b)\ \forall n$ ja $x_n \kohti c$. Tällöin $f_1(x_n) \kohti f(c)$ ja 
$f_2(x_n) \kohti f(c)$, koska $f_1$ ja $f_2$ ovat jatkuvia $c$:ssä, ja koska oletuksen mukaan 
$f_1(c)=f_2(c)=f(c)$. Olkoon nyt $\eps>0$. Koska $f_1(x_n) \kohti f(c)$ ja 
$f_2(x_n) \kohti f(c)$, niin Määritelmän \ref{jonon raja} perusteella on olemassa indeksit 
$N_1,N_2$, siten että
\[
\abs{f_1(x_n)-f(c)} < \eps\,\ \text{kun}\ n>N_1\,, \quad 
\abs{f_2(x_n)-f(c)} < \eps\,\ \text{kun}\ n>N_2\,.
\]
Tällöin jos $N = \max\{N_1,N_2\}$, niin pätee
\[
\abs{f(x_n)-f(c)} \le \max\{\abs{f_1(x_n)-f(c)},\,\abs{f_2(x_n)-f(c)}\} < \eps, 
                                                          \quad \text{kun}\ n > N,
\]
sillä jokaisella $n$ on joko $f(x_n)=f_1(x_n)$ tai $f(x_n)=f_2(x_n)$. Tässä $\eps>0$ oli 
mielivaltainen, joten Määritelmän \ref{jonon raja} perusteella $f(x_n) \kohti f(c)$. \loppu

%käänteisfunktion jatkuvuuteen liittyvä esimerkki 
Lause \ref{käänteisfunktion jatkuvuus} todistetaan myöhemmin Luvussa \ref{weierstrassin lause}.
Tällöin nähdään, että yleisempikin väittämä
\[ 
f:\ A \kohti B\,\ \text{jatkuva bijektio}\ \impl\ \inv{f}:\ B \kohti A\,\ \text{jatkuva bijektio} 
\]
on tosi esim.\ siinä tapauksessa, että $A$ koostuu äärellisen monesta erillisestä välistä,
kunhan jokainen väli on \pain{sul}j\pain{ettu}. Seuraavasta vastaesimerkistä nähdään, että tämä 
rajoitus on välttämätön. 
\begin{Exa}
Määritellään funktio $f$ joukossa
\begin{multicols}{2} \raggedcolumns
\[
D_f=A=[-1,0]\cup (1,2]
\]
seuraavasti:
\[
f(x)=\begin{cases}
x &,\text{ kun } x\in [-1,0] \\
x-1 \ &, \text{ kun } x\in (1,2]
\end{cases}
\]
\begin{figure}[H]
\setlength{\unitlength}{1cm}
\begin{center}
\begin{picture}(4,4)(-2,-2)
\put(-2,0){\vector(1,0){4}} \put(1.8,-0.4){$x$}
\put(0,-2){\vector(0,1){4}} \put(0.2,1.8){$y$}
\drawline(-1,-1)(0,0)
\drawline(1,0)(2,1)
\put(-0.1,-0.1){$\bullet$}
\end{picture}
%\caption{$y=f(x)$}
\end{center}
\end{figure}
\end{multicols}
Tällöin $f$ on koko määrittelyjoukossaan jatkuva ja $f:A \kohti [-1,1]$ on bijektio, joten myös 
$\inv{f}:[-1,1] \kohti A$ on bijektio (vrt. Luku \ref{käänteisfunktio}). Mutta \inv{f} ei ole 
jatkuva pisteessä $x=0$. \loppu
\end{Exa}
Esimerkissä on käänteisfunktion jatkuvuuden kannalta ongelmana, että määrittelyjoukon osaväli 
$(1,2]$ ei ole suljettu. Ongelmaa ei voi poistaa ottamalla $x=1$ mukaan määrittelyjoukkoon,
sillä jos asetetaan $f(1)=0$, niin $f(0)=f(1)$, jolloin $f$ ei ole injektio, ja jos asetetaan 
$f(1)=c \neq 0$, niin $f$ ei ole jatkuva pisteessä $x=1$.