\section{Käyrän tangentti ja normaali} \label{derivaatta geometriassa}
\alku \sectionmark{Tangentti ja normaali}
\index{tangentti (käyrän)|vahv} \index{normaali(vektori)!a@suoran, käyrän|vahv}
\index{kzyyrzy@käyrä|vahv} \index{parametrinen käyrä|vahv}

Derivaatan tavanomainen geometrinen tulkinta euklidisessa tasossa on:
\index{kulmakerroin}%
\[
\boxed{\kehys\quad \text{Derivaatta}=\text{tangentin kulmakerroin}. \quad}
\]
Käyrän $y=f(x)$ sivuaja eli \kor{tangentti} pisteessä $(c,f(c))$ on pisteiden $(c,f(c))$ ja 
$(c+\Delta x,f(c+\Delta x))$ kulkevan suoran eli
\index{sekantti (käyrän)}%
\kor{sekantin} 'raja-arvo', kun 
$\Delta x\kohti 0$. Pisteen $(c,f(c))$ kautta kulkeva suora, joka on tangenttia vastaan 
kohtisuora, on käyrän $y=f(x)$ \kor{normaali} ko. pisteessä.
\begin{figure}[H]
\setlength{\unitlength}{1cm}
\begin{center}
\begin{picture}(10,8)(-1,-1)
\put(-1,0){\vector(1,0){10}} \put(8.8,-0.4){$x$}
\put(0,-1){\vector(0,1){8}} \put(0.2,6.8){$y$}
\Thicklines
\curve(
   -1.0000,    0.4586,
   -0.5000,    0.7987,
         0,    1.0117,
    0.5000,    1.1249,
    1.0000,    1.1657,
    1.5000,    1.1614,
    2.0000,    1.1394,
    2.5000,    1.1269,
    3.0000,    1.1514,
    3.5000,    1.2402,
    4.0000,    1.4206,
    4.5000,    1.7200,
    5.0000,    2.1657,
    5.5000,    2.7851,
    6.0000,    3.6054,
    6.5000,    4.6542,
    7.0000,    5.9586)
\thinlines
\put(3.28,0.9){\line(1,1){3.8}}
\put(1.8,0.66){\line(3,1){6}}
\put(2.6,4.5){\line(1,-3){1.3}}
\put(2,4.7){normaali}
\put(7,3){tangentti}
\put(7,5){sekantti}
\dashline{0.1}(3.67,0)(3.67,1.25)
\dashline{0.1}(6,0)(6,3.6)
\put(3.57,-0.5){$c$} \put(5.9,-0.5){$c+\Delta x$}
\put(6.5,6.1){$y=f(x)$}
\end{picture}
%\caption{Derivaatan geometrinen tulkinta}
\end{center}
\end{figure}
Tangentin ja normaalin yhtälöt ovat
\begin{align*}
\text{Tangentti}\,: &\qquad y-f(c) = f'(c)(x-c). \\
\text{Normaali}\,:  &\qquad x-c = -f'(c)(y-f(c)).
\end{align*}
\index{kohtisuora leikkaus!a@käyrien}%
Sanotaan, että kaksi käyrää \kor{leikkaavat kohtisuorasti}, jos leikkauspisteessä tangentit
ovat kohtisuorassa toisiaan vastaan.
\begin{Exa}
Suora kulkee pisteen $(4,0)$ kautta ja leikkaa käyrän $y=x^2$ kohtisuorasti. Suoran yhtälö?
\end{Exa}
\ratk Suora on käyrän normaali leikkauspisteessä. Jos leikkauspiste on $(t,t^2)$, niin suoran 
yhtälö on siis
\[
x-t=-2t(y-t^2).
\]
Tämä kulkee pisteen $(4,0)$ kautta ehdolla
\[
2t^3+t-4=0 \ \impl \ t\approx 1.12817390.
\]
Suoran yhtälö: $\ y=k(x-4)$, $\ k\approx -0.44319409$. \loppu
\begin{Exa} \label{kohtisuora leikkaus} \index{kohtisuora leikkaus!b@käyräparvien}
Näytä, että \kor{käyräparvet} \index{kzyyrzy@käyräparvi}
\[
xy=a,\quad x^2-y^2=b,
\]
missä $a,b\in\R$, $a\neq 0$, leikkaavat toisensa kohtisuorasti.
\end{Exa}
\ratk Yhteisissä pisteissä $(x,y)$ on oltava $x\neq 0$, $y\neq 0$, koska $xy=a\neq 0$. Kun 
merkitään
\[
xy=a \ \ekv \ y=y(x)\quad (x\neq 0),
\]
niin implisiittisesti derivoimalla saadaan
\begin{align*}
\frac{d}{dx}[xy(x)]&=y(x)+xy'(x)=\frac{d}{dx}a=0 \\
&\impl \ y'(x)=-y(x)/x=-y/x.
\end{align*}
Vastaavasti voidaan merkitä
\[
x^2-y^2=b \ \ekv \ y=y(x),
\]
jolloin kyse on kaksihaaraisesta implisiittifunktiosta. Molemmille haaroille pätee 
derivoimissääntö
\begin{align*}
\frac{d}{dx}\left(x^2-[y(x)]^2\right)&= 2x-2y(x)y'(x)=\frac{d}{dx}b=0 \\
&\impl \ y'(x)=x/y(x)=x/y.
\end{align*}
Käyrien yhteisissä pisteissä $(x,y)$ tangenttien kulmakertoimien tulo on siis 
\[
-\frac{y}{x}\cdot\frac{x}{y} = -1,
\]
joten tangentit ovat kaikissa leikkauspisteissä kohtisuorat. \loppu
\index{zza@\sov!Heijastuslaki}%
\begin{Exa}: \vahv{Heijastuslaki}. Halutaan määrittää toistaiseksi tuntemattomalla välillä
$\,A\subset\R\,$ funktio $\,x \in A \map y(x) \ge 0\,$ ja vastaava käyränkaari
\[
S = \{\,(x,y) \in \Rkaksi \mid x \in A\ \ja\ y = y(x) \ge 0\,\}
\]
siten, että jokainen pisteestä $(-c,0)$ lähtenyt ja käyrästä heijastunut valonsäde kulkee
pisteen $(c,0)$ kautta ($c>0$). Muotoile ongelma differentiaaliyhtälöksi muotoa
\[ 
F(x,y,y') = 0, 
\]
\index{ellipsi}%
ja etsi ratkaisuja \kor{ellipsin} kaarien
\[
y=y(x) \ \ekv \ \frac{x^2}{a^2}+\frac{y^2}{b^2}=1
\]
joukosta ($a,b\in\R_+$).
\end{Exa}
\ratk Valonsäteen heijastuessa pisteessä $P=(x,y)$ on tulokulma sama kuin heijastuskulma
käyrän tangenttiin nähden.
\begin{figure}[H]
\setlength{\unitlength}{1cm}
\begin{center}
\begin{picture}(10,5)(-5,-1)
\put(-5,0){\vector(1,0){10}} \put(4.8,-0.4){$x$}
\put(0,0){\vector(0,1){4}} \put(0.2,3.8){$y$}
\path(-4,0)(1,3)(4,0)
\dashline{0.1}(1,3)(4,3)
\put(1,3){\line(5,-1){3}}\put(1,3){\line(-5,1){3}}
\put(4,2.4){\line(-2,3){0.1}}
\put(4,2.4){\line(-3,-2){0.15}}
\put(0,0){\vector(1,0){1}} \put(0,0){\vector(0,1){1}} \put(0.9,-0.5){$\vec i$}
\put(0.2,0.8){$\vec j$}
\multiput(-4,0)(8,0){2}{\line(0,-1){0.1}} \put(-4.4,-0.5){$-c$} \put(3.9,-0.5){$c$}
\put(-4.2,0.2){$A$} \put(3.9,0.2){$B$} \put(0.8,3.3){$P=(x,y)$} \put(4.2,2.2){$\vec t$}
\put(0.9,2.9){$\bullet$}
\put(1,3){\arc{3}{0}{0.197}}
\put(1,3){\arc{1}{0.197}{0.785}}
\put(1,3){\arc{1}{2.6}{3.35}}
\put(2.65,2.75){$\scriptstyle{\beta}$} \put(1.6,2.55){$\alpha$} \put(0.2,2.8){$\alpha$}
\end{picture}
%\caption{Valonsäteen heijastuminen}
\end{center}
\end{figure}
Käyrän tangentin suuntainen vektori heijastuspisteessä on (vrt.\ kuvio)
\[
\vec t=\vec i +y'\vec j,
\]
ja vektorit $\overrightarrow{AP}$ ja $\overrightarrow{PB}$ ovat (kuvio)
\[
\overrightarrow{AP}=(x+c)\vec i+y\vec j,\quad \overrightarrow{PB}=(c-x)\vec i -y\vec j.
\]
Vaadittu heijastusehto on
\begin{align*}
&\frac{\overrightarrow{AP}\cdot\vec t}{\abs{\overrightarrow{AP}}}=
\frac{\overrightarrow{PB}\cdot\vec t}{\abs{\overrightarrow{PB}}} \\
&\ekv \ \frac{(c+x)+yy'}{\sqrt{(c+x)^2+y^2}}=\frac{(c-x)-yy'}{\sqrt{(c-x)^2+y^2}}\,.
\end{align*}
Neliöön korottamalla ja sieventämällä tämä yksinkertaistuu yhtälöksi
\[
F(x,y,y') = xy^2(y')^2-(c^2-x^2+y^2)yy'-xy^2=0.
\]
Kokeillaan, toteutuuko tämä, kun
\[
y^2=b^2-\frac{b^2}{a^2}x^2 \ \impl \ y'=-\frac{b^2}{a^2}\,x/y.
\]
Näillä sijoituksilla differentiaaliyhtälö pelkistyy yhtälöksi
\[
\left(\frac{c^2+b^2}{a^2}-1\right) b^2x=0,
\]
joka toteutuu $\forall x$ ehdolla $\,a^2=c^2+b^2$. Tässä $b\in\R_+$ on vapaasti valittavissa,
joten ongelman ratkaisuja ovat
\[
y=y(x) = b\,\sqrt{1-\frac{x^2}{a^2}}\,,\quad x\in[-a,a]=A,\,\ a=\sqrt{c^2+b^2},\,\ b\in\R_+.
\]
\index{polttopiste (ellipsin)}%
Nämä ovat erimuotoisia ellipsin puolikaaria, joiden \kor{polttopisteinä} ovat $(\pm c,0)$. 
Kuvassa on $c=1$. \loppu
\begin{figure}[H]
\setlength{\unitlength}{1cm}
\begin{center}
\begin{picture}(10,4)(-5,0)
\put(-5,0){\vector(1,0){10}} \put(4.8,-0.4){$x$}
\put(0,0){\vector(0,1){4}} \put(0.2,3.8){$y$}
\curve(
   -4.26,    0,
   -4.2000,    0.4243,
   -4.1000,    0.7714,
   -4.0000,    1.0000,
   -3.9000,    1.1811,
   -3.6000,    1.5875,
   -3.3000,    1.8855,
   -3.0000,    2.1213,
   -2.7000,    2.3141,
   -2.4000,    2.4739,
   -2.1000,    2.6067,
   -1.8000,    2.7166,
   -1.5000,    2.8062,
   -1.2000,    2.8775,
   -0.9000,    2.9317,
   -0.6000,    2.9698,
   -0.3000,    2.9925,
         0,    3.0000,
    0.3000,    2.9925,
    0.6000,    2.9698,
    0.9000,    2.9317,
    1.2000,    2.8775,
    1.5000,    2.8062,
    1.8000,    2.7166,
    2.1000,    2.6067,
    2.4000,    2.4739,
    2.7000,    2.3141,
    3.0000,    2.1213,
    3.3000,    1.8855,
    3.6000,    1.5875,
    3.9000,    1.1811,
    4.0000,    1.0000,
    4.1000,    0.7714,
    4.2000,    0.4243,
    4.26,    0)
\curve(
  -3.3541,                  0,
  -3.3300,             0.1795,
  -3.3000,             0.2683,
  -3.2000,             0.4494,
  -3.1000,             0.5727,
  -3.0000,             0.6708,
  -2.7000,             0.8899,
  -2.4000,             1.0479,
  -1.8000,             1.2657,
  -1.2000,             1.4007,
  -0.6000,             1.4758,
        0,             1.5000,
   0.6000,             1.4758,
   1.2000,             1.4007,
   1.8000,             1.2657,
   2.4000,             1.0479,
   2.7000,             0.8899,
   3.0000,             0.6708,
   3.1000,             0.5727,
   3.2000,             0.4494,
   3.3000,             0.2683,
   3.3300,             0.1795,
   3.3541,                  0)
\multiput(-3,0)(6,0){2}{\line(0,-1){0.1}} \put(-3.4,-0.5){$-1$} \put(2.9,-0.5){$1$}
\put(-0.075,2.925){$\scriptstyle{\bullet}$} \put(0.2,2.6){$1$}
\put(0.5,1.6){$b=\tfrac{1}{2}$} \put(1,3){$b=1$}
\end{picture}
%\caption{$c=1$}
\end{center}
\end{figure}

\subsection*{Parametrisen käyrän tangentti}

Parametrisen käyrän, eli vektoriarvoisen funktion $t \in A\ \map \vec r\,(t) \in \R^d$, missä
$A \subset \R$ ja $d=2$ tai $d=3$ (vrt.\ Luku \ref{parametriset käyrät}), derivaatta $\dvr(t)$ 
määritellään tavanomaiseen tapaan, eli
\[ 
\dvr(t) = \lim_{\Delta t \kohti 0}\dfrac{1}{\Delta t}\left[\vec r\,(t+\Delta t)-\vec r\,(t)\right]
        = \begin{cases} \begin{aligned}
          x'(t)\vec i + y'(t)\vec j \quad\quad\quad\quad\ \ &\text{(tasokäyrä)} \\
          x'(t)\vec i + y'(t)\vec j + z'(t)\vec k \quad     &\text{(avaruuskäyrä)}
          \end{aligned} \end{cases} \]
sellaisissa pisteissä, joissa funktiot $x(t)$, $y(t)$ ja $z(t)$ ovat derivoituvia. Määritelmän
mukaan derivaattavektori on pisteiden $P \vastaa \vec r\,(t)$ ja
$Q \vastaa \vec r\,(t + \Delta t)$ kulkevan suoran suuntavektorin raja-arvo, sikäli kuin pisteet
$P$ ja $Q$ ovat erillisiä $\abs{\Delta t}$:n ollessa riittävän pieni. Viimeksi mainittu ehto on
voimassa ainakin, jos
\begin{equation} \label{pysähtymättömyysehto}
 \abs{\dvr(t)} \neq 0,
\end{equation}
sillä linearisoivan approksimaatioperiaatteen mukaisesti pätee
\[
\abs{\vec r\,(t+\Delta t)-\vec r\,(t)}\ =\ \abs{\dvr(t)}\abs{\Delta t}+g(t,\Delta t), \quad 
                                    \lim_{\Delta t \kohti 0} \frac{g(t,\Delta t)}{\Delta t}=0,
\]
jolloin ehdolla \eqref{pysähtymättömyysehto} on tässä vasen puoli $\neq 0$, kun
$\Delta t \in (-\delta,\delta)$ jollakin (riittävän pienellä) $\delta>0$. Ehdolla 
\eqref{pysähtymättömyysehto} voidaan siis vektori $\dvr(t) \neq \vec 0$ tulkita käyrän 
\index{tangenttivektori (käyrän)}%
$S = \{P(t) \vastaa \vec r\,(t) \mid t \in A\}$ \kor{tangenttivektoriksi} pisteessä 
$P(t) \vastaa \vec r\,(t)$. Vektorin $\dvr(t)$ suuntainen yksikkövektori $\tv(t)$ on nimeltään 
\kor{yksikkötangenttivektori}: \index{yksikkövektori!a@yksikkötangenttivektori}%
\begin{equation} \label{yksikkötangenttivektori} 
\boxed{\kehys\quad \tv(t) = \dfrac{\dvr(t)}{\abs{\dvr(t)}} \quad 
                     (\text{yksikkötangenttivektori}, \,\ \abs{\dvr(t)} \neq 0\,). \quad} 
\end{equation}
Huomattakoon, että yksikkötangenttivektori annetussa käyrän pisteessä $P$ on puhtaasti
g\pain{eometrinen} käsite: Mahdollisia $\tv$:n arvoja (sikäli kuin $\tv$ yleensä on 
määritelty) on vain kaksi. Ehdon \eqref{pysähtymättömyysehto} ollessa voimassa parametrisointi 
valitsee näistä toisen kaavan \eqref{yksikkötangenttivektori} mukaisesti, muuten $\tv$ 
\pain{ei} \pain{rii}p\pain{u} p\pain{arametrisoinnista}. 

Yksikkötangenttivektori voidaan tulkita selkeämmin geometrisesti, kun merkitään
\[
\Delta\vec r=\vec r\,(t+\Delta t)-\vec r\,(t).
\]
Jos pisteet $P \vastaa \vec r\,(t)$ ja $Q \vastaa \vec r\,(t + \Delta t)$ ovat erillisiä, niin 
$\Delta\vec r$ on pisteiden $P$ ja $Q$ kautta kulkevan käyrän sekantin suuntavektori. Ehdon 
\eqref{pysähtymättömyysehto} voimassa ollessa tarkkenee approksimaatio 
$\abs{\Delta\vec r\,} \approx \abs{\dvr(t)}\abs{\Delta t}$ rajalla $\Delta t \kohti 0$, jolloin 
yksikkötangenttivektorin molemmat arvot saadaan toispuolisina raja-arvoina
\begin{multicols}{2} \raggedcolumns
\[
\tv_\pm = \lim_{\Delta t\kohti 0^\pm} \frac{\Delta\vec r}{\abs{\Delta\vec r\,}}\,.
\]
\begin{figure}[H]
\setlength{\unitlength}{1cm}
\begin{center}
\begin{picture}(8,3)(0,1)
\put(0,3){\vector(3,-2){3}}
\put(0,3){\vector(1,0){7}}
\put(3,1){\vector(2,1){4}}
\curve(1,0.8,3,1,5,1.7,7,3,8,4)
\put(3,1){\vector(4,1){3.5}}
\put(6,3.2){$\scriptstyle{\vec r(t+\Delta t)}$}
\put(2.6,1.4){$\scriptstyle{\vec r(t)}$}
\put(6.3,1.5){$\scriptstyle{\tv}$}
\put(5.5,2.5){$\scriptstyle{\Delta\vec r}$}
\put(2.95,0.95){$\scriptstyle{\bullet}$}
\put(6.94,2.93){$\scriptstyle{\bullet}$}
\end{picture}
\end{center}
\end{figure}
\end{multicols}
Tutkittaessa käyrän geometriaa parametrisoinnin avulla asetetaan ehto
\eqref{pysähtymättömyysehto} usein lähtökohtaisesti 'mukavuusehtona' koko tarkasteltavalla 
välillä, jolloin kaava \eqref{yksikkötangenttivektori} on käytettävissä. Tyypillisesti 
oletetaan, että $\vec r\,(t)$:n koordinaattifunktiot $x(t)$, $y(t)$, $z(t)$ ovat
j\pain{atkuvasti} \pain{derivoituvia} suljetulla välillä $A = [a,b]$ ja että ehto
\eqref{pysähtymättömyysehto} on voimassa koko välillä, mukaanlukien päätepisteet, joissa
derivaatta $\dvr$ tulkitaan toispuoliseksi.
\begin{Exa} Ruuviviivan 
$\,\vec r\,(\varphi)=R\cos\varphi\vec i+R\sin\varphi\vec j+a\varphi\vec k\ (R>0,\ a \neq 0)$
yksikkötangenttivektori pisteessä $\,P(\varphi) = (x(\varphi),y(\varphi),z(\varphi))\,$ on
\[ 
\tv(\varphi) = \pm \dfrac{-R\sin\varphi\vec i + R\cos\varphi\vec j + a\vec k}{\sqrt{R^2+a^2}}
                = \pm \dfrac{1}{\sqrt{R^2+a^2}}\,(-y\vec i + x\vec j + a\vec k).  
\]
Tässä on $\,\abs{\dvr(\varphi)} = \sqrt{R^2+a^2} =$ vakio. \loppu 
\end{Exa}
\begin{Exa} Käyrä $\,y=x^2,\,z=x^3\,$ leikkaa tason $T$ kohtisuorasti pisteessä $(1,1,1)$. 
Tason yhtälö? 
\end{Exa}
\ratk Käyrän eräs parametrisointi on 
\[ 
x(t)=t,\ y(t)=t^2,\ z(t)=t^3\ \ekv\ \vec r\,(t) = t\vec i + t^2\vec j + t^3\vec k. 
\]
Taso $T$ kulkee pisteen $(1,1,1) \vastaa \vec r\,(1)$ kautta ja sen normaalivektori on 
$\vec n = \dvr(1) = \vec i + 2\vec j + 3\vec k$, joten tason yhtälö on 
(vrt.\ Luku \ref{suorat ja tasot})
\[ 
(x-1)+2(y-1)+3(z-1) = 0 \qekv x+2y+3z-6 = 0. \loppu 
\]

\subsection*{Nopeusvektori}
\index{nopeusvektori|vahv}

Jos parametrisessa käyrässä $t \map \vec r\,(t)$ parametri $t$ on fysikaalinen aikamuuttuja ja 
$\vec r\,(t)$ on avaruudessa liikkuvan pisteen (partikkelin yms.) paikka hetkellä $t$, niin
ko.\ pisteen \pain{no}p\pain{eus}(vektori) ja \pain {ratano}p\pain{eus} eli \pain{vauhti} 
määritellään:
\[ \boxed{\begin{aligned} 
  \ykehys\quad \text{\pain{No}p\pain{eusvektori}:} \quad   
           &\vec v(t) = \dvr(t) = x'(t)\vec i + y'(t)\vec j + z'(t)\vec k. \quad \\
  \text{\pain{Vauhti}:} \quad\quad\quad\quad\ 
           &v(t) = \abs{\dvr(t)}. \akehys
          \end{aligned} } \]
Aikaparametrisoinnissa ehto \eqref{pysähtymättömyysehto} on siis 'pysähtymiskielto'.
\begin{Exa} \label{ympyräliike} Partikkeli liikkuu pitkin origokeskistä $R$-säteistä 
ympyräviivaa siten, että napakulma hetkellä $t$ on $\varphi(t)$. Nopeus ja vauhti hetkellä 
$t$\,? 
\end{Exa}
\ratk Partikkelin paikkavektori hetkellä $t$ on 
$\vec r\,(t) = R\cos\varphi(t)\vec i + R\sin\varphi(t)\vec j$, joten
\[ 
\vec v(t) = R\varphi'(t)\,[-\sin\varphi(t)\vec i + \cos\varphi(t)\vec j\,] 
          = \begin{cases} \begin{aligned} 
            v(t)\tv_+(t), \quad &\text{jos}\ \varphi'(t) \ge 0, \\
            v(t)\tv_-(t), \quad &\text{jos}\ \varphi'(t) < 0,
            \end{aligned} \end{cases} 
\]
missä $v(t) = R\abs{\varphi'(t)}$ on vauhti ja 
$\tv_\pm(t) = \pm[-\sin\varphi(t)\vec i + \cos\varphi(t)\vec j\,]$ on liikkeen suuntainen
yksikkövektori = liikeradan yksikkötangenttivektori. \loppu
\begin{Exa}: \vahv{Sykloidi}. \index{sykloidi} \index{zza@\sov!Sykloidi}
$R$-säteinen pyörä vierii liukumatta pitkin $x$-akselia siten, että pyörän keskipisteen
liikenopeus on $v_0\vec i$, $v_0=\text{vakio}$. Pyörän ulkokehän piste $P$ on hetkellä $t=0$
origossa. Mikä on $P$:n nopeus ajan funktiona?
\end{Exa}
\ratk 
\begin{multicols}{2} \raggedcolumns
Pisteen $P$ liikerata on sykloidi, jonka aikaparametrisaatio on 
(vrt.\ Luku \ref{parametriset käyrät})
\begin{align*}
x(t) &= v_0t-R\sin \varphi(t), \\
y(t) &= R-R\cos \varphi(t),
\end{align*}
missä
\[
R\varphi(t)=v_0t,
\]
\begin{figure}[H]
\setlength{\unitlength}{1cm}
\begin{center}
\begin{picture}(5,2)(0,1)
\put(0,0){\vector(1,0){4}} \put(3.8,-0.5){$x$}
\put(0,0){\vector(0,1){3}} \put(0.2,2.8){$y$}
\put(2,1.25){\circle{2.5}}
\dashline{0.2}(0,2.5)(4,2.5) \put(-0.5,2.4){$\scriptstyle{2R}$}
\dashline{0.1}(2,1.25)(0.8,1.6)
\put(2,1.25){\vector(1,0){1}} \put(2.6,1.4){$\scriptstyle{v_0\vec i}$}
\dashline{0.1}(2,0)(2,1.25)
\put(2,1.25){\arc{0.6}{1.59}{3.43}}
\put(1.2,0.85){$\scriptstyle{\varphi(t)}$}
\put(1.93,1.18){$\scriptstyle{\bullet}$} \put(0.73,1.53){$\scriptstyle{\bullet}$} 
\put(0.5,1.7){$\scriptstyle{P}$}
\end{picture}
\end{center}
\end{figure}
\end{multicols}
joten $P$:n paikkavektori hetkellä $t$ on
\[
\vec r\,(t)=\left[v_0t-R\sin\frac{v_0t}{R}\right]\,\vec i 
                + R\left[1-\cos\frac{v_0t}{R}\right]\,\vec j.
\]
Nopeus ja vauhti hetkellä $t$ ovat
\begin{align*}
&\vec v(t) = \dvr(t) = v_0\left[\left(1-\cos\frac{v_0t}{R}\right)\vec i 
                                   + \sin\frac{v_0t}{R}\vec j\right], \\
&v(t) = \abs{\dvr(t)} = \abs{v_0}\sqrt{2-2\cos(\frac{v_0t}{R})}
                      = 2\abs{v_0}\Bigl|\sin\dfrac{v_0t}{2R}\Bigr|.
\end{align*}
Vauhdin minimiarvo $v_{\text{min}} = 0$ saavutetaan aina kun piste $P$ koskettaa $x$-akselia ja
maksimiarvo $v_{\text{max}} = 2\abs{v_0}$ aina kun $P$ on korkeudella $2R$. \loppu
\begin{figure}[H]
\setlength{\unitlength}{1cm}
\begin{center}
\begin{picture}(8,3)(-0.5,0)
\put(0,0){\vector(1,0){7.5}} \put(7.3,-0.5){$x$}
\put(0,0){\vector(0,1){3}} \put(0.2,2.8){$y$}
\dashline{0.2}(0,2)(7.5,2) \put(-0.5,1.9){$\scriptstyle{2R}$}
\curve(
      0,         0,
    0.0206,    0.1224,
    0.1585,    0.4597,
    0.5025,    0.9293,
    1.0907,    1.4161,
    1.9015,    1.8011,
    2.8589,    1.9900,
    3.8508,    1.9365,
    4.7568,    1.6536,
    5.4775,    1.2108,
    5.9589,    0.7163,
    6.2055,    0.2913,
    6.2794,    0.0398,
    6.2849,    0.0234,
    6.3430,    0.2461,
    6.5620,    0.6534,
    7.0106,    1.1455)
\put(6.05,-0.4){$\scriptstyle{2\pi R}$}
\end{picture}
\end{center}
\end{figure}

\pagebreak

\Harj
\begin{enumerate}

\item
Määritä seuraavien tasokäyrien tangentin ja normaalin yhtälöt perusmuodossa $ax+by+c=0$
annetussa pisteessä: \newline
a) \ $y=x^3-2x^2+x+1, \quad (x,y)=(2,3)$ \newline
b) \ $y=\ln x, \quad (x,y)=(e,1)$ \newline
c) \ $x^2y^3-x^3y^2=12, \quad (x,y)=(-1,2)$ \newline
d) \ $x\sin(xy-y^2)=x^2-1, \quad (x,y)=(1,1)$ \newline
e) \ $x^y=y^x, \quad (x,y)=(2,4)$

\item
a) Määritä $a,b,c$ siten, että käyrät $y=x^2+ax+b$ ja $y=cx-x^2$ sivuavat toisiaan (eli niillä
on yhteinen tangentti) pisteessä $(1,3)$. \vspace{1mm}\newline
b) Määritä pisteen $(-4,23/2)$ kautta kulkevien käyrän $9y=x^2$ normaalien yhtälöt.
\vspace{1mm}\newline
c) Mihin käyrän $y=x^3$ pisteeseen asetettu normaali leikkaa $x$-akselin pisteessä $(4,0)$?
\vspace{1mm}\newline
d) Määritä käyrien $y=\sinh x$ ja $y=\cosh x$ pisteseen $x=a$ asetettujen tangenttien ja
normaalien leikkauspisteet. \vspace{1mm}\newline
e) Suora kulkee pisteen $(3,0)$ kautta ja leikkaa käyrän $y=e^x$ kohtisuorasti. Määritä suoran
yhtälö, tarvittaessa numeerisin apukeinoin. \vspace{1mm}\newline
f) Mikä suora leikkaa kohtisuorasti käyrät $y=e^x$ ja $y=\ln x$\,? Mikä on käyrien lyhin
etäisyys?

\item \index{logaritminen spiraali}
a) Määritä \kor{logaritmisen spiraalin} $\,r=e^\varphi,\ \varphi\in\R\,$ (napakoordinaatit)
tangentin ja normaalin yhtälöt pisteessä $(r,\varphi)=(1,0)$. \vspace{1mm}\newline
b) Näytä implisiittisellä derivoinnilla, että käyrällä $\,S:\ x^5+x^2y^3+y^5=1\,$
on ainakin yksi vaakasuora (eli $x$-akselin suuntainen) ja yksi pystysuora
($y$-akselin suuntainen) tangentti. Missä käyrän pisteissä nämä sijaitsevat?
 
\item
a) Millä ehdolla käyrät $y=e^{ax}$ ja $y=e^{bx}$ leikkaavat kohtisuorasti? Missä kulmassa
käyrät leikkaavat, jos $a=1$ ja $b=2$\,? \vspace{1mm}\newline
b) Laske käyrien $y=\Arcsin x$ ja $y=\Arccos x$ (tangenttien) välinen kulma käyrien 
leikkauspisteessä. \vspace{1mm}\newline
c) Näytä, että sikäli kuin käyrät $y=ae^x$ ja $y=\sqrt{b-2x}\ $ ($a,b\in\R$) leikkaavat, niin
ne leikkaavat kohtisuorasti. Millä ehdolla käyrät leikkaavat?

\item
Avaruussuora $S$ on avaruuskäyrän
\[
\vec r\,(t)=7\sqrt{2}\cos t\,\vec i+\sqrt{2}\sin t(3\vec i-2\vec j+6\vec k)
\]
tangentti pisteessä $(4,2,-6)$. Määritä $S$:n yhtälö parametrimuodossa.

\item
Määritä seuraavien avaruuskäyrien yksikkötangenttivektori annetussa käyrän pisteessä $P$ sekä
sen avaruustason yhtälö, jonka käyrä leikkaa kohtisuorasti $P$:ssä: \newline
a) \ $\vec r=t^3\,\vec i+(2t-t^2)\,\vec j+(3t-2t^4)\,\vec k,\quad P=(1,1,1)$ \newline
b) \ $\vec r=e^t\,\vec i-\ln(t+1)\,\vec j-\cos t\,\vec k,\quad P=(1,0,-1)$ \newline
c) \ $x=2\sqrt{2}\cos t,\ y=\sqrt{2}\sin t,\ z=4t,\quad P=(-2,1,3\pi)$

\item
Näytä, että ruuviviivan $\vec r\,(t)=\cos t\,\vec i+\sin t\,\vec j+t\,\vec k,\ t\in\R$
tangenttien ja $xy$-tason leikkauspisteet muodostavat tasokäyrän
\[
x=\cos t+t\sin t, \quad y=\sin t-t\cos t, \quad t\in\R.
\]

\item
Tasolla liikuu pistemäinen kappale siten, että hetkellä $t$ kappale on pisteessä 
$P(t)=(x(t),y(t))$, missä $x(t)=\sin f(t)$, $y(t)=1-\cos f(t)$, ja edelleen 
\[
f(t)=\begin{cases} \,t^2(3-t)^2, &\text{kun}\ t\in[0,3], \\ 0, &\text{muulloin}. \end{cases}
\] 
Millä ajan hetkillä kappaleen vauhti (=nopeuden itseisarvo) on suurimmillaan, ja millainen
geometrinen jälki kappaleen liikkeestä jää?

\item (*)
Määritä (tarvittaessa numeerisin keinoin) suoran $S$ yhtälö tiedoista: \newline
a) \ $S$ sivuaa käyriä $y=x^2$ ja $y=\ln x$. \newline
b) \ $S$ leikkaa kohtisuorasti käyrät $y=x^2$ ja $y=\ln x$.

\item (*)
Avaruuskäyrä
\[
S: \quad \begin{cases} \,x^3+2y^3=3x^2yz, \\ \,y^3+z^3=2xy \end{cases}
\]
leikkaa kohtisuorasti avaruustason $T$ pistessä $(1,1,1)$. Määritä $T$:n yhtälö
perusmuodossa $ax+by+cz+d=0$.

\item (*) \index{zzb@\nim!Palloaberraatio}
(Palloaberraatio) Avaruudessa $xy$-tasoa pitkin negatiivisen $y$-akselin suuntaan etenevä
valonsäde heijastuu pisteessä $P$ pallopeilistä 
\[
S:\quad x^2+(y-R)^2+z^2=R^2, \quad 0 \le y \le R, 
\]
jolloin heijastunut säde leikkaa $y$-akselin pisteessä $y=c$. Laske $c$:n lauseke 
$P$:n $x$-koordinaatin funktiona. Totea, että pienillä suhteen $|x|/R$ arvoilla on likimain
$c(x) \approx R/2$ ja että poikkeama (nk.\ palloaberraatio) on likimain
\[
c(x)-\frac{R}{2} \,\approx\, -\frac{3R}{4}\left(\frac{x}{R}\right)^2.
\]

\item (*) \label{H-dif-2: tutka} \index{zzb@\nim!Tutka}
(Tutka) Origosta lähtevät radioaallot  heijastuvat pyörähdyspinnasta, jonka symmetria-akseli on
$y$-kaseli ja profiili $xy$-tasossa on käyrä $y=y(x)$, $x\in \R$. Profiili on valittu siten,
että  heijastuneet aallot kulkevat positiivisen $y$-akselin suuntaan heijastuspisteestä
riippumatta. Näytä, että tämä ehto voidaan esittää differentiaaliyhtälönä 
\[
(xy'-y)^2=x^2+y^2.
\]
Etsi mahdolliset polynomiratkaisut muotoa $y(x)=ax^2+b$.

\item (*) \index{zzb@\nim!Polttolasi}
(Polttolasi) Lasista on valmistettu linssi, jonka optinen akseli (symmetria-akseli) on
$y$-akseli ja profiili $xy$-tasossa on
\[
A=\{(x,y)\in\Rkaksi \mid x\in[-a,a]\ \ja\ f(x) \le y \le b\},
\]
\begin{multicols}{2} \raggedcolumns
missä $f$ on parillinen funktio ja $f(0)=0$. Käyrän $y=f(x)$ muoto halutaan sellaiseksi, että
linssi kokoaa kaikki negatiivisen $y$-akselin suuntaan kulkevat, linssin läpäisseet valonsäteet
pisteeseen $(0,-c)$. Taittumislaki linssin kaarevalla pinnalla on $k\sin\alpha_1=\sin\alpha_2$,
missä $\alpha_1$ on säteen tulokulma (lasissa) pinnan normaalin suhteen, $\alpha_2$ on 
lähtökulma, ja $k>1$ on lasin ja ilman välinen taitekerroin. 
\begin{figure}[H]
\setlength{\unitlength}{1cm}
\begin{picture}(5,5.5)(0,0.5)
\thicklines
\put(3,6){\arc{6}{0.785}{2.356}}
\path(0.879,3.879)(5.121,3.879)
\thinlines
\path(4,5.5)(4,3.172)(3,1)
\path(3.333,5.057)(4.667,1.286)
\put(4,3.172){\arc{1}{1.231}{2.002}} \put(4,3.172){\arc{2.1}{-1.911}{-1.571}}
\put(3.6,4.5){$\alpha_1$} \put(3.75,2.2){$\alpha_2$} 
\put(2.93,0.93){$\scriptstyle{\bullet}$} \put(2.3,0.9){$-c$}
\put(3,3){\vector(1,0){2.5}} \put(3,0.5){\vector(0,1){5}}
\put(5.7,2.9){$x$} \put(3.2,5.5){$y$}
\end{picture}
\end{figure}
\end{multicols}
Näytä, että funktion $y=f(x)$ on toteutettava differentiaaliyhtälö
\[
\left[k^2x^2+(k^2-1)(y+c)^2\right](y')^2-2x(y+c)y'=x^2.
\]

\end{enumerate}