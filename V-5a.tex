%Esimerkit ja harjoitustehtävät vanhasta luvusta 'Kaarenpituus: $\sin$ ja $\cos$'
\begin{Exa} Missä pisteissä funktio $f(x)=2\abs{\sin x}-\cos x)$ saavuttaa pienimmän ja
suurimman arvonsa välillä $[0,2\pi]$? 
\end{Exa}
\ratk Funktio on derivoituva väleillä $(0,\pi)$ ja $(\pi,2\pi)$, joilla $\sin x$ ei vaihda 
merkkiään:
\[
f'(x) = \begin{cases}
        \dif(2\sin x-\cos x) = 2\cos x + \sin x,    \quad &x \in (0,\pi), \\
        \dif(-2\sin x - \cos x = -2\cos x + \sin x, \quad &x \in (\pi,2\pi).
        \end{cases}
\]
Derivaatan nollakohdat ovat pisteissä, joissa 
\begin{align*}
x \in (0,\pi)\ \ja\ 2\cos x + \sin x = 0    
                        &\qimpl x=x_1 \in (\pi/2,\pi)\ \ja\ \tan x_1 = - 2 \\
                        &\qimpl \sin x_1 = 2/\sqrt{5}, \quad \cos x_1 = - 1/\sqrt{5} \\
                        &\qimpl f(x_1) = \sqrt{5}, \\
x \in (\pi,2\pi)\ \ja\ -2\cos x + \sin x = 0 
                        &\qimpl x = x_2 \in (\pi,3\pi/2)\ \ja \tan x_2 = 2 \\
                        &\qimpl \sin x_2 = - 2/\sqrt{5}, \quad \cos x_2 = - 1/\sqrt{5} \\
                        &\qimpl f(x_2) = \sqrt{5}. 
\end{align*}
Vertailemalla arvoihin $f(0)=f(2\pi)=-1$ ja $f(\pi)=1$ todetaan, että $f$ saavuttaa välillä 
$[0,2\pi]$ maksimiarvonsa $f_{max}=\sqrt{5}$ pisteissä $x_1 = \Arctan(-2)+\pi = 2.03444393..$ ja
$x_2=\Arctan 2+\pi = 4.24874137..$ ja minimiarvonsa $f_{min}=-1$ välin päätepisteissä. \loppu
\begin{Exa} Näytä, että $f(x)=x-\sin x$ on aidosti kasvava $\R$:ssä. \end{Exa}
\ratk Koska $f'(x)=1-\cos x \ge 0\ \forall x$, Lauseen \ref{monotonisuuskriteeri} mukaan $f$ on
jokaisella välillä $[a,b]$ kasvava. Saman lauseen mukaan $f$ on jokaisella välillä $[a,b]$ myös
aidosti kasvava, koska joukko
\[
X = [a,b] \cap \{x\in\R \mid f'(x)=0\} = [a,b] \cap \{k \cdot 2\pi,\ k \in \Z\}
\]
on äärellinen. Siis $f$ on aidosti kasvava $\R$:ssä. \loppu

\Harj
\begin{enumerate}

\item \label{H-V-5: numeerinen kaarenpituus}
Tutki, kuinka tarkasti kaaren
\[
S=\{P=(x,y) \mid y=x^2\ \ja\ x\in[0,1]\}
\]
pituus saadaan lasketuksi käyttämällä pisteistöä $\{(x_i,f(x_i)),\ i=0 \ldots 10\}$, missä 
$x_i=i/10$. Tarkka arvo on $s=1.47894..$

\item
Origosta siirrytään käyrää $r=4\cos\varphi$ (napakoordinaatit) seuraten pisteeseen, jonka
$x$-koordinaatti $=3$. Mikä on matkan pituus lyhintä reittiä?

\item
Näytä, että jos $a,b\in\R$ ja $b \neq 0$, niin 
$\,\displaystyle{\lim_{x \kohti 0}\frac{\sin ax}{\sin bx}=\frac{a}{b}}\,$.

\item
Määritä raja-arvot
\begin{align*}
&\text{a)}\ \lim_{x \kohti 0} \frac{x\sin x}{1-\cos x} \qquad
 \text{b)}\ \lim_{x \kohti 0} \frac{x\sin 2x}{1-\cos 2x} \qquad
 \text{c)}\ \lim_{x \kohti 0^+} \frac{x-\sqrt{x}}{\sqrt{\sin x}} \\
&\text{d)}\ \lim_{x\kohti\frac{\pi}{2}}(1+\cos 2x)\tan^2 x \qquad
 \text{e)}\ \lim_{x\kohti\infty} x\sin\frac{1}{x}
\end{align*}

\item
Näytä induktiolla, että $\dif^n\tan x=p(\tan x)$, missä $p$ on polynomi astetta $n+1$.

\item
Johda derivoimissääntöjen ja Lauseen \ref{yksinkertaisin dy} avulla kaava
\[
\Arctan x + \Arccot x=\frac{\pi}{2}\,, \quad x\in\R.
\]

\item
Totea funktio
\[
f(x)=\begin{cases} 
     x^2\sin\dfrac{1}{x^2}\,, &\text{kun}\ x \neq 0 \\ 0, &\text{kun}\ x=0
     \end{cases} \]
esimerkiksi funktiosta, joka on derivoituva jokaisessa pisteessä $x\in\R$, mutta derivaatta
ei ole jokaisessa pisteessä jatkuva.

\item
Millä väleillä funktio $f(x)=x+4\sin x$ on aidosti kasvava ja millä aidosti vähenevä?
Hahmottele funktion kuvaaja.

\item
Määritä seuraavien implisiittifunktioiden $y(x)$ derivaatat annetuissa pisteissä:
\begin{align*}
&\text{a)}\ y+2\sin y+\cos y=x,\ \ (x,y)=(1,0) \\
&\text{b)}\ 2x+y-\sqrt{2}\sin(xy)=\frac{\pi}{2}\,,\ \ (x,y)=\left(\frac{\pi}{4}\,,1\right) \\
&\text{c)}\ x\sin(xy-y^2)=x^2-1,\ \ (x,y)=(1,1) \\
&\text{d)}\ \tan(xy^2)=\frac{2xy}{\pi}\,,\ \ (x,y)=\left(-\pi,\frac{1}{2}\right)
\end{align*}

\item
Määritä napa- tai pallokoordinaatistoon siirtymällä seuraavien funktioiden maksimi- ja 
minimiarvot annetussa joukossa $A$. Anna myös karteesisessa koordinaatistossa pisteet, 
joissa nämä arvot saavutetaan.
\begin{align*}
&\text{a)}\ f(x,y)=x^3y^4,\ \ A=\{(x,y) \mid x^2+y^2 \le 1\} \\
%&\text{b)}\ f(x,y)=x^2y^4,\ \ A=\{(x,y) \mid x^2+y^2 \le 25\} \\
&\text{b)}\ f(x,y,z)=xyz^3,\ \ A=\{(x,y,z) \mid x^2+y^2+z^2 \le 1\}
\end{align*}

\item
Määritä funktion $f(x)=4\abs{\cos x}-3\sin x+2\cos x\,$ derivaatan epäjatkuvuuskohdat, 
paikalliset ääriarvokohdat ja absoluuttiset maksimi- ja minimiarvot välillä $[0,2\pi]$. 
Hahmottele käyrän $y=f(x)$ kulku.

\item
Ratkaise alkuarvotehtävä $\,y(0)=0,\ y'=\sin x+\cos x,\ x\in\R$.

\item (*)
Näytä sopivalla muuttujan vaihdolla, että \newline
$\D
\text{a)}\ \ \lim_{x\kohti\infty} x\left(\frac{\pi}{2}-\Arctan x\right)=1, \qquad 
\text{b)}\ \ \lim_{x \kohti 1^-} \frac{\pi-2\Arcsin x}{\sqrt{1-x}}=2\sqrt{2}$.

\item (*)
Näytä derivoimissääntöjen ja Lauseen \ref{toiseksi yksinkertaisin dy} avulla, että 
pätee:
\[
\Arctan x=\sum_{k=0}^\infty \frac{(-1)^{k}}{2k+1}\,x^{2k+1},\ \ x\in(-1,1).
\]

\item(*)
Kaarenpituuden määritelmän mukaisesti saadaan yksikköympyrän kaarenpituudelle alalikiarvo
laskemalla sellaisen $n$-kulmion piiri, jonka kärjet ovat yksikköympyrällä. Olkoon $n$-kulmio
säännöllinen ja piiri $=s_n$. Näytä, että kun valitaan $n=2^{k+1},\ k=1,2, \ldots\,$, niin
luvut 
$a_k=s_n/2,\ n=2^{k+1}$ saadaan lasketuksi palautuvana lukujonona
\[
a_0=2, \quad a_{k+1}=\frac{2a_k}{\sqrt{2+\sqrt{4-(2^{-k}a_k)^2}}}\,, \quad k=0,1,\ldots
\]
Laske $a_k$ kun $k=1 \ldots 10$ ja vertaa lukuun $\pi$.
 
\end{enumerate}