\section{Yhden muuttujan funktiot} \label{yhden muuttujan funktiot}
\alku
\index{funktio A!b@reaalifunktio|vahv}
\index{yhden muuttujan funktio|vahv}

Yhden (reaali)muuttujan (reaaliarvoisella) funktiolla eli reaalifunktiolla tarkoitetaan
funktiota tyyppiä
\[
f:\DF_f \kohti \R, \quad \DF_f \subset \R.
\]
\index{mzyzy@määrittelyjoukko}%
Tässä $\DF_f$ on $f$:n \kor{määrittelyjoukko} (lähtöjoukko, engl. domain) ja joukko
\[
\RF_f=\{y\in\R \mid y=f(x)\,\ \text{jollakin}\ x\in\DF_f\}
\]
\index{arvojoukko}%
on $f$:n \kor{arvojoukko} (engl.\ range eli 'kantama').\footnote[2]{Suomenkielisissä
teksteissä määrittely- ja arvojoukoille käytetään myös merkintöjä $M_f,A_f$.} Funktion $f$ 
\index{kuvaaja}%
\kor{kuvaaja} (engl.\ graph) joukossa $A \subset \DF_f$ on euklidisen tason (yleensä
karteesisen) koordinaatiston avulla määritelty pistejoukko
\[
G_{f,A}=\{\,P \vastaa (x,y) \in \R^2 \mid x \in A\,\ja\, y=f(x)\,\} \subset \Ekaksi.
\]
Kuvaajan tarkoituksena on 'geometrisoida' funktio niin, että näköhavainnot tulevat 
mahdollisiksi.\footnote[3]{Kuvaaja on geometrinen vastine reaalifunktion joukko-opilliselle
määritelmälle $\Rkaksi$:n osa\-joukkona, ks.\ Luku \ref{trigonometriset funktiot}.}
Esimerkiksi arvojoukko $\RF_f$ on kuvaajasta helppo hahmottaa, ja sellainen
usein luontevalta tuntuva sanonta kuin '$f$ pisteessä $x$' ($=f(x)$) sisältää myös 
geometrisointiajatuksen ($x \vastaa P \in E^1$, vrt.\ Luku \ref{tasonvektorit}).
\setlength{\unitlength}{1mm}
\begin{figure}[H]
\begin{center}
\begin{picture}(130,70)(-50,-20)
\put(0,-20){\vector(0,1){70}} \put(3,47){$y$}
\put(-50,0){\vector(1,0){130}} \put(77,-5){$x$}
\spline(-40,10)(-20,30)(0,20)(20,-20)(40,30)(50,39.4)(60,30)
\put(41,30){$\bullet$} \put(-1,30){$\bullet$} \dashline{2}(-1,31)(41,31) 
\put(41,-1){$\bullet$} \dashline{2}(42,31)(42,0)
\dashline{2}(-40,10)(-40,0) \dashline{2}(60,30)(60,0) \dashline{2}(50,37)(0,37) 
\dashline{2}(20,-9)(0,-9)
\put(8,22){\vector(-1,1){7}}\put(10,21){$f$ pisteessä $x$}
\put(50,-9){\vector(-1,1){7}}\put(52,-10){piste $x$}
\put(-40,-12.2){$\underbrace{\hspace{100mm}}_{\D \DF_f}$}
\put(-12,13){$\RF_f \left\{ \begin{array}{c} \vspace{40mm} \end{array} \right.$}
\end{picture}
%\caption{Funktion 'geometrisointi'}
\end{center}
\end{figure}

Reaalimuuttujan funktiota $f$ tutkitaan useimmiten jollakin avoimella, suljetulla tai 
puoliavoimella välillä, jolle $f$:n määrittelyjoukko voidaan haluttaessa ajatella rajatuksi
ko.\ tarkastelussa. Kerrattakoon Luvusta \ref{reaaliluvut} merkinnät
\begin{align*}
\text{avoin väli:} \quad       &(a,b)\ =\ \{x\in\R \mid a<x<b\}, \\
\text{suljettu väli:} \quad    &[a,b]\,\ =\ \{x\in\R \mid a\leq x\leq b\}, \\
\text{puoliavoin väli:} \quad  &(a,b]\,\ \text{tai}\,\ [a,b)
\intertext{ja myös yleisessä käytössä olevat 'äärettömän välin' merkinnät}
              (0,\infty)\ =\,\ &\{x\in\R \mid x>0\} = \R_+, \\
             (-\infty,0)\ =\,\ &\{x\in\R \mid x<0\} = \R_-, \\
        (-\infty,\infty)\ =\,\ &\R.
\end{align*}
Usein yhden muuttujan reaalifunktioista ilmoitetaan vain laskusääntö (liittämissääntö)
muodossa 'funktio $f(x)$'. Tällöin oletetaan (ellei toisin mainita), että määrittelyjoukko
$\DF_f$ on suurin mahdollinen ts.
\[
\DF_f=\{x\in\R \ | \ y=f(x) \text{ määritelty yksikäsitteisesti ja } y\in\R\}.
\]
\begin{Exa} \label{reaalifunktioita} \index{funktio B!d@paloittain määritelty}
\index{paloittainen!a@funktion määrittely}
\begin{align*}
&\text{a)}\ \ f(x)=x^4+x^2+1 \qquad 
 \text{b)}\ \ f(x)=x^2/(x-1)^2 \qquad 
 \text{c)}\ \ f(x)=\cot x \\
&\text{d)}\ \ f(x)=x^{3/4} \quad\
 \text{e)}\ \ f(x) = \begin{cases} \,\cos x,     &\text{kun}\ x<0 \\ 
                                   \,x-x^2,\,\   &\text{kun}\ 0 \le x \le 1 \\
                                   \,2-\sqrt{x}, &\text{kun}\ x>1
                      \end{cases} \quad\
 \text{f)}\ \ f(x) = \sum_{k=1}^\infty \frac{x^k}{k^2}
\end{align*}
\begin{itemize}
\item[a)] Polynomi: $\ \DF_f=\R$, $\,\RF_f=[1,\infty)$.
\item[b)] \kor{Rationaalifunktio}: $\ \DF_f=\{x\in\R \mid x \neq 1\}$, $\,\RF_f=[0,\infty)$.
\item[c)] Trigonometrinen funktio: 
           $\ \DF_f=\{x\in\R \mid x/\pi\not\in\Z\}$, $\,\RF_f=\R$.
\item[d)] \kor{Potenssifunktio}: $\ \DF_f=\RF_f=[0,\infty)$.
\item[e)] \kor{Paloittain määritelty} funktio: $\ \DF_f = \R$, $\,\RF_f = (-\infty,1]$.
\item[f)] Potenssisarja: Määrittelyjoukko on $\,\DF_f = [-1,1]$
          (vrt.\ Luku \ref{potenssisarja}). Arvojoukko on vaikeampi määrittää, mutta
          osoittautuu: $\,\RF_f = [-\tfrac{\pi^2}{12},\tfrac{\pi^2}{6}]$. (Sivuutetaan
          perustelut.) \loppu
\end{itemize}
\end{Exa}
\index{rationaalifunktio} \index{potenssifunktio}%
Esimerkin b-kohdan rationaalifunktion yleisempi muoto on $f(x)=p(x)/q(x)$, missä $p$ ja $q$
ovat (reaalikertoimisia) polynomeja. Määrittelyjoukko on tällöin
$\DF_f=\{x\in\R \mid q(x) \neq 0\}$. Potenssifunktion (esimerkin d-kohta) yleinen muoto on
$f(x)=x^\alpha$, missä (toistaiseksi) $\alpha\in\Q$. Määrittelyjoukko on joko
$\DF_f=[0,\infty)$ ($\alpha>0,\ \alpha\not\in\N$), 
$\DF_f=(0,\infty)=\R_+$ ($\alpha<0,\ \alpha\not\in\Z$),
$\DF_f=\{x\in\R\ |\ x \neq 0\}$ ($\alpha\in\Z,\ \alpha \le 0$) tai
$\DF_f=\R$ ($\alpha\in\N$). 
\begin{Exa} \label{erään polynomin arvojoukko} Määritä funktion $f(x) = x^2-7x+11$ arvojoukko
välillä $[1,4]$. \end{Exa}
\ratk Tehtävän asettelun mukaisesti rajataan funktion määrittelyjoukko tässä väliksi
$A = [1,4]$, jolloin arvojoukolle luonteva merkintä on $f(A)$. Koska
\[ 
f(x) = \left(x - \dfrac{7}{2}\right)^2 - \dfrac{49}{4} + 11 
                                       = \left(x - \dfrac{7}{2}\right)^2 - \dfrac{5}{4}\,, 
\]
niin $f(A)$:n minimi on
\[ 
f_{\text{min}} = \min\,\{f(x) \mid x \in [1,4]\,\} = f(7/2) = -5/4, 
\]
ja $f(A)$:n maksimi saavutetaan mahdollisimman etäällä pisteestä $x = 7/2$, eli pisteessä 
$x=1$\,:
\[ 
f_{\text{max}} = \max\,\{f(x) \mid x \in [1,4]\,\} = f(1) = 5. 
\]
Tähän asti on päätelty: $y \in f(A)\,\impl\,-5/4 \le y \le 5$, eli $f(A) \subset [-5/4,5]$. 
Toisaalta jos $y \in [-5/4,5]$, niin yhtälöllä $f(x)=y$ on ratkaisu
\[
x = \dfrac{7}{2} - \sqrt{y + \dfrac{5}{4}}\,,
\]
joka on välillä $[1,4]$. Siis pätee $y \in [-5/4,5]\,\impl\,y \in f(A)$, eli 
$[-5/4,5] \subset f(A)$. Koska on sekä $f(A) \subset [-5/4,5]\,$ että 
$\,[-5/4,5] \subset f(A)$, niin on 
\[
f(A) = [-5/4,5]. \loppu 
\]
Esimerkissä on kyse tyypillisestä 'funktiotutkimuksesta', jossa on annettu joukko
$A \subset \DF_f$ (reaalifunktion tapauksessa usein väli) ja on määrättävä $B=f(A)$ eli
funktion arvojoukko, kun määrittelyjoukko rajataan $A$:ksi. Ongelma voi olla asetettu myös
käänteisesti niin, että tunnetaan joukko $B$, ja on määrättävä joukko 
$A = \{x \in \DF_f \mid f(x) \in B\}$. Esimerkiksi funktion nollakohtia määrättäessä on 
ratkaistava käänteinen ongelma, kun $B=\{0\}$, eli on \pain{ratkaistava} y\pain{htälö} 
$f(x)=0$. Jos $B=[0,\infty)$, niin on \pain{ratkaistava} \pain{e}p\pain{ä}y\pain{htälö}
$f(x) \ge 0$. 
\jatko \begin{Exa} (jatko) Jos esimerkissä asetetaan $\DF_f=\R$ ja $B=\{0,1\}$, niin
\begin{align*}
A &= \{x\in\R \mid f(x) \in B\} \\
  &= \{x\in\R \mid x^2-7x+11=0\,\tai\,x^2-7x+11=1\} \\
  &= \{2,\,\tfrac{1}{2}(7-\sqrt{5}),\,\tfrac{1}{2}(7+\sqrt{5}),\,5\}. \loppu
\end{align*}
\end{Exa}

\subsection*{Monotoniset funktiot}
\index{funktio B!a@monotoninen|vahv}

Yhtälöitä ja epäyhtälöitä ratkaistaessa, ja muutenkin funktioita tutkittaessa, on käytännössä 
suurta hyötyä seuraavasta yhden muuttujan funktioille 'luonnetta' antavasta määritelmästä
(vrt.\ Määritelmä \ref{monotoninen jono} lukujonoille).
\begin{Def} \label{monotoninen funktio} \index{monotoninen!b@reaalifunktio|emph}
\index{aidosti kasvava, vähenevä, monotoninen!b@reaalifunktio|emph}
Yhden reaalimuuttujan funktio on \kor{kasvava} (engl.\ increasing) \kor{välillä}
$A \subset \DF_f$, jos
\[
\forall x_1,x_2 \in A\,\bigl[\,x_1 < x_2 \ \impl \ f(x_1) \leq f(x_2)\,\bigr]
\]
ja \kor{aidosti} (engl. strictly) \kor{kasvava}, jos
\[
\forall x_1,x_2 \in A\,\bigl[\,x_1 < x_2 \ \impl \ f(x_1) < f(x_2)\,\bigr].
\]
Vastaavasti $f$ on \kor{vähenevä} (engl.\ decreasing) \kor{välillä} $A \subset \DF_f$, jos
\[
\forall x_1,x_2 \in A\,\bigl[\,x_1 < x_2 \ \impl \ f(x_1) \geq f(x_2)\,\bigr]
\]
ja \kor{aidosti vähenevä}, jos
\[
\forall x_1,x_2 \in A\,\bigl[\,x_1 < x_2 \ \impl \ f(x_1) > f(x_2)\,\bigr].
\]
Jos $f$ on välillä A jompaa kumpaa tyyppiä, niin sanotaan, että $f$ on ko.\ välillä (aidosti) 
\kor{monotoninen}. \end{Def}
\begin{figure}[H]
\setlength{\unitlength}{1mm}
\begin{center}
\begin{picture}(140,45)(0,10)
\put(20,20){\vector(1,0){40}} \put(58,16){$x$}
\put(20,20){\vector(0,1){30}} \put(22,48){$y$}
\put(80,20){\vector(1,0){40}} \put(118,16){$x$}
\put(80,20){\vector(0,1){30}} \put(82,48){$y$} 
\curve(20,30,25,32,30,38)
\drawline(30,38)(40,38)
\drawline(40,38)(45,45)
\curve(80,45,90,32,105,25)
\put(95,32){$y=f(x)$}
\put(35,32){$y=f(x)$}
\put(30,10){$f$ kasvava}
\put(83,10){$f$ aidosti vähenevä}
\end{picture}
%\caption{Esimerkkejä}
\end{center}
\begin{Exa} Identiteetistä
\[
\sqrt{x_1}-\sqrt{x_2}=\frac{x_1-x_2}{\sqrt{x_1}+\sqrt{x_2}}\,, 
                               \quad x_1,x_2 \ge 0, \ x_1 \neq x_2
\]
nähdään, että funktio $f(x)=\sqrt{x}$ on välillä $[0,\infty)$ (eli määrittelyjoukossaan)
aidosti kasvava. \loppu
\end{Exa}
\end{figure}
\begin{multicols}{2} \raggedcolumns
Jos $f$ on välillä $A\subset\DF_f$ aidosti kasvava ja $c\in A$, niin epäyhtälöllä
\[
f(x) \leq c
\]
on välille $A$ rajattuna helppo ratkaisu:
\[
x\in A \cap (-\infty,a],\ \text{ missä } f(a)=c.
\]
\begin{figure}[H]
\setlength{\unitlength}{1mm}
\begin{center}
\begin{picture}(40,35)(-5,-5)
\put(-5,0){\vector(1,0){40}} \put(33,-4){$x$}
\put(0,-5){\vector(0,1){30}} \put(2,23){$y$}
\curve(0,2,15,10,30,25)
\drawline(15,10)(15,0)
\drawline(15,10)(0,10)
\put(14,-4){$a$} \put(-4,9){$c$}
\dottedline[$\shortmid$]{1}(1,0.2)(15,0.2)
\end{picture}
%\caption{$f$ aidosti kasvava}
\end{center}
\end{figure}
\end{multicols}
\jatko\begin{Exa} (jatko) Jos $a>0$, niin epäyhtälön $\sqrt{x} \le a$ ratkaisu on
\[
\{x\in\R \mid \sqrt{x} \le a\} \,=\, \{x\in[0,\infty) \mid x \le a^2\} 
                               \,=\, [0,a^2\,]. \loppu
\]
\end{Exa}
Jos funktio ei ole koko tarkasteltavalla välillä monotoninen, on funktiotutkimuksen
ensimmäinen askel usein välin jakaminen sellaisiin osaväleihin, joilla monotonisuus toteutuu.
\begin{Exa}
$f(x)=\sqrt{\abs{x}} \quad (\DF_f=\R)$ on aidosti vähenevä välillä $(-\infty,0]$ ja aidosti 
kasvava välillä $[0,\infty)$. \loppu
\end{Exa}
\begin{Exa} Jos $f(x) = -x^2+6x+1$, niin kirjoittamalla
\[ f(x) = -(x-3)^2 + 10 \]
nähdään, että $f$ on aidosti kasvava välillä $(-\infty,3]$ ja aidosti vähenevä välillä 
$[3,\infty)$. \loppu \end{Exa} 
\begin{Exa}
Trigonometristen funktioiden määritelmien (Luku \ref{trigonometriset funktiot}) perusteella 
seuraavat väittämät ovat uskottavia (myös tosia).
\begin{itemize}
\item[$\sin x$:] Aidosti monotoninen väleillä 
                 $[(k-\frac{1}{2})\pi,(k+\frac{1}{2})\pi],\ k\in\Z$\,: kasvava kun $k$ on
                 parillinen ja vähenevä kun $k$ on pariton.
\item[$\cos x$:] Aidosti monotoninen väleillä $[k\pi,(k+1)\pi],\ k\in\Z$\,: vähenevä kun $k$
                 on parillinen ja kasvava kun $k$ on pariton.
\end{itemize}
\begin{figure}[H]
\setlength{\unitlength}{1cm}
\begin{picture}(14,4)(-2,-2)
\put(-2,0){\vector(1,0){14}} \put(11.8,-0.4){$x$}
\put(0,-2){\vector(0,1){4}} \put(0.2,1.8){$y$}
%\linethickness{0.5mm}
\multiput(3.14,0)(3.14,0){3}{\drawline(0,-0.1)(0,0.1)}
\put(0.1,-0.4){$0$} \put(3.05,-0.4){$\pi$} \put(6.10,-0.4){$2\pi$} \put(9.20,-0.4){$3\pi$}
\curve(
   -1.5708,   -1.0000,
   -1.0708,   -0.8776,
   -0.5708,   -0.5403,
   -0.0708,   -0.0707,
    0.4292,    0.4161,
    0.9292,    0.8011,
    1.4292,    0.9900,
    1.9292,    0.9365,
    2.4292,    0.6536,
    2.9292,    0.2108,
    3.4292,   -0.2837,
    3.9292,   -0.7087,
    4.4292,   -0.9602,
    4.9292,   -0.9766,
    5.4292,   -0.7539,
    5.9292,   -0.3466,
    6.4292,    0.1455,
    6.9292,    0.6020,
    7.4292,    0.9111,
    7.9292,    0.9972,
    8.4292,    0.8391,
    8.9292,    0.4755,
    9.4292,   -0.0044,
    9.9292,   -0.4833,
   10.4292,   -0.8439,
   10.9292,   -0.9978)
\curvedashes[1mm]{0,1,2}
\curve(
   -1.5708,    0.0000,
   -1.0708,    0.4794,
   -0.5708,    0.8415,
   -0.0708,    0.9975,
    0.4292,    0.9093,
    0.9292,    0.5985,
    1.4292,    0.1411,
    1.9292,   -0.3508,
    2.4292,   -0.7568,
    2.9292,   -0.9775,
    3.4292,   -0.9589,
    3.9292,   -0.7055,
    4.4292,   -0.2794,
    4.9292,    0.2151,
    5.4292,    0.6570,
    5.9292,    0.9380,
    6.4292,    0.9894,
    6.9292,    0.7985,
    7.4292,    0.4121,
    7.9292,   -0.0752,
    8.4292,   -0.5440,
    8.9292,   -0.8797,
    9.4292,   -1.0000,
    9.9292,   -0.8755,
   10.4292,   -0.5366,
   10.9292,   -0.0663)
\put(1,1.2){$y=\sin x$}
\put(5.5,1.2){$y=\cos x$}
\end{picture} 
\end{figure}
\begin{itemize}
\item[$\tan x$:] Aidosti kasvava väleillä
                 $((k-\frac{1}{2})\pi,(k+\frac{1}{2})\pi), \ k \in \Z$.
\item[$\cot x$:] Aidosti vähenevä väleillä $((k\pi,(k+1)\pi), \ k \in \Z$. \loppu
\end{itemize}
\begin{figure}[H]
\setlength{\unitlength}{1cm}
\begin{picture}(14,4)(-2,-2)
\put(-2,0){\vector(1,0){14}} \put(11.8,-0.4){$x$}
\put(0,-2){\vector(0,1){4}} \put(0.2,1.8){$y$}
\multiput(3.14,0)(3.14,0){3}{\drawline(0,-0.1)(0,0.1)}
\put(0.1,-0.4){$0$} \put(3.05,-0.4){$\pi$} \put(6.10,-0.4){$2\pi$} \put(9.20,-0.4){$3\pi$}
\multiput(0,0)(3.14,0){4}{
\curve(
   -1.1,-2,     
   -1.0708,   -1.8305,
   -0.9708,   -1.4617,
   -0.8708,   -1.1872,
   -0.7708,   -0.9712,
   -0.6708,   -0.7936,
   -0.5708,   -0.6421,
   -0.4708,   -0.5090,
   -0.3708,   -0.3888,
   -0.2708,   -0.2776,
   -0.1708,   -0.1725,
   -0.0708,   -0.0709,
    0.0292,    0.0292,
    0.1292,    0.1299,
    0.2292,    0.2333,
    0.3292,    0.3416,
    0.4292,    0.4577,
    0.5292,    0.5848,
    0.6292,    0.7279,
    0.7292,    0.8935,
    0.8292,    1.0917,
    0.9292,    1.3386,
    1.0292,    1.6622,
        1.1,2)}
\curvedashes[1mm]{0,1,2}
\multiput(0,0)(3.14,0){4}{
\curve(
    0.5000,   1.8305,
    0.6000,    1.4617,
    0.7000,    1.1872,
    0.8000,    0.9712,
    0.9000,    0.7936,
    1.0000,    0.6421,
    1.1000,   0.5090,
    1.2000,   0.3888,
    1.3000,    0.2776,
    1.4000,   0.1725,
    1.5000,    0.0709,
    1.6000,  -0.0292,
    1.7000,  -0.1299,
    1.8000,   -0.2333,
    1.9000,   -0.3416,
    2.0000,   -0.4577,
    2.1000,   -0.5848,
    2.2000,   -0.7279,
    2.3000,   -0.8935,
    2.4000,   -1.0917,
    2.5000,   -1.3386,
    2.6000,   -1.6622,
    2.7000,   -2.1154)}
\put(-2.2,-0.4){$y=\tan x$}
\put(1.3,0.5){$y=\cot x$}
\end{picture}
%\caption{Trigonometriset funktiot}
\end{figure}
\end{Exa}

\subsection*{Yhdistetty funktio}
\index{funktio B!e@yhdistetty|vahv}
\index{yhdistetty funktio|vahv}

Kahden funktion $f,g$ \kor{yhdistetty} (engl. composite) \kor{funktio} merkitään
$f \circ g$ ja määritellään laskusäännöllä
\[
(f \circ g)(x)=f(g(x)).
\]
Tavallinen ääntämistapa on hieman arkinen 'f pallo g'. 
\begin{Exa}
Palautuva lukujono muotoa
\[
x_0 \in \R, \quad x_n=f(x_{n-1}), \quad n=1,2,\ldots
\]
voidaan tulkita 'sisäkkäisten', ts. yhdistettyjen funktioiden avulla:
\begin{align*}
x_1 &= f(x_0), \\ 
x_2 &= f(x_1) = f(f(x_0))=(f \circ f)(x_0), \\
x_3 &= f(f(f(x_0)))=(f \circ f \circ f)(x_0), \quad \text{jne.} \loppu
\end{align*}
\end{Exa}
%Jos esimerkiksi $f(x)=1/(1+x)$, niin syntyy nk. \kor{ketjumurtoluku}
%\[
%\left.\begin{aligned}
%x_n=\cfrac{1}{1+
%     \cfrac{1}{1+
%      \cfrac{1}{1+\dotsb}
%        }}& \\
%        & \cfrac{\ddots \quad }{1+x_0}
%\end{aligned} \quad \right\} n \text{ kpl} \quad\loppu
%\]
Yhdistetyn funktion määrittelyjoukko on aina rajattava niin, että funktiot eivät 'riitele'.
Tällöin suurimmaksi mahdolliseksi määrittelyjoukoksi (joka yleensä oletetaan, ellei toisin 
mainita) tulee
\[
\DF_{f\circ g}=\{x \in \DF_g \ | \ g(x) \in \DF_f\} \subset \DF_g.
\]
Jos näin määritelty joukko $\DF_{f \circ g}$ on tyhjä, ei yhdistettyä funktiota voi määritellä.
\begin{Exa} Määrittele yhdistetyt funktiot $f \circ g$ ja $g \circ f$, kun 
$f(x)=\sqrt{a-x},\quad$ $g(x)=\sqrt{x-b}\ (a,b\in\R)$. \end{Exa}
\ratk Laskusäännöt ovat
\[
(f\circ g)(x) = \sqrt{a-\sqrt{x-b}}, \quad (g\circ f)(x) = \sqrt{\sqrt{a-x}-b}.
\]
Koska $\DF_f=(-\infty,a]$, $\DF_g=[b,\infty)$, on
\[
\DF_{\,f \circ g}=\{x \in [b,\infty) \mid \sqrt{x-b} \in (-\infty,a]\,\}.
\]
Jos $a<0$, niin $\DF_{f\circ g}=\emptyset$, muuten
\[
\DF_{f\circ g}\ =\ \{\,x\in\R \mid x \geq b \ \ja \ x-b \leq a^2\,\}\ 
                =\ [b,b+a^2] \quad (a \geq 0).
\]
Siis
\[
\DF_{f\circ g}=\begin{cases}
               \,\emptyset, &\text{ jos } a <0, \\
               \,[b,b+a^2], &\text{ jos } a \geq 0.
               \end{cases}
\]
Vastaavalla tavalla päätellään
\begin{align*}
\DF_{g\circ f}&=\{\,x\in (-\infty,a] \ | \ \sqrt{a-x} \in [b,\infty)\,\} \\
              &=\begin{cases}
                \,(-\infty,a],     &\text{ jos } b \le 0, \\
                \,(-\infty,a-b^2], &\text{ jos } b > 0.
                \end{cases} \qquad\quad \loppu
\end{align*}

\subsection*{Muuttujan vaihto}
\index{funktio B!g@muuttujan vaihto|vahv}
\index{muuttujan vaihto (sijoitus)|vahv}

Jos tutkimuskohteena oleva funktio $f$ joukossa $A\subset\DF_f$ on esitettävissä
yhdistettynä funktiona $f(x)=g(v(x))$, niin usein auttaa \kor{muuttujan vaihto} eli
\kor{sijoitus} $t=v(x)$. Tällöin voidaan siirtyä tutkimaan (mahdollisesti yksinkertaisempaa)
funktiota $g(t)$ joukossa $B=v(A)$.
\begin{Exa} Määritä funktion $f(x)=x-7\sqrt{x}+11$ arvojoukko $f(A)$ välillä $A=[1,16]$.
\end{Exa}
\ratk Sijoituksella $t=\sqrt{x}$ tutkimuskohteeksi tulee funktio $g(t)=t^2-7t+11$ välillä
$B=[1,4]$. Siis $f(A)=g(B)=[-5/4,5]$ (Esimerkki \ref{erään polynomin arvojoukko}). \loppu


\subsection*{Funktioiden yhdistely laskutoimituksilla}
\index{funktio B!f@yhdistely laskutoimituksilla|vahv}

Reaaliarvoisia funktioita on mahdollista yhdistellä peruslaskutoimituksilla samalla tavoin kuin 
lukujonoja. Tällöin ajatellaan, että kun lukujonoja lasketaan yhteen, kerrotaan ja jaetaan 
termeittäin, niin funktioita yhdistellään vastaavalla tavalla \kor{pisteittäin}. Näin ajatellen
saadaan määritellyksi funktioiden väliset peruslaskutoimitukset. Yhden reaalimuuttujan
funktioille täsmällisempi määritelmä on seuraava:
\begin{Def} (\vahv{Funktioiden yhdistely}) \label{funktioiden yhdistelysäännöt}
\index{laskuoperaatiot!f@reaalifunktioiden|emph}
Jos $f:\DF_f \rightarrow \R$, $g:\DF_g\rightarrow\R$, $\DF_f,\DF_g \subset \R$, niin funktiot
$\lambda f$ $(\lambda\in\R)$, $f+g$, $fg$ ja $f/g$ määritellään
\[
\begin{array}{clll}
(1) & (\lambda f)(x)&=\ \lambda f(x), \quad & x \in \DF_f \\ \\
(2) & (f+g)(x)&=\ f(x)+g(x), \quad & x\in \DF_f \cap \DF_g \\ \\ 
(3) & (fg)(x)&=\ f(x)g(x), \quad & x\in \DF_f \cap \DF_g \\ \\
(4) & (f/g)(x)&=\ f(x)/g(x), \quad & x\in \DF_f \cap \DF_g\ \wedge\ g(x) \neq 0
\end{array} 
\]
\end{Def}
Ainoa uusi piirre lukujonoihin nähden on, että funktioita yhdisteltäessä on määrittelyjoukkoa 
rajoitettava, ellei ole $\DF_f=\DF_g$. Perusmuotoisten 'jonofunktioiden' tapauksessa tätä
ongelmaa ei ollut, koska määrittelyjoukko oli aina sama $(=\N)$. Huomattakoon erityisesti, että
funktion $f/g$ määrittelyjoukko on ym. määritelmän mukaisesti
\[
\DF_{f/g}=\{x\in\R \ | \ x\in\DF_f\ \wedge\ x\in\DF_g\ \wedge\ g(x)\neq 0\}.
\]
Tässä vaatimus $g(x)\neq 0$ esiintyi jo lukujonojen yhteydessä, vrt. Lause 
\ref{raja-arvojen yhdistelysäännöt}.
\begin{Exa}
Määritelmän \ref{funktioiden yhdistelysäännöt} säännön (4) mukaisesti on
\[
\tan=\sin/\cos, \quad \cot = \cos/\sin. \loppu
\]
\end{Exa}

\subsection*{Parilliset ja parittomat funktiot}
\index{funktio B!b@parillinen, pariton|vahv}
\index{parillinen, pariton!b@funktio|vahv}

\begin{Def} \label{parilliset ja parittomat funktiot} Jos $f: \DF_f \kohti \R$ ja $f$:n 
määrittelyjoukko $\DF_f \subset \R$ on origon suhteen symmetrinen, ts. pätee 
$-x \in \DF_f\ \forall x\in\DF_f$, niin sanotaan, että $f$ on \kor{parillinen} (engl. even), jos
\[
f(-x)=f(x) \quad \forall x\in\DF_f,
\]
ja \kor{pariton} (engl. odd), jos
\[
f(-x)=-f(x) \quad \forall x\in\DF_f.
\] \end{Def}
\begin{Exa} Trigonometrisistä funktioista kosini on parillinen ja sini pariton. Potenssifunktio
$f(x) = x^n,\ n \in \Z$, on parillinen/pariton kun $n$ on parillinen/pariton. Funktio 
$f(x)=0\ \forall x\in\DF_f$ (esim.\ $\DF_f=\R$) on funktioista ainoa, joka on sekä parillinen
että pariton. \loppu 
\end{Exa}

Yhdisteltäessä funktioita Määritelmän \ref{funktioiden yhdistelysäännöt} mukaisesti ovat 
seuraavat säännöt helposti todennettavissa:
\begin{itemize}
\item[(1)]  $f$ ja $g$ parillisia/parittomia $\ \impl\ $ $f+g$, $\lambda f$ ja $1/f$ 
            parillisia/parittomia
\item[(2a)] $f$ ja $g$ parillisia/parittomia $\ \impl\ $ $fg$ parillinen
\item[(2b)] $f$ parillinen ja $g$ pariton $\ \impl\ $ $fg$ pariton
\end{itemize}
\jatko \begin{Exa} (jatko) Ym.\ sääntöjen perusteella (ja muutenkin) voidaan päätellä: 
a) Trigonometriset funktiot $\tan$ ja $\cot$ ovat parittomia. b) Polynomi 
$f(x) = \sum_{k=0}^n a_k x^k$ on parillinen/pariton täsmälleen kun $a_k = 0$ jokaisella 
parittomalla/parillisella indeksin $k$ arvolla. \loppu 
\end{Exa} 
Jos funktio on origon suhteen symmetrisesti määritelty, mutta parillisuuden suhteen 'epäpuhdas',
niin se voidaan aina esittää parillisen ja parittoman funktion summana. Nimittäin 
$f = f_+ + f_-$, missä
\[
f_+(x)=\frac{1}{2}[f(x)+f(-x)], \quad f_-(x)=\frac{1}{2}[f(x)-f(-x)].
\]
\begin{Exa}
\[
f(x)\ =\ \begin{cases} 0, & \text{kun } x \leq 0 \\ x, & \text{kun } x > 0 \end{cases}\ \
      =\ \ \frac{1}{2}\,\abs{x} + \frac{1}{2}x\,\quad \forall x \in \R. \loppu
\]
\end{Exa}
\begin{figure}[H]
\setlength{\unitlength}{1mm}
\begin{center}
\begin{picture}(120,35)
\put(18,20){\vector(1,0){24}} \put(41,16){$x$}
\put(30,20){\vector(0,1){12}} \put(32,30){$y$}
\put(48,20){\vector(1,0){24}} \put(71,16){$x$}
\put(60,20){\vector(0,1){12}} \put(62,30){$y$} 
\put(78,20){\vector(1,0){24}} \put(101,16){$x$}
\put(90,20){\vector(0,1){12}} \put(92,30){$y$}
\linethickness{0.5mm}
\curve(18,20,30,20)
\curve(30,20,42,32)
\curve(48,26,60,20)
\curve(60,20,72,26)
\curve(78,14,102,26)
\put(29,5){$f$}
\put(59,5){$f_+$}
\put(89,5){$f_-$}
\end{picture}
%\caption{Funktion esittäminen parillisen ja parittoman funktion summana}
\end{center}
\end{figure}

\subsection*{Jaksolliset funktiot}

\begin{Def} \index{funktio B!c@jaksollinen|emph} \index{jaksollinen funktio|emph}
\index{periodinen funktio} 
Reaalifunktio $f$ on \kor{jaksollinen} eli \kor{periodinen}, jos jollakin
$a\in\R_+$ pätee $\,x \pm a\in\DF_f\ \forall x\in\DF_f$ ja
\[
f(x+a)=f(x) \quad \forall x\in\DF_f.
\]
Tällöin $a$ on $f$:n \kor{jakso}.
\end{Def}
Määritelmän mukaisesti myös jokainen jakson monikerta on jakso. Jaksoista pienin on nimeltään
\kor{perusjakso} (usein vain 'jaksoksi' sanottu).
\begin{Exa} Funktiot $\,\abs{\sin x}$, $\,\abs{\cos x}$, $\,\tan x$ ja $\,\cot x\,$ ovat 
jaksollisia, \newline (perus)jaksona $a=\pi$. \loppu
\end{Exa} 
\begin{Exa} Jos $\DF_f=\R$, $f$:n jakso on $a=1$ ja $f(x)=x$, kun $x\in[0,1)$, niin
\[
f(x) = x-k, \quad \text{kun}\ x\in[k,k+1),\ k\in\Z.
\]
\end{Exa} 
\begin{figure}[H]
\setlength{\unitlength}{1cm}
\begin{picture}(14,3)(-2,-1)
\put(-2,0){\vector(1,0){14}} \put(11.8,-0.4){$x$}
\put(0,-2){\vector(0,1){4}} \put(0.2,1.8){$y$}
%\multiput(3.14,0)(3.14,0){3}{\drawline(0,-0.1)(0,0.1)}
\multiput(-1,0)(1,0){12}{\line(1,1){1}}
\multiput(-1.07,-0.07)(1,0){12}{$\scriptstyle{\bullet}$}
\put(-1.2,-0.4){$-1$} \put(0.1,-0.4){$0$} \put(1,-0.4){$1$} \put(2,-0.4){$2$}
\end{picture}
\end{figure}

\Harj
\begin{enumerate}

\item
Määritä algebran keinoin seuraavien reaalifunktioiden arvojoukot: \newline
a) \ $x^2+2x+8,\quad$
b) \ $1-x-4x^2 \quad$
c) \ $1/(2+x+x^2) \quad$
d) \ $1/(1-\sqrt{x})$ \newline
e) \ $x^2/(1-x^2) \quad$
f) \ $(x+1)/(x+2) \quad$
g) \ $\abs{x}+\abs{x+1} \quad$
h) \ $\abs{x}-\abs{x+2}$ \newline
i) \ $\sum_{k=0}^\infty x^k \quad$
j) \ $\sum_{k=0}^\infty (-1)^k 2^k x^k \quad$ 
k) \ $\sum_{k=0}^\infty x^{2k} \quad$
l) \ $\sum_{k=0}^\infty (-1)^k x^{2k}$

\item
Selvitä algebran keinoin, millä väleillä seuraavat funktiot ovat aidosti kasvavia ja millä
aidosti väheneviä: \,\ a)\, $1/x^4$, \,\ b)\, $x/(x+1)$, \,\ c)\, $\abs{x^2+x-2}$, \newline 
d)\, $1/(x^2+2x+2)$, \,\ e)\, $1/(x^2+3x+2)$, \,\ f)\, $x^4/(2-x^4)$ \,\ 
g)\, $\max\{0,\sin x\}$

\item
Mitkä ovat seuraavien yhdistettyjen funktioiden määrittely- ja arvojoukot? \newline
a) \ $\sqrt{8-2x} \quad$ 
b) \ $\sqrt{1-x-x^2}\quad$
c) \ $1/(1-\sqrt{2-x}) \quad$
d) \ $\cos(\sin x)$

\item
Määritä yhdistettyjen funktioiden $f \circ g$ ja $g \circ f$ laskusäännöt ja määrittelyjoukot
seuraavissa tapauksissa: \newline
a) \ $f(x)=1/\sqrt{x+1},\,\ g(x)=\sqrt{x-1}$ \newline
b) \ $f(x)=\sqrt{x+1},\,\ g(x)=x/(1-2x) \quad$ \newline
c) \ $f(x)=\sqrt{2x+1},\,\ g(x)=\sin x$

\item
a) Olkoon $f(x)=1-4\sqrt{x}$. Mitkä ovat funktioiden $f \circ f$ ja $f \circ f \circ f$
määrittelyjoukot? \ b) Olkoon $g(x)=\frac{1}{2}x+1$. Millainen reaalifunktio on 
$f(x)=\lim_n g_n(x)$, missä $g_1=g$, $g_2=g \circ g$, $g_3=g \circ g \circ g$, jne.\,?

\item
Määritä algebran keinoin seuraavien funktioiden arvojoukot: \newline
a) \ $1-\sqrt[4]{17x}+\sqrt{x} \quad$
b) \ $2+5x^{48}-x^{96} \quad$
c) \ $x-4\sqrt{x}+3-4\abs{\sqrt{x}-1}$ \newline
d) \ $x^4/(2-x^4) \quad$
e) \ $\cos x-7\sin x \quad$
f) \ $\sin x + \cos 2x \quad$ 
g) \ $2\sin x - \abs{\cos 2x}$ 

\item
a) Näytä, että jos $g$ on kasvava välillä $A\subset\DF_g$ ja $f$ on kasvava välillä 
$B=g(A)\subset\DF_f$, niin $f \circ g$ on kasvava välillä $A$. Sovella väittämää funktioon
$f(x)=\sqrt{x^2+x-2}$ välillä $[1,\infty)$. \newline
b) Näytä, että jos $f$ ja $g$ ovat (aidosti) kasvavia välillä $A\subset\DF_f\cap\DF_g$, niin
samoin on funktio $f+g$. Miten on funktion $fg$ laita?

\item
Määrittele $f+g$, $fg$ ja $f/g$ (määrittelyjoukko ja sievennetty laskusääntö), kun \ 
a) \ $f(x)=g(x)=x+1$, \ \ 
b) \ $f(x)=\abs{x}+x,\ g(x)=\abs{x}-x$, \newline
c) \ $f(x)=\sin x \sin 2x,\ g(x)=2\cos^3 x$.

\item
Näytä: a) Funktion jako parilliseen ja parittomaan osaan on yksikäsitteinen.
b) Jos $g$ on parillinen, niin $f \circ g$ on parillinen tai ei määritelty.
c) Jos $f$ on parillinen/pariton ja $g$ on pariton, niin $f \circ g$ on parillinen/pariton
   tai ei määritelty.

\item
Jaa seuraavat funktiot parilliseen ja parittomaan osaan: \newline
a) \ $f(x)=2+x-3x^2+x^4+x^5+\sin x-3\cos x$ \newline
b) \ $f(x)=|x+1|+|x-1|-x+2x^2$\newline
c) \ $f(x)=\sin(x+\frac{\pi}{4})+\cos(x+\frac{\pi}{3})$

\item \label{H-IV-1: näyttöjä}
a) Näytä, että jos $f$ on (aidosti) kasvava/vähenevä väleillä $A_1$ ja $A_2$ ja
$A_1 \cap A_2 \neq \emptyset$, niin $f$ on (aidosti) kasvava/vähenevä välillä 
$A=A_1 \cup A_2$. \newline
b) Olkoon $\DF_f=\R$ ja $f$ (aidosti) kasvava välillä $[0,\infty)$. Näytä, että jos lisäksi
$f$ on parillinen/pariton, niin $f$ on (aidosti) vähenevä välillä $(-\infty,0]$\,/\, 
(aidosti) kasvava $\R$:ssä.

\item
Funktio $f(x)=\sin\frac{x}{5}+\cos\frac{x}{7}$ on jaksollinen. Mikä on perusjakso? 

\item
Funktion $g$ jakso on $a=2$ ja $g(x)=1-\abs{x}$, kun $x\in[-1,1]$. Määritä funktion
$f(x)=g(x)+cx-5$ nollakohdat, kun\, a) $c=1$, \ b) $c=1/2$.

\item
Määritellään porrasfunktio
\[
f(x)=k, \quad \text{kun}\ \ 2k-2 \le x < 2k,\ k\in\Z.
\]
Määritä $a\in\R$ ja jaksollinen funktio $g$ siten, että $f(x)=ax+g(x),\ x\in\R$.

\item (*) \label{H-IV-1: funktioalgebran haasteita}
Todista pelkin algebran keinoin: \vspace{1mm}\newline
a) Funktion $f(x)=(3x^2+3)/(x^2+x+1)$ arvojoukko on $\RF_f=[2,6]$. \newline
b) Funktio $f(x)=x^2/(x+1)$ on aidosti kasvava väleillä $(-\infty,-2]$ ja $[0,\infty)$
ja aidosti vähenevä väleillä $[-2,-1)$ ja $(-1,0]$.

\end{enumerate}