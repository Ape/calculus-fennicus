\chapter{Reaalimuuttujien funktiot}

Matemaattisten funktioiden p��tyypit ovat
\begin{itemize}
\item \kor{yhden} reaali\kor{muuttujan} reaaliarvoiset \kor{funktiot}, eli reaalifunktiot
\item \kor{useamman} reaali\kor{muuttujan} reaaliarvoiset \kor{funktiot}
\item yhden tai useamman reaalimuuttujan \kor{vektoriarvoiset funktiot}
\item \kor{kompleksifunktiot}, eli kompleksimuuttujan kompleksiarvoiset funktiot
\end{itemize}
T�ss� luvussa aloitetaan funktoiden tutkimus tarkastelemalla yhden tai useamman, toistaiseksi
kahden tai kolmen, reaalimuuttujan reaaliarvoisia funktioita sek� yhden tai kahden muuttujan
vektoriarvoisia funktioita. Tarkasteltaville funktiotyypeille on yhteist� niiden saama 'n�kyv�'
muoto, kun luvut, lukuparit ja lukukolmikot muuttujina tai vektorit funktion arvoina ymm�rret��n
euklidisten pisteavaruuksien tai vastaavien vektoriavaruuksien olioina.

Funktioiden tutkimus on matematiikassa hyvin keskeist�, siksi my�s t�h�n liittyv� k�sitteist�
ja keinovalikoima on huomattavan laaja. T�ss� luvussa ei koko 'teknologiaa' oteta viel� k�ytt��n,
vaan rajoitutaan toistaiseksi kaikkein yksinkertaisimpiin algebran ja geometrian keinoihin.
Toisaalta sovelluksia (etenkin fysiikan sovelluksia) ajatellen t�ss� luvussa tarkasteltavien
funktioiden tyyppivalikoima on jo melko edustava. Tarkoituksena on t�m�n valikoiman puitteissa
k�yd� l�pi mm.\ sellaiset funktioiden algebran k�sitteet kuin funktioiden 
\kor{algebralliset yhdistelyt}, \kor{yhdistetty funktio}, \kor{k��nteisfunktio} ja
\kor{implisiittifunktio}. Selvimmin 'geometrisia funktioita' ovat Luvussa
\ref{parametriset k�yr�t} esitelt�v�t \kor{parametriset k�yr�t} ja \kor{parametriset pinnat}.