\section{*Tensorit, vektorit ja skalaarit}  \label{tensorit}
\sectionmark{*Tensorit}
\alku

Kuten edellä on todettu, on neliömuodolla $\mx^T\mA\mx$ se ominaisuus, että koordinaatistoa 
kierrettäessä neliömuodon matriisi muuntuu similaarisesti:
\[
\mx\hookrightarrow\my=\mC^T\mx \ \impl \ \mA\hookrightarrow\mB=\mC^T\mA\mC.
\]
On myös nähty, että pinnan kaarevuusmatriisilla on tämä sama ominaisuus, kun koordinaatiston 
kierto tapahtuu pinnan normaalin ympäri. Yleisesti sanotaan sellaista oliota, jolla on 
jokaisessa koordinaatistossa matriisin olomuoto, ja joka muuntuu koordinaatistoa kierrettäessä
\index{tensori}%
ym.\ säännön mukaan, \kor{tensoriksi}.\footnote[2]{Tensori on matematiikassa tässä esitettyä
yleisempi käsite. Tensorit liitetään, ei ainoastaan karteesisen koordinaatiston kiertoihin,
vaan yleisempien käyräviivaisten kordinaatistojen välisiin muunnoksiin. Muunnoksiin liittyvä
differentiaalilaskenta, nk.\ \kor{tensorilaskenta}, on oma matematiikan lajinsa. Tensoreilla ja
tensorilaskennalla on paljon käyttöä fysiikassa.}
Neliömuoto, matriisinsa kautta nähtynä, on siis tensori, itse asiassa \kor{symmetrinen tensori}.
Myös pinnan kaarevuus on symmetrinen tensori, ja kaarevuusmatriisin sijasta puhutaankin yleensä
\index{kaarevuustensori}%
\kor{kaarevuustensorista}. Näin puhuen kaarevuus tulee selvemmin ymmärretyksi pinnan
paikalliseksi geometriseksi ominaisuudeksi, joka vain näyttäytyy erilaisena eri
koordinaatistoissa.

Jos $f=f(x,y,z)$ on kahdesti jatkuvasti derivoituva, niin tietyssä pisteessä laskettu Hessen 
matriisi
\[
\mA=\begin{bmatrix} 
    f_{xx} & f_{xy} & f_{xz} \\ f_{xy} & f_{yy} & f_{yz} \\ f_{xz} & f_{yz} & f_{zz} 
    \end{bmatrix}
\]
muuntuu koordinaatistoa kierrettäessä ym. similaarisuussäännön mukaisesti. --- Tämä nähdään 
approksimoimalla $f$:ää toisen asteen Taylorin polynomilla. Silloin voidaan ajatella, että $f$:n
'toinen derivaatta' ei ole mihinkään yksittäiseen koordinaatistoon sidottu toisten 
osittaisderivaattojen matriisi, vaan pikemmin \pain{tensori}, jonka ilmiasuja nämä matriisit 
ovat (!).

Myös \pain{lineaarikuvaus} $\,A:\R^n\kohti\R^n$ voidaan ymmärtää tensorina, sillä jos
\[
A:\ \mx \map \my \ \ekv \ \my=\mA\mx,
\]
niin nähdään, että koordinaattimuunnoksessa
\[
\mx=\mC\mx',\quad \my=\mC\my'\quad (\inv{\mC}=\mC^T)
\]
lineaarikuvaus saa muodon
\[
\my'=\mB\mx',\quad \mB=\mC^T\mA\mC.
\]
\index{zza@\sov!Magneettinen permeabiliteetti}%
\begin{Exa} \vahv{Magneettinen permeabiliteetti}.
Sähkömagnetiikassa magneettivuon tiheyden ($\vec B$) ja magneettikentän voimakkuuden ($\vec H$)
välinen yksinkertaisin materiaalilaki on
\[
\vec B=\mu\vec H,
\]
missä materiaalivakio $\mu$ on väliaineen \pain{ma}g\pain{neettinen} p\pain{ermeabiliteetti}.
Tämä laki on sama kaikissa koordinaatistoissa. Yleisemmin, jos materiaali on epäisotrooppinen,
voi $\vec B$:n ja $\vec H$:n riippuvuus olla yleisempi lineaarikuvaus muotoa
\[
\mB=\boldsymbol{\mu}\mH \qekv \begin{bmatrix} B_1 \\ B_2 \\ B_3 \end{bmatrix} =
\begin{bmatrix} 
\mu_{11}&\mu_{12}&\mu_{13} \\ \mu_{21}&\mu_{22}&\mu_{23} \\ \mu_{31}&\mu_{32}&\mu_{33} 
\end{bmatrix}
\begin{bmatrix} H_1 \\ H_2 \\ H_3 \end{bmatrix}.
\]
Tällöin $\boldsymbol{\mu}$ on tensori, eli matriisi $(\mu_{ij})$ muuntuu koordinaatistoa 
kierrettäessä similaarisesti. \loppu
\end{Exa}
\begin{Exa}
Kahdesti jatkuvasti derivoituvasta funktiosta $f=f(x,y)$ tiedetään, että $f_{xx}(0,0)=24$, 
$f_{yy}(0,0)=-24$ ja $f_{xy}(0,0)=-7$. Laske $f_{\xi\xi}(0,0)$, $f_{\xi\eta}(0,0)$ ja 
$f_{\eta\eta}(0,0)$ koordinaatistossa $(\xi,\eta)$, jonka kantavektorit ovat
\[
\vec e_\xi=\frac{1}{5}(4\vec i + 3\vec j),\quad \vec e_\eta=\frac{1}{5}(-3\vec i +4\vec j).
\]
\end{Exa}
\ratk Koordinaattimuunnos $(x,y)\hookrightarrow (\xi,\eta)$ määräytyy ehdosta
\[
%       &x\vec i+y\vec j=\xi\vec e_\xi+\eta\vec e_\eta
%                =\frac{1}{5}(4\xi-3\eta)\vec i+\frac{1}{5}(3\xi+4\eta)\vec j \\
\begin{bmatrix} x\\y \end{bmatrix}=\frac{1}{5}\begin{rmatrix} 4&-3\\3&4 \end{rmatrix}
\begin{bmatrix} \xi \\ \eta \end{bmatrix}=\mC\begin{bmatrix} \xi \\ \eta \end{bmatrix}.
\]
Koska $f$:n toisen kertaluvun osittaisderivaatat muodostavat tensorin, on
\begin{align*}
\begin{bmatrix} 
f_{\xi\xi}(0,0) & f_{\xi\eta}(0,0) \\ f_{\xi\eta}(0,0) & f_{\eta\eta}(0,0) 
\end{bmatrix}
&=\mC^T\begin{bmatrix} f_{xx}(0,0)&f_{xy}(0,0) \\ f_{xy}(0,0)&f_{yy}(0,0) \end{bmatrix}\mC \\
&=\frac{1}{25}\begin{rmatrix} 4&3\\-3&4 \end{rmatrix} 
              \begin{rmatrix} 24&-7\\-7&-24 \end{rmatrix}
              \begin{rmatrix} 4 & -3 \\ 3 & 4 \end{rmatrix} \\
&=\begin{rmatrix} 0&-25\\-25&0 \end{rmatrix}.
\end{align*}
Siis $f_{\xi\xi}(0,0)=f_{\eta\eta}(0,0)=0$, $f_{\xi\eta}=-25$. --- Eräs oletukset toteuttava 
funktio on
\[
f(x,y)=(4x+3y)(3x-4y)=-25\xi\eta. \loppu
\]

\subsection*{Vektorit}
\index{vektoria@vektori (geometrinen)!a@tason \Ekaksi|vahv}
\index{vektoria@vektori (geometrinen)!b@avaruuden \Ekolme|vahv}

Jos tensori on olio, joka 'näyttää matriisilta' jokaisessa (karteesisessa) koordinaatistossa,
niin tason tai avaruuden vektori voidaan vastaavasti tulkita olioksi, jonka olomuoto on lukupari
tai lukukolmikkko jokaisessa koordinaatistossa. Kuten tensori, tämä pari tai kolmikko muuntuu
koordinaatistoa kierrettäessä systemaattisella tavalla. Esimerkiksi jos 
$\vec v=x\vec i + y\vec j + z\vec k$ on avaruusvektori, niin muunnossääntö 
$\mx \vastaa (x,y,z) \map (x',y',z') \vastaa \mx'$ siirryttäessä kierrettyyn koordinaatistoon,
jonka kantavektorien koordinaatit kannassa $\{\vec i,\vec j,\vec k\}$ ovat 
ortogonaalimatriisin $\mC$ sarakkeet, on (vrt.\ Luku \ref{lineaarikuvaukset})
\[
\mC\mx'=\mx \qekv \mx'=\mC^T\mx.
\]
Siis ei ainoastaan tensorin, vaan myös vektorin 'perimmäinen olemus' paljastuu vasta 
koordinaatistoa kierrettäessä (!). --- Havaitaan myös, että symmetrisellä tensorilla ja 
vektorilla on se yhteinen piirre, että löytyy koordinaatisto, tai koordinaatistoja, joissa 
tensorin/vektorin ilmiasu matriisina/$\R^n$:n alkiona saa mahdollisimman yksinkertaisen muodon.
Symmetrisen tensorin tapauksessa tämä yksinkertaisin muoto on diagonaalimatriisi. Tason tai 
avaruuden vektorin yksinkertaisin muoto on lukupari $(a,0)$ tai lukukolmikko $(a,0,0)$, missä 
$a>0$ on vektorin pituus. Tällaiseen koordinaatistoon siirryttäessä  palataan itse asiassa 
vektorin alkuperäiseen geometriseen ideaan --- vektorihan oli alunperin suunnattu jana. Vektorin
suunnan pysyvyys koordinaatistoa kierrettäessä varmistetaan em.\ muunnossäännöllä. Sen mukaan
siis vektorin ilmiasu $\R^n$:n alkiona \pain{ei} ole pysyvä, vaan nimenomaan muuttuu 
koordinaatistoa kierrettäessä, muunnossäännön ilmaisemalla systemaattisella tavalla.
\begin{Exa} Voidaan kuvitella (etenkin tapauksissa $n=2$ ja $n=3$), että euklidinen avaruus 
$\R^n$ on pisteavaruuden $E^n$ koordinaattiavaruus jossakin valitussa (karteesisessa) 
koordinaatistossa. Jos nyt valittua pistettä $P \in E^n$ edustaa vektori $\mx$ (paikkavektori),
niin koordinaatistoa kierrettäessä $P$:n koordinaatit muuntuvat kaavan $\mx'=\mC^T\mx$
mukaisesti. Muunnoskaava varmistaa, että piste 'pysyy paikallaan', eli pisteen paikkavektori on
vektori juuri tämän vaatimuksen vuoksi. Vastaavasti jos tarkastellaan funktion $f(\mx)$
muuntumista koordinaatiston kierrossa, niin muunnossäännöllä $g(\mx')=f(\mC\mx')$ tarkoitetaan,
että 'funktio pysyy paikallaan', ts.\ funktion arvo tietyssä (paikallaan pysyvässä) pisteessä
ei muutu. \loppu \end{Exa}
\begin{Exa} Olkoon $f=f(x,y,z)$ on differentioituva funktio. Kierretään koordinaatistoa 
muunnoksella $\mC\mx'=\mx\ \ekv\ \mx'=\mC^T\mx\ (\mx=[x,y,z]^T,\ \mx'=[x',y',z']^T)$ ja 
merkitään $g(\mx')=f(\mx)=f(\mC\mx')$. Olkoon $\nabla f(0,0,0)= a_1\vec i+a_2\vec j+a_3\vec k$.
Tällöin jos $\ma=[a_1,a_2,a_3]^T$, niin differentioituvuuden määritelmän mukaisesti on
\[
f(x,y,z)=f(0,0,0)+\ma^T\mx + o(\abs{\mx}).
\]
Ketjusääntöjen perusteella myös $g$ on differentioituva origossa, joten jollakin 
$\mb=[b_1,b_2,b_2]^T$ on
\[
g(x',y',z')=g(0,0,0)+\mb^T\mx' + o(\abs{\mx'})=f(0,0,0)+\mb^T\mx' + o(\abs{\mx'}).
\]
Kun näissä hajotelmissa oletetaan, että $\mC\mx'=\mx$, niin $\abs{\mx'}=\abs{\mx}$ ja 
$g(x',y',z')=f(x,y,z)$, joten on oltava
\[
\ma^T\mx=\mb^T\mx'=\mb^T\mC^T\mx=(\mC\mb)^T\mx \qimpl \mC\mb=\ma.
\]
Tulos merkitsee, että funktion gradientti (tulkittuna pystyvektorina) muuntuu koordinaatistoa
kierrettäessä kuten vektori. Samaan tulokseen tullaan myös laskemalla $g$:n gradientti suoraan
ketjusääntöjen avulla kaavasta $g(\mx')=f(\mC\mx')$. \loppu
\end{Exa}
Funktion gradientti on jo aiemminkin tulkittu vektoriksi, mutta tällöin tulkinta perustui vain
gradientin 'ulkonäköön' kiinteässä koordinaatistossa. Nyt voidaan siis vahvistaa, että tulkinta
kestää myös koordinaatiston kierron:
\[
\boxed{\quad\kehys \text{Gradientti \kor{on} vektori}. \quad}
\]

\subsection*{Skalaarit}
\index{skalaari|vahv}

Jos tensorin ja vektorin todellinen luonne paljastuu vasta koordinaatiston kierrossa, niin 
samoin on \pain{skalaarin} laita. Skalaari on olio, jonka ilmiasu jokaisessa koordinaatistossa
on luku (tässä reaaliluku), ja nimenomaan aina \pain{sama} luku. Esimerkiksi jos 
$f=f(\mx),\ \mx\in\R^n$, on reaaliarvoinen funktio, niin $f$:n  arvo tietyssä 
(kiinteäksi ajatellussa) pisteessä on skalaari. Vektoriin liittyvä skalaari on vektorin pituus,
koska pituus on reaaliluku, joka ei muutu koordinaatiston kierrossa. Kahteen vektoriin 
$\mx,\my\in\R^n$ liittyvä skalaari on (nimensä mukaisesti) \pain{skalaaritulo}, sillä jos 
$\mx'=\mC^T\mx$ ja $\my'=\mC^T\my$, niin
\[
(\mx')^T\my' = (\mC^T\mx)^T\mC^T\my = \mx^T\mC\mC^T\my = \mx^T\mI\my = \mx^T\my.
\]
Tensoreihin liittyvä mielenkiintoinen skalaari on tensorin \kor{jälki}:
\begin{Def} \label{tensorin jälki} \index{jzy@jälki (tensorin)|emph}
Jos $T$ on $\R^n$:n tensori, jonka matriisi annetussa ortonormeeratussa koordinaatistossa on
$\mA=(a_{ij})$, niin $T$:n \kor{jälki} (engl.\ trace) on
\[
tr\,T = \sum_{i=1}^n a_{ii}\,.
\]
\end{Def}
\begin{Lause} \label{jälki on skalaari} Tensorin jälki on skalaari. \end{Lause}
\tod Olkoon $\mC$ ortogonaalimatriisi kokoa $n \times n$ ja $\mB=\mC^T\mA\mC$. Lause väittää,
että
\[
\sum_{i=1}^n [\mB]_{ii} = \sum_{i=1}^n [\mA]_ {ii}\,.
\]
Lähdetään matriisitulon määritelmästä ja vaihdetaan summausjärjestystä\,:
\begin{align*}
\sum_{i=1}^n [\mB]_{ii} 
         &= \sum_{i=1}^n\sum_{j=1}^n[\mC^T]_{ij}\,[\mA\mC]_{ji} \\
         &= \sum_{i=1}^n\sum_{j=1}^n[\mC^T]_{ij}\sum_{k=1}^n[\mA]_{jk}\,[\mC]_{ki} \\
         &= \sum_{j=1}^n\sum_{k=1}^n[\mA]_{jk}\left(\sum_{i=1}^n[\mC]_{ki}\,[\mC^T]_{ij}\right).
\end{align*}
Tässä on $\mC$:n ortogonaalisuuden perusteella
\[
\sum_{i=1}^n[\mC]_{ki}\,[\mC^T]_{ij} = [\mC\mC^T]_{kj} = [\mI]_{kj} = \delta_{kj}\,,
\]
joten
\[
\sum_{i=1}^n [\mB]_{ii} = \sum_{j=1}^n\sum_{k=1}^n[\mA]_{jk}\delta_{kj} 
                        = \sum_{j=1}^n[\mA]_{jj}\,. \loppu
\]

\begin{Exa} Tarkastellaan funktion $u=u(x,y)\,$ 'toista derivaattaa', eli tensoria $T$, jonka
matriisi on
\[
\mA = \begin{bmatrix} u_{xx}(x,y) & u_{xy}(x,y) \\ u_{xy}(x,y) & u_{yy}(x,y) \end{bmatrix}.
\]
Kun koordinaatistoa kierretään muunnoksella $\mC[\xi,\eta]^T=[x,y]^T$ ja merkitään 
$v(\xi,\eta)=u(x(\xi,\eta),y(\xi,\eta))$, niin $T$:n matriisi kierretyssä koordinaatistossa on
\[
\begin{bmatrix} 
v_{\xi\xi}(\xi,\eta) & v_{\xi\eta}(\xi,\eta) \\ v_{\xi\eta}(\xi,\eta) & v_{\eta\eta}(\xi,\eta)
\end{bmatrix}
= \mC^T \begin{bmatrix} 
        u_{xx}(x,y) & u_{xy}(x,y) \\ u_{xy}(x,y) & u_{yy}(x,y) 
        \end{bmatrix} \mC.
\]
Lauseen \ref{jälki on skalaari} mukaan pätee
\[
(\partial_\xi^2+\partial_\eta^2)v(\xi,\eta)\,=\,(\partial_x^2+\partial_y^2)u(x,y). \loppu
\]
\end{Exa}
Esimerkistä voidaan vetää se (yleisemminkin $\R^n$:ssä pätevä) merkittävä johtopäätös, että
Laplacen operaattori $\Delta$ 'näyttää samalta' kaikissa karteesisissa koordinaatistoissa.
Koska $\Delta$ myös operoi sekä skalaareihin että vektoreihin kuten skalaari, niin voidaan
kahdessa eri merkityksessä sanoa\,:
\[
\boxed{\quad\kehys \text{$\Delta$ \kor{on skalaarinen operaattori}}. \quad}
\]

\Harj
\begin{enumerate}

\item
Kahdesti jatkuvasti derivoituvasta funktiosta $f(x,y)$ tiedetään: $f_{xx}(0,0)=0$,
$f_{yy}(0,0)=-2$ ja $f_{xy}(0,0)=2$. Laske $f_{\xi\xi}(0,0)$ ja $f_{\eta\eta}(0,0)$ sellaisessa
kierretyssä koordinaatistossa, jossa $f_{\xi\eta}(0,0)=0$.

\item
Eräässä materiaalissa magneettikentän tiheyden ja magneettikentän voimakkuuden välillä on
riippuvuus $\mB=\boldsymbol{\mu}\mH$, missä $\boldsymbol{\mu}$ on tensori, jonka
matriisiesitys on
\[
\boldsymbol{\mu}= \begin{bmatrix} \mu&0&0 \\ 0&2\mu&0 \\ 0&0&3\mu \end{bmatrix}
\]
koordinaatistossa, jonka kantavektorit ovat
\[
\vec e_1=\frac{1}{5}(3\vec i+ 4\vec j\,), \quad 
\vec e_2=\frac{1}{25}(12\vec i-9\vec j+20\vec k\,), \quad 
\vec e_3=\frac{1}{25}(16\vec i-12\vec j-15\vec k\,). 
\]
Millainen on $\boldsymbol{\mu}$:n esitysmuoto peruskoordinaatiston kannassa
$\{\vec i,\vec j,\vec k\}$\,?

\item
Näytä, että jos $\vec a$ ja $\vec F$ ovat vektorikenttiä (eli vektoriarvoisia funktioita),
niin \, a) $\nabla\cdot\vec F$ on skalaarikenttä, \ b) $(\vec a\cdot\nabla)\vec F$ on
vektorikenttä.

\item (*)
a) Näytä, että differentiaalioperaattori
\[
A=a\partial_x^2+b\partial_y^2+c\partial_x\partial_y,
\]
missä $a$, $b$ ja $c$ ovat skalaareita ($a,b,c\in\R$), säilyy muuttumattomana $\Rkaksi$:n
koordinaatiston kierrossa täsmälleen kun $a=b$ ja $c=0$,
eli kun $A=a\Delta$. \vspace{1mm}\newline
b) Millaiseksi osittaisdifferentialiyhtälö $u_{xx}-u_{yy}=0$ muuntuu kierretyssä 
koordinaatistossa, jonka kantavektorit ovat
\[
\vec e_1=\frac{1}{\sqrt{2}}(\vec i+\vec j\,), \quad 
\vec e_2=\frac{1}{\sqrt{2}}(-\vec i+\vec j\,)\,?
\]
c) Jos $\vec a=\vec i+\vec j$, niin millaisen muodon differentiaalioperaattori
\[
(\vec a\cdot\nabla)^2 = (\partial_x)^2+(\partial_y)^2+2\partial_x\partial_y
\]
saa b-kohdan kierretyssä koordinaatistossa?

\end{enumerate}