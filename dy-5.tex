\section{Lineaariset, vakiokertoimiset DY:t.  \\ Eulerin differentiaaliyhtälö} 
\label{vakikertoimiset ja Eulerin DYt}
\sectionmark{Vakiokertoimiset ja Eulerin DY:t}
\alku
\index{lineaarinen differentiaaliyhtälö!d@2.\ kertaluvun|vahv}
\index{lineaarinen differentiaaliyhtälö!f@vakiokertoiminen|vahv}
\index{vakiokertoiminen DY|vahv}

Edellisessä luvussa ratkaistiin jo ensimmäisen kertaluvun lineaarinen ja vakiokertoiminen
differentiaaliyhtälö $y'+ay=f(x)$. Sovelluksissa yleinen on myös vastaava toisen kertaluvun
differentiaaliyhtälö, jonka yleinen muoto on
\begin{equation} \label{linvak-ty}
y''+ay'+by=f(x), \quad a,b\in\R.
\end{equation}
Tämän ratkaisemiseksi tarkastellaan ensin vastaavaa homogeenista yhtälöä
\begin{equation} \label{linvak-hy}
y''+ay'+by=0.
\end{equation}
Kun tämän ratkaisua etsitään yritteellä
\[
y(x)=e^{rx},
\]
niin sijoittamalla yhtälöön saadaan vakion $r$ määrämiseksi
\index{karakteristinen yhtälö (polynomi)!a@differentiaaliyhtälön}%
\kor{karakteristinen yhtälö}
\[
\boxed{\kehys\quad r^2+ar+b=0 \quad\text{(karakteristinen yhtälö).}\quad}
\]
Karakteristisen yhtälön juuret ovat
\[
r=-\frac{a}{2}\pm\sqrt{\frac{a^2}{4}-b}\,,
\]
jolloin on kolme mahdollisuutta:

\underline{1. \ $a^2-4b>0$}. \ Tässä tapauksessa karakteristisella yhtälöllä on kaksi erisuurta 
reaalijuurta $r_1$ ja $r_2$ ja homogeeniyhtälöllä \eqref{linvak-hy} siis ratkaisut
\[
y_1(x)=e^{r_1x},\quad y_2(x)=e^{r_2x}.
\]
Ratkaisu on myös $y(x)=C_1y_1(x)+C_2y_2(x)\ (C_1,C_2\in\R)$, sillä jos yhtälö \eqref{linvak-hy}
kirjoitetaan $\dyf y=0$, niin $\dyf$ on lineaarinen operaattori, jolloin on
$\dyf(C_1y_1+C_2y_2)=C_1\dyf y_1+C_2\dyf y_2=0$. Siis jokainen $y(x)$ muotoa
\[
y(x)=C_1e^{r_1x}+C_2e^{r_2x}, \quad C_1,C_2\in\R
\]
on homogeeniyhtälön \eqref{linvak-hy} ratkaisu. Näytetään nyt, että kyseessä on yhtälön
\eqref{linvak-hy} yleinen ratkaisu eli mikä tahansa ratkaisu on tätä muotoa. Tätä silmällä
pitäen kirjoitetaan ensin karakteristinen yhtälö muotoon
\[
r^2+ar+b \,=\, (r-r_1)(r-r_2) \,=\, r^2-(r_1+r_2)r+r_1r_2 \,=\, 0,
\]
missä siis $a=-r_1-r_2$ ja $b=r_1r_2$. Olkoon $y(x)$ homogeeniyhtälön \eqref{linvak-hy}
ratkaisu. Tällöin derivoimalla funktio $u(x)=y'(x)-r_2y(x)$ todetaan, että
\[
u'-r_1u = y''-(r_1+r_2)y'+r_1r_2y = y''+ay'+by=0.
\]
Siis $u(x)$ ja $y(x)$ ovat ratkaistavissa differentiaaliyhtälöryhmästä
\[
\begin{cases}
\,u'-r_1u=0, \\ \,y'-r_2y=u(x).
\end{cases}
\]
Kun tässä ensimmäisen yhtälön yleinen ratkaisu $u(x)=C_1e^{r_1x}\ (C_1\in\R)$ sijoitetaan
jälkimmäiseen yhtälöön, niin todetaan tämän yksittäisratkaisuksi
$y_0(x)=(r_1-r_2)^{-1}C_1e^{r_1x}$ (vrt.\ edellinen luku) ja yleiseksi ratkaisuksi siis
\[
y(x)=\frac{C_1}{r_1-r_2}\,e^{r_1x}+C_2e^{r_2x}.
\]
Tässä voidaan $C_1$:n tilalle kirjoittaa yhtä hyvin $(r_1-r_2)C_1$, joten todetaan, että
jokainen yhtälön \eqref{linvak-hy} ratkaisu on muotoa $y(x)=C_1e^{r_1x}+C_2e^{r_2x}$,
$C_1,C_2\in\R$. Jokainen tällainen funktio oli myös ratkaisu, joten kyseessä on yleinen
ratkaisu.

\underline{2. \ $a^2-4b=0$}. \ Tässä tapauksessa karakteristisella yhtälöllä on kaksoisjuuri 
$r=-a/2$, joten em.\ yhtälöryhmässä on $r_1=r_2=r$. Kun ensimmäisen yhtälön yleinen ratkaisu
$u(x)=C_1e^{rx}\ (C_1\in\R)$ sijoitetaan jälkimmäiseen, niin tämän yksittäisratkaisu on
$y_0(x)=C_1xe^{rx}$ (vrt.\ edellinen luku) ja yleinen ratkaisu siis
\[
y(x)=(C_1x+C_2)e^{rx},\quad r=-\frac{a}{2}\,.
\]
Tämä on myös yhtälön \eqref{linvak-hy} yleinen ratkaisu.

\underline{3. \ $a^2-4b<0$}. \ Tässä tapauksessa karakteristisen yhtälön juuret muodostavat
konjugaattiparin
\[
r_{1,2}=\alpha\pm i\beta,\quad \alpha=-\frac{a}{2}, \ \beta=\sqrt{b-\frac{a^2}{4}}\,.
\]
Sijoituksella $y(x)=e^{\alpha x}u(x)$ yhtälö \eqref{linvak-hy} muuntuu muotoon
\[
u''+\beta^2 u = 0.
\]
Tämä on ratkaistavissa 1.\ kertalukuun palutuvana DY:nä (ks.\ Luku \ref{toisen kertaluvun dy}),
mutta suoremminkin voi päätellä, että ratkaisuja ovat
$u_1(x)=\cos\beta x$ ja $u_2(x)=\sin\beta x$ ja yleinen ratkaisu siis ilmeisesti
$u(x)=C_1\cos\beta x+C_2\sin\beta x,\ C_1,C_2\in\R$. Näin saadaan yhtälön \eqref{linvak-hy}
yleiseksi ratkaisuksi
\[
\kehys y(x)=e^{\alpha x}(C_1\cos\beta x + C_2\sin\beta x).
\]
\begin{Exa}
Ratkaise reuna-arvotehtävä
\[
\begin{cases}
y''-y'+2y=0,\quad x\in (0,1), \\
y(0)=1, \ y(1)=0.
\end{cases}
\]
\end{Exa}
\ratk Karakteristisen yhtälön $r^2-r-2$ juuret ovat $r_1=2$, $r_2=-1$, joten 
differentiaaliyhtälön yleinen ratkaisu on
\[
y(x)=C_1e^{2x}+C_2e^{-x}.
\]
Reunaehdot toteutuvat, kun
\[
\left\{ \begin{alignedat}{3}
&C_1 \ + \ & &C_2 & &=1 \\
e^2&C_1 \ + \ & e^{-1}&C_2 & &=0
\end{alignedat} \right. \,\ \ekv \,\ 
\begin{cases}
\,C_1=-1/(e^3-1), \\
\,C_2=\,e^3/(e^3-1).
\end{cases}
\]
Siis reuna-arvotehtävän ratkaisu on
\[
y(x)=\frac{1}{e^3-1}(-e^{2x}+e^{3-x}). \loppu
\] 

\subsection*{Täydellinen vakiokertoiminen yhtälö}
\index{lineaarinen differentiaaliyhtälö!b@täydellinen|vahv}

Koska differentiaaliyhtälö \eqref{linvak-ty} on lineaarinen, niin sen ratkaisulle pätee sama
yleisperiaate kuin ensimmäisen kertaluvun lineaariselle DY:lle: Täydellisen yhtälön yleinen
ratkaisu = homogeenisen yhtälön yleinen ratkaisu + täydellisen yhtälön yksittäisratkaisu
$y_0(x)$. Yksittäisratkaisun määräämiseksi tarkastellaan tässä yhteydessä vain 'sivistyneen
arvauksen' menetelmiä, jotka toimivat silloin, kun $f(x)$ yhtälössä \eqref{linvak-ty} on
riittävän yksinkertaista muotoa. (Yleisempi menetelmä esitetään seuraavassa luvussa; ks.\ myös
Harj.teht.\,\ref{H-dy-5: linvak-ty}.)
\begin{Exa}
Ratkaise $y''+y=x^2+2e^{2x}$.
\end{Exa}
\ratk Yritetään yksittäisratkaisua muodossa
\[
y(x)=Ax^2+Bx+C+De^{2x}.
\]
Sijoittamalla yhtälöön todetaan tämä ratkaisuksi kun $A=1$, $B=0$, $C=-2$ ja $D=2/5$.
Homogeenisen yhtälön yleinen ratkaisu on
\[
y(x)=C_1\cos x+C_2\sin x,
\]
joten täydellisen yhtälön yleinen ratkaisu on
\[
y(x)=C_1\cos x+C_2\sin x+x^2-2+\frac{2}{5}e^{2x}. \loppu
\]

Jos yleisemmin yhtälön \eqref{linvak-ty} oikea puoli $R(x)$ on muotoa
\[
\text{a)} \ \, R(x)=x^ne^{\alpha x}, \quad \text{tai} \quad 
\text{b)} \ \, R(x)=x^n(A\cos\omega x + B\sin\omega x),
\]
missä $n\in \{0,1,2,\ldots\}$, $\alpha\in\R$, ja $\omega\in\R$, $\omega\neq 0$, niin yhtälön
\eqref{linvak-ty} yksittäisratkaisu on löydettävissä vastaavasti muodossa
\begin{itemize}
\item[a)] $y(x)=p(x)e^{\alpha x}$,
\item[b)] $y(x)=p(x)\cos\omega x+q(x)\sin\omega x$,
\end{itemize}
missä $p$ ja $q$ ovat polynomeja. Pääsääntöisesti $p$ ja $q$ ovat astetta $n$. Poikkeuksen
muodostavat ne tapaukset, joissa pääsäännön mukainen yrite sattuu olemaan homogeenisen yhtälön
ratkaisu. Tapauksessa a) tämä on mahdollista kun $n=0$ tai $n=1$, tapauksessa b) kun $n=0$. 
Tällöin polynomin astetta on nostettava yhdellä, tapauksessa a) mahdollisesti kahdella, jotta
yksittäisratkaisu löytyisi.
\begin{Exa}
Ratkaise alkuarvotehtävä
\[
\begin{cases} \,y''+2y'+y=e^{-x},\,\ x\in\R, \\ \,y(1)=y'(1)=0. \end{cases}
\]
\end{Exa}
\ratk Karaktristisella yhtälöllä on kaksoisjuuri $r=-1$, joten homogeenisen yhtälön yleinen
ratkaisu on $y(x)=(C_1+C_2x)e^{-x}$. Koska sekä $f(x)=e^{-x}$ että $xe^{-x}$ ovat homogeenisen
yhtälön ratkaisuja, niin yksittäisratkaisua on etsittävä muodossa
\[
y(x)=(Ax^2+Bx+C)\,e^{-x}.
\]
Tämä osoittautuu ratkaisuksi, kun valitaan $A=1/2$ ja $B,C\in\R$, eli saatiin suoraan yleinen
ratkaisu
\[
y(x)=\Bigl(\frac{1}{2}x^2+Bx+C\Bigr)e^{-x}, \quad B,C\in\R.
\]
(Samaan tulokseen olisi tultu, jos yritteessä olisi valittu 'viisaammin' $B=C=0$ ja lisätty 
homogeenisen yhtälön yleinen ratkaisu vasta jälkikäteen.) Alkuehdot toteutuvat, kun
$A=-1$ ja $B=1/2$, joten alkuarvotehtävän ratkaisu on
\[
y(x) = \frac{1}{2}(x-1)^2e^{-x}.  \loppu
\]

\subsection*{Kompleksiarvoiset ratkaisut}
\index{lineaarinen differentiaaliyhtälö!j@kompleksiarvoiset ratkaisut|vahv}
\index{kompleksiarvoinen ratkaisu (DY:n)|vahv}

Vakiokertoimisia differentiaaliyhtälöitä ratkaistaessa on usein kätevää suorittaa laskut 
kompleksiarvoisia funktioita käyttäen silloinkin, kun pyritään reaaliseen lopputulokseen.
Menetelmä on kätevä erityisesti silloin, kun karakteristisen yhtälön juuret ovat 
kompleksilukuja, tai kun täydellisen yhtälön \eqref{linvak-ty} oikealla puolella esiintyy
trigonometrisia funktioita. Kompleksifunktioilla laskettaessa hyväksytään homogeenisen yhtälön
ratkaisuyritteessä
\[
y(x)=e^{rx}
\]
suoraan myös kompleksiset $r$:n arvot. Kyseessä on tällöin reaalimuuttujan kompleksiarvoinen
funktio, jonka derivaatta määritellään normaaliin tapaan eli erotusosamäärän raja-arvona. Kun
derivaatan määritelmässä merkitään po.\ funktion tapauksessa
\[
z=rx,\quad \Delta z=r\Delta x,
\]
niin nähdään, että
\[
\frac{y(x+\Delta x)-y(x)}{\Delta x}=r\,\frac{e^{z+\Delta z}-e^z}{\Delta z}\,.
\]
Tässä $\Delta x\kohti 0 \ \impl \ \Delta z\kohti 0$, joten kompleksifunktion $e^z$ 
derivoimissäännön perusteella (ks.\ Luku \ref{kompleksinen eksponenttifunktio}) voidaan todeta,
että pätee odotetusti
\[
\boxed{\quad \frac{d}{dx}\,e^{rx}=re^{rx},\quad r\in\C. \quad}
\]
\begin{Exa} Ratkaise differentiaaliyhtälö \eqref{linvak-hy} kompleksifunktioiden avulla, kun
$a^2-4b<0$.
\end{Exa}
\ratk Karakteristisen yhtälön juuret ovat $r_{1,2}=\alpha \pm i\beta$, joten yleinen ratkaisu
saadaan aiempaan tapaan ratkaisemalla yhtälöryhmä
\[
\begin{cases}
\,u'-r_1u=0, \\ \,y'-r_2y=u(x).
\end{cases}
\]
Kuten aiemmin (vrt.\ tapaus $r_1,r_2\in\R$ edellä) saadaan yleiseksi ratkaisuksi
\begin{align*}
y(x) &= C_1e^{r_1 x}+C_2e^{r_2 x} \\
     &= e^{\alpha x}(C_1e^{i\beta x}+C_2e^{-i\beta x}),\quad C_1,C_2\in\C.
\end{align*}
--- Huomattakoon, että tässä ei kertoimia $C_1,C_2$ ole syytä rajoittaa reaalisiksi. Kun
huomioidaan Eulerin kaava
\[
e^{\pm i\beta x}=\cos\beta x\pm i\sin \beta x,
\]
niin nähdään, että ratkaisu on esitettävissä yhtäpitävästi muodossa
\begin{align*}
&y(x)=e^{\alpha x}(D_1\cos\beta x+D_2\sin\beta x), \\[1mm]
&\text{missä} \quad
\begin{cases} \,D_1 = C_1+C_2, \\ \,D_2 =i(C_1-C_2). \end{cases}
\end{align*}
Tästä nähdään, että valitsemalla $C_2=\overline{C}_1$ ($= C_1$:n konjugaatti) saadaan yleinen
reaalinen ratkaisu aiemmin esitetyssä muodossa. \loppu

Esimerkin menettelyllä voidaan hakea homogeenisen yhtälön \eqref{linvak-hy} yleinen ratkaisu
myös kompleksikertoimisessa tapauksessa ($a,b\in\C$), mikäli sellainen tilanne eteen tulisi.
Tutkitaan sen sijaan sovelluksissa hyvin yleistä laskentatapaa, jossa reaalikertoimiselle
täydelliselle yhtälölle
\[
\dyf y = y''+ay'+by = \begin{cases} \,\sin\omega x \\ \,\cos\omega x \end{cases}
\]
etsitään yksittäisratkaisu käyttäen hyväksi kompleksifunktioita. Menetelmä perustuu seuraavaan
yksinkertaiseen havaintoon: Jos $f(x)=f_1(x)+if_2(x)$, missä $f_1$ ja $f_2$ ovat reaaliarvoisia,
niin $y$ on yhtälön $\dyf y=f$ yksittäisratkaisu täsmälleen kun $y=y_1+iy_2$, missä
$\dyf y_1=f_1$ ja $\dyf y_2=f_2$. (Tämä on helposti todettavissa $\dyf$:n lineaarisuuden ja
kertoimien $a,b$ reaalisuuden perusteella --- kompleksikertoimisessa tapauksessa sääntö ei 
päde.)
\begin{Exa} Etsi yksittäisratkaisu differentiaaliyhtälölle $y''-y'+y=\sin\omega x$ 
($\omega\neq 0$) käyttäen kompleksifunktioita.
\end{Exa}
\ratk Koska $\sin \omega x = \text{Im} \, (e^{i\omega x})$, niin probleema voidaan ratkaista
etsimällä kompleksiarvoinen yksittäisratkaisu differentiaaliyhtälölle
\[
y''-y'+y=e^{i\omega x},
\]
ja ottamalla ratkaisusta imaginaariosa. Kompleksinen ratkaisu löytyy helposti sijottamalla
yrite $y(x)=Ae^{i\omega x}\ (A\in\C)$ yhtälöön:
\begin{align*}
     &(-\omega^2-i\omega+1)Ae^{i\omega x}=e^{i\omega x}\quad\forall x \\
     &\qimpl A=\frac{1}{1-\omega^2-i\omega}=\frac{1-\omega^2+i\omega}{\omega^4-\omega^2+1} \\
     &\qimpl y(x) \,=\, \frac{1-\omega^2+i\omega}{\omega^4-\omega^2+1}\,e^{i\omega x}
          \,=\, \frac{1-\omega^2+i\omega}{\omega^4-\omega^2+1}\,(\cos\omega x + i\sin\omega x).
\end{align*}
Kysytty ratkaisu on tämän imaginaariosa, eli
\[
y(x)=\frac{1}{\omega^4-\omega^2+1}\,[\,\omega\cos\omega x + (1-\omega^2)\sin\omega x\,]. \loppu
\]

Esimerkissä olisi luonnollisesti tultu toimeen myös reaalisella yritteellä 
$y(x)=A\cos\omega x+B\sin\omega x\ (A,B\in\R)$, mutta laskusta olisi tullut ikävämpi.

\subsection*{Eulerin differentiaaliyhtälö}
\index{lineaarinen differentiaaliyhtälö!fa@Eulerin|vahv}
\index{Eulerin!c@differentiaaliyhtälö|vahv}

Differentiaaliyhtälöä muotoa
\[
x^2y''+axy'+by=f(x)
\]
sanotaan (toisen kertaluvun) \kor{Eulerin} DY-tyypiksi. Perusmuodossa
\[
y''+ax^{-1}y'+bx^{-2}y=x^{-2}f(x)
\]
on $x=0$ kertoimien epäjatkuvuuspiste, joten Eulerin differentiaaliyhtälöä on tarkasteltava
erikseen väleillä $(-\infty,0)$ ja $(0,\infty)$ --- sovelluksissa yleensä välillä $(0,\infty)$.
Kummallakin välillä yhtälö palautuu vakiokertoimiseksi sijoituksella
\[
\abs{x}=e^t \ \ekv \ t=\ln\abs{x},
\]
sillä kun esim. välillä $(0,\infty)$ kirjoitetaan
\[
y(x)=y(e^t)=u(t)=u(\ln x),
\]
niin saadaan
\begin{align*}
y'(x)  &\,=\, \frac{1}{x}u'(\ln x)=\frac{1}{x}u'(t), \\
y''(x) &\,=\, \frac{1}{x^2} u''(\ln x)-\frac{1}{x^2}u'(\ln x)
        \,=\, \frac{1}{x^2}[u''(t)-u'(t)],
\end{align*}
joten yhtälö saadaan vakiokertoimiseen muotoon
\[
u''(t)+(a-1)u'(t)+bu(t)=f(e^t).
\]
Välillä $(-\infty,0)$ tulee oikealle puolelle $f(-e^t)$, muuten tulos on sama.

Homogeenista Eulerin yhtälöä ei käytännössä tarvitse muuntaa vakiokertoimiseksi, sillä
ratkaisua voi etsiä suoraan yritteellä
\[
y(x)=x^r.
\]
Tällöin $r$:n määräämiseksi saadaan toisen asteen karakteristinen yhtälö. Riippuen siitä, 
millaisia juuret ovat, yleiseksi ratkaisuksi tulee jokin seuraavista:
\begin{itemize}
\item[a)] $y(x)=C_1x^{r_1}+C_2x^{r_2}$,
\item[b)] $y(x)=x^r(C_1+C_2\ln\abs{x})$,
\item[c)] $y(x)=x^\alpha[\,C_1\cos (\beta\ln\abs{x})+C_2\sin (\beta\ln\abs{x})\,]$.
\end{itemize}
Tapauksessa a) juuret ovat reaaliset ja erisuuruiset, tapauksessa b) on reaalinen kaksoisjuuri,
ja tapauksessa c) juurina on konjugaattipari $\alpha\pm i\beta$. Väitetyt yleisen ratkaisun
muodot voi päätellä em.\ muunnoksen avulla.

Täydellinen Eulerin yhtälö ratkaistaan suotuisissa tapauksissa 'sivistyneellä arvauksella'.
Arvauksen muodon voi johtaa ajatellen muunnosta vakiokertoimiseen tilanteeseen.
\begin{Exa}
Ratkaise alkuarvotehtävä
\[
\begin{cases} \,x^2y''+3xy'+y=1/x,\,\ x>0, \\ \,y(1)=1,\ y'(1)=0. \end{cases}
\]
\end{Exa}
\ratk Homogeenisessa yhtälössä sijoitus $y(x)=x^r$ johtaa karakteristiseen yhtälöön
\[
r(r-1)+3r+1=r^2+2r+1=0.
\]
Tällä on kaksoisjuuri $r=-1$, joten homogeenisen yhtälön yleinen ratkaisu on
\[
y(x)=\frac{1}{x}(C_1+C_2\ln x).
\]
Koska oikea puoli $f(x)=x^{-1}$ on homogeenisen yhtälön ratkaisu, samoin kuin 
$x^{-1}\ln\abs{x}$, niin täydellisen yhtälön yksittäisratkaisu on etsittävä muodossa
\begin{align*}
y_0(x)   &= Ax^{-1}(\ln x)^2.
\intertext{Kun sijoitetaan yhtälöön tämä sekä derivaatat}
y_0'(x)  &= Ax^{-2}[\,-(\ln x)^2+2\ln x\,], \\
y_0''(x) &= Ax^{-3}[\,2(\ln x)^2-6\ln x+2\,],
\end{align*}
niin saadaan ratkaisu, kun $A=1/2$. Siis yleinen ratkaisu on
\[
y(x)=\frac{1}{x}\left(C_1+C_2\ln x+\frac{1}{2}(\ln x)^2\right).
\]
Alkuehdot toteutuvat, kun valitaan $C_1=C_2=1$. \loppu

\subsection*{Korkeamman kertaluvun vakiokertoimiset DY:t}
\index{lineaarinen differentiaaliyhtälö!e@korkeamman kertaluvun|vahv}

Yleinen lineaarinen, vakiokertoiminen, homogeeninen, $n$:nnen kertaluvun differentiaaliyhtälö
\[
y^{(n)}+a_{n-1}y^{(n-1)}+\cdots +a_0y=0
\]
voidaan ratkaista yritteellä $y(x)=e^{rx}$, kuten tapauksissa $n=1,2$. Saadaan
\index{karakteristinen yhtälö (polynomi)!a@differentiaaliyhtälön}%
karakteristinen yhtälö
\[
r^n+a_{n-1}r^{n-1}+\cdots+a_0=0,
\]
jonka kutakin juurta vastaa juuren kertaluvun mukainen määrä lineaarisesti riippumattomia 
ratkaisuja. Jos kyseessä on $m$-kertainen reaalijuuri, niin nämä ratkaisut ovat
\[
x^ke^{rx},\quad k=0\ldots m-1.
\]
Jos kyseessä on $m$-kertainen konjugaattipari $\alpha\pm i\beta$, niin vastaavat ratkaisut ovat
\[
x^ke^{\alpha x}\cos\beta x,\quad x^ke^{\alpha x}\sin\beta x,\quad k=0\ldots m-1.
\]
Homogeenisen yhtälön yleinen ratkaisu saadaan kaikkien näiden ratkaisujen (yhteensä $n$ kpl)
lineaarisena yhdistelynä.
\begin{Exa}
Ratkaise $\,y'''-y=x^2,\,\ x\in\R$.
\end{Exa}
\ratk Yksittäisratkaisu on $y(x)=-x^2$. Homogeenista yhtälöä ratkaistaessa tulee
karakteristiseksi yhtälöksi $\,r^3-1=0$. Juuret ovat
\[
r_1=1, \quad r_{2,3}= -\frac{1}{2} \pm i\frac{\sqrt{3}}{2}\,,
\]
joten yleinen ratkaisu on
\[
y(x)=C_1e^x
    +e^{-x/2}\left(C_2\cos\frac{\sqrt{3}x}{2}+C_3\sin\frac{\sqrt{3}x}{2}\right)-x^2. \loppu
\]

Kuten esimerkissä, voidaan täydellisen yhtälön ratkaisu usein löytää kokeilemalla. Yleisempiin
menetelmiin ei juuri käytännön tarvetta olekaan. 

Todettakoon lopuksi, että yleinen $n$:nnen kertaluvun homogeeninen Eulerin DY
\[
x^{n}y^{(n)}+a_{n-1}x^{n-1}y^{(n-1)}+ \ldots + a_0y = 0
\]
ratkaistaan yritteellä $y(x)=x^r$ samaan tapaan kuin edellä.

\Harj
\begin{enumerate}

\item
Ratkaise (yleinen ratkaisu kohdissa a)--n), yleinen ratkaisu kaikilla lukupareilla
$(a,b)\in\Rkaksi$ kohdissa o)--t), alku- tai reuna-arvotehtävän ratkaisu 
kohdissa u)--ö)): \vspace{1mm}\newline
a) \ $y''+y'-30y=0 \qquad\qquad$
b) \ $y''+4y'+y= 0$ \newline
c) \ $y''+6y'+10y=0 \qquad\quad\ \ $
d) \ $y''+4y'+6y=0$ \newline
e) \ $y''+4y'+5y=3x-2 \qquad$
f) \ $y''-y'-2y=x^2$ \newline
g) \ $y''-7y'+6y=\sin x \qquad\,\ \ $
h) \ $y''+4y=\sin 3x$ \newline
i) \ $y''+2y'+5y=e^{-x} \qquad\quad\ \ $
j) \ $y''+y'-2y=x+e^x$ \newline
k) \ $y''-4y=xe^{2x} \qquad\qquad\quad\,\ $
l) \ $y''+4y'+4y=(x-1)^2e^{-2x}$ \newline
m) \ $y''-6y'+9y=(xe^x)^3 \quad\ \ $
n) \ $y''+4y'+4y=e^{-2x}+\sin x$ \newline
o) \ $y''+ay'=e^{bx} \qquad\qquad\quad\ \ $
p) \ $y''-a^2y=e^{bx}$ \newline
q) \ $y''+a^2y=e^{bx} \qquad\qquad\quad\,\ $
r) \ $y''+a^2y=\sin bx$ \newline
s) \ $y''+a^2y=x\sin bx \qquad\quad\ \ $
t) \ $y''+2y'+(1+a^2)y=e^{bx}$ \newline
u) \ $y''+2y'+y=0,\,\ y(0)=0,\ y'(0)=1$ \newline
v) \ $y''+2y'+2y=0,\,\ y(0)=1,\ y'(0)=0$ \newline
x) \ $y''-4y'-5y=0,\,\ y(0)=1,\ \lim_{x\kohti\infty}y(x)=0$ \newline
y) \ $y''+3y'+2y=0,\,\ y(0)=y(1)=1$ \newline
x) \ $y''-4y'+5y=\sin x,\,\ y(0)=0,\ y'(0)=-1$ \newline
å) \ $y''-3y'+x^2-1=0,\,\ y(0)=1,\ y'(0)=0$ \newline
ä) \ $y''-5y'+6y=e^x,\,\ y(0)=y(1)=0$ \newline
ö) \ $4y''+8y'+5y=x^2,\,\ y(0)=y(\pi)=1$

\item
Etsi seuraaville differentiaaliyhtälöille ensin kompleksiarvoinen yksittäisratkaisu, kun
yhtälön oikea puoli on $f(x)=e^{i\omega x}$ ($\omega\in\R,\ \omega \neq 0$), ja määritä sen
avulla reaalinen yksittäisratkaisu. \vspace{1mm}\newline
a) \ $y''-2y'+2y=\sin\omega x \qquad$ 
b) \ $y''-3y'+2y=\sin\omega x$ \newline
c) \ $y''+4y'+4y=\cos\omega x \qquad$
d) \ $y''+y'+y=\sin\omega x-2\cos\omega x$

\item \index{zzb@\nim!Resonanssi}
(Resonanssi) Kappaleeseen, jonka massa $=m$ ja paikka $=y(t)$ hetkellä $t$, kohdistuu
jousivoima ja lisäksi ulkoinen kuorma $f(t)$, jolloin $y(t)$ toteuttaa liikeyhälön
$\,my''=-ky+f(t)$. Määritä $y(t),\ t \ge 0$ alkuehdoilla $y(0)=y'(0)=0$, kun
$f(t)=F\sin\omega t$ ($F=$ vakio). Tarkastele erikseen tapaukset \,a) $\,\omega\neq\omega_0\,$
ja \,b) $\omega=\omega_0$, missä $\,\omega_0=\sqrt{k/m}\,$ (nk.\ resonanssitaajuus).

\item \index{zzb@\nim!Tzz@Töyssy}
(Töyssy) Auton etupyörä osuu töyssyyn, jolloin pyörän liikeyhtälö pystysuunnassa on
$\,my''=F_1+F_2$, missä $m=$ pyörän massa, $F_1=-ky$ on jousivoima ja $F_2=-cy'$ on
iskunvaimentimen vastusvoima. Hahmottele ratkaisu $y(t)$ (aikayksikkö s) alkuehdoilla
$y(0)=0$ m, $y'(0)=60$ m/s, kun $m=15$ kg, $k=6 \cdot 10^5$ kg/s$^2$ ja iskunvaimennin on
a) uusi: $c=7500$ kg/s, b) pian vaihdettava: $c=6000$ kg/s,\, c) heti vaihdettava:
$c=4500$ kg/s.

\item
Ratkaise (kohdassa i) kaikilla $(a,b)\in\Rkaksi$): \vspace{1mm}\newline
a) \ $x^2y''-xy'+y=0 \qquad\qquad\ \ $
b) \ $x^2y''-6xy'+7y=0$ \newline
c) \ $x^2y''-2xy'+2y=x \qquad\quad\ \ $
d) \ $x^2y''+xy'-y=(x+1)^2/x^2$ \newline
e) \ $x^2y''+xy'+y=x^3-2x \qquad$
f) \ $x^2y''-3xy'+4y=x^2\ln\abs{x}$ \newline
g) \ $(x^2+2x+1)y''+(x+1)y'+y=x^2$ \newline
h) \ $(3x+2)^2y''+(9x+6)y'-36y=81x+18$ \newline
i) \,\ $x^2y''+(2a+1)xy'+by=0$

\item
a) Määritä kaikki välillä $(0,\infty)$ derivoituvat funktiot, jotka toteuttavat yhtälön
$\,x^2y'(x)=2\int_0^x y(t)\,dt,\ x>0$. \newline
b) Ratkaise differentiaaliyhtälö $\,x^2y'''+2(x^2-x)y''+(x^2-2x+2)y'=x^3$ tekemällä sijoitus
$u=y'/x$.

\item
Ratkaise (yleinen ratkaisu tai alkuarvotehtävän ratkaisu): \vspace{1mm}\newline
a) \ $y^{(5)}+2y'''+y'=0 \qquad\quad\,$
b) \ $y^{(7)}+3y^{(6)}+3y^{(5)}+y^{(4)}=0$ \newline
c) \ $y'''+3y''-2y=\sin x \qquad$
d) \ $y^{(4)}+3y'''+3y''+y'=2e^{-2x}-2x$ \newline
e) \ $y^{(4)}+a^4y=x^2,\,\ a \ge 0$ \newline
f) \ $y'''-(a+2)y''+(2a+1)y'-ay=x+1,\,\ a\in\R$ \newline
g) \ $x^3y'''+2ax^2y''-4a(xy'-y)=0,\,\ a\in\R$ \newline
h) \ $y'''-5y''+17y'-13y=0,\,\ y(0)=y'(0)=0,\ y''(0)=24$ \newline
i) \ $y'''-y''-y'+y=e^x,\,\ y(0)=y'(0)=y''(0)=0$ \newline
j) \ $y'''-ay'=0,\,\ y(0)=0,\ y'(0)=1,\ y''(0)=2,\,\ a\in\R$ \newline
k) \ $x^3y'''+xy'-y=\sqrt{x},\,\ y(1)=1,\ y'(1)=y''(1)=0$

\item (*) \index{zzb@\nim!Pystyammunta}
(Pystyammunta) Luoti ammutaan suoraan ylöspäin lähtönopeudella $800$ m/s. Lennon aikana
luotiin vaikuttaa nopeuteen verrannollinen vastusvoima, jolloin nousukorkeus $y(t)$ toteuttaa
differentiaaliyhtälön
\[
y''+ay'+g=0, 
\]
missä $g=9.8$ m/s$^2$, $a=0.10$ s$^{-1}$ luodin nousuvaiheessa ja $a=0.20$ s$^{-1}$
putoamisvaiheessa. Määritä luodin lennon lakikorkeus, lentoon kuluva aika ja luodin paluunopeus
sen pudotessa maahan. 

\item (*) \label{H-dy-5: linvak-ty}
Vakiokertoimisessa differentiaaliyhtälössä $\,y''+ay'+by=f(x)\,$ olkoon karakteristisen
yhtälön juuret $r_1,r_2\in\C$. Kirjoittamalla differentiaaliyhtälö systeemiksi funktioille $y$
ja $u=y'-r_2y$ näytä, että yhtälön yksittäisratkaisu alkuehdoilla $y(x_0)=y'(x_0)=0$ on
\[
y_0(x)=\int_{x_0}^x e^{r_1(x-t)}\left[\int_{x_0}^t e^{r_2(t-s)}f(s)\,ds\right]dt.
\]
Tarkista kaavan toimivuus, kun $a=-2$, $b=1$, $x_0=0$ ja $f(x)=x^2-4x+2$.

\end{enumerate}