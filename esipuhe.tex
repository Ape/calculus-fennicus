\chapter*{Esipuhe}

Tämä kirja on syntynyt kirjoittajan TKK:ssa (myöhemmin Aalto-yliopistossa) pitämien matematiikan
luentojen pohjalta vuosina 2000--2013. Kirja on aiemmin ilmestynyt kaksiosaisena kattamaan
TKK:n ensimmäisen lukuvuoden laajan matematiikan peruskurssit I--II.

Yliopistotasoisen (varmaan muunkin tasoisen) matematiikan perusopetuksen uudistaja joutuu 
väistämättä monien vaikeiden kysymysten ja valintojen eteen. Toisaalta matematiikka tuntuu 
olevan kuin kirkko, jossa perustotuudet ovat pysyviä ja liturgisiakin muutoksia vastustetaan
kiivaasti. Toisaalta kuitenkin ympäröivä maailma muuttuu koko ajan ja muutospaineet
kohdistuvat ennen pitkää myös matematiikkaan. Kirjoittaja on erityisesti joutunut pohtimaan,
millainen pitkän aikavälin vaikutus tietokoneilla ja tehokkailla laskimilla mahdollisesti on
matematiikan opetukseen yliopistotasolla. --- Kun numeerisia ja symbolisia manipulaatioita
voi suorittaa kätevästi koneella, niin millainen on se matematiikan taito, jota pitäisi
opettaa ihmisille? Olisiko ehkä syytä korostaa aiempaa enemmän matematiikan käytännöllistä
puolta ja sovelluksia? Vai päinvastoin matematiikkaa puhtaana abstraktin ajattelun taitona? 

Jääköön lukijan pääteltäväksi, onko mainituilla mietteillä ollut lopputulokseen jokin 
vaikutus (ja jos, niin minkä suuntainen). Mainittakoon kuitenkin yksi tälle oppikirjalle 
ominainen, matematiikan vakiintuneesta opetusperinteestä poikkeava piirre: Kirjassa
esitellään lukujen muodostamat lukujonot ja sarjat heti aluksi, jolloin reaaliluvut voidaan
määritellä sellaisina kuin ne laskimien ja tietokoneiden maailmassa 'näkyvät' eli äärettöminä
desimaalilukuina.

Kansainvälisenä vertailukohtana tälle kirjalle voi mainita amerikkalaiset Calculus-kirjat. 
Näihin verrattuna tämä kirja on tavoitteiltaan kunnianhimoisempi. Pyrkimyksenä on esittää
aivan perusteista lähtevä, yhtenäinen ja loogisesti etenevä johdatus matematiikan
perusideoihin ja moderniin laskutekniikkaan. Loogista aukottomuutta silmällä pitäen
abstraktiotasoa on paikoin nostettu tyypillisestä, esim.\ mainituille Calculus-kirjoille
ominaisesta 'katutasosta'. Kirja onkin tarkoitettu melko vaativalle yleisölle, jolle
matematiikka on paitsi laskutekniikkaa ja 'kaavoja' myös älyllinen haaste ja aito
kiinnostuksen kohde. 

Otaniemen kotikentällä tämän kirjan edeltäjiä ovat apulaisprofessori Harri Rikkosen 1960- ja
1970-lukujen vaihteessa ja lehtori Simo K.\ Kivelän 1980- ja 1990-luvuilla kirjoittamat 
suomenkieliset oppikirjat tai luentomonisteet. Perinteen vaikutus näkyy myös tässä kirjassa
--- perinteestä on todella vaikea irrottautua. Perimätiedon ohella hyvin hyödyllisiä ovat
olleet ne lukuisat elävät keskustelut, joita kirjoittaja on käynyt TKK:n matematiikan
laitoksen henkilökuntaan kuuluvien kanssa. Monista kirjan syntyvaiheissa aktiivisista
keskustelukumppaneista on syytä mainita erityisesti lehtori (nyk.\ emer.) Simo Kivelä, joka
ansaitsee 'Latex-guruna' vielä erilliskiitoksen lukuisista matemaattista tekstinkäsittelyä
koskevista neuvoista ja teknisestä avusta. Kiitoksen ansaitsee myös lehtori Pekka
Alestalo kirjaamistaan yli sadasta kriittisestä kommentista vuodelta 2006, jolloin hän toimi
laajojen peruskurssien I-II sijaisluennoitsijana.

Kirjan ensimmäinen, käsin kirjoitettu versio syntyi lukuvuonna 2000-2001. Ko.\ vuoden
yleisöön kuulunut tekn.\ yo.\ (nyk.\ TkT) Antti H. Niemi työsti tekstistä ensimmäisen
tietokoneistetun version kesällä 2002. Kirjan muodon teksti sai ensimmäisen kerran
lukuvuonna 2005-2006. Uudistettuja painoksia on ilmestynyt vuosina 2007 ja 2009. Nyt
ilmestyvässä 'lopullisessa' versiossa on kirjan sisältöä edelleen korjailtu ja
virtaviivaistettu, ja myös harjoitustehtäviä on huomattavasti muokattu.

Kirjoittajan ensi kosketus TKK:n laajaan matematiikkaan tapahtui opiskelijana syksyllä 1968.
Silloista Matematiikan pitkää peruskurssia luennoi Harri Rikkonen päärakennuksen D-salissa
--- samassa, jossa kirjoittaja piti ensimmäiset laajan peruskurssin luentonsa 32 vuotta
myöhemmin. Noista ajoista on moni asia tätä kirjoitettaessa muuttunut: Enää ei ole TKK:ta
eikä (syksyn 2013 opetusuudistuksen jälkeen) enää laajaa matematiikkaakaan. Välitöntä
kurssikirjakäyttöä ei tällä kirjalla siis enää ole. Jääköön kirja kuitenkin ainakin
verkkojulkaisuna muistuttamaan TKK:ssa omintakeisesti syntyneestä ja puoli vuosisataa
jatkuneesta laajan matematiikan opetusperinteestä.

Kiitän Aalto-yliopiston Matematiikan ja systeemianalyysin laitosta taloudellisesta tuesta, joka
on kannustanut oppikirjan viimeistelyyn vielä emerituksenakin. Koko luentoyleisöäni syksystä
2000 kevääseen 2013 muistelen lämmöllä ja kiitän tämän kirjan myötä vielä kerran myös
kriittisestä palautteesta.

Helsingissä 15.6.2015

Juhani Pitkäranta

  
