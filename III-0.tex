\chapter{Kompleksiluvut}

Siirtyminen reaaliluvista kompleksilukuihin on matemaattisen analyysin merkitt�vimpi� ja samalla
merkillisimpi� aluevaltauksia. Kyse on lukualueen laajennuksesta, ts. siirtymisest� j�lleen 
uudelle 'todellisuuden' tasolle. Vaikka laajennusta voi pit�� vain kuvitelmana, niin t�m� 
kuvitelma on yksinkertaistanut matemaattista ajattelua siin� m��rin, ett� sill� on lopulta ollut
syv�llinen vaikutus kaikkeen matematiikkaan, my�s k�yt�nn�n laskentamenetelmiin. Matemaattisen
analyysin perinteess� lukualueen laajennus korostuu k�sitteiss� \kor{reaalianalyysi} ja 
\kor{kompleksianalyysi}. Molemmat ovat nyky��n hyvin laajoja (ja hieman ep�m��r�isi�) 
matematiikan alueita. Kompleksianalyysin osa-alueista maininnan arvoinen on kompleksimuuttujan
funktioiden teoria eli \kor{funktioteoria}\footnote[2]{Funktioteorian tutkimusperinne on 
Suomessa vahva. T�t� matematiikan suuntausta edusti my�s Suomen historian tunnetuin 
matemaatikko, akateemikko \hist{Rolf Nevanlinna} (1895-1980). \index{Nevanlinna, R.|av}}.
