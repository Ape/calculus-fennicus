\section{Usean muuttujan funktiot: Jatkuvuus ja raja-arvot} 
\label{usean muuttujan jatkuvuus}
\sectionmark{Usean muuttujan jatkuvuus}
\alku
\index{jatkuvuusb@jatkuvuus (usean muuttujan)|vahv}
\index{funktion raja-arvo|vahv}

Kahden ja kolmen reaalimuuttujan funktioiden algebraa on tarkasteltu aiemmin Luvussa 
\ref{kahden ja kolmen muuttujan funktiot}. Tässä ja seuraavissa luvuissa kohteena ovat kahden
ja kolmen muuttujan funktioiden lisäksi yleisemmät $n$ reaalimuuttujan reaaliarvoiset funktiot 
muotoa
\[
y=f(x_1,\ldots,x_n).
\]
Tässä voi käyttää myös matriisilaskun merkintää
\[
y=f(\mx), \quad \mx\in\R^n
\] 
\index{funktio A!d@$n$ muuttujan (vektorim.)} \index{vektorimuuttujan funktio}%
($\mx$ pysty- tai vaakavektori) ja puhua \kor{vektorimuuttujan} funktiosta. Mahdollista
(ja matemaattisissa teksteissä  tavallistakin) on matriisilaskun merkinnän sijasta käyttää myös
vektorimuuttujalle yksinkertaista symbolia $x$, eli merkitä
\[
y=f(x),\quad x=(x_1,\ldots,x_n)\in\R^n.
\]
Jatkossa käytetään eri merkintätapoja rinnakkain, jolloin matriisialgebran merkinnöissä $\mx$ 
tarkoittaa pystyvektoria. Kahden ja kolmen muuttujan tapauksissa merkitään vektorimuuttuja
vanhaan tapaan $(x,y)$ tai $(x,y,z)$. 

Funktion j\pain{atkuvuuden} ja \pain{ra}j\pain{a}-\pain{arvon} käsitteet, jotka toistaiseksi
on liitetty vain yhden muuttujan funktioihin 
(ks.\ Luvut \ref{jatkuvuuden käsite}--\ref{funktion raja-arvo}), ovat yleistettävissä melko
suoraviivaisesti useamman muuttujan funktioita koskeviksi. Aloitetaan kahden muuttujan
funktioista.

\subsection*{Funktio $f(x,y)$}

Kahden muuttujan funktion jatkuvuuden ja raja-arvon määritelmien alustukseksi tarvitaan
\begin{Def} \label{lukuparijonon suppeneminen} \index{raja-arvo!c@lukuparien jonon|emph}
\index{suppeneminen!ab@lukuparien jonon|emph}
Jono $\seq{(x_n,y_n)}$, missä 
$(x_n,y_n) \in \R^2,\ n\in\N$, \kor{suppenee kohti} (lähestyy) \kor{lukuparia} (pistettä) 
$(x,y)$ täsmälleen kun $x_n \kohti x$ ja $y_n \kohti y$. 
\end{Def}
Kun lukuparien jonon suppenemiselle käytetään aiempaan tapaan merkintää '\kohti', niin on siis 
sovittu:
\[ 
(x_n,y_n) \kohti (x,y) \qekv x_n \kohti x\ \ \ja\ \ y_n \kohti y. 
\]
Tämän kanssa yhtäpitävä sopimus on 
(ks.\ Harj.teht.\,\ref{jonon raja-arvo}:\,\ref{H-I-6: lukujonopari}d)
\[
(x_n,y_n)\kohti(x,y) \qekv (x_n-x)^2+(y_n-y)^2 \kohti 0.
\]
Tämän mukaan siis $(x_n,y_n)\kohti(x,y)$ tarkoittaa yksinkertaisesti, että pisteen
$(x_n,y_n)$ (geometrinen) etäisyys pisteestä $(x,y)$ lähestyy (lukujonona) $0$:aa, kun
$n\kohti\infty$. Jos erityisesti $(x,y)=(0,0)$, niin polaarimuunnoksen $x_n=r_n\cos\varphi_n$,
$y_n=r_n\sin\varphi_n$ perusteella on
\[
(x_n,y_n)\kohti(0,0) \qekv r_n \kohti 0 \qquad \text{(polaarikoordinaatisto)}.
\]
Koska tämä ei aseta mitään rajoituksia jonolle $\seq{\varphi_n}$, niin lähestyminen voi
tapahtua esim.\ pitkin puolisuoraa, jolla $\varphi_n=\varphi=$ vakio, tai se voi olla
suunnaltaan 'hyppelehtivää', esim.\ spiraalimaista. Joka tapauksessa mahdollisia
lähestymis\-\pain{suuntia} (vastaten puolisuoria) on äärettömän monta. --- Tämä on olennainen
ero verrattuna yhden muuttujan tilanteeseen, jossa pistettä voi lähestyä vain kahdesta eri
suunnasta.

Kun kaksiulotteinen 'lähestyminen' $(x_n,y_n)\kohti(x,y)$ on näin määritelty, on
Määritelmien \ref{funktion jatkuvuus} (jatkuvuus) ja \ref{funktion raja-arvon määritelmä}
(funktion raja-arvo) yleistäminen kahden muuttujan tilanteeseen suoraviivaista:
\begin{Def} \label{kahden muuttujan jatkuvuus} 
\index{jatkuvuusb@jatkuvuus (usean muuttujan)!a@funktion $f(x,y)$|emph}
Funktio $f:\ \DF_f\subset\R^2\kohti\R$ on
\kor{jatkuva} pisteessä $(x,y)\in\DF_f$ täsmälleen kun kaikille reaalilukuparien jonoille
$\seq{(x_n,y_n)}$ pätee
\[
(x_n,y_n)\in\DF_f\,\ \forall n\,\ \ja\,\ (x_n,y_n)\kohti(x,y)\,\ 
                                  \impl\,\ f(x_n,y_n) \kohti f(x,y).
\]
\end{Def}
\begin{Def} \label{kahden muuttujan raja-arvo}
\index{raja-arvo!d@kahden muuttujan funktion|emph}
Funktiolla $f(x,y)$ on pisteessä $(a,b)$
\kor{raja-arvo} $A\in\R$, jos jokaiselle reaalilukuparien jonolle $\seq{(x_n,y_n)}$ pätee
\[
(a,b)\neq(x_n,y_n)\in\DF_f\,\ \forall n\,\ \ja\,\ (x_n,y_n) \kohti (a,b)\,\ 
                                             \impl\,\ f(x_n,y_n) \kohti A, 
\]
ja oletus on voimassa jollekin jonolle $\seq{(x_n,y_n)}$. Raja-arvo merkitään
\[
\lim_{(x,y)\kohti(a,b)}f(x,y)=A.
\]
\end{Def}
Kuten yhden muuttujan tapauksessa, Määritelmä \ref{kahden muuttujan raja-arvo} estää raja-arvon
määrittelyn sellaisessa (eristetyssä) pistessä $(a,b)$, jota kohti lähestyminen
$(x_n,y_n)\kohti(a,b)$ joukosta $\DF_f$ käsin ei ole mahdollista lisäehdolla
$(x_n,y_n)\neq(a,b)\ \forall n$. Määritelmässä \ref{kahden muuttujan jatkuvuus} tätä lisäehtoa
ei ole, joten jos $(a,b)\in\DF_f$ on $\DF_f$:n eristetty piste, niin $f$ on jatkuva pisteessä
$(a,b)$ (vrt.\ Harj.teht.\,\ref{jatkuvuuden käsite}:\,\ref{H-V-1: eristetty piste}).
\begin{Exa} \label{udif-1: esim 1} Määritellään funktiot
\[
f_1(x,y)=\frac{x^2y}{x^2+y^2}\,, \quad
f_2(x,y)=\frac{xy}{x^2+y^2}\,, \quad (x,y)\in\R^2,\ (x,y)\neq(0,0).
\]
Tutki, ovatko funktiot jatkuvia origossa, kun asetetaan $f_1(0,0)=f_2(0,0)=0$.
\end{Exa}
\ratk Polaarimuunnoksilla
\begin{align*}
&g_1(r,\varphi) \,=\, f_1(r\cos\varphi,r\sin\varphi) \,=\, r\cos^2\varphi\sin\varphi, \\
&g_2(r,\varphi) \,=\, f_2(r\cos\varphi,r\sin\varphi) \,=\, \cos\varphi\sin\varphi
\end{align*}
päätellään: Jos $(x_n,y_n) \vastaa (r_n,\varphi_n)$, niin
\[
r_n \kohti 0 \qimpl |f_1(x_n,y_n)| \,=\, |g_1(r_n,\varphi_n)|
                                   \,\le\, r_n \,\kohti\, 0 \,=\, f(0,0),
\]
joten $f_1$ on jatkuva pisteessä $(0,0)$. Sen sijaan $f_2$ on epäjatkuva origossa, sillä
$r_n \kohti 0\ \not\impl\ g_2(r_n,\varphi_n) \kohti 0$.  \loppu

Raja-arvoille ja jatkuvuudelle pätevät samanlaiset yhdistelysäännöt kuin yhden muuttujan 
tapauksessa, sillä näiden tulosten taustalla oleva logiikka ja algebra ei olennaisesti riipu 
muuttujien lukumäärästä. Jatkuvuuden yhdistelytulokset ovat seuraavat:
\index{jatkuvuusb@jatkuvuus (usean muuttujan)!b@yhdistelysäännöt|emph}%
\begin{Lause} \label{yhdistelylause 1} Jos $f:\ D_f \kohti \R,\ D_f \subset \R^2$, ja
$g:\ D_g \kohti \R,\ D_g \subset \R^2$, ovat jatkuvia pisteessä $(x,y) \in D_f \cap D_g$, niin 
myös $\lambda f\ (\lambda \in \R)$, $f+g$ ja $fg$ ovat jatkuvia pisteessä $(x,y)$. Jos lisäksi
$g(x,y) \neq 0$, niin myös $f/g$ on jatkuva pisteessä $(x,y)$. 
\end{Lause}
\begin{Lause} \label{yhdistelylause 2} Jos $f: D_f \kohti \R,\ D_f \subset \R^2$, on jatkuva 
pisteessä $(x,y) \in D_f$ ja $g: D_g \kohti \R,\ D_g \subset \R$, on jatkuva pisteessä
$f(x,y)\in\DF_g$, niin yhdistetty funktio $g \circ f$ on jatkuva pisteessä $(x,y)$. 
\end{Lause}

Määritelmän \ref{kahden muuttujan jatkuvuus} mukainen jatkuvuus yksittäisessä pisteessä
laajenee luonnollisella tavalla jatkuvuudeksi joukossa: $f$ on jatkuva joukossa
$A\subset\DF_f$, jos $f$ on jatkuva $A$:n jokaisessa pisteessä. Kuten yhden muuttujan
tapauksessa, tavanomaiset 'yhden lausekkeen funktiot' ovat jatkuvia koko määrittelyjoukossaan.
\jatko \begin{Exa} (jatko) Jos esimerkissä jätetään $f_1(0,0)$ ja $f_2(0,0)$ erikseen
määrittelemättä, niin esimerkin funktiot ovat Lauseen \ref{yhdistelylause 1} perusteella
jatkuvia koko yhteisessä määritelyjoukossaan
\[
\DF_{f_1} = \DF_{f_2} = \{(x,y)\in\R^2 \mid (x,y)\neq(0,0)\}.
\]
Funktiolla $f_1$ on origossa raja-arvo
\[
\lim_{(x,y)\kohti(0,0)}f_1(x,y)=0
\]
(Määritelmä \ref{kahden muuttujan raja-arvo}), joten $f_1$ saadaan origossa (ja siis koko
$\R^2$:ssa) jatkuvaksi asettamalla $f_1(0,0)=0$ (funktion jatkaminen!). Funktiolla $f_2$ ei
tätä raja-arvoa ole, joten $f_2$:n jatkaminen origoon ei ole mahdollista. \loppu
\end{Exa}
\begin{Exa} Rationaalifunktion $f(x,y)=p(x,y)/q(x,y)$ ($p$ ja $q$ polynomeja) määrittelyjoukko
on $\DF_f=\{(x,y)\in\R^2 \mid q(x,y) \ne 0\}$. Lauseen \ref{yhdistelylause 1} mukaan $p$ ja $q$
ovat jatkuvia koko määrittelyjoukossaan ($=\R^2$), samoin $f$. \loppu
\end{Exa}
\begin{Exa} Missä pisteissä funktio $f(x,y) = \sqrt{y-x^2}/(x-y^2)$ on a) määritelty,
b) jatkuva? 
\end{Exa}
\ratk a) Määrittelyjoukko on
$\,D_f = \{\,(x,y) \in \R^2 \mid y \ge x^2\ \ja\ x \neq y^2\,\}$. \newline 
b) Kyseessä on yhdistetty ja yhdistelty funktio $f = (g \circ f_1)/f_2$, missä 
\[ \begin{aligned} &f_1(x,y)=y-x^2,\ D_{f_1} = \R^2, \quad f_2(x,y)=x-y^2,\ D_{f_2} = \R^2, \\
                   &g(t) = \sqrt{t},\ D_g = [0,\infty) \subset \R. 
   \end{aligned} \]
Lauseiden \ref{yhdistelylause 1} ja \ref{yhdistelylause 2} perusteella $f$ on jatkuva koko 
määrittelyjoukossaan. \loppu 

Kuten yhden muuttujan tapauksessa, myös kahden muuttujan funktion jatkuvuus ja raja-arvo
voidaan määritellä vaihtoehtoisesti vetoamatta luku(pari)jonoi\-hin. Vaihtoehtoisessa
'$(\eps,\delta)$-määritelmässä' tarvitaan avointa väliä $(x-\delta,x+\delta)$ vastaava
\index{ympzy@($\delta$-)ympäristö}%
\kor{ympäristö}, tarkemmin \kor{pisteen avoin $\delta$-ympäristö}, jota merkitään
$U_\delta(x,y)$ ($\delta>0$). Kuten yhdessä ulottuvuudessa, tällä tarkoitetaan joukkoa, joka
ympäröi pistettä $\delta$:n verran, tai $\delta$:aan verrannollisen matkan, j\pain{oka}
\pain{suuntaan}. Tyypillisesti $U_\delta(x,y)$ ajatellaan joko neliön tai kiekon muotoiseksi,
eli
\begin{align*}
\text{joko}&: \quad U_\delta(x,y)=(x-\delta,x+\delta)\times(y-\delta,y+\delta), \\
 \text{tai}&: \quad U_\delta(x,y)=\{(x',y')\in\R^2 \mid (x'-x)^2+(y'-y)^2<\delta^2\}.
\end{align*}
Jatkuvuuden vaihtoehtoinen määritelmä on (vrt.\ Määritelmä \ref{vaihtoehtoinen jatkuvuus})
\begin{Def} \label{kahden muuttujan vaihtoehtoinen jatkuvuus}
\index{jatkuvuusb@jatkuvuus (usean muuttujan)!a@funktion $f(x,y)$|emph}
Funktio $f:\ \DF_f\subset\R^2\kohti\R$ on jatkuva pistessä $(x,y)\in\DF_f$ täsmälleen kun
jokaisella $\eps>0$ on olemassa $\delta>0$ siten, että
\[
|f(x',y')-f(x,y)| < \eps\ \ \forall (x',y') \in U_\delta(x,y)\cap\DF_f.
\]
\end{Def}
Määritelmät \ref{kahden muuttujan jatkuvuus} ja \ref{kahden muuttujan vaihtoehtoinen jatkuvuus}
voidaan osoittaa yhtäpitäviksi samaan tapaan kuin yhden muuttujan tapauksessa (vrt.\ Lause
\ref{jatkuvuuskriteerien yhtäpitävyys}).

Ympäristö $U_\delta(x,y)$ on nimensä mukaisesti esimerkki \kor{avoimesta} joukosta, jonka
yleisempi määritelmä on (vrt.\ Määritelmä \ref{analyyttinen funktio} joukoille $A\in\C$)
\begin{Def} \label{avoin joukko} \index{avoin joukko|emph } 
Joukko $A\subset\R^2$ on \kor{avoin}, jos jokaisella$(x,y) \in A$ on olemassa $\delta>0$ siten,
että $U_\delta(x,y) \subset A$.
\end{Def}
 
\subsection*{Funktio $f(x,y,z)$}

Määritelmillä \ref{kahden muuttujan jatkuvuus} ja \ref{kahden muuttujan raja-arvo} on ilmeiset
vastineensa kolmen muuttujan funktioille $f(x,y,z)$. Esimerkiksi jatkuvuusehto pisteessä
$(x,y,z)\in\DF_f$ on \index{jatkuvuusb@jatkuvuus (usean muuttujan)!c@funktion $f(x,y,z)$}%
\[
(x_n,y_n,z_n)\in\DF_f\,\ \forall n\,\ \ja\,\ (x_n,y_n,z_n)\kohti(x,y,z)\,\
                                      \impl\,\ f(x_n,y_n,z_n) \kohti f(x,y,z),
\]
missä (vrt.\ Harj.teht.\,\ref{jonon raja-arvo}:\,\ref{H-I-6: lukujonopari}e)
\begin{align*}
(x_n,y_n,z_n)\kohti(x,y,z) &\qekv x_n \kohti x\ \ja\ y_n \kohti y\ \ja\ z_n \kohti z \\
                           &\qekv (x_n-x)^2+(y_n-y)^2+(z_n-z)^2 \kohti 0.
\end{align*}
Erityisesti jos $(x,y,z)=(0,0,0)$, niin pallokoordinaattimuunnoksen $(x_n,y_n,z_n)$
$\vastaa(r_n,\theta_n\varphi_n)$ perusteella on
\[
(x_n,y_n,z_n)\kohti(0,0,0) \qekv r_n \kohti 0 \qquad \text{(pallokoordinaatisto)}.
\]
\begin{Exa} Täsmälleen millä ehdoilla luvuille $\alpha,\beta,\gamma\in\R$ on
\[
\lim_{(x,y,z)\kohti(0,0,0)}f(x,y,z)=0, \quad \text{kun} \quad
f(x,y,z)=\frac{|x|^\alpha|y|^\beta|z|^\gamma}{x^2+y^2+z^2}\,?
\]
\end{Exa}
\ratk Tehdään pallokoordinaatimuunnos:
\[
f(x,y,z) = g(r,\theta,\varphi) = r^{\alpha+\beta+\gamma-2}|
                                 \sin\theta|^{\alpha+\beta}|\cos\theta|^\gamma
                                 |\cos\varphi|^\alpha|\sin\varphi|^\beta.
\]
Päätellään, että $\,r_n \kohti 0\ \impl\ g(r_n,\theta_n,\varphi_n) \kohti 0\,$ on tosi
täsmälleen ehdoilla
\[
\alpha+\beta+\gamma > 2,\,\ \alpha \ge 0,\,\ \beta \ge 0,\,\ \gamma \ge 0. \loppu
\]

\subsection*{Funktio $f(\mx),\  \mx\in\R^n$}

Yleiselle $n$ muuttujan reaaliarvoiselle funktiolle $f$ jatkuvuuseehto pisteessä
$\mx\in\DF_f$ on
\index{jatkuvuusb@jatkuvuus (usean muuttujan)!d@funktion $f(\mx),\ \mx\in\R^n$}%
\[
\mx_k\in\DF_f\,\ \forall k\,\ \mx_k \kohti \mx\,\ \impl\,\ f(\mx_k) \kohti f(\mx),
\]
\index{suppeneminen!ac@$\R^n$:n jonon} \index{raja-arvo!e@$\R^n$:n jonon}%
missä $\seq{\mx_k}$ on $\R^n$:n vektorijono ja \kor{suppeneminen} $\mx_k \kohti \mx$ kohti
\kor{raja-arvoa} $\mx\in\R^n$ tarkoittaa:
\[
\mx_k \kohti \mx \qekv (\mx_k)_i \kohti (\mx)_i=x_i\,,\,\ i=1 \ldots n
                 \qekv |\mx_k-\mx| \kohti 0,
\]
missä $|\cdot|$ on $\R^n$:n euklidinen normi. Määritelmien
\ref{kahden muuttujan vaihtoehtoinen jatkuvuus} ja \ref{avoin joukko} $n$-ulotteisissa
vastineissa ympäristö $U_\delta(\mx)$ tulkitaan joko $n$-ulotteiseksi kuutioksi, jonka särmän
pituus $=2\delta$, tai $n$-ulotteiseksi kuulaksi, jonka säde $=\delta$.

\subsection*{*Jatkuvuus kompaktissa joukossa}

Asetetaan (vrt.\ Määritelmät \ref{avoimet ym. joukot} ja \ref{jatkuvuus kompaktissa joukossa})
\begin{Def} \label{kompakti joukko - Rn} Joukko $K\subset\R^n$ on
\begin{itemize} \index{suljettu joukko|emph} \index{rajoitettu!b@joukko|emph}
\index{kompakti joukko|emph}
\item[-] \kor{suljettu}, jos kaikille $\R^n$:n vektorijonoille $\seq{\mx_k}$ pätee: \newline
         $\,\mx_k \in K\ \forall k\ \ja\ \mx_k \kohti \mx\in\R^n\ \impl\ \mx \in K$,
\item[-] \kor{rajoitettu}, jos $\exists C\in\R_+$ siten, että 
         $\,|\mx| \le C\ \forall \mx \in K$,
\item[-] \kor{kompakti}, jos $K$ on suljettu ja rajoitettu.
\end{itemize}
\end{Def}
Suljetun joukon vaihtoehtoinen määritelmä on (vrt.\ Lause \ref{avoin vs suljettu})

\begin{Lause} \label{avoin vs suljettu - Rn} \index{komplementti (joukon)|emph}
Joukko $A\subset\R^n$ on suljettu täsmälleen kun $A$:n \kor{komplementti}
$\complement(A)=\{\mx\in\R^n \mid \mx \not\in A\}$ on avoin.
\end{Lause}
\tod Harj.teht.\,\ref{H-udif-1: avoin vs suljettu}.
\begin{Exa} Kompakteja joukkoja $K\subset\R^n$ ovat esimerkiksi (suljettu) $n$-ulot\-teinen
suorakulmainen särmiö
\[
K=[a_1,b_1]\times[a_2,b_2]\times\ldots\times[a_n,b_n]
\]
ja (suljettu) $n$-ulotteinen kuula
\[
K = \{\mx\in\R^n \mid |\mx-\mx_0| \le R\}.
\]
Myös jokainen äärellinen $\R^n$:n osajoukko on kompakti
(Harj.teht.\,\ref{H-udif-1: äärellinen joukko}). \loppu
\end{Exa}
Kompaktin joukon käsite on keskeinen seuraavassa Weierstrassin lauseen
\ref{Weierstrassin peruslause} yleistyksessä, joka tapuksessa $n=1$ on todistettu aiemmin
Lauseena \ref{weierstrass}. Yleisempää todistusta ei esitetä; todettakoon ainoastaan, että
todistuksen logiikka on hyvin samanlainen kuin yhden muuttujan tapauksessa, ks.\ Luku
\ref{jatkuvuuden logiikka}. Asetetaan ensin (vrt.\ Määritelmä
\ref{jatkuvuus kompaktissa joukossa})
\begin{Def} \label{jatkuvuus kompaktissa joukossa - Rn} 
\index{jatkuvuusb@jatkuvuus (usean muuttujan)!e@kompaktissa joukossa|emph}
Funktio $f:\ \DF_f\subset\R^n\kohti\R$ on \kor{jatkuva kompaktissa joukossa} $K\subset\DF_f$,
jos jokaiselle $\R^n$:n vektorijonolle $\seq{\mx_k}$ pätee
\[
\mx_k \in K\ \forall k\,\ \ja\,\ \mx_k \kohti \mx\,\ \impl\,\ f(\mx_k) \kohti f(\mx).
\]
\end{Def}
\begin{*Lause} \label{weierstrass - Rn} \index{Weierstrassin lause|emph}
Jos funktio $f:\ \DF_f\subset\R^n\kohti\R$ on jatkuva
kompaktissa joukossa $K\subset\DF_f$, niin $f$ saavuttaa $K$:ssa pienimmän ja suurimman arvonsa.
\end{*Lause}

\Harj
\begin{enumerate}

\item
Kohdissa a)--h) määritä raja-arvo tai päättele, ettei raja-arvoa ole. Kohdissa i)--j) määritä
kaikki kertoimien $a,b,c$ arvot, joilla raja-arvo on olemassa. Siirtyminen napa- tai 
pallokoordinaatistoon voi auttaa.
\begin{align*}
&\text{a)}\ \lim_{(x,y)\kohti(2,-1)} (xy+y^2) \qquad 
 \text{b)}\ \lim_{(x,y)\kohti(0,0)} \sqrt{x^2+2y^2} \qquad
 \text{c)}\ \lim_{(x,y)\kohti(0,0)} \frac{x^2+y^2}{y} \\
&\text{d)}\ \lim_{(x,y)\kohti(0,0)}\,\frac{x}{x^2+y^2} \qquad\ 
 \text{e)}\ \lim_{(x,y)\kohti(0,0)}\,\frac{y^3}{x^2+y^2} \qquad\quad\ \
 \text{f)}\ \lim_{(x,y)\kohti(0,0)}\,\frac{x^2y^2}{x^2+y^4} \\
&\text{g)}\ \lim_{(x,y)\kohti(0,0)}\,\frac{\sin(xy)}{x^2+y^2} \qquad\ 
 \text{h)}\ \lim_{(x,y,z)\kohti(0,0,0)}\,\frac{\sin(xyz)}{x^2+y^2+z^2} \\
&\text{i)}\ \lim_{(x,y)\kohti(0,0)}\,\frac{xy}{ax^2+bxy+cy^2} \qquad
 \text{j)}\ \lim_{(x,y,z)\kohti(0,0,0)}\ \frac{x^2+y^2-z^2}{ax^2+by^2+cz^2}
\end{align*}

\item
Kohdissa a)--i) jatka funktion $f$ määritelmä niin, että funktiosta tulee jatkuva
mahdollisimman suuressa $\R^2$:n tai $\R^3$:n osajoukossa. Kohdissa j)--m) määritä kaikki
pisteet, joissa $f$ on epäjatkuva. 
\begin{align*}
&\text{a)}\ \ f(x,y)=\frac{x\sqrt[4]{\abs{y}}}{|x|+|y|} \qquad
 \text{b)}\ \ f(x,y)=\frac{\sqrt{(1+x^2)(1+y^2)}-1}{x^2+y^2} \\
&\text{c)}\ \ f(x,y)=\frac{x^3-8y^3}{x-2y} \qquad
 \text{d)}\ \ f(x,y)=\frac{x+y}{x^3+y^3} \qquad
 \text{e)}\ \ f(x,y)=\frac{x^2+y}{x^4-y^2} \\
&\text{f)}\ \ f(x,y)=\frac{\sin(x+y)}{x^2-y^2} \qquad
 \text{g)}\ \ f(x,y)=\frac{\sin(x^3+y^3)}{x+y} \\
&\text{h)}\ \ f(x,y,z)=\frac{1-\cos(xyz)}{x^2y^2z^2} \qquad
 \text{i)}\ \ f(x,y,z)=\frac{x^2-y^2+z^2+2xz}{x+y+z} \\
&\text{j)}\ \ f(x,y)=\begin{cases}
              4x^2+y^2-47, &\text{kun}\ x^2+y^2<25 \\ x^2-2y^2+28, &\text{kun}\ x^2+y^2 \ge 25
              \end{cases} \\
&\text{k)}\ \ f(x,y)=\begin{cases}
                     x^2+xy-2y^2, &\text{kun}\ 0<\abs{y}<x \\ 0, &\text{muulloin}
                     \end{cases} \\
&\text{l)}\ \ f(x,y)=
 \begin{cases}
 y\sin\left(\dfrac{1}{x}+\dfrac{1}{y}\right), &\text{kun}\ x \neq 0\,\ \text{ja}\,\ y \neq 0 \\
 0, &\text{kun}\ x=0\,\ \text{tai}\,\ y=0
 \end{cases} \\
&\text{m)}\ \ f(x,y,z)=\begin{cases}
                       \dfrac{\sin(xyz)}{\sin(xy)}\,, &\text{kun}\ xy \neq 0 \\
                       z, &\text{kun}\ xy=0
                       \end{cases}
\end{align*}

\item (*) \label{H-udif-1: kiero raja-arvo}
Määritellään $f:\,\Rkaksi\map\R$ seuraavasti:
\[
f(x,y)=\begin{cases} 
       \,1, &\text{kun}\ x^2<y<2 x^2\\ \,0, &\text{muulloin}
       \end{cases}
\]
Osoita, että funktiolla on sama raja-arvo origossa lähestyttäessä origoa mitä
tahansa suoraa $S:\ ax+by=0\ (a,b\in\R)$ pitkin, mutta siitä huolimatta ei ole olemassa
raja-arvoa $\,\lim_{(x,y)\to (0,0)} f(x,y)$.

\item (*) \label{H-udif-1: avoin vs suljettu}
Todista Lause \ref{avoin vs suljettu - Rn}.

\item (*)
Olkoon $f(x,y)=p(x,y)/q(x,y)$, missä $p$ ja $q$ ovat määriteltyjä ja jatkuvia koko $\R^2$:ssa
(esim.\ polynomeja). \vspace{1mm}\newline
a) Näytä suoraan Lauseen \ref{kahden muuttujan vaihtoehtoinen jatkuvuus} avulla, että
$\DF_f\subset\R^2$ on avoin. \vspace{1mm}\newline
b) Päättele $\DF_f$ avoimeksi osoittamalla ensin Määritelmään \ref{kompakti joukko - Rn}
nojaten, että $\{(x,y)\in\R^2 \mid q(x,y)=0\}$ on suljettu joukko ja vetoamalla Lauseeseen
\ref{avoin vs suljettu - Rn}.

\item (*) \label{H-udif-1: äärellinen joukko}
Olkoon $\ma_i\in\R^n,\ i=1 \ldots m\,\ (m\in\N)$, $\ma_i\neq\ma_j$ kun $i \neq j$, ja
$\,K=\{\ma_1,\ldots,\ma_m\}$. \vspace{1mm}\newline
a) Näytä, että $K\subset\R^n$ on kompakti joukko. \vspace{1mm}\newline
b) Näytä, että jos $f:\ \DF_f\subset\R^n\kohti\R$ on mikä tahansa funktio, jolle pätee
$\ma_i\in\DF_f,\ i=1 \ldots m$, niin $f$ on jatkuva $K$:ssa Määritelmän
\ref{jatkuvuus kompaktissa joukossa - Rn} mukaisesti. \vspace{1mm}\newline
c) Tarkista, että Lauseen \ref{weierstrass - Rn} väittämä on sopusoinnussa b)-kohdan tuloksen
kanssa.

\end{enumerate}