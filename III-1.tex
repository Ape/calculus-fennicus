\section{Osoitinkunta} \label{osoitinkunta}
\alku
\index{osoitinkunta|vahv}
\index{laskuoperaatiot!d@osoittimien|vahv}

Lähtökohtana on euklidinen taso ja siihen pystytetty karteesinen koordinaatisto,
koordinaatteina $(x,y)$. Kutsutaan tällä kertaa \kor{osoittimeksi} suuntajanaa, jolla on 
tietty pituus ja suunta. Pituutta merkitään symbolilla $r$ ja suunta mitataan $x$-akselin
suunnasta vastapäivään kiertäen
\index{vaihekulma}%
\kor{vaihekulmalla} (napakulmalla) $\varphi$. Merkitään
\[
\ptp = r \vkulma{\varphi}\,.
\]
\begin{figure}[H]
\begin{center}
\import{kuvat/}{kuvaII-25.pstex_t}
\end{center}
\end{figure}
Osoitin on siis toistaiseksi täsmälleen samanlainen olio kuin vektori: sillä on pituus ja 
suunta. Osoittimet myös samastetaan kuten vektorit. On huomioitava ainoastaan, että
$\varphi$ ja $\varphi+2\pi$ vastaavat samaa suuntaa, joten samastussäännöt ovat
\index{samastus '$=$'!f@osoittimien}% 
\[
r_1\vkulma{\varphi_1}=r_2\vkulma{\varphi_2} \qekv
\begin{cases}
\,r_1=r_2=0\ \ \text{tai} \\ 
\,r_1=r_2>0\,\ \wedge\,\ \varphi_1-\varphi_2=k \cdot 2\pi,\ \ k\in\Z.
\end{cases}
\]
Tapauksessa $r=0$ samastetaan siis kaikki suunnat, kuten nollavektorissa. Osoitinta
kutsutaankin tässä tapauksessa
\index{nollaosoitin}%
\kor{nollaosoittimeksi} ja merkitään
\[
0 \vkulma{\varphi} = \pointer{0}.
\]
Osoittimien joukkoa merkitään jatkossa $\Pkunta$:llä:
\[
\Pkunta=\{\ptp = r\vkulma{\varphi} \mid r\in[0,\infty),\ \varphi\in\R\}.
\]

Osoittimien yhteenlaskuoperaatio $(+)$ määritellään samalla tavoin kuin vektoreille, eli
kolmiodiagrammin avulla:
\begin{figure}[H]
\begin{center}
\import{kuvat/}{kuvaII-26.pstex_t}
\end{center}
\end{figure}
Myös skalaareilla eli reaaliluvuilla kertominen suoritetaan samoin kuin vektoreilla.
Erityisesti jos $\lambda \in \R$ ja $\lambda > 0$, niin
\[
\lambda(r \vkulma{\varphi}) = \lambda r \vkulma{\varphi}\,, \quad \lambda > 0.
\]
Tapauksessa $\lambda=0$ on kertolaskun tulos nollaosoitin. Jos $\lambda \in \R$ ja 
$\lambda < 0$, niin määritellään (vrt. vektorit)
\begin{align*}
\lambda (r \vkulma{\varphi}) &= \abs{\lambda} \cdot (-1) (r \vkulma{\varphi}) \\
                             &= \abs{\lambda} r \vkulma{(\varphi + \pi)}\,, \quad \lambda < 0.
\end{align*}
Tässä $r\vkulma(\varphi + \pi)$ on yhteenlaskun määritelmän mukaisesti osoittimen 
$\ptp = r \vkulma{\varphi}\,$
\index{vastaosoitin}%
\kor{vastaosoitin}:
\[
-\ptp = (-1)\ptp = r\vkulma{(\varphi + \pi)}\,.
\]
Tähänastisen perusteella osoittimien yhteenlaskusta ja skaalauksesta muodostuva algebra 
$(\Pkunta,+,\R)$ näyttää yksinkertaisesti tason vektoriavaruudelta. Näin onkin toistaiseksi,
ja vektorianalogia voidaan viedä hiukan pidemmällekin: Otetaan käyttöön tason vektoriavaruuden
ortonormeerattua kantaa $\{\vec i, \vec j\}$ vastaava osoitinkanta, jossa kantaosoittimia
merkittäköön
\begin{multicols}{2} \raggedcolumns
\begin{align*}\vec i\ \vast\ \ptr &= 1 \vkulma{0}, \\
\vec j\ \vast\ \pti &= 1 \vkulma{\pi/2}.
\end{align*}
\begin{figure}[H]
\setlength{\unitlength}{1cm}
\begin{center}
\begin{picture}(2,2.5)(0,0)
\put(0,0){\vector(1,0){2}} \put(1.75,-0.5){$\ptr$}
\put(0,0){\vector(0,1){2}} \put(0.2,1.6){$\pti$}
\end{picture}
\end{center}
\end{figure}
\end{multicols}
Jos nyt $\ptp = r \vkulma{\varphi} \in \Pkunta$, niin $\ptp$ voidaan ilmaista kantaosoittimien 
avulla yksikäsitteisesti muodossa
\begin{align} \label{vaellus1}
\ptp = r \vkulma{\varphi}\, &=\, r\cos{\varphi}\,\ptr + r\, \sin{\varphi}\,\pti \notag \\ 
                            &=\, x \ptr + y \pti.
\end{align}
\begin{figure}[H]
\begin{center}
\import{kuvat/}{kuvaII-27.pstex_t}
\end{center}
\end{figure}
Toisaalta jos tunnetaan $\ptp$:n koordinaatit $x,y$ osoitinkannassa, niin esitysmuoto 
$\ptp = r \vkulma{\varphi}$ saadaan lasketuksi kaavoilla
\begin{equation} \label{vaellus2}
r=\sqrt{x^2+y^2}\,, \quad \begin{cases}
                          \,\cos{\varphi} = x/r, \\ \,\sin{\varphi} = y/r.
                          \end{cases}
\end{equation}
Tämän mukaan $\varphi$ määräytyy $2\pi$:n monikertaa vaille yksikäsitteisesti, jos $r>0$
(eli $(x,y)\neq(0,0)$), joten osoitin määräytyy sovittujen samastussääntöjen nojalla
yksikäsitteisesti. Jatkossa käytetään osoittimen eri esityismuodoista nimityksiä
\begin{align*}
\ptp = r \vkulma{\varphi} \qquad\qquad &\text{\kor{polaarimuoto} (polaariesitys)}, \\
\ptp = x \ptr + y \pti    \qquad       &\text{\kor{komponenttimuoto}}.
\end{align*} 
Siirtyminen esitysmuodosta toiseen tapahtuu siis säännöillä \eqref{vaellus1} ja
\eqref{vaellus2}. Koska komponenttiesitys luo kääntäen yksikäsitteisen vastaavuuden
$\Pkunta \vast \Rkaksi$, niin osoittimia voidaan ajatella abstraktisti myös reaalilukujen
pareina:
\[
\ptp = x\ptr+y\pti\ \vast\ (x,y)\in\Rkaksi.
\]

Osoittimen komponenttimuoto on erityisen kätevä yhteenlaskussa:
\[
\boxed{\begin{aligned}
\quad \ptp_1 &= r_1 \vkulma{\theta_1} \ \ja \ \ptp_2 = r_2 \vkulma{\theta_2} \\
             &\impl\ \ptp_1 + \ptp_2 = (r_1 \cos{\theta_1} + r_2 \cos{\theta_2})\,\ptr 
                                     + (r_1 \sin{\theta_1} + r_2 \sin{\theta_2})\,\pti.
                                                                            \akehys\quad
\end{aligned} } \]
Polaarimuotoon päästään tästä takaisin säännöillä (\ref{vaellus2}) --- lopputulosta ei
yleisessä muodossa kannata kirjoittaa.

Tähän asti osoitinavaruuden ja tason vektoriavaruuden välinen analogia on täydellinen.
Tultaessa osoittimien \pain{kertolaskuun} tiet kuitenkin eroavat. Osoittimille ei määritellä
skalaarituloa eikä ristituloakaan, vaan toisen tyyppinen tulo, jota jatkossa sanotaan
\index{osoitintulo}%
\kor{osoitintuloksi}. Määritelmä on:
\[
\boxed{\kehys\quad \ptp_1 = r_1 \vkulma{\varphi_1} \ \ja \ \ptp_2 = r_2 \vkulma{\varphi} \qimpl 
                 \ptp_1\cdot\ptp_2 = r_1 r_2 \vkulma{(\varphi_1 + \varphi_2)}. \quad}
\]
Osoitintulo on siis funktio tyyppiä $\,\Pkunta \times \Pkunta \kohti \Pkunta$ --- kyse on
hieman omalaatuisesta reaalilukujen kertolaskun (pituudet) ja yhteenlaskun (suunnat)
yhdistelmästä (vrt.\ Harj.teht.\,\ref{kunta}:\ref{H-I-2: Big Ben}).

Osoitintulon määritelmästä seuraa suoraan normaali vaihdantalaki
\[
\ptp_1\ptp_2 = \ptp_2\ptp_1
\]
(tässä jätetty kertomerkki pois tavalliseen tapaan). Myös liitäntälaki on voimassa
(Harj.teht.\,\ref{H-III-1: osoitintulon liitäntälaki}). Lisäksi nähdään, että
\index{ykkösosoitin}%
\kor{ykkösosoitin}, eli kertolaskun ykkösalkio, on
\[
\pointer{1} = \ptr = 1 \vkulma{0}\,.
\]
Jokaisella $\ptp \in \Pkunta, \ \ptp \neq \pointer{0}$, on myös
\index{kzyzy@käänteisosoitin}%
\kor{käänteisosoitin} $\ptpinv$, joka toteuttaa
\[
\ptp\cdot\ptpinv = \pointer{1} = 1 \vkulma{0}\,.
\]
Nimittäin jos $\ptp = r \vkulma{\varphi},\ r \neq 0$, niin 
\begin{multicols}{2} \raggedcolumns
\[
\ptpinv = r^{-1}\vkulma{-\varphi}\,.
\]
\begin{figure}[H]
%\begin{center}
\import{kuvat/}{kuvaII-28.pstex_t}
%\end{center}
\end{figure}
\end{multicols}
Osoitintulon määritelmästä seuraa välittömästi, että jokaisella $n\in\N$ pätee potenssiin 
korotuksen laskusääntö
\[
\boxed{\kehys\quad (\ptp)^n = r^n\vkulma{n\varphi}. \quad}
\]
Tapauksessa $\ptp\neq\pointer{0}$ tämä on pätevä jokaisella $n\in\Z$, kun sovitaan normaaliin 
tapaan, että
\[
(\ptp)^0=\pointer{1}, \quad\ (\ptp)^{-n}=[\ptpinv]^n,\ \ n\in\N \quad (\ptp\neq\pointer{0}).
\]
\begin{Exa} Määritä osoitinyhtälön $(\ptp)^3=-\ptp$ kaikki ratkaisut.
\end{Exa}
\ratk Osoitintulon määritelmän ja samastussääntöjen nojalla päätellään
\begin{align*}
      &(\ptp)^3=-\ptp \\[2mm]
\qekv &r^3\vkulma{3\varphi} \,=\, r\vkulma{(\varphi+\pi)} \\[2mm]
\qekv &r^3=r=0 \quad \text{tai} \quad 
       r^3=r>0\ \ \ja\ \ 3\varphi=(\varphi+\pi)+k \cdot 2\pi,\,\ \ k\in\Z \\[1mm]
\qekv &r=0 \quad \text{tai} \quad r=1\ \ \ja\ \ \varphi=\frac{\pi}{2}+k\cdot\pi,\ \ k\in\Z.
\end{align*}
Tapauksessa $r=1$ saadaan erilaisia osoittimia vain $k$:n arvoilla $0,1$, joten ratkaisut
ovat
\[
\ptp=\pointer{0},\ \ptp=\pm\pti. \loppu
\]

Osoitintulon ja osoittimien yhteenlaskun määritelmistä seuraa vielä, että pätee myös
osittelulaki
\[
\ptp \cdot (\ptq_1 + \ptq_2) = \ptp \cdot \ptq_1 + \ptp \cdot \ptq_2\,.
\]
Tämä on seuraus siitä, että jos $\ptp = r \vkulma{\varphi}$, niin kertolasku $\ptp \cdot \ptq$
voidaan kirjoittaa muotoon
\[
\ptp \cdot \ptq = r (1\vkulma{\varphi}) \cdot \ptq.
\]
Tässä osoittimella $(1\vkulma{\varphi})$ kertominen on sama kuin kierto kulman $\varphi$
verran. Sekä tälle operaatiolle että skalaarilla $r$ kertomiselle pätee yhteenlaskun suhteen 
osittelulaki (vrt.\ ristitulon vastaavan osittelulain todistus Luvussa \ref{ristitulo}), joten 
väitetty osittelulaki seuraa.

Kun nyt $\Pkunta$:ssä on tullut määritellyksi sekä yhteenlasku että kertolasku, jotka 
toteuttavat normaalit vaihdanta-, liitäntä- ja osittelulait, ja lisäksi on konstruoitu
yhteenlaskun nolla-alkio ja vasta-alkio sekä kertolaskun ykkösalkio ja käänteisalkio, niin on 
tullut osoitetuksi, että $(\Pkunta,+,\cdot)$ on itse asiassa \pain{kunta}. Nähdään siis, että 
osoitintulon mukaan ottaminen muuttaa osoitinalgebran varsin radikaalisti. Osoittimia voikin
luonnehtia 'pyöriviksi luvuiksi', joita vain lasketaan yhteen kuten vektoreita. --- Erillistä
skalaarilla kertomisoperaatiota ei osoitintulon määrittelyn jälkeen enää tarvita, sillä
skalaarin $\lambda\in\R$ voi tulkita osoittimeksi
\[
\pointer{\lambda} \,=\, \begin{cases}
                        \,\lambda \vkulma{0},        &\text{jos }\ \lambda \ge 0, \\
                        \,\abs{\lambda}\vkulma{\pi}, &\text{jos }\ \lambda <0,
                        \end{cases}
\]
jolloin skaalaus on osoitintulon erikoistapaus: $\lambda\ptp=\pointer{\lambda}\ptp$.
 
Osoitintulolle saadaan varsin yksinkertainen esitystapa myös komponenttimuodossa. Nimittäin
jos
\begin{align*}
\ptp_1 &= r_1 \cos{\varphi_1} \ptr + r_1 \sin{\varphi_1} \pti = x_1 \ptr + y_1 \pti, \\
\ptp_2 &= r_2 \cos{\varphi_2} \ptr + r_2 \sin{\varphi_2} \pti = x_2 \ptr + y_2 \pti,
\end{align*}
niin osoitintulon määritelmästä ja trigonometristen funktioiden yhteenlaskukaavoista 
(ks.\ Luku \ref{trigonometriset funktiot}) seuraa
\begin{align*}
\ptp_1 \cdot \ptp_2 &= r_1r_2 \cos({\varphi_1 + \varphi_2})\,\ptr 
                                + r_1r_2 \sin({\varphi_1 + \varphi_2})\,\pti \\
                    &= r_1r_2 (\cos{\varphi_1} \cos{\varphi_2} 
                                - \sin{\varphi_1} \sin{\varphi_2})\,\ptr \\
                    & \qquad + r_1r_2 (\sin{\varphi_1} \cos{\varphi_2} 
                                + \cos{\varphi_1} \sin{\varphi_2})\,\pti \\
                    &= (x_1x_2-y_1y_2)\ptr + (x_1y_2 + y_1x_2)\pti.
\end{align*}
Näin ollen osoitintuloa vastaa $\Rkaksi$:ssä määritelty tulo
\begin{equation} \label{kertolasku}
\boxed{\kehys\quad (x_1,y_1) \cdot (x_2,y_2) = (x_1x_2-y_1y_2,\,x_1y_2 + y_1x_2). \quad}
\end{equation}
Yhdessä jo aiemmin määritellyn (vektorien) yhteenlaskun kanssa, ts.
\begin{equation} \label{yhteenlasku}
\boxed{\kehys\quad (x_1,y_1) +(x_2,y_2) = (x_1 + x_2,\,y_1 + y_2) \quad}
\end{equation}
on lukupareista saatu aikaan algebra $(\Rkaksi,+,\cdot)$, joka siis on kunta (!). Kunnan
$(\Rkaksi,+,\cdot)\,$ nolla- ja ykkösalkiot ovat
\[
(0,0)\vastaa\pointer{0}, \quad (1,0)\vastaa\pointer{1}.
\]

\Harj
\begin{enumerate}

\item \label{H-III-1: osoitintulon liitäntälaki}
Todista osoitintulon liitäntälaki $\,(\ptp_1\ptp_2)\ptp_3=\ptp_1(\ptp_2\ptp_3)$.

\item
Näytä suoraan osoitintulon määritelmästä, että pätee: \newline
a) \ $-\ptp=(-\pointer{1})\ptp \quad$ 
b) \ $\pointer{0}\ptp=\pointer{0} \quad$
c) \ $\ptp\ptq=\pointer{0}\ \impl\ \ptp=\pointer{0}\ \tai\ \ptq=\pointer{0}$

\item
Määritä seuraavien osoitinyhtälöiden kaikki ratkaisut: \newline
a) \ $(\ptp)^2=-\pointer{1} \quad$ 
b) \ $2(\ptp)^3=\pointer{1} \quad$
c) \ $(\ptp)^4+3\ptp=\pointer{0} \quad$
d) \ $(\ptp)^4=4\vkulma\pi$

\item
Olkoon $(x,y)\in\Rkaksi,\ (x,y) \neq (0,0)$. Lähtien lukuparien tulon määritelmästä 
määritä $(a,b)\in\Rkaksi$ siten, että $(x,y)\cdot(a,b)=(1,0)$, ts.\ $(a,b)=(x,y)^{-1}$.
 
\end{enumerate}