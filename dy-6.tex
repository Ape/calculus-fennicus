\section{*Yleinen 2.\ kertaluvun lineaarinen DY} \label{2. kertaluvun lineaarinen DY}
\sectionmark{\ Yleinen 2. kertaluvun lineaarinen DY}
\alku
\index{lineaarinen differentiaaliyhtälö!d@2.\ kertaluvun|vahv}

Yleinen toisen kertaluvun lineaarinen differentiaaliyhtälö on muotoa
\begin{equation} \label{lin-2: ty}
y''+P(x)y'+Q(x)y=R(x).
\end{equation}
Perusoletus on, että sekä
\index{kerroin (DY:n)} \index{oikea puoli (DY:n)}%
\kor{kerroinfunktiot} $P$ ja $Q$ että yhtälön \kor{oikea puoli} $R$ 
ovat jatkuvia tarkasteltavalla välillä $(a,b)$. Ratkaisuna pidetään tällöin jokaista ko.\
välillä kahdesti derivoituvaa funktiota $y(x)$, joka toteuttaa yhtälön. --- Yhtälöstä on
tällöin luettavissa, että myös $y''$ on jatkuva välillä $(a,b)$. Jatkossa tarkastellaan aluksi
yhtälön \eqref{lin-2: ty} homogeenista erikoistapausta, jossa $R(x)=0$.

\subsection*{Homogeeninen yhtälö: Ratkaisuavaruus}
\index{lineaarinen differentiaaliyhtälö!a@homogeeninen|vahv}
\index{ratkaisuavaruus (DY:n)|vahv}

Kun homogeeninen differentiaaliyhtälö kirjoitetaan
\begin{equation} \label{lin-2: hy}
\dyf y=y''+P(x)y'+Q(x)y=0,
\end{equation}
niin operaattori $\dyf$ on jälleen lineaarinen:
\index{differentiaalioperaattori!b@lineaarisen DY:n}%
\[
\dyf(c_1y_1+c_2y_2)=c_1\dyf y_1+c_2\dyf y_2, \quad c_1,c_2\in\R.
\]
Siis jos $y_1$ ja $y_2$ ovat yhtälön \eqref{lin-2: hy} ratkaisuja, niin ratkaisu on myös mikä
tahansa näiden lineaarinen yhdistely
\[
y(x)=C_1y_1(x)+C_2y_2(x),\quad C_1,C_2\in\R.
\]
Ratkeavuusteorian keskeisin tulos on, että tämä itse asiassa on yhtälön \eqref{lin-2: hy}
yleinen ratkaisu edellyttäen ainoastaan, että funktiot $y_1$ ja $y_2$ ovat
\index{lineaarinen riippumattomuus}%
\kor{lineaarisesti riippumattomat} välillä $(a,b)$. Tällä tarkoitetaan, että pätee (vrt.\ Luku 
\ref{funktioavaruus})
\[
c_1y_1(x)+c_2y_2(x)=0 \quad \forall x\in (a,b) \ \impl \ c_1=c_2=0.
\]
\index{lineaarinen riippuvuus}%
Jos tämä ei päde, niin sanotaan, että $y_1$ ja $y_2$ ovat \kor{lineaarisesti riippuvat} välillä
$(a,b)$. Tässä tapauksessa on joko $y_2=cy_1$ tai $y_1=cy_2$ ko.\ välillä jollakin $c\in\R$. 
\begin{*Lause} \label{lin-2: lause 1} Jos differentiaaliyhtälössä \eqref{lin-2: hy} 
kerroinfunktiot $P$ ja $Q$ ovat jatkuvia välillä $(a,b)$, niin yhtälöllä on ko. välillä kaksi
lineaarisesti riippumatonta ratkaisua. Edelleen jos $y_1$ ja $y_2$ ovat mitkä tahansa kaksi
yhtälön \eqref{lin-2: hy} lineaarisesti riippumatonta ratkaisua välillä $(a,b)$, niin yleinen
ratkaisu ko.\ välillä on
\[
y(x)=C_1y_1(x)+C_2y_2(x),\quad C_1,C_2\in\R.
\]
\end{*Lause}
\begin{Exa} \label{lin-2: lause 1 - esim}
Eulerin differentiaaliyhtälön
\[
x^2y''-3xy'-3y=0
\]
perusmuodossa \eqref{lin-2: hy} kerroinfunktiot ovat $P(x)=-3/x$ ja $Q(x)=-3/x^2$. Nämä ovat
jatkuvia väleillä $(-\infty,0)$ ja $(0,\infty)$, ja näillä väleillä yleinen ratkaisu on
(ks.\ edellinen luku)
\[
y(x)=C_1x+C_2x^3, \quad C_1,C_2\in\R.
\]
Tulos on sopusoinnussa Lauseen \ref{lin-2: lause 1} kanssa, sillä $y_1(x)=x$ ja $y_2(x)=x^3$
ovat lineaarisesti riippumattomat jokaisella välillä $(a,b)$:
\[
c_1x+c_2x^3 = 0 \ \ \forall x \in (a,b) \qekv c_1=c_2 = 0. \loppu
\]
\end{Exa}
\jatko \begin{Exa} (jatko) Jos esimerkin differentiaaliyhtälöä $\,x^2y''-3xy'-3y=0$
tarkastellaan välillä $(-\infty,\infty)$, niin ratkaisuksi havaitaan myös
\[
y_3(x) = |x|^3 = \begin{cases} 
                 \,x^3, &\text{kun}\ x \ge 0, \\ \,-x^3, &\text{kun}\ x<0,
                 \end{cases}
\]
sillä tämä on kahdesti derivoituva ja toteuttaa yhtälön koko $\R$:ssä. Tätä funktiota ei
kuitenkaan voi ilmaista muodossa $C_1y_1(x)+C_2y_2(x)$, sillä $y_1(x)=x$ ja $y_2(x)=x^3$
(samoin niiden lineaariset yhdistelyt) ovat parittomia, kun taas $y_3$ on parillinen. --- Tulos
ei ole ristiriidassa Lauseen \ref{lin-2: lause 1} kanssa, sillä lauseen oletus
kerroinfunktioiden $P$ ja $Q$ jatkuvuudesta ei toteudu välillä $(-\infty,\infty)$. Huomattakoon
myös, että väleillä $(0,\infty)$ ja $(-\infty,0)$ on $y_3(x)=\pm y_2(x)$, joten näillä
väleillä $y_3$ ei tuo mitään uutta yleiseen ratkaisuun. \loppu
\end{Exa}
Lauseen \ref{lin-2: lause 1} tulos on tulkittavissa niin, että yhtälön \eqref{lin-2: hy}
ratkaisujen joukko $\mathcal{Y}$ on tehdyin oletuksin \pain{2--ulotteinen} \pain{vektoriavaruus}
(funktioavaruus, vrt.\ Luku \ref{funktioavaruus}). Tämän avaruuden \pain{kanta} on mikä tahansa
lineaarisesti riippumaton ratkaisupari $\{y_1,y_2\}$, jolloin koko ratkaisuavaruus voidaan
esittää muodossa
\[
\mathcal{Y}=\{y=c_1y_1+c_2y_2 \ | \ c_1,c_2\in\R\}.
\]
Lauseen \ref{lin-2: lause 1} (osittainen) todistus esitetään jäljempänä. Tätä ennen 
tarkastellaan lauseen seuraamuksia ajatellen yhtälön \eqref{lin-2: hy} ratkaisemista
kvadratuureilla.

\subsection*{Homogeeninen yhtälö: Ratkaiseminen kvadratuureilla}
\index{lineaarinen differentiaaliyhtälö!a@homogeeninen|vahv}

Ensinnäkin joudutaan toteamaan, että yhtälö \eqref{lin-2: hy} \pain{ei} aina ratkea
kvadratuureilla. Kvadratuureihin palautumaton on esimerkksi niinkin yksinkertainen yhtälö kuin
\[
y''=xy.
\]
Tämän huonon uutisen jälkeen todettakoon, että jos yhtälölle \eqref{lin-2: hy} on 
keksittävissä edes y\pain{ksi} ei-triviaali (eli nollasta poikkeava) ratkaisu $y_1$, niin toinen, 
$y_1$:stä lineaarisesti riippumaton ratkaisu --- ja niin muodoin yleinen ratkaisu --- on
konstruoitavissa kvadratuureilla. Jatkossa rajoitutaan tähän tapaukseen, eli oletetaan yksi
ratkaisu $y_1 \neq 0$ tunnetuksi.

Jos tunnetaan yhtälön \eqref{lin-2: hy} ratkaisu $y_1\neq 0$, niin toista ratkaisua voidaan 
etsiä tuttuun tapaan \pain{vakion} \pain{varioinnilla}, eli muodossa
\[
y(x)=C(x)y_1(x).
\]
Tällöin on 
\[
y' = y_1C'+y_1'C, \quad y'' = y_1C''+2y_1'C'+y_1''C.
\]
Koska $\dyf y_1=0$, seuraa $C$:lle differentiaaliyhtälö
\[
y_1(x)C''+[2y_1'(x)+P(x)y_1(x)]C'=0.
\]
Sijoituksella $u=C'$ tästä tulee separoituva:
\[
u'+K(x)u=0,\quad K(x)=2\,\frac{y_1'(x)}{y_1(x)}+P(x)=2\frac{d}{dx}\ln\abs{y_1(x)}+P(x).
\]
Separoimalla ja integroimalla saadaan
\begin{align*}
\int\frac{du}{u}  &= -\int K(x)\,dx \\
\ekv \ \ln\abs{u} &= -2\ln\abs{y_1(x)}-\int P(x)\,dx + A \quad (A\in\R).
\end{align*}
Valitsemalla $A=0$ (yksi ratkaisu $u\neq 0$ riittää!) saadaan
\[
u(x)=\frac{e^{-\int P(x)dx}}{[y_1(x)]^2}=C'(x).
\]
Näin ollen eräs ratkaisu on
\begin{align*}
y_2(x) &= y_1(x)\int u(x)\,dx \\
       &= y_1(x) \int \frac{e^{-\int P(x)dx}}{[y_1(x)]^2}\,dx.
\end{align*}
Ratkaisu on pätevä ainakin $y_1$:n nollakohtien välisillä avoimilla väleillä. Tällaisilla
väleillä suhde $u=y_2/y_1$ ei selvästikään ole vakio (koska on positiivisen funktion
integraalifunktio), joten löydetty ratkaisu on $y_1$:stä lineaarisesti riippumaton.
Homogeenisen yhtälön \eqref{lin-2: hy} yleinen ratkaisu on tällöin (ainakin mainituilla
väleillä) kirjoitettavissa
\[
y(x)=C_1y_1(x)+C_2y_2(x).
\]
\begin{Exa} \label{lin-2: Ex4}
Differentiaaliyhtälöllä
\[
y''+xy'-y=0
\]
on ilmeinen ratkaisu $\,y_1(x)=x$, joten etsitään toista muodossa
\[
y(x)=xC(x) \qimpl y'=xC'+C, \quad y''=xC''+2C'.
\]
Sijoitus yhtälöön antaa
\[
xu'+(x^2+2)u=0, \quad u=C'.
\]
Ratkaisu separoimalla (ol.\ $x>0$):
\begin{align*}
&\int\frac{du}{u} \,=\, \ln u
                  \,=\, -\int\left(x+\frac{2}{x}\right)dx 
                  \,=\, \ln\left(x^{-2}e^{-x^2/2}\right)+\ln A \\
&\qimpl u=C'=Ax^{-2}e^{-x^2/2}.
\end{align*}
Valitsemalla $A=1$, asettamalle ehto $\lim_{x\kohti\infty}C(x)=0$ ja integroimalla osittain
saadaan
\[
C(x) = -\int_x^\infty u(t)\,dt = -x^{-1}e^{-x^2/2}+\int_x^\infty e^{-t^2/2}\,dt.
\]
Yleinen ratkaisu on näin ollen
\[
y(x) \,=\, C_1x + C_2\,xC(x)
     \,=\, C_1x + C_2\left(-e^{-x^2/2}+x\int_x^\infty e^{-t^2/2}\,dt\right), \quad C_1,C_2\in\R. 
\]
Vaikka laskun välivaiheissa oletettiin $x>0$, on ratkaisu pätevä koko $\R$:ssä. \loppu
\end{Exa}

\subsection*{Homogeeninen yhtälö: ratkeavuusteoria}
\index{lineaarinen differentiaaliyhtälö!a@homogeeninen|vahv}

Palataan Lauseeseen \ref{lin-2: lause 1}, jota ei ole todistettu. Lauseen väittämistä
syvällisin koskee --- kuten tavallista --- ratkaisujen $y_1$ ja $y_2$ olemassaoloa. Jatkossa
johdetaan tämä tulos, samoin kuin Lauseen  \ref{lin-2: lause 1} muut väittämät, seuraavasta
peruslauseesta, joka puolestaan on erikoistapaus paljon yleisemmästä alkuarvotehtävän
ratkeavuutta koskevasta väittämästä. Väittämää ei tässä vaiheessa todisteta, vaan asiaan
palataan myöhemmin luvussa \ref{Picard-Lindelöfin lause}. Puheena olevaan erikoistapaukseen
sovellettuna ratkeavuusväittämä on seuraava: 
\begin{*Lause} \label{lin-2: lause 2} \index{differentiaaliyhtälön!h@ratkeavuus|emph} 
Jos $P$ ja $Q$ ovat jatkuvia välillä $(a,b)$ ja $x_0\in (a,b)$, niin alkuarvotehtävällä
\[
\left\{ \begin{aligned}
&y''+P(x)y'+Q(x)y=0,\quad x\in (a,b), \\
&y(x_0)\,=A, \\
&y'(x_0) =B
\end{aligned} \right.
\]
on yksikäsitteinen ratkaisu jokaisella $A,B\in\R$.
\end{*Lause}

Jatkossa siis todistetaan väittämä: \ 
$\text{Lause \ref{lin-2: lause 2}}\ \impl\ \text{Lause \ref{lin-2: lause 1}}$. Todistuksessa
näyttelee keskeistä roolia seuraava käsite.
\begin{Def} \index{Wronskin determinantti|emph} Välillä $(a,b)$ derivoituvien funktioiden
$y_1$ ja $y_2$ \kor{Wronskin determinantti} on
\[
W_{12}(x)=\begin{vmatrix}
y_1(x) & y_2(x) \\
y_1'(x) & y_2'(x)
\end{vmatrix} = (y_1y_2'-y_2y_1')(x), \quad x\in(a,b).
\]
\end{Def}
\begin{Prop} \label{lin-2: prop} Jos $y_1$ ja $y_2$ ovat derivoituvia välillä $(a,b)$ ja 
$W_{12}(x_0)=(y_1y_2'-y_2y_1')(x_0)\neq 0$ jollakin $x_0\in (a,b)$, niin $y_1$ ja $y_2$ ovat
lineaarisesti riippumattomat välillä $(a,b)$.
\end{Prop}
\tod Jos joillakin $c_1,c_2\in\R$ pätee
\[
c_1y_1(x)+c_2y_2(x)=0,\quad x\in (a,b),
\]
niin derivoimalla seuraa, että myös
\[
c_1y_1'(x)+c_2y_2'(x)=0,\quad x\in (a,b).
\]
Kun nämä yhtälöt kirjoitetaan pisteessä $x_0$, niin saadaan yhtälöryhmä
\[ \left\{ \begin{aligned}
\,y_1(x_0)c_1+y_2(x_0)c_2 &= 0, \\ y_1'(x_0)c_1+y_2'(x_0)c_2 &= 0.
           \end{aligned} \right. \]
Koska $W_{12}(x_0)\neq 0$, niin yhtälöryhmän ainoa ratkaisu on 
$c_1=c_2=0$.\footnote[2]{Jos $\,a_{ij},x_i,b_i\in\R,\ i,j=1,2$ ja 
$\,W=a_{11}a_{22}-a_{21}a_{12} \neq 0$, niin pätee
\[
\left\{ \begin{aligned} 
        a_{11}x_1+a_{12}x_2 &= b_1 \\ a_{21}x_1+a_{22}x_2 &= b_2 
        \end{aligned} \right.
\qekv \left\{ \begin{aligned} 
        x_1 &= W^{-1}(a_{22}b_1-a_{12}b_2), \\ x_2 &= W^{-1}(a_{11}b_2-a_{21}b_1). 
        \end{aligned}\right.
\]}
Siis oletuksesta $\,c_1y_1(x)+c_2y_2(x)=0,\ x\in(a,b)\,$ seuraa, että $c_1=c_2=0$, joten $y_1$
ja $y_2$ ovat lineaarisesti riippumattomat välillä $(a,b)$. \loppu
\begin{Exa} Funktiot $y_1(x)=x^3$ ja $y_2(x)=|x|^3$ ovat (kehdestikin) derivoituvia ja
lineaarisesti riippumattomia $\R$:ssä (vrt.\ Esimerkki \ref{lin-2: lause 1 - esim} edellä).
Laskemalla näiden funktioiden Wronskin determinantti todetaan, että
$W_{12}(x)=0\ \forall x\in\R$. Tämän (vasta)esimerkin perusteella päätellään:
\begin{align*}
&\text{$y_1$ ja $y_2$ lineaarisesti riippumattomat välillä $(a,b)$} \\[1mm]
&\qquad\qquad\ \ \not\impl\quad W_{12}(x_0) \neq 0 \quad \text{jollakin}\ x_0\in(a,b). \loppu
\end{align*}
\end{Exa}
Esimerkin perusteella Proposition \ref{lin-2: prop} (implikaatio)väittämä ei päde kääntäen. Jos
sen sijaan oletuksia vahvistetaan niin, että $y_1$ ja $y_2$ oletetaan differentiaaliyhtälön
\eqref{lin-2: hy} ratkaisuiksi välillä $(a,b)$, niin käänteinenkin väittämä on tosi. Tällaisella
oletuksella saadaan seuraava, Propositiota \ref{lin-2: prop} huomattavasti vahvempi tulos.
\begin{Lause} \label{lin-2: lause 3} Jos $y_1$ ja $y_2$ ovat differentiaaliyhtälön 
\eqref{lin-2: hy} ratkaisuja välillä $(a,b)$ \ \ ($P$ ja $Q$ jatkuvia välillä $(a,b)$) ja 
$W_{12}(x)$ on funktioiden $y_1,y_2$ Wronskin determinantti välillä $(a,b)$, niin pätee
\begin{align*}
      &W_{12}(x_0)=0 \quad \text{jollakin}\ x_0\in (a,b) \\[1mm]
\qekv &W_{12}(x)\ =0 \quad \text{jokaisella}\ x\in(a,b) \\[1mm]
\qekv &\text{$y_1$ ja $y_2$ lineaarisesti riippuvat välillä $(a,b)$}.
\end{align*}
\end{Lause}
\tod Oletuksien mukaan
\[
\begin{cases}
y_1''+P(x)y_1'+Q(x)y_1=0,\quad &x\in (a,b), \\
y_2''+P(x)y_2'+Q(x)y_2=0,\quad &x\in (a,b).
\end{cases}
\]
Kertomalla ensimmäinen yhtälö $y_2$:lla ja toinen $y_1$:llä ja vähentämällä seuraa
\[
y_1y_2''-y_2y_1''+P(x)(y_1y_2'-y_2y_1')=0,\quad x\in (a,b).
\]
Mutta
\[
y_1y_2''-y_2y_1''=\frac{d}{dx}(y_1y_2'-y_2y_1').
\]
Siis on päätelty, että $y(x)=W_{12}(x)$ on jokaisella $x_0\in(a,b)$ ratkaisu alkuarvotehtävälle
\begin{equation} \label{Wronskin yhtälö}
\begin{cases}
\,y'+P(x)y=0,\quad x\in (a,b) \\ \,y(x_0)=W_{12}(x_0)
\end{cases}
\end{equation}
Koska ratkaisu on (vrt.\ Luku \ref{lineaarinen 1. kertaluvun DY})
\[
W_{12}(x)=W_{12}(x_0)\,e^{-\int_{x_0}^x P(t)dt},\quad x\in (a,b)
\]
ja koska $e^t \neq 0\ \forall t\in\R$, niin seuraa ensimmäinen osaväittämä: \ 
$W_{12}(x_0)=0\ \ekv\ W_{12}(x)=0\,\ \forall x\in(a,b)$.

Tapauksessa $y_1(x)=0\ \forall x\in(a,b)$ on toinenkin osaväittämä tosi, joten oletetaan, että
$y_1(x_0) \neq 0$ jollakin $x_0\in(a,b)$. Tällöin on $y_1$:n jatkuvuuden nojalla 
$y_1(x) \neq 0$ jollakin välillä $(x_0-\delta,x_0+\delta)\subset(a,b)$, $\delta>0$. Tällä 
välillä $y_2/y_1$ on derivoituva ja
\[
\frac{d}{dx}\left[\frac{y_2(x)}{y_1(x)}\right]=\frac{W_{12}(x)}{[y_1(x)]^2}\,.
\]
Näin ollen jos $\,W_{12}(x)=0\,$ välillä $(a,b)$, niin jollakin $C\in\R$ pätee
\[
\frac{y_2(x)}{y_1(x)}=C\ \ekv\ -Cy_1(x)+y_2(x)= 0 \quad \text{välillä}\ (x_0-\delta,x_0+\delta).
\]
Kun merkitään $\,y(x)=-Cy_1(x)+y_2(x),\ x\in(a,b)$, niin on siis $y(x)=0$ mainitulla osavälillä,
jolloin on erityisesti $y(x_0)=y'(x_0)=0$. Koska $y$ on myös differentiaaliyhtälön 
\eqref{lin-2: hy} ratkaisu välillä $(a,b)$, niin $y$ ratkaisee siis Lauseen \ref{lin-2: lause 2}
alkuarvotehtävän, kun $A=B=0$. Koska ilmeinen ratkaisu on myös $y=0$ ja koska ratkaisu on 
mainitun lauseen mukaan yksikäsitteinen, niin on siis oltava 
$\,y(x)=-Cy_1(x)+y_2(x)=0,\ x\in(a,b)$. Tällöin $y_1$ ja $y_2$ ovat lineaarisesti riippuvat
välillä $(a,b)$. Päättely perustui oletukseen, että $\,W_{12}(x)=0\,$ välillä $(a,b)$, joten
toisen osaväittämän osa \fbox{$\impl$} tuli todistetuksi. Jäljelle jäävä osa \fbox{$\Leftarrow$}
on jo todistettu, sillä tämä on Proposition \ref{lin-2: prop} väittämän loogisesti
ekvivalenttinen muoto. \loppu

Nyt ollaan valmiita esittämään (Lauseeseen \ref{lin-2: lause 2} nojaava)
 
\underline{Lauseen \ref{lin-2: lause 1} todistus} \ Olkoot $y_1$ ja $y_2$ Lauseen 
\ref{lin-2: lause 2} alkuarvotehtävän ratkaisut, kun\, a) $A=1,\ B=0$, \ b) $A=0,\ B=1$.
Tällöin funktioiden $y_1$ ja $y_2$ Wronskin determinantti $x_0$:ssa on
\[
W_{12}(x_0)=\begin{vmatrix}
1 & 0 \\
0 & 1
\end{vmatrix}=1.
\]
Proposition \ref{lin-2: prop} mukaan $y_1$ ja $y_2$ ovat lineaarisesti riippumattomat välillä
$(a,b)$, joten ensimmäinen osaväittämä on todistettu.

Toisen osaväittämän todistamiseksi oletetaan, että $y_1$ ja $y_2$ ovat mitkä tahansa kaksi 
differentiaaliyhtälön \eqref{lin-2: hy} lineaarisesti riippumatonta ratkaisua välillä $(a,b)$
ja että $u$ on kolmas saman DY:n ratkaisu. Tällöin jos voidaan määritellä kertoimet $c_1$ ja 
$c_2$ siten, että funktiolle $\,y(x)=c_1y_1(x)+c_2y_2(x)\,$ pätee 
$\,y(x_0)=u(x_0),\ y'(x_0)=u'(x_0)$, niin $u$ ja $y$ ovat saman alkuarvotehtävän ratkaisuja,
jolloin Lauseen \ref{lin-2: lause 2} mukaan on oltava $u(x)=y(x),\ x\in(a,b)$. Kertoimet
määräytyvät yhtälöryhmästä
\[ \left\{ \begin{aligned}
y_1(x_0)c_1+y_2(x_0)c_2 &= u(x_0), \\ y'_1(x_0)c_1+y'_2(x_0)c_2 &= u'(x_0).
           \end{aligned} \right. \]
Tämä ratkeaa yksikäsitteisesti ehdolla $\,y_1(x_0)y'_2(x_0)-y'_1(x_0)y_2(x_0) \neq 0$, eli
ehdolla $W_{12}(x_0) \neq 0$, missä $\,W_{12}\,$ on funktioiden $y_1$ ja $y_2$ Wronskin
determinantti. Mutta oletuksen ja Lauseen \ref{lin-2: lause 3} perusteella
$W_{12}(x_0) \neq 0$ jokaisella $x_0\in(a,b)$. Siis $u$ on esitettävissä funktioiden $y_1$ ja
$y_2$ lineaarisena yhdistelynä. \loppu

\subsection*{Täydellinen yhtälö: vakioiden variointi}
\index{lineaarinen differentiaaliyhtälö!b@täydellinen}
\index{lineaarinen differentiaaliyhtälö!g@vakio(ide)n variointi|vahv}
\index{vakio(ide)n variointi|vahv}

Jos homogeenisen yhtälön \eqref{lin-2: hy} yleinen ratkaisu tunnetaan, niin täydellisen
yhtälön \eqref{lin-2: ty} ratkaisemiseksi riittää jälleen löytää tälle yksittäisratkaisu.
Yleinen menetelmä perustuu vakioiden variointiin: Jos homogeenisen yhtälön yleinen ratkaisu
on $y(x)=C_1y_1(x)+C_2y_2(x)$, niin täydelliselle yhtälölle löydetään yksittäisratkaisu
muodossa
\[
y(x)=C_1(x)y_1(x)+C_2(x)y_2(x).
\]
Tällöin
\[
y'=C_1(x)y_1'(x)+C_2(x)y_2'(x)+C_1'(x)y_1(x)+C_2'(x)y_2(x).
\]
Tämän lausekkeen yksinkertaistamiseksi asetetaan lisäehto
\[
C_1'(x)y_1(x)+C_2'(x)y_2(x)=0
\]
(osoittautuu mahdolliseksi!), jolloin
\begin{align*}
y'  &= C_1(x)y_1'(x)+C_2(x)y_2'(x), \\
y'' &= C_1(x)y_1''(x)+C_2(x)y_2''(x) +C_1'(x)y_1'(x)+C_2'(x)y_2'(x)
\end{align*}
ja näin ollen
\begin{align*}
\dyf y &= y''+P(x)y'+Q(x)y \\
       &= C_1(x)\dyf y_1+C_2(x)\dyf y_2+C_1'(x)y_1'(x)+C_2'(x)y_2'(x).
\end{align*}
Tässä on $\dyf y_1=\dyf y_2=0$, joten oletettu lisäehto huomioiden on saatu yhtälöryhmä
\[
\begin{cases}
\,y_1(x)C_1'+y_2(x)C_2'=0, \\
\,y_1'(x)C_1'+y_2'(x)C_2'=R(x).
\end{cases}
\]
Koska tässä on (Lause \ref{lin-2: lause 3})
\[
(y_1y_2'-y_2y_1')(x)=W_{12}(x)\neq 0,
\]
niin yhtälöryhmä ratkeaa:
\[
C_1'=-\frac{y_2(x)R(x)}{W_{12}(x)}\,,\quad C_2'=\frac{y_1(x)R(x)}{W_{12}(x)}\,.
\]
Integroimalla $C_1(x)$ ja $C_2(x)$ näistä yhtälöistä on löydetty täydellisen yhtälön
\eqref{lin-2: ty} yksittäisratkaisu. Yhtälön \eqref{lin-2: ty} yleiseksi ratkaisuksi tulee näin
muodoin
\[
y(x)=C_1y_1(x)+C_2y_2(x)
    -y_1(x)\int\frac{y_2(x)R(x)}{W_{12}(x)}\,dx+y_2(x)\int\frac{y_1(x)R(x)}{W_{12}(x)}\,dx.
\]

\begin{Exa}
Ratkaise Eulerin differentiaaliyhtälö
\[
x^2y''+4xy'+2y=f(x), \quad x>0.
\]
\end{Exa}
\ratk Homogeenisen yhtälön yleinen ratkaisu on (ks.\ edellinen luku)
\[
y(x) \,=\, \frac{C_1}{x}+\frac{C_2}{x^2} \,=\, C_1y_1(x)+C_2y_2(x).
\]
Em.\ laskukaavoissa on
\[
R(x)=\frac{f(x)}{x^2}\,, \quad W_{12}(x)=(y_1y_2'-y_1'y_2)(x)=-\frac{1}{x^4}\,,
\] 
joten vakioiden variointi antaa
\begin{align*}
&C_1'= f(x) \qimpl C_1(x)=\int f(x)\,dx, \\
&C_2'=-xf(x) \qimpl C_2(x)=-\int xf(x)\,dx.
\end{align*}
Yksittäisratkaisu on siis
\[
y(x) = \frac{1}{x}\int f(x)\,dx - \frac{1}{x^2}\int xf(x)\,dx.
\]
Tämä on myös yleinen ratkaisu, kun oikealla puolella integraaleihin sisällytetään
määräämättömät integroimisvakiot. \loppu


\subsection*{Tunnettuja differentiaaliyhtälöitä}

Toisen kertaluvun lineaarinen, homogeeninen differentiaaliyhtälö on sovelluksissa yleinen
differentiaaliyhtälön tyyppi. Seuraavassa muutamia tunnettuja differentiaaliyhtälöitä, jotka on
nimetty niitä tutkineiden matemaatikkojen mukaan.
\index{lineaarinen differentiaaliyhtälö!fb@Besselin, Hermiten, Laguerren, $\quad$
       Legendren, T\v{s}eby\v{s}evin}             
%\index{lineaarinen differentiaaliyhtälö!l@Besselin DY}
%\index{lineaarinen differentiaaliyhtälö!l@Hermiten DY}
%\index{lineaarinen differentiaaliyhtälö!l@Laguerren DY}
%\index{lineaarinen differentiaaliyhtälö!l@Legendren DY}
%\index{lineaarinen differentiaaliyhtälö!l@T\v{s}eby\v{s}evin DY}
\index{Besselin differentiaaliyhtälö}
\index{Hermiten!c@differentiaaliyhtälö}
\index{Laguerren differentiaaliyhtälö}
\index{Legendren differentiaaliyhtälö}
\index{Tsebysev@T\v{s}eby\v{s}evin differentiaaliyhtälö}%
\vspace{3mm}\newline
\kor{Legendre}: $\qquad(x^2-1)y''+2xy'-n(n+1)y=0$ \vspace{3mm}\newline
\kor{T\v{s}eby\v{s}ev}: $\qquad (1-x^2)y''-xy'+n^2y=0$ \vspace{3mm}\newline
\kor{Hermite}: $\qquad\ y''-2xy'+2ny=0$ \vspace{3mm}\newline
\kor{Laguerre}: $\qquad xy''+(1-x)y'+ny=0$ \vspace{3mm}\newline
\kor{Bessel}: $\quad\qquad x^2y''+xy'+(x^2-p^2)y=0$ \vspace{3mm}\newline
Näistä muilla paitsi Besselin differentiaaliyhtälöllä on ratkaisuna polynomi astetta $n$, kun
$n\in\N\cup\{0\}$ (sovelluksien kannalta kiinnostavin tapaus). Ko.\ polynomit on nimetty
samoin kuin differentiaaliyhtälöt. Besselin differentiaaliyhtälössä $p \ge 0$ on reaalinen
paramatetri

\subsection*{Sarjaratkaisut}
\index{lineaarinen differentiaaliyhtälö!k@sarjaratkaisut|vahv}
\index{sarjaratkaisu (lineaarisen DY:n)|vahv}

Jos toisen kertaluvun lineaarisella, homogeenisella differentiaaliyhtälöllä ei ole yhtään 
'tunnistettavaa' ratkaisua, niin ratkaisemista voidaan yrittää \kor{sarjamenetelmällä}.
Tällöin ratkaisua etsitään joko potenssisarjana tai potenssisarjoja sisältävänä lausekkeena.
Menetelmän perusidea on sama kuin integroinnin sarjamenetelmässä 
(ks.\ Luku \ref{osamurtokehitelmät}).

Esimerkkinä sarjamenetelmän soveltamisesta olkoon Besselin differentiaaliyhtälö, jonka
perusmuoto on
\[
y''+\frac{1}{x}y'+(1-\frac{p^2}{x^2})y=0.
\]
Yleinen ratkaisu (sovelluksissa yleensä välillä $(0,\infty)$) kirjoitetaan muodossa
\[
y(x)=C_1 J_p(x)+C_2 Y_p(x),
\]
missä $J_p$ ja $Y_p$ ovat
\index{Besselin funktio}%
\kor{Besselin funktioita}, tarkemmin \kor{ensimmäisen lajin} ($J_p$)
ja \kor{toisen lajin} ($Y_p$) Besselin funktioita. (Tapauksessa $p=1/2$ nämä ovat 
poikkeuksellisesti alkeisfunktioita, ks.\ Harj.teht.\,\ref{H-dy-6: Besselin erikoistapaus}.)
Funktio $J_p$ on muotoa
\[
J_p(x)=\sum_{k=0}^\infty a_kx^{2k+p}=x^p F(x),
\]
missä potenssisarja $F(x)=\sum_{k=0}^\infty a_kx^{2k}$ suppenee kaikkialla. Muilla kuin 
kokonaislukuarvoilla on vastaavasti
\[
Y_p(x)=\sum_{k=0}^\infty a_kx^{2k-p}=x^{-p} G(x),\quad p\notin \{0,1,2,\ldots\}
\] 
missä $G$:n sarja jälleen suppenee kaikkialla. Jos $p$ on kokonaisluku, niin $Y_p$ sisältää
myös logaritmisen termin muotoa $x^{-p}H(x)\ln\abs{x}$, missä $H(x)$ on ilmaistavissa 
potenssisarjana.

Olkoon $p=0$, ja yritetään sarjaratkaisua $y(x)=\sum_{k=0}^\infty a_kx^k$.
Sijoitus yhtälöön antaa
\[
\sum_{k=2}^\infty k(k-1)a_kx^{k-2}+\sum_{k=1}^\infty ka_kx^{k-2}+\sum_{k=0}^\infty a_kx^k=0.
\]
Kun kahdessa ensimmäisessä summassa vaihdetaan summausindeksiksi $k$:n tilalle $k+2$, saadaan
\[
\sum_{k=0}^\infty (k+2)(k+1)a_{k+2}x^k+\bigl[\,a_1x^{-1}+\sum_{k=0}^\infty (k+2)a_{k+2}x^k\,\bigr] 
                                                 + \sum_{k=0}^\infty a_kx^k=0 \quad \forall x
\]
eli yhdistämällä summat
\[
a_1x^{-1}+\sum_{k=0}^\infty [(k+2)^2a_{k+2}+a_k]x^k = 0\quad\forall x.
\]
Nähdään, että tämä toteutuu, kun valitaan
\begin{align*}
a_1 &= a_3=a_5=\ldots =0 \\
a_0 &= 1,\quad a_{k+2}=-\frac{1}{(k+2)^2}\,a_k,\quad k=0,2,4 \ldots
\end{align*}
Kertoimille $a_{2k}$ saadaan (pienen mietiskelyn jälkeen) lauseke
\[
a_{2k}=(-1)^k\frac{1}{(k!)^2\cdot 4^k},\quad k=0,1,\ldots
\]
Ratkaisu, nimeltään $J_0$, on siis
\[
J_0(x)=\sum_{k=0}^\infty (-1)^k\frac{1}{(k!)^2}\left(\frac{x}{2}\right)^{2k}.
\]
Tämä sarja suppenee (ja on termeittäin derivoitavissa) kaikkialla, joten on löydetty
koko $\R$:ssä pätevä ratkaisu differentiaaliyhtölölle $xy''+y'+xy=0$. Toisen lajin ratkaisu
$Y_0(x)$ on pätevä vain erikseen väleillä $(-\infty,0)$ ja $(0,\infty)$. Tämäkin on 
ilmaistavissa potenssisarjojen avulla (ks. Harj.teht.\,\ref{H-dy-6: Besselin toinen laji}).

\Harj
\begin{enumerate}

\item
Minkä toisen kertaluvun lineaarisen ja homogeenisen differentiaaliyhtälön ratkaisuja
(jollakin avoimella välillä) ovat
a) $y_1(x)=x+1$ ja $y_2(x)=x^2$, b) $y_1(x)=x$ ja $y_2(x)=e^x$, c) $y_1(x)=x^2$ ja 
$y_2(x)=\sin x\,$?

\item
Ratkaise käyttäen annettua, yksittäisratkaisua koskevaa lisätietoa: \vspace{1mm}\newline
a) \ $xy''-(x+3)y'+y=0; \quad$ polynomi \newline
b) \ $x^2(\ln x-1)y''-xy'+y=0; \quad$ polynomi \newline
c) \ $(x^2-2x-1)y''-(2x-1)y'+2y=0; \quad$ polynomi \newline
d) \ $ y''+(\tan x-2\cot x)y'+2(\cot^2x)y=0; \quad \sin x$ \newline
e) \ $y''+2(1-\tan^2x)y=0; \quad \cos^2x$ \newline
f) \ $(x-2)y''-(4x-7)y'+(4x-6)y=0; \quad e^{ax}$

\item
Differentiaaliyhtälöllä $\,y''+P(x)y'+Q(x)y=0\,$ on helposti arvattava ei-triviaali ratkaisu,
jos $1+P(x)+Q(x)=0$. Ratkaise tämän (ja tarvittaessa toisenkin) arvauksen perusteella:
\vspace{1mm}\newline
a) \ $(x-1)y''-xy'+y=0$ \newline
b) \ $(2x-x^2)y''+(x^2-2)y'+(2-2x)y=0$ \newline
c) \ $y''-(2+2x+x^{-1})y'+(1+2x+x^{-1})y=0$ \newline
d) \ $(x+1)y''-xy'-y=(x+1)^2$ 

\item
Oletetaan, että differentiaaliyhtälön $\,y''+P(x)y'+Q(x)y=0\,$ kerroinfunktioille pätee:
$P$ on derivoituva ja $P'$ ja $Q$ ovat jatkuvia tarkasteltavalla välillä $(a,b)$. Tällöin
voidaan sijoituksella $y(x)=K(x)u(x)$ muuntaa DY muotoon
\[
u''+I(x)u=0.
\]
Miten $K(x)$ on valittava ja mikä on $I(x)$:n lauseke?

\item
Ratkaise vakioiden varioinnilla:
\vspace{1mm}\newline
a) \ $y''+y=\tan x \qquad\qquad\qquad\,$
b) \ $y''-y=1/(e^x+1)$ \newline
c) \ $y''+2y'+y=1/(x^2+1) \quad\,\ $
d) \ $x^2y''+2xy'-2y=x^2e^x$ \newline
e) \ $y''+5y'+4y=f(x) \qquad\quad\ \ $
f) \ $y''+4y'+4y=f(x)$ \newline
g) \ $y''+4y'+5y=f(x) \qquad\quad\,\ $
h) \ $x^2y''+5xy'+3y=f(x)$ \newline
i) \ $x^2y''+5xy'+4y=f(x) \qquad\ $
j) \ $x^2y''-2y=f(x)$

\item \index{Legendren polynomi}
a) \kor{Legendren polynomi} astetta $n$ määritellään
\[
P_n(x)=\frac{1}{2^n n!}\frac{d^n}{dx^n}\,(x^2-1)^n.
\]
Totea, että $P_n$ on polynomi astetta $n$ ja näytä, että tämä on Legendren 
differentiaaliyhtälön ratkaisu $n$:n arvoilla $0,1,2,3$. \vspace{1mm}\newline
b) Määritä Hermiten ja Laguerren polynomit astetta $n \le 2$, ts.\ kyseisten
differentiaaliyhtälöiden polynomiratkaisut, kun $n=0,1,2$. \vspace{1mm}\newline
Määritä\, c) Legendren,\, d) T\v{s}eby\v{s}evin,\, e) Hermiten,\, f) Laguerren
DY:n yleinen ratkaisu, kun $n=1$.

\item \label{H-dy-6: Besselin erikoistapaus}
Tapauksessa $p=1/2\,$ Besselin differentiaaliyhtälö on ratkaistavissa alkeisfunktioilla.
Määritä ratkaisu sijoituksella $\,y(x)=u(x)/\sqrt{x}$.

\item 
Seuraavien differentiaaliyhtälöiden yleinen ratkaisu on esitettävissä pisteen $x=0$ lähellä 
muodossa $y(x)=C_1\sum_{k=0}^\infty a_k x^k+C_2\sum_{k=0}^\infty b_k x^k$, missä $a_0=b_1=1$ ja
$a_1=b_0=0$. Määritä molempien potenssisarjojen neljä ensimmäistä nollasta poikkeavaa
termiä. \vspace{1.5mm}\newline
a) \ $y''+2xy'+5y=0 \quad$
b) \ $y''+2xy+6y=0 \quad$
c) \ $y''+xy'-x^2y=0 \quad$ \newline
d) \ $y''+(x+1)y'+(x-1)y=0 \quad$
e) \ $(1-x^2)y''-xy'+ay=0,\ a\in\R$

\item (*) \index{Tsebysev@T\v{s}eby\v{s}evin polynomi}
\kor{T\v{s}eby\v{s}evin polynomi} astetta $n$ määritellään välillä $[-1,1]$ kaavalla
\[
T_n(x)=\cos\,(n\Arccos x), \quad x\in[-1,1].
\]
a) Näytä, että $T_n$ on polynomi astetta $n$ (sijoita $x=\cos\theta$). \ b) Näytä, että $T_n$
on T\v{s}eby\v{s}evin differentiaaliyhtälön ratkaisu välillä $(-1,1)$. \ c) Laske polynomin
$T_5$ kertoimet ja hahmottele $T_5(x)$ graafisesti välillä $[-2,2]$. Mitkä ovat $T_5$:n
nollakohdat ja missä pisteissä on $T_5(x)=\pm 1\,$?

\item (*) \label{H-dy-6: Besselin toinen laji}
Näytä, että jos $p=0$, niin Besselin differentiaaliyhtälöllä on välillä $(0,\infty)$
(toisen lajin) ratkaisu muotoa $Y_0(x)=J_0(x)\ln x+F(x)$, missä $F$ on ilmaistavissa kaikkialla
suppenevana potenssisarjana.

\item (*)
Näytä, että differentiaaliyhtälön $y''=xy$ yleinen ratkaisu on esitettävissä potenssisarjana
muodossa
\[
y(x)=C_1 \sum_{k=0}^\infty a_k x^{3k} + C_2 \sum_{k=0}^\infty b_k x^{3k+1}, \quad a_0=b_0=1.
\]
Määrittele kerroinjonot $\seq{a_k}$ ja $\seq{b_k}$ palautuvina lukujonoina ja päättele, että
sarja suppenee kaikkialla. Millainen on yleisemmin differentiaaliyhtälön $y''=x^my\,$ yleinen
sarjaratkaisu, kun $m\in\N\,$?

\end{enumerate}