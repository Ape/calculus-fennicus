\chapter[Usean muuttujan differentiaalilaskenta]{Usean muuttujan \\ 
differentiaalilaskenta}

Tässä luvussa tarkastelun kohteena ovat \kor{vektorimuuttujan}, eli useamman reaalimuuttujan
funktiot tyyppiä $f:\,\R^n\kohti\R$ ja vektorimuuttujan \kor{vektoriarvoiset} funktiot
tyyppiä $\mf:\,\R^n\kohti\R^m$. Edellisten erikoistapaukset, kahden ja kolmen
\mbox{reaalimuuttujan} funktiot, ovat ennestään tuttuja Luvusta
\ref{kahden ja kolmen muuttujan funktiot}. Myös vektoriarvoisia funktioita on tavattu jo
aiemmin, sillä sekä Luvussa \ref{parametriset käyrät} esitellyt parametriset käyrät ja
parametriset pinnat että Luvuissa \ref{lineaarikuvaukset}--\ref{affiinikuvaukset} käsitellyt
lineaari- ja affiinikuvaukset ovat tällaisten erikoistapauksia. Vektoriarvoisia funktioita ovat
myös tason ja avaruuden \kor{vektorikentät}, joissa vektori ymmärretään
geometris--fysikaalisena 'nuolivektorina'.

Luvussa \ref{usean muuttujan jatkuvuus} yleistetään Luvuista
\ref{jatkuvuuden käsite}--\ref{funktion raja-arvo} tutut jatkuvuuden ja raja-arvon käsitteet
useamman muuttujan funktioille. Tämän jälkeen luvun keskeisen sisällön muodostavat erilaiset
\pain{derivaatan} käsitteen laajennukset useamman muuttujan tilanteisiin ja näihin
laajennuksiin liittyvä laskutekniikka (differentiaalilaskenta) ja sovellukset. Yleisempinä
derivaatan käsitteinä esitellään tässä luvussa \kor{osittaisderivaatat} ja niistä muodostettu
\kor{gradientti} (Luvut \ref{osittaisderivaatat}--\ref{gradientti}), vektorikenttiin liittyvät
\kor{divergenssi} ja \kor{roottori}
(Luvut \ref{divergenssi ja roottori}--\ref{div ja rot käyräviivaisissa}) ja lopulta yleisempiin
vektoriarvoisiin funktioihin eli \kor{epälineaarisiin kuvauksiin} liittyen 
\kor{Jacobin matriisi} (Luku \ref{jacobiaani}). Sovelluksina tarkastellaan mm.\ geometrisia
tehtäviä kuten \kor{pinnan tangenttitason} määrittämistä (Luku \ref{gradientti}), fysiikan 
\kor{osittaisdifferentiaaliyhtälöitä} 
(Luvut \ref{divergenssi ja roottori}--\ref{div ja rot käyräviivaisissa}), 
\kor{epälineaarisia yhtälöryhmiä} ja niiden numeerista ratkaisemista (Luku \ref{jacobiaani})
ja \kor{optimoinnin} ongelmia (Luku \ref{usean muuttujan ääriarvotehtävät}). Luvussa
\ref{käänteisfunktiolause} esitellään kolme epälineaarisiin yhtälöryhmiin liittyvää
\kor{epälineaarisen analyysin} vahvaa lausetta. Viimeisessä osaluvussa
(Luku \ref{usean muuttujan taylorin polynomit}) määritellään \kor{Taylorin polynomi} useamman
muuttujan funktioille. 