\section{*Picardin--Lindelöfin lause} \label{Picard-Lindelöfin lause}
\alku

Palataan tutkimaan ensimmäisen kertaluvun alkuarvotehtävää
\begin{equation} \label{Picard-1}
\begin{cases} \,y'=f(x,y), \\ \,y(x_0)=y_0 \end{cases}
\end{equation}
tällä kertaa teoreettisemmalta kannalta. Asetetaan kysymys: Millä, funktiolle $f$ ja luvuille
$x_0,y_0$, asetettavilla ehdoilla voidaan taata, että alkuarvotehtävällä on yksikäsitteinen
ratkaisu pisteen $x_0$ ympäristössä? Vastauksen antaa seuraava kuuluisa lause.
\begin{*Lause} \label{picard-lindelöf} \index{Picardin--Lindelöfin lause|emph}
\index{differentiaaliyhtälön!h@ratkeavuus|emph}%
\vahv{(Picard--Lindelöf\,\footnote[2]{Picardin--Lindelöfin lauseen alkuperäisversion esitti
ja todisti tässä esitettyä yleisemmässä muodossa ranskalainen matemaatikko \hist{Emile Picard}
(1853-1941) vuonna 1893. Suomalainen matemaatikko \hist{Ernst Lindelöf} (1870-1946) tarkensi
tulosta hiukan myöhemmin. Lauseen väittämän jälkimmäinen osa (koskien ratkeavuutta koko
tarkasteltavalla välillä) on Lindelöfin käsialaa.
\index{Picard, E.|av} \index{Lindelöf, E.|av}})} Olkoon $T=[a,b]\times [c,d]\subset\Rkaksi$ ja
oletetaan, että $f:T\kohti\R$ toteuttaa ehdot
\begin{itemize}
\item[(i)]  Funktio $x \map f(x,y)$ on jatkuva välillä $[a,b]$ jokaisella $y\in[c,d]$.
\item[(ii)] Funktio $y \map f(x,y)$ on jokaisella $x\in[a,b]$ Lipschitz-jatkuva välillä $[c,d]$
            vakiolla, joka ei riipu $x$:stä, ts.\ $\exists L\in\R_+$ siten, että pätee
            \[
            |f(x,y_1)-f(x,y_2)| \le L\abs{y_1-y_2} \quad\forall (x,y_1)\in T, \ (x,y_2)\in T.
            \]
\end{itemize}
Tällöin, jos $(x_0,y_0)\in(a,b)\times(c,d)$, niin on olemassa $\delta>0$ siten, että
alkuarvotehtävällä
\[
\begin{cases} \,y'=f(x,y),\quad x\in (x_0-\delta,x_0+\delta), \\ \,y(x_0)=y_0 \end{cases}
\]
on yksikäsitteinen ratkaisu. Edelleen jos oletukset ovat voimassa, kun välin $[c,d]$ tilalla
on $\R$ ($T=[a,b]\times\R$), niin ratkaisu on olemassa ja yksikäsitteinen koko avoimella
välillä $(a,b)$, kun $(x_0,y_0)\in(a,b)\times\R$.
\end{*Lause}

Seuraavassa käydään lyhyesti läpi Lauseen \ref{picard-lindelöf} todistukset ideat ja
päävaiheet. (Todistuksen tarkkaan läpiviemiseen ovat käsitteelliset eväämme hieman vajavaiset.)
Lähtökohtana on ajatus, että alkuarvotehtävän ratkaisu konstruoidaan nk.\ 
\index{kiintopisteiteraatio!Picardin iteraatio|(} \index{Picardin iteraatio|(}%
\kor{Picardin iteraatiolla}. Tässä puolestaan idea on seuraava: Kun ollaan lähellä pistettä
$x_0$, niin voidaan olettaa, että $y(x)$ on lähellä $y_0$:aa. Ratkaisun ensimmäinen
approksimaatio on tämän mukaan vakio:
\[
y(x)\approx y_0(x)=y_0.
\]

Parannetaan tätä nyt asteittain ratkaisemalla palautuvasti alkuarvotehtävät
\[
\begin{cases}
\,y_k'(x)=f(x,y_{k-1}(x)), \\
\,y_k(x_0)=y_0,\quad k=1,2,\ldots
\end{cases}
\]
Koska tässä $y_{k-1}$ tunnetaan aikaisemmilta iteraatiokierroksilta (tai alkuehdosta), saadaan
jokainen $y_k$ määrätyksi palautuvasti tunnetun funktion integraalina:
\begin{equation} \label{Picard-2}
y_0(x) = y_0, \quad y_k(x) = y_0+\int_{x_0}^x f(t,y_{k-1}(t))dt,\quad k=1,2,\ldots
\end{equation}
Näin on määritelty Picardin iteraatio. --- Huomattakoon, että tässä on itse asiassa kyse
\pain{kiinto}p\pain{isteiteraatiosta}, missä 'pisteen' sijasta etsitään funktiota (vrt.\ Luku
\ref{kiintopisteiteraatio}). Iteraatiota sovelletaan alkuarvotehtävän integroituun muotoon
\[ 
y(x) = y_0+\int_{x_0}^x f(t,y(t))\,dt,
\]
\index{integraaliyhtälö}%
eli Picardin iteraatiossa on kyse tämän \kor{integraaliyhtälön} iteratiivisesta 
ratkaisemisesta. --- Katkaisemalla iteraatio saadaan likimääräinen ratkaisu, kuten
kiintopisteiteraatioissa yleensä.

Picardin iteraatiota voi siis pitää alkuarvotehtävän likimääräisen ratkaisemisen menetelmänä.
Toisin kuin edellisessä luvussa tarkastellut numeeriset menetelmät, Picardin iteraatio on
kuitenkin lähinnä 'ajattelumenetelmä', sillä käytännössä iteraatioon sisältyvät integraalit
tulevat yleensä nopeasti hankaliksi.
\begin{Exa} Laske kaksi ensimmäistä Picardin iteraattia $\,(y_k(x), \ k=1,2\,)$
alkuarvotehtävälle
\[
\begin{cases}
\,y'=x+y^2, \\ \,y(0)=1.
\end{cases}
\]
\end{Exa}
\ratk Tässä on $f(x,y)=x+y^2$, $x_0=0$, $y_0=1$, joten $y_0(x)=1$ ja
\begin{align*}
y_1(x) &= 1+\int_0^x (t+1)\,dt=\underline{\underline{1+x+\frac{1}{2}x^2}}, \\
y_2(x) &= 1+\int_0^x \left[t+\left(1+t+\frac{1}{2}t^2\right)^2\right]dt \\
&= \underline{\underline{1+x+\frac{3}{2}x^2+\frac{2}{3}x^3+\frac{1}{4}x^4+\frac{1}{20}x^5}}.
\end{align*}
Seuraavaa iteraatti $y_3(x)$ olisi jo polynomi astetta $11$. \loppu

Palataan Lauseen \ref{picard-lindelöf} todistukseen, jossa siis ideana on konstruoida
alkuarvotehtävän ratkaisu Picardin iteraatiolla \eqref{Picard-2}. Tarkoituksena on tällöin
osoittaa, että iteraatio suppenee kohti välillä $[x_0-\delta,x_0+\delta]$ jatkuvaa funktiota
$y$, joka toteuttaa
\begin{equation} \label{Picard-3}
y(x)=y_0+\int_{x_0}^x f(t,y(t))dt,\quad x\in [x_0-\delta,x_0+\delta].
\end{equation}
Parametrin $\delta$ valinnalla ($\delta$ riittävän pieni) taataan, että 
$[x_0-\delta,x_0+\delta]\subset[a,b]$ ja että $y(x)$ ei 'karkaa' väliltä $[c,d]$, ts.\ 
$\,(x,y(x)) \in T\ \ \forall x \in [x_0-\delta,x_0+\delta]$ (vrt.\ vastaava ajatus Lauseen 
\ref{separoituvan DY:n ratkaisu} todistuksessa). Tällöin oletuksen (i) mukaan yhdistetty 
funktio $x \map f(x,y(x))$ on jatkuva välillä $[x_0-\delta,x_0+\delta]$, jolloin yhtälöstä 
\eqref{Picard-3} ja Analyysin peruslauseesta seuraa, että $y$ on jatkuvasti derivoituva välillä 
$[x_0-\delta,x_0+\delta]$. Tällöin $y$ on myös alkuarvotehtävän \eqref{Picard-1} ratkaisu 
välillä $(x_0-\delta,x_0+\delta)$.

Iteraation \eqref{Picard-2} suppenemisen toteamiseksi otetaan käyttöön välillä 
$[x_0-\delta,x_0+\delta]$ määriteltyä jatkuvaa funktiota mittaava \kor{maksiminormi}
\index{maksiminormib@maksiminormi (funktion)} \index{normi!xb@maksiminormi (funktion)}%
\[ 
\norm{y} = \max\,\{\,\abs{y(x)} \mid  x_0-\delta \le x \le x_0+\delta\,\}. 
\]
Tälle (niinkuin normeille yleensä, vrt.\ Luku \ref{abstrakti skalaaritulo}) pätee 
\pain{kolmioe}p\pain{ä}y\pain{htälö}
\[
\norm{y_1+y_2} \le \norm{y_1}+\norm{y_2}.
\]

Lähdetään nyt Lauseen \ref{picard-lindelöf} oletuksesta (ii) ja arvioidaan
\begin{align*}
\abs{y_{k+1}(x)-y_k(x)}\ 
     &=\ \left| \int_{x_0}^x [f(t,y_k(t))-f(t,y_{k-1}(t))]\,dt \right| \\ 
     &\le\ \left| \int_{x_0}^x L \abs{y_k(t)-y_{k-1}(t)}\,dt \right| \\[2mm]
     &\le\ L \abs{x-x_0} \norm{y_k-y_{k-1}} \\[4mm]
     &\le\ L\delta\,\norm{y_k-y_{k-1}}, \quad x \in [x_0-\delta,x_0+\delta], \quad k=1,2, \ldots
\end{align*}  
Tulos on kirjoitettavissa
\[
\norm{y_{k+1}-y_k}\ \le\ L\delta\,\norm{y_k-y_{k-1}}, \quad k=1,2,\ldots
\]
Merkitään nyt $q=L\delta$, ja oletetaan jatkossa, että $q < 1$ (järjestettävissä valitsemalla
$\delta<1/L$). Käyttämällä saatua arviota palautuvasti seuraa
\[
\norm{y_{k+1}-y_k}\ \le\ q^k\norm{y_1-y_0}, \quad k=1,2,\ldots
\]
Oletuksen (i) mukaan funtkio $g(x)=f(x,y_0)$ on jatkuva välillä $[a,b]$, joten iteraatiokaavan
\eqref{Picard-2} perusteella
\[
\norm{y_1-y_0} = \max_{x\in[x_0-\delta,x_0+\delta]}\,\Bigl|\int_{x_0}^x f(t,y_0)\,dt\Bigr| 
                 \le C\delta,
\]
missä $C$ on $\abs{g}$:n maksimiarvo välillä $[a,b]$. Yhdistämällä saadut arviot ja käyttämällä
kolmioepäyhtälöä seuraa
\begin{align*}
\norm{y_n-y_k}\ &=\ \norm{(y_n-y_{n-1})+(y_{n-1}-y_{n-2})+ \ldots + (y_{k+1}-y_k)} \\[3mm]
                &\le\ \norm{y_n-y_{n-1}}+ \norm{y_{n-1}-y_{n-2}} + \ldots 
                                                                 + \norm{y_{k+1}-y_k} \\[3mm]
                &\le\ (q^{n-1}+q^{n-2} + \ldots + q^k)\,C\delta \\
                &<\ C\delta q^k\sum_{i=0}^\infty q^i\,
                 =\,\frac{C\delta}{1-q}\,q^k, \quad 0 \le k < n.
\end{align*}
Tästä voidaan päätellä ensinnäkin, että jos $\delta$ on riittävän pieni, niin
$\norm{y_n-y_0} \le C\delta/(1-\delta) \le \min\{y_0-c,\,d-y_0\}\ \forall n$, jolloin
$y_n(x)$:n pako väliltä $[c,d]$ on estetty. Toiseksi päätellään, että jono $\seq{y_k(x)}$ on
Cauchy jokaisella $x \in [x_0-\delta,x_0+\delta]$ ja siis suppenee:
\[ 
y_k(x) \kohti y(x)\in[c,d], \quad x \in [x_0-\delta,x_0+\delta]. 
\]
\index{tasainen suppeneminen}%
On edelleen pääteltävissä, että tämä suppeneminen on itse asiassa \kor{tasaista} välillä
$[x_0-\delta,x_0+\delta]$, ts.\ pätee
\[
\norm{y_k-y}\,=\,\max_{x\in[x_0-\delta,\,x_0+\delta]} \abs{y_k(x)-y(x)} 
                 \kohti\ 0, \quad \text{kun}\ k \kohti \infty,
\]
ja että raja-arvona konstruoitu $y(x)$ on jatkuva välillä $[x_0-\delta,x_0+\delta]$. (Tässä on
todistuksen käsitteellisesti vaativin kohta!) Ottamalla lopuksi iteraatiokaavassa 
\eqref{Picard-2} puolittain raja-arvo, kun $k\kohti\infty$, päätellään, että $y(x)$ ratkaisee 
probleeman \eqref{Picard-3}, jolloin Lauseen \ref{picard-lindelöf} ensimmäisestä väittämästä
on todistettu ratkaisun olemassaoloa koskeva osa. Ratkaisun yksikäsitteisyys on todettavissa
suoraviivaisemmin, ks.\ Harj.teht.\,\ref{H-dy-8: ratkaisun yksikäsitteisyys}. 

Lauseen \ref{picard-lindelöf} toista väittämää todistettaessa ei jonon $\seq{y_k(x)}$ 
pako-ongelmaa ole, joten väli $[x_0-\delta,x_0+\delta]$ voidaan korvata välillä $[a,b]$.
Tarkentamalla em.\ laskua osoittautuu, että voidaan myös arvioida
\[
\abs{y_{k+1}(x)-y_k(x)}\ \le\ C\,\frac{L^k\abs{x-x_0}^k}{k!}\,, \quad x\in[a,b].
\]
Tutkittaessa jonon $\seq{y_k(x)}$ suppenemista tulee siis vertailukohdaksi nyt geometrisen
sarjan sijasta eksponenttifunktion sarja
\[ 
e^{L\abs{x-x_0}} = \sum_{k=0}^\infty \frac{L^k\abs{x-x_0}^k}{k!}\,. 
\]
Tämä suppenee kaikkialla, joten voidaan päätellä samalla tavoin kuin edellä, että funktio
$y(x)=\lim_k y_k(x)$ on integraaliyhtälön \eqref{Picard-3} ratkaisu, tällä kertaa koko välillä
 $[a,b]$. Alkuarvotehtävä \eqref{Picard-1} on näin ratkaistu yksikäsitteisesti välillä $(a,b)$.
\index{kiintopisteiteraatio!Picardin iteraatio|)} \index{Picardin iteraatio|)}

\begin{Exa}
Alkuarvotehtävä
\[
\begin{cases} \,y'=\sin y, \\ \,y(x_0)=y_0 \end{cases}
\]
toteuttaa Lauseen \ref{picard-lindelöf} ehdot, kun $T=\R^2$, sillä $f(y)=\sin y$ on kaikilla 
väleillä Lipschitz-jatkuva vakiolla $L=1$. Alkuarvotehtävällä on siis yksikäsitteinen ratkaisu
välillä $(-\infty,\infty)$ jokaisella $(x_0,y_0)\in\R^2$. Jos valitaan esim.\ $x_0=0$, niin
nähdään, että $y_0$:n eri arvoja vastaavat ratkaisut löytyvät kaikki Esimerkissä 
\ref{separoituva DY}:\ref{muuan separoituva dy} laskettujen ratkaisujen joukosta. Näin on tullut
varmistetuksi, että mainitussa esimerkissä löydettiin kaikki differentiaaliyhtälön $y'=\sin y$
ratkaisut. \loppu
\end{Exa}
\begin{Exa}
Alkuarvotehtävälle
\[
\begin{cases} \,y'=\sqrt{\abs{y}}, \\ \,y(0)=0 \end{cases}
\]
Lauseen \ref{picard-lindelöf} ehdot eivät toteudu millään joukon $T=(a,b)\times(c,d)$
valinnalla, sillä on oltava $0\in(a,b)$, jolloin $f(y)=\sqrt{\abs{y}}$ ei ole Lipschitz-jatkuva
välillä $[a,b]$. Ratkaisu ei olekaan yksikäsitteinen 
(ks.\ Esimerkki \ref{DY-käsitteet}:\,\ref{erikoinen dy}b). \loppu
\end{Exa}
\begin{Exa}
Jos lineaarisessa alkuarvotehtävässä
\[
\begin{cases}
\,y'+P(x)y=R(x), \\ \,y(x_0)=y_0
\end{cases}
\]
$P$ ja $Q$ ovat jatkuvia välillä $[a,b]$, niin Lauseen \ref{picard-lindelöf} ehdot ovat 
voimassa, kun valitaan $T=[a,b]\times\R$. Nimittäin
\begin{itemize}
\item[(i)]  Funktio $x \map f(x,y)=-P(x)y+R(x)$ \ on jatkuva välillä $[a,b]$ jokaisella
            $y\in\R$.
\item[(ii)] $\abs{f(x,y_1)-f(x,y_2)}=\abs{P(x)}\abs{y_1-y_2}\leq L\abs{y_1-y_2},\quad 
             (x,y_1), \ (x,y_2) \in T$, missä $L=\D \max_{x\in [a,b]} \abs{P(x)}$.
\end{itemize}
Alkuarvotehtävä siis ratkeaa yksikäsitteisesti välillä $(a,b)$, kun $x_0\in (a,b)$.
(Sama pääteltiin Luvussa \ref{lineaarinen 1. kertaluvun DY} toisin keinoin.) \loppu
\end{Exa}

\pagebreak
\subsection*{Differentiaaliyhtälösysteemin ratkeavuus}
\index{differentiaaliyhtälön!h@ratkeavuus|vahv}
\index{differentiaaliyhtälö!g@--systeemi|vahv}

Picardin-Lindelöfin lauseen todellinen hienous piilee siinä, että lauseen väittämä voidaan
yleistää hyvin suoraviivaisesti koskemaan yleistä normaalimuotoista 
differentiaaliyhtälösysteemiä (vrt.\ Luku \ref{DY-käsitteet})
\[ \left\{ \begin{aligned} 
           y'_1(x) &= f_1(x,y_1, \ldots, y_n), \\
           y'_2(x) &= f_2(x,y_1, \ldots, y_n), \\
                  &\vdots \\
           y'_n(x) &= f_n(x,y_1 \ldots, y_n).
           \end{aligned} \right. \]
Alkuehdoiksi oletetaan
\[ 
y_i(x_0) = A_i, \quad i = 1 \ldots n, 
\]
missä $x_0\in(a,b)$. Tarkastellaan esimerkkinä alkuarvotehtävää kokoa $n=2\,$:
\begin{equation} \label{Picard-4}
\left\{ \begin{aligned}
&y' = f_1(x,y,u), \\
&u' = f_2(x,y,u), \quad x\in(a,b), \\
&y(x_0)=y_0,\ u(x_0)=u_0.
        \end{aligned} \right.
\end{equation}
Picardin-Lindelöfin lauseen muotoilussa tälle ongelmalle asetetaan 
\[
T=[a,b]\times[c_1,d_1]\times[c_2,d_2]\subset\R^3 \quad 
        \text{tai} \quad T=[a,b]\times\R^2\subset\R^3
\]
ja oletetaan
\begin{itemize}
\item[(i)]  Funktiot $\,x \map f_1(x,y,u)\,$ ja $\,x \map f_2(x,y,u)\,$ ovat jatkuvia välillä
            $[a,b]$ jokaisella $\,(y,u)\in[c_1,d_1]\times[c_2,d_2]\,$ tai jokaisella
            $\,(y,u)\in\Rkaksi$.
\item[(ii)] $f_1$ ja $f_2$ toteuttavat muuttujien $y,u$ suhteen Lipschitz-ehdon: \newline
            Jollakin $L\in\R_+$ pätee
            \begin{align*}
            \abs{f_i(x,y_1,u_1)-f_i(x,y_2,u_2)} 
                     &\le L\max\{\abs{y_1-y_2},\,\abs{u_1-u_2}\}, \quad i=1,2 \\
                     &\qquad\forall (x,y_1,u_1)\in T, \ (x,y_2,u_2)\in T.
            \end{align*}
\end{itemize}
Näillä oletuksilla päädytään vastaavaan väittämään kuin edellä, ts.\ alkuarvotehtävä 
\eqref{Picard-4} ratkeaa yksikäsitteisesti, joko $x_0$:n lähellä välillä 
$(x_0-\delta,x_0+\delta)$, tai vahvemmin oletuksin koko välillä $(a,b)$. Myös todistus on 
perusidealtaan sama kuin edellä: Lähtökohtana on ratkaisun konstruoiminen Picardin iteraatiolla
\[ \left\{ \begin{aligned} 
y_k(x) &= y_0+\int_{x_0}^x f_1(t,y_{k-1}(t),u_{k-1}(t))\,dt, \\
u_k(x) &= u_0+\int_{x_0}^x f_2(t,y_{k-1}(t),u_{k-1}(t))\,dt, \quad k=1,2, \ldots 
           \end{aligned} \right. \]

Palataan lopuksi Luvussa \ref{2. kertaluvun lineaarinen DY} tarkasteltuun toisen kertaluvun
lineaariseen ja homogeeniseen alkuarvotehtävään
\begin{equation} \label{Picard-5}
\begin{cases}
\,y''+P(x)y'+Q(x)y=0, \\ \,y(0)=A, \ y'(0)=B.
\end{cases}
\end{equation}
Kun tässä kirjoitetaan $u=y'$, niin päästään systeemimuotoon \eqref{Picard-4}, missä
\[
f_1(x,y,u)=u, \quad f_2(x,y,u) = -Q(x)y-P(x)u. 
\]
Jos $P$ ja $Q$ ovat jatkuvia välillä $(a,b)$ (kuten oletettiin Luvussa 
\ref{2. kertaluvun lineaarinen DY}), niin $x \map f_1(x,y,u)$ ja $x \map f_2(x,y,u)$ ovat
jatkuvia välillä $[a_1,b_1]$ jokaisella $(y,u)\in\Rkaksi$, kun $[a_1,b_1]\subset(a,b)$. Koska
\begin{align*}
f_1(x,y_1,u_1)-f_1(x,y_2,u_2)\ &=\ u_1-u_2, \\
f_2(x,y_1,u_1)-f_2(x,y_2,u_2)\ &=\ -Q(x)(y_1-y_2)-P(x)(u_1-u_2), 
\end{align*}
niin nähdään, että myös ehto (ii) toteutuu, kun asetetaan
\[ 
L=\max\{1,\,L_1\}, \quad L_1 = \max_{x\in[a_1,b_1]}\bigl\{\abs{P(x)}+\abs{Q(x)}\bigr\}. 
\]
Näin ollen Picardin--Lindelöfin lauseen (vahvemman väittämän) mukaan alkuarvotehtävä 
\eqref{Picard-5} ratkeaa yksikäsitteisesti välillä $(a_1,b_1)$. Koska tämä on totta aina kun
$[a_1,b_1] \subset (a,b)$, niin tehtävällä on yksikäsitteinen ratkaisu koko avoimella välillä
$(a,b)$. Lause \ref{lin-2: lause 2} on näin tullut todistetuksi Picardin--Lindelöfin lauseen
seuraamuksena.

\Harj
\begin{enumerate}

\item
Mitä voidaan Picardin--Lindelöfin lauseen valossa sanoa seuraavien alkuarvotehtävien 
ratkeavuudesta ja ratkaisujen yksikäsitteisyydestä $\R$:ssä tai $\R$:n osaväleillä?
\begin{align*}
&\text{a)}\ \ y'=\cos y+\frac{x}{y^2+1}\,,\,\ y(0)=A\in\R \\
&\text{b)}\ \ y'=\frac{y}{\cosh x}+e^{x-y^2},\,\ y(0)=A\in\R \\ 
&\text{c)}\ \ y'=\frac{y}{\cos x}+e^{x-y^2},\,\ y(0)=A\in\R \\
&\text{d)}\ \ y'=8xy+\frac{\sin(xy)}{x^2+2x-35}\,,\,\ y(0)=A\in\R \\
&\text{e)}\ \ y''=\cos(xy),\,\ y(0)=A\in\R,\ y'(0)=B\in\R
\end{align*}

\item
Määritä seuraaville alkuarvotehtäville kaikki alkupisteen lähiympäristössä pätevät ratkaisut.
Vertaa Picardin--Lindelöfin lauseen väittämään. \vspace{1mm}\newline
a) \ $y'=y^2,\,\ y(0)=1 \qquad\qquad\quad\,\ $
b) \ $y'=y^2,\,\ y(0)=1000$ \newline
c) \ $y'=y\sqrt{\abs{y}},\,\ y(0)=1 \qquad\quad\,\ \ $
d) \ $y'=\sqrt[4]{\abs{y}},\,\ y(0)=1$ \newline 
e) \ $y'=\sqrt[4]{\abs{y}},\,\ y(0)=0.0001 \qquad\,$
f) \ $y'=\sqrt[4]{\abs{y}},\,\ y(0)=0$

\item 
Halutaan laskea Neperin luku $e$ lähtien tiedosta, että $e=y(1)$, missä $y(x)$ ratkaisee
alkuarvotehtävän $\,y'=y,\ y(0)=1$. Millainen laskukaava saadaan luvulle $e$, kun tehtävä 
ratkaistaan Picardin iteraatiolla?

\item 
Laske seuraavissa alkuarvotehtävissä Picardin iteraatit $y_1(x)$ ja $y_2(x)$ ja vertaa 
tarkkaan ratkaisuun. 
(Vrt.\ Harj.teht.\ \ref{DYn numeeriset menetelmät}:\ref{H-dy-7: Euler-kokeita}.)
\[
\text{a)}\ \ \begin{cases} \,y'=-y^2, \\ \,y(0)=1 \end{cases}
\text{b)}\ \ \begin{cases} \,y'=y^2, \\ \,y(0)=1 \end{cases}
\text{c)}\ \ \begin{cases} \,y'=-2xy^2, \\ \,y(0)=1 \end{cases}
\text{d)}\ \ \begin{cases} \,y'=2xy^2, \\ \,y(0)=1 \end{cases}
\]

\item
Laske alkuarvotehtävälle
\[
\text{a)}\ \ \begin{cases} \,y''=-y, \\ \,y(0)=1,\ y'(0)=0 \end{cases} \quad
\text{b)}\ \ \begin{cases} \,y''=-y, \\ \,y(0)=0,\ y'(0)=1 \end{cases}
\]
likimääräinen ratkaisu iteroimalla $2n$ kertaa ($n\in\N$) Picardin iteraatiolla tehtävän
systeemimuodosta.

\item (*) \label{H-dy-8: ratkaisun yksikäsitteisyys}
Näytä, että Lauseen \ref{picard-lindelöf} oletuksin alkuarvotehtävän \eqref{Picard-1} ratkaisu
(jos olemassa) on yksikäsitteinen välillä $(x_0-\delta,x_0+\delta)$, kun $\delta$ on riittävän
pieni. \kor{Vihje}: Oleta $y_1(x)$ ja $y_2(x)$ ratkaisuiksi, lähde tehtävän integraalimuodosta
\eqref{Picard-3}, käytä oletusta (ii) ja valitse $\delta<1/L$.


\end{enumerate}