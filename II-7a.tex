\subsection*{Voiman momentti}

Ristituloa k�ytet��n fysiikassa erityisesti py�rimisliikkeen kuvauksessa, kuten 
(\pain{kulmano}p\pain{euden}) ja py�rimisliikett� aikaansaavan voiman \pain{momentin} 
ilmaisemiseen. Olkoon j�ykk� kappale tuettu pisteest� $O$ ja vaikuttakoon sen pinnan pisteess� 
$P$ voima $\vec F$ (vektori!). T�ll�in jos merkit��n $\vec r = \overrightarrow{OP}$, niin 
voiman momentti pisteen $O$ suhteen on vektori
\[
\vec M = \vec r \times \vec F.
\]
\begin{figure}[H]
\begin{center}
\import{kuvat/}{kuvaII-15.pstex_t}
\end{center}
\end{figure}
Jos voimia ja vaikutuspisteit� on useita, m��r�t��n kokonaismomentti vektorien yhteenlaskulla:
\[
\vec M=(\vec r_1 \times \vec F_1) + (\vec r_2 \times \vec F_2) + \cdots = \sum_i\vec M_i.
\]
Vektori $\vec M$ (mik�li $\neq \vec 0$) m��r�� kappaleen py�rimisakselin (kulmanopeusvektorin)
suunnan voimien vaikuttaessa. (Ehto $\vec M = \vec 0$ on tasapainoehto.)