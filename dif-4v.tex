\subsection*{Taylorin lauseen todistus}

Seuraavassa esitetään kaksi erilaista todistustapaa Taylorin lauseelle. Ensimmäinen todistus 
noudattaa 'pitkää kaavaa', jossa lauseen väittämä pyritään, paitsi todistamaan, myös
johtamaan. Tämä todistustapa on melko havainnollinen ja tekee väittämästä itsestään varsin
'uskottavan'. Tulos tosin jää hieman vaillinaiseksi: Piste $\xi$ pystytään (ilman
lisäponnistuksia) sijoittamaan vain suljetulle välille $[x_0,x]$ tai $[x,x_0]$, ja lisäksi
on oletettava $f^{(n+1)}$ jatkuvaksi välillä $(a,b)$. Toisena todistusvaihtoehtona esitetään
hyvin lyhyt ja elegantti todistus. Kummassakin todistustavassa päättelyn kulmakivi on
Differentiaalilaskun väliarvolause tai sen perusmuoto Rollen lause
(Lauseet \ref{toinen väliarvolause} ja \ref{Rollen lause}).

\underline{\vahv{Todistus 1.}} Oletetaan Lauseen \ref{Taylor} oletusten lisäksi, että
$f^{(n+1)}$ on jatkuva välillä $(a,b)$. Olkoon $x_0 <x < b$ ja tarkastellaan välillä $[x_0,x]$
funktiota
\[
g(t)=f(t)-T_n(t,x_0).
\]
Tällöin Lause \ref{toinen väliarvolause} sovellettuna funktioon $g^{(n)}$ väittää, että
\[
g^{(n)}(t)-g^{(n)}(x_0)=g^{(n+1)}(\xi)\,(t-x_0),\quad t\in (x_0,x],
\]
missä $\,\xi=\xi(t)\in (x_0,t)$. Koska $g^{(n)}(x_0)=0$ ja koska
\[
\frac{d^{n+1}}{dt^{n+1}}T_n(t,x_0)=0,
\]
niin seuraa
\[
g^{(n)}(t)=f^{(n+1)}(\xi)(t-x_0), \quad t\in (x_0,x].
\]
Kun merkitään
\[
m=\min_{t\in [x_0,x]} f^{(n+1)}(t), \quad M=\max_{t\in [x_0,x]} f^{(n+1)}(t),
\]
on päätelty, että
\begin{equation} \label{Taylor-välitulos 1}
m(t-x_0) \le g^{(n)}(t) \le M(t-x_0),\quad t\in [x_0,x].
\end{equation}
Mikäli $n\geq 1$, jatketaan päättelyä kirjoittamalla tulos \eqref{Taylor-välitulos 1} 
ekvivalenttiin muotoon
\[
\begin{cases}
\,\frac{d}{dt}[g^{(n-1)}(t)-\frac{1}{2}m(t-x_0)^2\,] &\ge\, 0,\ \ t\in [x_0,x], \\
\,\frac{d}{dt}[g^{(n-1)}(t)-\frac{1}{2}M(t-x_0)^2]   &\le\, 0,\ \ t\in [x_0,x].
\end{cases}
\]
Tästä nähdään, että (vrt. Lause \ref{monotonisuuskriteeri})
\begin{itemize}
\item[(i)] $g^{(n-1)}(t)-\frac{1}{2}m(t-x_0)^2$ \quad   on kasvava välillä $[x_0,x]$,
\item[(ii)] $g^{(n-1)}(t)-\frac{1}{2}M(t-x_0)^2$ \,\ \  on vähenevä välillä $[x_0,x]$.
\end{itemize}
Koska $g^{(n-1)}(x_0)=0$, päätellään että
\begin{equation} \label{Taylor-välitulos 2}
\frac{1}{2}m(t-x_0)^2\leq g^{(n-1)}(t)\leq \frac{1}{2}M(t-x_0)^2,\quad t\in [x_0,x].
\end{equation}
Mikäli $n \ge 2$, jatketaan päättelyä kirjoittamalla tulos \eqref{Taylor-välitulos 2} 
ekvivalenttiin muotoon
\[
\begin{cases}
\,\frac{d}{dt}[g^{(n-2)}(t)-\frac{1}{3!}m(t-x_0)^3] &\ge 0,\ \ t\in [x_0,x], \\
\,\frac{d}{dt}[g^{(n-2)}(t)-\frac{1}{3!}M(t-x_0)^3] &\le 0,\ \ t\in [x_0,x],
\end{cases}
\]
mistä päätellään samalla tavoin kuin edellä, että
\begin{equation} \label{Taylor-välitulos 3}
\frac{1}{3!}m(t-x_0)^3\leq g^{(n-2)}(t)\leq \frac{1}{3!}M(t-x_0)^3,\quad t\in [x_0,x].
\end{equation}
Näin jatkaen (induktio) päädytään lopulta tulokseen
\begin{equation} \label{Taylor-välitulos 4}
\frac{m}{(n+1)!}(t-x_0)^{n+1}\leq g(t)\leq \frac{M}{(n+1)!}(t-x_0)^{n+1},\quad t\in [x_0,x],
\end{equation}
joka $n$:n arvoilla $0,1,2$ jo saatiin välituloksina 
\eqref{Taylor-välitulos 1},\,\eqref{Taylor-välitulos 2},\,\eqref{Taylor-välitulos 3}.

Asetetaan lopuksi $t=x$ päättelyn lopputuloksessa \eqref{Taylor-välitulos 4} ja kirjoitetaan
tulos muotoon
\[
g(x)=\frac{c}{(n+1)!}\,(x-x_0)^{n+1},\quad c\in [m,M].
\]
Koska tässä $m$ on $f^{(n+1)}$:n minimiarvo ja $M$ maksimiarvo välillä $[x_0,x]$ ja koska 
$f^{(n+1)}$ on jatkuva ko.\ välillä, niin Lauseen \ref{ensimmäinen väliarvolause} mukaan on
\[
c=f^{(n+1)}(\xi) \quad \text{jollakin}\ \xi\in [x_0,x].
\]
Taylorin lauseen väittämä (muodossa $\xi\in[x_0\,,x]$) on näin todistettu tapauksessa $x>x_0$.
Tapauksessa $x<x_0$ on päättely vastaava. \loppu

\vspace{0.5cm}
\underline{\vahv{Todistus 2.}} \ Tässäkin tapauksessa otetaan muuttujaksi $t$ ja ajatellaan
$x \neq x_0$ kiinteäksi. Olkoon
\[
g(t) = f(t) - T_n(t,x_0) - H(t-x_0)^{n+1}, \quad H = \frac{f(x)-T_n(x,x_0)}{(x-x_0)^{n+1}}\,. 
\]
Tällöin on $g^{(k)}(x_0)=0,\ k=0 \ldots n$, ja lisäksi $g(x)=0$. Koska $g(x_0)=g(x)=0$,
niin Lauseen \ref{Rollen lause} mukaan on $g'(\xi_1)=0$ jollakin $\xi_1 \in (x_0,x)$ 
(ol.\ $x_0<x$). Tällöin koska $g'(x_0)=g'(\xi_1)=0$, niin saman lauseen mukaan on
$g''(\xi_2)=0$ jollakin $\xi_2 \in (x_0,\xi_1)$. Jatkamalla samalla tavoin päätellään, että 
$g^{(n)}(x_0)=g^{(n)}(\xi_n)=0$ jollakin $\xi_n \in (x_0,\xi_{n-1})$, jolloin Lauseen 
\ref{Rollen lause} mukaan on $g^{(n+1)}(\xi_{n+1})=0$ jollakin 
$\xi_{n+1} \in (x_0,\xi_n) \subset (x_0,x)$. Mutta $g^{(n+1)}(t) = f^{(n+1)}(t)-H(n+1)!$
--- Siis $f^{(n+1)}(\xi_{n+1})-H(n+1)!=0\ \impl$ väite. \loppu