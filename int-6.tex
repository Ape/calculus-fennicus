\section{Analyysin peruslause} \label{analyysin peruslause}
\alku
Palataan Luvussa \ref{määrätty integraali} tarkasteltuun probleemaan (P), jossa etsittiin
annetun funktion $f$ integraalifunktiota $y(x)$ välillä $(a,b)$ lisäehdoilla, että $y(x)$ on
jatkuva välillä $[a,b]$ ja $y(a)=0$. Kysymys probleeman ratkeavuudesta jätettiin tuolloin
avoimeksi. Näytetään nyt, että riittävä ehto ratkeavuudelle on, että funktio $f$ on
 j\pain{atkuva} \pain{välillä} $[a,b]$. Kyseessä on eräs matemaattisen analyysin
perustavimmista tuloksista, ja se tunnetaankin nimellä
\kor{Analyysin peruslause}\footnote[2]{Englanninkielinen nimi on 'The Fundamental Theorem of
Calculus'. Lauseen muotoilut kirjallisuudessa ovat vaihtelevia. --- Usein peruslauseena
esitetään vain Lauseen \ref{Analyysin peruslause} toinen osaväittämä, jolloin lause jää
sisällöltään melko kevyeksi, ks.\ Lause \ref{Analyysin peruslause 2} jäljempänä.}.
\begin{*Lause} \label{Analyysin peruslause} \vahv{(Analyysin peruslause)}
\index{Analyysin peruslause|emph} \index{Riemann-integroituvuus!b@jatkuvan funktion|emph}
Jos $f$ on jatkuva välillä $[a,b]$, niin
\begin{enumerate}
\item $f$ on Riemann-integroituva välillä $[a,b]$.
\item Funktio $F(x)=\int_a^x f(t)\, dt$ on jatkuvasti derivoituva välillä $[a,b]\,$ ja \\
      $F'(x)=f(x)$ ko.\ välillä.
\end{enumerate}
\end{*Lause}
Seuraavassa todistetaan Lauseen \ref{Analyysin peruslause} ensimmäinen (vaativampi) osaväittämä
heikennetyssä muodossa, uudistamalla Luvussa \ref{määrätty integraali} tehty Lipschitz-oletus
\[
\abs{f(x_1)-f(x_2)}\leq L\abs{x_1-x_2}\quad\forall x_1,x_2\in [a,b].
\]
\begin{Lause} \label{Analyysin peruslause 1} Välillä $[a,b]$ Lipschitz-jatkuva funktio on
Riemann-integroituva.% ko.\ välillä.
\end{Lause}
\tod Olkoon $X_n=\{x_k,\ k=0 \ldots n\}$ välin $[a,b]$ jako. Koska $f$ on (ainakin) jatkuva
välillä $[a,b]$, niin $f$ saavuttaa väleillä $[x_{k-1},x_k]$ maksimi- ja minimi-\linebreak
arvonsa (Lause\ref{Weierstrassin peruslause}), joten jakoon $X_n$ liittyvät Riemannin ylä- ja
alasummat \linebreak (ks.\ edellinen luku) voidaan kirjoittaa
\[
\overline{\sigma}(f,X_n)=\sum_{k=1}^n f(\xi_k)(x_k-x_{k-1}), \quad\,\
\underline{\sigma}(f,X_n)=\sum_{k=1}^n f(\eta_k)(x_k-x_{k-1}),
\]
missä $\xi_k,\eta_k\in[x_{k-1},x_k]$. Tämän ja oletetun Lipschitz-jatkuvuuden perusteella on
\begin{align*}
0 \,\le\, \overline{\sigma}(f,X_n)-\underline{\sigma}(f,X_n)
 &\,=\, \sum_{k=1}^n[f(\xi_k)-f(\eta_k)](x_k-x_{k-1}) \\
 &\,\le\, \sum_{k=1}^n L|\xi_k-\eta_k|(x_k-x_{k-1}) \\
 &\,\le\, \sum_{k=1}^n Lh_{X_n}(x_k-x_{k-1}) \,=\, L(b-a)h_{X_n}.
\end{align*}
On päätelty, että jos $\seq{X_n}$ on jono jakoja, jolle $h_{X_n} \kohti 0$, niin
\[
\overline{\sigma}(f,X_n)-\underline{\sigma}(f,X_n)\,=\,\mathcal{O}(h_{X_n})\,\kohti\,0.
\]
Tämän ja Lauseiden \ref{ylä- ja alaintegraalit raja-arvoina} ja \ref{Riemann-integroituvuus}
perusteella seuraa väite. \loppu

Analyysin peruslauseen 1.\ väittämän mukaisesti Lause \ref{Analyysin peruslause 1} on tosi myös,
jos $f$ on pelkästään jatkuva välillä $[a,b]$. Todistuksen pääidea on tässäkin tapauksessa sama,
mutta todistuksen on nojattava syvällisempään jatkuvuuden logiikkaan, tarkemmin sanoen
\pain{tasaiseen} j\pain{atkuvuuteen} (ks.\ Luku \ref{jatkuvuuden logiikka}). Todettakoon tässä
ainoastaan, että pelkän jatkuvuusoletuksen perusteella em.\ todistuskonstruktion
loppupäätelmäksi tulee
\[
\overline{\sigma}(f,X_n)-\underline{\sigma}(f,X_n) = o(1)
                  \,\kohti\,0, \quad \text{kun}\ h_{X_n} \kohti 0.
\]
Väitetty integroituvuus (Analyysin peruslauseen 1.\ väittämä) tulee siis näinkin todistetuksi.

Analyysin peruslauseen 2. väittämän todistamiseksi tarkastellaan (sovelluksiakin silmällä
pitäen, ks.\ esimerkki jäljempänä) yleisempää tilannetta, jossa funktio $f$ on integroituva 
välillä $[a,b]$ mutta ei välttämättä jatkuva koko välillä $[a,b]$. 
Tällöinkin funktio $F(x) = \int_a^x f(t)\,dt$ on määritelty koko välillä $[a,b]$ 
(Lause \ref{integroituvuus osavälillä}). Seuraavan lauseen toinen väittämä todistaa Lauseen
\ref{Analyysin peruslause} toisen osaväittämän, joten Analyysin peruslause tulee samalla
kokonaaan todistetuksi.
\begin{Lause} \label{Analyysin peruslause 2} Olkoon $f$ määritelty, rajoitettu ja
Riemann-integroituva välillä $[a,b]$ ja $F(x)=\int_a^x f(t)\,dt$, $\,x\in[a,b]$. Tällöin pätee:
\begin{enumerate}
\item Jos $\abs{f(x)}\leq M \ \forall x\in [a,b]$, niin $F$ on välillä $[a,b]$ Lipschitz-jatkuva
      vakiolla $L=M$.
\item Jos $f$:llä on oikeanpuoleinen raja-arvo $f(x^+)$ pisteessä $x\in[a,b)$, niin $F$ on 
      pisteessä $x$ oikealta derivoituva ja $D_+F(x)=f(x^+)$. Vastaavasti jos $f$:llä on 
      vasemmanpuoleinen raja-arvo $f(x^-)$ pisteessä $x\in(a,b]$, niin $F$ on vasemmalta 
      derivoituva pisteessä $x$ ja $D_-F(x)=f(x^-)$.
\end{enumerate}
\end{Lause}
\tod \ 1. Jos $\,-M \le f(x) \le M,\ x\in[a,b]$, niin integraalien vertailuperiaatteen
(Lause \ref{integraalien vertailuperiaate}) nojalla pätee jokaisella osavälillä 
$[x_1,x_2]\subset[a,b]\ (x_1<x_2)$
\[
-M(x_2-x_1) = \int_{x_1}^{x_2} (-M)\,dt \le \int_{x_1}^{x_2} f(t)\,dt 
              \le \int_{x_1}^{x_2} M\,dt = M(x_2-x_1).
\]
Integraalin additiivisuuden (Lause \ref{integraalin additiivisuus}) perusteella seuraa
\[
\abs{F(x_2)-F(x_1)} = \left|\int_{x_1}^{x_2} f(t)\,dt\right| 
                      \le M \abs{x_1-x_2}\ \ \forall x_1,x_2\in[a,b]\ \impl\ \text{väite 1.}
\]

2. Olkoon oikeanpuoleinen raja-arvo $f(x^+)$ olemassa pisteessä $x\in[a,b)$ ja olkoon $\eps>0$.
Tällöin on olemassa $\delta\in(0,b-x]$ siten, että $\abs{f(t)-f(x^+)}<\eps$, kun 
$t\in[x,x+\delta)$ (Lause \ref{approksimaatiolause}). Tällöin jos $0<\Delta x<\delta$, niin
integraalin lineaarisuuden, additiivisuuden ja vertailuperiaatteen nojalla
\begin{multline*}
\left|\frac{F(x+\Delta x)-F(x)}{\Delta x}-f(x^+)\right| 
               \,=\, \left|\frac{1}{\Delta x}\int_x^{x+\Delta x} [f(t)-f(x^+)]\,dt\right| \\
               \,\le\, \frac{1}{\Delta x}\int_x^{x+\Delta x} |f(t)-f(x^+)|\,dt
               \,\le\, \frac{1}{\Delta x}\int_x^{x+\Delta x} \eps\,dt \,=\,\eps.
\end{multline*}
Koska tässä $\eps>0$ oli mielivaltainen ja epäyhtälö on pätevä jokaisella 
$\Delta x\in(0,\delta)$, missä $\delta>0$, niin $F$ on määritelmän mukaan oikealta derivoituva
pisteessä $x$ ja $D_+F(x)=f(x^+)$. Väittämän toinen osa todistetaan vastaavasti. \loppu
\begin{Exa}: \vahv{Liikelaki}. \index{zza@\sov!Liikelaki} \label{liikelaki} \ Jos kappaleen
(massa = $m$) suoraviivaisessa liikkeessä vaikuttaa liikesuuntaan voima $f(t)$ ($t$=aika),
niin \pain{liikemäärän} \pain{säil}y\pain{mislaki} aikavälillä $[0,\infty)$ on
\[
mv(t)-mv(0)= \int_0^t f(t')dt',
\]
missä $v(t)=$ kappaleen nopeus hetkellä $t$ ja oikealla puolella oleva (Riemannin) integraali
on voiman $f$ \pain{im}p\pain{ulssi} aikavälillä $[0,t]$. Jos $f$ on jatkuva pisteessä $t>0$,
niin Lauseen \ref{Analyysin peruslause 2} perusteella liikemäärän säilymislaista seuraa
(puolittain derivoimalla) \pain{liike}y\pain{htälö}
\[
mv'(t)=f(t).
\]
Jos $f$ on pisteessä $t$ epäjatkuva (fysikaalisesti mahdollista!), ei liikeyhtälö ole (yleensä)
voimassa ko.\ hetkellä. Liikemäärän säilymislaki sen sijaan säilyttää pätevyytensä, edellyttäen
ainoastaan, että $f$ on (Riemann-)integroituva aikaväleillä $[0,t]$. Liikemäärän
säilymislakia voidaan näin ollen pitää 'alkuperäisenä' fysiikan lakina ja liikeyhtälöä 
pikemminkin tämän seuraamuksena mainitun lisäoletuksen ($f$ jatkuva pisteessä $t$) vallitessa. 
Esimerkiksi jos $v(0)=v_0$ ja
\[
f(t)=\begin{cases}
F,   &\text{ kun } t\in [t_1,t_2], \\
\,0, &\text{ muuten},
\end{cases}
\]
missä $F \neq 0$ on vakio ja $0<t_1<t_2\,$, niin liikemäärän säilymislain mukainen
(fysikaalisesti oikea!) ratkaisu on
\begin{align*}
v(t) = \begin{cases}
v_0,                         &\text{kun}\,\ t\in [0,t_1], \\[3mm]
v_0+\dfrac{F}{m}\,(t-t_1),   &\text{kun}\,\ t\in [t_1,t_2], \\[3mm]
v_0+\dfrac{F}{m}\,(t_2-t_1), &\text{kun}\,\ t\in [t_2,\infty).
\end{cases}
\end{align*}
Liikeyhtälö $mv'(t)=f(t)$ on tässä tapauksessa voimassa väleillä $(0,t_1)$, $(t_1,t_2)$ ja
$(t_2,\infty)$. Pisteissä $t_1$ ja $t_2$ nopeus $v(t)$ on Lauseen \ref{Analyysin peruslause 2}
mukaisesti jatkuva (koska $f$ on rajoitettu) ja sekä vasemmalta että oikealta derivoituva.
Koska $f$:llä on näissä
pisteissä hyppyepäjatkuvuus, ovat toispuoliset derivaatat erisuuret, ja näin ollen $v(t)$ ei
ole derivoituva pisteissä $t_1$ ja $t_2$. \loppu
\end{Exa}
\begin{figure}[H]
\setlength{\unitlength}{1cm}
\begin{center}
\begin{picture}(10,3)(-1,-0.4)
\put(-1,0){\vector(1,0){10}} \put(8.8,-0.5){$t$}
\put(0,-1){\vector(0,1){4}} \put(0.2,2.8){$v(t)$}
\path(0,1)(3,1)(6,2)(9,2) 
\put(0,1){\line(-1,0){0.1}} \put(-0.5,0.9){$v_0$}
\put(3,0){\line(0,-1){0.1}} \put(2.9,-0.5){$t_1$}
\put(6,0){\line(0,-1){0.1}} \put(5.9,-0.5){$t_2$}
\end{picture}
\end{center}
\end{figure}

\subsection*{Funktion $\,\int_{g_1(x)}^{g_2(x)} f(t)\,dt$ derivaatta}

Integraalin additiivisuuden nojalla otsikon funktio voidaan esittää muodossa \linebreak
$\,F(g_2(x))-F(g_1(x))$, missä $F(x)=\int_a^x f(t)\,dt$. Yhdistetyn funktion derivoimissäännön
ja Lauseen \ref{Analyysin peruslause 2} perusteella päädytään derivoimissääntöön
\[
\boxed{\quad \frac{d}{dx}\int_{g_1(x)}^{g_2(x)} f(t)\, dt 
                      = f(g_2(x))g_2'(x)-f(g_1(x))g_1'(x) \quad}
\]
seuraavin oletuksin:
\begin{itemize}
\item[(i)]   $g_1(t)$ ja $g_2(t)$ ovat derivoituvia pisteessä $t=x$.
\item[(ii)]  $f(t)$ on määritelty, rajoitettu ja Riemann-integroituva jollakin välillä $[a,b]$,
             missä $\,a<\min\{g_1(x),g_2(x)\}\,$ ja $b>\max\{g_1(x),g_2(x)\}$.
\item[(iii)] $f(t)$ on jatkuva pisteissä $g_1(x)$ ja $g_2(x)$.
\end{itemize}
\begin{Exa} Jokaisella $x \neq 0$ (myös kun $x=1$\,(!)) pätee
\[
\frac{d}{dx}\int_x^{1/x} \frac{e^t}{t}\,dt\,
               =\,-\frac{1}{x^2}\left(x\,e^{1/x}\right)-\frac{e^x}{x}\,
               =\,-\frac{1}{x}(e^x+e^{1/x}),
\]
sillä oletukset (i)--(iii) ovat voimassa. Samasta syystä pätee jokaisella $x>0$
\[
\frac{d}{dx}\int_x^{x^2} \frac{e^t}{t}\,dt\,
               =\,2x \cdot \frac{e^{x^2}}{x^2}-\frac{e^x}{x}\,
               =\,\frac{1}{x}(2e^{x^2}-e^x).
\]
Sen sijaan jos $x \le 0$, niin tämä lasku ei ole pätevä, koska tällöin ei oletus (ii) 
(tapauksessa $x=0$ ei myöskään (iii)) ole voimassa. \loppu
\end{Exa}

\subsection*{Osittaisintegrointi ja sijoitus määrätyssä integraalissa}
\index{osittaisintegrointi|vahv}
\index{muuttujan vaihto (sijoitus)!b@integraalissa|vahv}

Määrätylle integraalille pätevät riittävin säännöllisyysoletuksin seuraavat laskukaavat, 
vrt.\ vastaavat määräämättömän integraalin kaavat Luvussa \ref{osittaisintegrointi}. 
\begin{enumerate}
\item \pain{Osittaisinte}g\pain{rointikaava}
\[
\int_a^b f'(x)g(x)\, dx=\sijoitus{a}{b} f(x)g(x)-\int_a^b f(x)g'(x)\, dx.
\]
\item \pain{Si}j\pain{oituskaava}
\[
\int_a^b f(x)\, dx = \int_\alpha^\beta f(u(t))u'(t)\, dt,\quad u(\alpha)=a, \ u(\beta)=b.
\]
\end{enumerate}
Osittaisintegrointikaava on pätevä silloin kun $f$ ja $g$ ovat välillä $[a,b]$ jatkuvasti 
derivoituvia. Tällöin $f'g$ ja $fg'$ ovat jatkuvia välillä $[a,b]$, jolloin Analyysin
peruslauseen ja integraalin lineaarisuuden perusteella pätee
\begin{align*}
\int_a^b f'(x)g(x)\,dx+\int_a^b f(x)g'(x)\,dx
            &= \int_a^b [f'(x)g(x)+f(x)g'(x)]\,dx \\     
            &= \int_a^b \frac{d}{dx}\,f(x)g(x)\,dx = \sijoitus{a}{b} f(x)g(x).
\end{align*}

Määrätyn integraalin sijoituskaava vaatii hieman pitemmät perustelut. Olkoon esimerkiksi
$\alpha<\beta$ (kaavan pätevyys ei tätä edellytä). Tällöin on oletettava
\begin{itemize}
\item[(i)] $u$ on jatkuvasti derivoituva välillä $[\alpha,\beta]$,
\item[(ii)] $f$ on jatkuva välillä $[A,B]$, jolle pätee
$t\in [\alpha,\beta] \ \impl \ u(t)\in [A,B]$.
\end{itemize}
Kun merkitään
\[
F(x)=\int_a^x f(s)ds, \quad x\in[A,B],
\]
niin Analyysin peruslauseen mukaan on
\[
F'(x)=f(x),\quad x\in [A,B].
\]
Tällöin on oletuksien (i)-(ii) perusteella
\[
\frac{d}{dt}F(u(t))=f(u(t))u'(t),\quad t\in [\alpha,\beta],
\]
joten
\begin{align*}
\int_\alpha^\beta f(u(t))u'(t)\, dt &= \sijoitus{\alpha}{\beta} F(u(t)) \\
&= F(u(\beta))-F(u(\alpha)) \\
&= F(b)-F(a)=\int_a^b f(x)\, dx.
\end{align*}
Tapauksessa $\alpha >\beta$ on puhuttava välistä $[\beta,\alpha]$, muuten perustelut ovat samat.
Huomattakoon, että koska funktion $u$ injektiviisyttä ei vaadittu, eivät $\alpha$ ja $\beta$
välttämättä ole yksikäsitteiset, vrt.\ kuvio.
\begin{figure}[H]
\setlength{\unitlength}{1cm}
\begin{center}
\begin{picture}(12,6)
\put(0,0){\vector(1,0){12}} \put(11.8,-0.4){$t$}
\put(0,0){\vector(0,1){6}} \put(0.2,5.8){$x$}
% f(x)=0.05(x-2)(x-8)(x-10)+2
\curve(
    1.50,   0.619,
    2.00,   2.000,
    2.50,   3.031,
    3.00,   3.750,
    3.50,   4.194,
    4.00,   4.400,
    4.50,   4.406,
    5.00,   4.250,
    5.50,   3.989,
    6.00,   3.600,
    6.50,   3.181,
    7.00,   2.750,
    7.50,   2.344,
    8.00,   2.000,
    8.50,   1.756,
    9.00,   1.650,
    9.50,   1.719,
   10.00,   2.000,
   10.50,   2.531,
   11.00,   3.350)
%\put(0.9,0.85){$\bullet$}\put(4.9,2.85){$\bullet$}
%\drawline(1,1)(5,3)
%\drawline(3,1.1)(5,2.1)
\dashline{0.2}(0,2)(11,2)
\dashline{0.2}(2,0)(2,2) \dashline{0.2}(8,0)(8,2) \dashline{0.2}(10,0)(10,2)
\dashline{0.2}(0,3.75)(7,3.75)
\dashline{0.2}(3,0)(3,3.75) \dashline{0.2}(5.8,0)(5.8,3.75)
\put(0,1){\line(-1,0){0.1}} \put(0,5){\line(-1,0){0.1}}
\put(-0.6,0.9){$A$} \put(-0.6,4.9){$B$}
\put(-0.45,1.9){$a$} \put(-0.43,3.65){$b$}
\put(1.9,-0.4){$\alpha$} \put(7.9,-0.4){$\alpha$} \put(9.9,-0.4){$\alpha$}
\put(2.9,-0.5){$\beta$} \put(5.71,-0.5){$\beta$}
\put(11,3.6){$x=u(t)$}
\end{picture}
\end{center}
\end{figure}
\begin{Exa}
Kun integraalissa
\[
\int_0^2 \frac{e^{-x}}{1+\sqrt{x}}\, dx
\]
tehdään sijoitus
\[
\sqrt{x}=t \ge 0\ \ \ekv\ \ x=u(t)=t^2,
\]
niin
\[
dx=u'(t)dt=2tdt
\]
ja
\[
u(\alpha)=0\ \impl \alpha=0, \quad u(\beta)=2\ \impl\ \beta=\sqrt{2},
\]
joten integraali saa muodon
\[
\int_0^2 \frac{e^{-x}}{1+\sqrt{x}}\, dx=\int_0^{\sqrt{2}} \frac{e^{-t^2}}{1+t}\,2t\,dt
                                       =2\int_0^{\sqrt{2}} \frac{te^{-t^2}}{1+t}\,dt. \loppu
\]
\end{Exa}
Esimerkin muunnetussa integraalissa (toisin kuin alkuperäisessä) integroitava funktio on 
säännöllinen (sileä) koko integroimisvälillä. Integroitavan funktion säännöllisyys on yleisesti
eduksi silloin, kun integraalin arvo lasketaan numeerisilla menetelmillä, ks.\ Luku 
\ref{numeerinen integrointi} jäljempänä.

\subsection*{Integraalilaskun väliarvolause}

Seuraava lause on väliarvolauseiden sarjan kolmas ja viimeinen, vrt.\ Lauseet 
\ref{ensimmäinen väliarvolause} ja \ref{toinen väliarvolause}. Verrattuna aiempiin
väliarvolauseisiin tämä lause ei ole kovin itsenäinen, sillä se seuraa helposti Analyysin
peruslauseesta ja Differentiaalilaskun väliarvolauseesta
(Harj.teht.\,\ref{H-int-6: todistuksia}a).
\begin{Lause} \label{kolmas väliarvolause} \vahv{(Integraalilaskun väliarvolause)}
\index{vzy@väliarvolauseet!c@integraalilaskun|emph} \index{Integraalilaskun väliarvolause|emph}
Jos f on jatkuva välillä $[a,b]$, niin jollakin $\xi\in(a,b)$ on
\[
\int_a^b f(x)\,dx = f(\xi)(b-a).
\]
\end{Lause}
Seuraavaa Lauseen \ref{kolmas väliarvolause} yleisempää (myös itsenäisempää) muotoa sanotaan
integraalilaskun \kor{yleistetyksi} väliarvolauseeksi.
\begin{Lause} \label{kolmas väliarvolause - yleistys}
Jos $f$ on jatkuva välillä $[a,b]$, $g$ on Riemann--integroituva välillä $[a,b]$ ja joko
$g(x) \ge 0$ tai $g(x) \le 0$ koko välillä $[a,b]$, niin jollakin $\xi \in[a,b]$ on
\[
\int_a^b f(x)g(x)\,dx=f(\xi)\int_a^b g(x)\,dx.
\]
\end{Lause}
\tod Jos $g(x)\geq 0$ ja $f_{\text{min}}$ ja $f_{\text{max}}$ ovat $f$:n minimi- ja maksimiarvot
välillä $[a,b]$, niin päätellään
\begin{align*}
&f_{\text{min}}\,g(x) \le f(x)g(x) \le f_{\text{max}}\,g(x),\quad x\in [a,b] \\[1mm]
&\impl\; f_{\text{min}} \int_a^b g(x)\,dx \le \int_a^b f(x)g(x)\,dx 
                                  \le f_{\text{max}} \int_a^b g(x)\,dx \\
&\impl\; \int_a^b f(x)g(x)\,dx = \eta \int_a^b g(x)\,dx, \quad 
                                    \eta\in [f_{\text{min}},f_{\text{max}}]. 
\end{align*}
Ensimmäisen väliarvolauseen (Lause \ref{ensimmäinen väliarvolause}) mukaan on tässä 
$\eta=f(\xi)$ jollakin $\xi\in [a,b]$, joten väite seuraa. Tapauksessa $g(t)\leq 0$ voidaan
käyttää jo todistettua väittämää, kun $g$:n tilalla on $-g$. \loppu

Funktion $f$ 
\index{keskiarvo!a@funktion}%
\kor{keskiarvoksi} (engl.\ average value) välillä $[a,b]$ sanotaan lukua
\[
K(f) = \frac{1}{b-a}\int_a^b f(x)\,dx.
\]
Jos $\,\D g(x) \ge 0\ \forall x\in[a,b]\,\ \text{ja}\,\ \int_a^b g(x)\,dx>0$, niin lukua
\[
K_g(f)= \frac{\int_a^b f(x)g(x)\, dx}{\int_a^b g(x)\,dx}
\]
sanotaan $f$:n \kor{painotetuksi keskiarvoksi} (engl.\ weighted average) välillä $[a,b]$ ja
\index{painotettu keskiarvo}%
funktiota $g$ tällöin \kor{painofunktioksi}. Lauseen \ref{kolmas väliarvolause - yleistys}
mukaan välillä $[a,b]$ jatkuvalle funktiolle pätee $K_g(f)=f(\xi)$ jollakin $\xi\in[a,b]$.

\subsection*{Taylorin lauseen integraalimuoto}

Monissa sovelluksissa (esim.\ numeeristen menetelmien virhettä arvioitaessa, ks.\
Harj.teht.\,\ref{H-int-6: virhekaava}) seuraava Taylorin lauseen vaihtoehtoinen
\kor{integraalimuoto} on hyvin kätevä --- vrt.\ Lause \ref{Taylor}, jota kutsutaan Taylorin
lauseen \kor{väliarvomuodoksi}.
\begin{Lause} \label{Taylor-integraali} \index{Taylorin lause!a@integraalimuoto|emph}
\vahv{(Taylorin lause -- integraalimuoto)} Jos $f$ on $n+1$ kertaa jatkuvasti derivoituva 
välillä $[x_0,x]$ $(x>x_0)$ tai $[x,x_0]$ ($x<x_0$) ja $T_n(x,x_0)$ on $f$:n Taylorin polynomi
astetta $n$ pisteessä $x_0$, niin
\[
f(x)-T_n(x,x_0)=\int_{x_0}^x \frac{1}{n!}(x-t)^n f^{(n+1)}(t)\,dt.
\]
\end{Lause}
\tod Lähdetään ilmeisestä identiteetistä
\[
f(x)=f(x_0)+\int_{x_0}^x f'(t)\,dt,
\]
joka on sama kuin lauseen väittämä tapauksessa $n=0$. Jos tässä $f$ on kahdesti jatkuvasti
derivoituva integroimisvälillä (oletus tapauksessa $n=1$), voidaan integroida osittain
seuraavasti:
\begin{align*}
f(x) &= f(x_0)-\sijoitus{x_0}{x} (x-t)f'(t)+\int_{x_0}^x (x-t)f''(t)\,dt \\
&= f(x_0)+f'(x_0)(x-x_0)+\int_0^x (x-t)f''(t)\,dt.
\end{align*}
Näin on todistettu väittämä tapauksessa $n=1$. Jatkamalla osittaisintegrointia nähdään 
vastaavasti, että väittämä on tosi, kun $n=2$, jne.\ (Yleinen todistus:
Harj.teht.\,\ref{H-int-6: todistuksia}b.) \loppu

\Harj
\begin{enumerate}

\item
Jos $f$ jatkuva välillä $[a,b]$, niin mikä on funktion
\[
g(x)=\int_a^b [f(t)-x\,]^2\,dt
\]
pienin arvo ja missä se saavutetaan?

\item
Laske:
\begin{align*}
&\text{a)}\,\ \frac{d}{dx}\int_2^x \frac{\sin t}{t}\,dt \qquad
 \text{b)}\,\ \frac{d}{dt}\int_t^3 \frac{\sin x}{x}\,dx \qquad
 \text{c)}\,\ \frac{d}{dx}\int_{x^2}^x \frac{\sin t}{t}\,dt \\
&\text{d)}\,\ \frac{d}{dx}x^2\int_0^{x^2} \frac{\sin u}{u}\,du \qquad
 \text{e)}\,\ \frac{d}{dt} \int_{-t^{-1}}^t \frac{\cos y}{1+y^2}\,dy \\
&\text{f)}\,\ \frac{d}{d\theta}\int_{\sin\theta}^{\cos\theta} \frac{1}{1-x^2}\,dx \qquad
 \text{g)}\,\ \frac{d}{dx}F(\sqrt{x}),\,\ F(t)=\int_0^t \cos(x^2)\,dx \\
&\text{h)}\,\ H'(2),\,\ H(x)=3x\int_4^{x^2} e^{-\sqrt{t}}\,dt
\end{align*}

\item
Ratkaise $y(x)$:
\[
\text{a)}\ \ y(x)=1-\int_0^x y(t)\,dt \qquad
\text{b)}\ \ y(x)=\pi\left(1+\int_1^x y(t)\,dt\right)
\]

\item
Laske integroimalla osittain kohdissa a)--f) ja sijoituksella kohdissa g)--l):
\begin{align*}
&\text{a)}\ \ \int_0^1 x e^{-x}\,dx \qquad
 \text{b)}\ \ \int_0^\pi x\cos x\,dx \qquad
 \text{c)}\ \ \int_0^\pi x\sin x\,dx \\
&\text{d)}\ \ \int_0^2 \sqrt{x}\ln x\,dx \qquad
 \text{e)}\ \ \int_0^1 \Arctan x\,dx \qquad
 \text{f)}\ \ \int_0^\pi e^{-x}\sin x\,dx \\
&\text{g)}\ \ \int_{-2}^2 \frac{1}{2x^2+4x+3}\,dx \qquad
 \text{h)}\ \ \int_0^{\ln 2} \frac{e^x}{1+e^x}\,dx \qquad
 \text{i)}\ \ \int_1^4 \frac{\sqrt{x}}{\sqrt[4]{x}+1}\,dx \\
&\text{j)}\ \ \int_0^1 \sqrt{\frac{1-x}{1+x}}\,dx \qquad
 \text{k)}\ \ \int_0^{\pi/4} \frac{1}{1+\tan x}\,dx \qquad
 \text{l)}\ \ \int_0^{\pi} \frac{\sin x}{2-\cos x}\,dx
\end{align*}

\item
Olkoon $R>0$ ja $ab \neq 0$. Laske
\[
\text{a)}\ \ \int_0^R x^2\sqrt{R^2-x^2}\,dx, \qquad
\text{b)}\ \ \int_0^{\pi/2} \frac{1}{a^2\cos^2 x+b^2\sin^2 x}\,dx.
\]

\item
Yksinkertaista seuraavat funktiolausekkeet sijoitusta käyttäen ($x>0$).
\[
\text{a)}\ \ f(x)=\int_0^x \sqrt{t}\,\sqrt{x^2-t^2}\,dx \qquad
\text{b)}\ \ f(x)=\int_{-x/2}^{x/2} \frac{1}{(x^2-t^2)^{3/2}}\,dt
\]

\item
Laske funktion $f(x)=\sin^2x$ \ a) keskiarvo, \,b) funktiolla $g(x)=x$ painotettu keskiarvo
välillä $[0,\pi]$.


\item \label{H-int-6: todistuksia}
a) Todista Integraalilaskun väliarvolause soveltamalla Differentiaalilaskun väliarvolausetta
funktioon $F(x)=\int_a^x f(t)\,dt$. \ b) Todista Lause \ref{Taylor-integraali} induktiolla. \
c) Johda Lauseesta \ref{Taylor-integraali} Taylorin lauseen väliarvomuoto (Lause \ref{Taylor})
käyttämällä hyväksi Lausetta \ref{kolmas väliarvolause - yleistys}.

\item (*)
Onko funktiolla $\D\,F(x)=\int_0^{2x-x^2}\cos\left(\frac{1}{1+t^2}\right)dt\,$
suurin tai pienin arvo? Perustele!

\item (*) \index{osittaissummaus}
Näytä oikeaksi \kor{osittaissummauksen} kaava
\[
\sum_{k=1}^n f_k g_k = \sijoitus{k=1}{k=n} F_k g_k - \sum_{k=1}^{n-1} F_k\,(g_{k+1}-g_k), \quad
                       F_k = \sum_{i=1}^k f_i.
\]             
Mikä on vastaava kaava määrätylle integraalille?

\item (*) \label{H-int-6: virhekaava}
Olkoon $f$ kolmesti jatkuvasti derivoituva välillä $[a-h,a+h]$. Johda Lauseen
\ref{Taylor-integraali} avulla keskeisdifferenssiapproksimaation virhekaava
\begin{align*}
&f'(a)-\frac{f(a+h)-f(a-h)}{2h} \,=\, -\frac{1}{h}\int_{a-h}^{a+h} k(t)f'''(t)\,dt, \\[2mm]
&\text{missä}\quad k(t)=\begin{cases}
      \,\frac{1}{2}(a-h-t)^2, &\text{kun}\ \ t\in[\,a-h,a\,], \\
      \,\frac{1}{2}(a+h-t)^2, &\text{kun}\ \ t\in[\,a,a+h\,].
      \end{cases}
\end{align*}
Johda tästä edelleen virhekaavalle väliarvomuoto
(vrt.\ Propositio \ref{keskeisdifferenssin tarkkuus}) 
\[
f'(a)-\frac{f(a+h)-f(a-h)}{2h} \,=\, -\frac{1}{6}\,h^2 f'''(\xi), \quad\ \xi\in[a-h,a+h].
\]

\end{enumerate}