\section{Kompleksinen eksponenttifunktio. \\ Hyperboliset funktiot} 
\label{kompleksinen eksponenttifunktio}
\sectionmark{Kompleksinen eksponenttifunktio}
\alku
\index{analyyttinen kompleksifunktio|vahv}
\index{funktio A!e@kompleksifunktio|vahv}

Peruseksponenttifunktio voidaan laajentaa kompleksimuuttujan funktioksi $\exp: \C\kohti\C$
niin, että sen keskeiset ominaisuudet säilyvät. Määritelmä kompleksialueella --- joka toimii
myös reaalimuuttujaan rajoitettuna --- on seuraava.
\begin{Def} \index{kompleksimuuttujan!c@eksponenttifunktio|emph}
\index{funktio C!f@$\exp$ ($e^x$, $e^z$)|emph}
Peruseksponenttifunktio $\exp (z)=e^z$ on funktio, jolla on ominaisuudet (aksioomat)
\index{eksponenttifunktio!a@aksioomat|emph}
\begin{itemize}
\item[(C1)] $\exp (z)$ on määritelty $\forall z\in\C$,
\item[(C2)] $\exp(x+i0)=e^x\ \forall x\in\R$,
\item[(C3)] $\exp(z)$ on derivoituva origossa,
\item[(C4)] $\exp (z_1+z_2)=\exp(z_1)\exp(z_2)\quad\forall z_1,z_2\in\C$.
\end{itemize}
\end{Def}
Aksiooman C2 mukaan $e^z$ on reaalisen eksponenttifunktion laajennus kompleksitasoon.
Aksiooma C3 (vrt.\ Luku \ref{analyyttiset funktiot}) vastaa edellisen luvun aksioomaa E3'.
Kompleksiselle eksponenttifunktille tämä aksiooman muoto on välttämätön, sillä pelkkä
jatkuvuusoletus (vastaten reaalisen eksponenttifunktion aksioomaa E3) ei takaisi funktion
$\exp(z)$ yksikäsitteisyyttä kompleksifunktiona
(ks.\ Harj.teht.\,\ref{H-VII-3: E(z) vaihtoehto}). Aksioomilla C1--C4 sen sijaan $\exp(z)$
tulee yksikäsitteisesti määritellyksi. Osoitetaan tässä ainoastaan, että aksioomien
mukainen funktio $\,\exp(z)\,$ on olemassa, ja annetaan samalla funktiolle koko $\C$:ssä
pätevä laskusääntö.
\begin{Lause} \label{funktio exp(z)}
Funktio
\begin{equation} \label{exp(z):n laskukaava}
\boxed{\kehys\quad e^{x+iy}=e^x(\cos y+i\sin y),\quad z=x+iy\in\C \quad}
\end{equation}
toteuttaa aksioomat C1--C4.
\end{Lause}
\tod (C4) \  Jos $z_1=x_1+iy_1$, $z_2=x_2+iy_2$, niin määritelmän \eqref{exp(z):n laskukaava} 
ja kompleksialgebran laskusääntöjen (ks.\ Luku \ref{kompleksiluvuilla laskeminen})
perusteella 
\begin{align*}
e^{z_1+z_2} &\,=\, e^{x_1+x_2}[1\vkulma(y_1+y_2)] \\
           &\,=\, e^{x_1}\cdot e^{x_2}\cdot (1\vkulma{y_1})\cdot (1\vkulma{y_2}) \\
           &\,=\, [e^{x_1}\cdot (1\vkulma{y_1})]\,[e^{x_2}\cdot (1\vkulma{y_2})] \\
           &\,=\, e^{z_1}\cdot e^{z_2}.
\end{align*}

(C3) \ Käytetään hajotelmia
\begin{align*}
e^x &= 1+x+ r_1(x), \\
\cos y &= 1+ r_2(y), \\
\sin y &= y + r_3(y),
\end{align*}
missä $\lim_{t \kohti 0} t^{-1}r_i(t)=0$ (vrt.\ Luvut \ref{kaarenpituus} ja 
\ref{exp(x) ja ln(x)}). Näiden ja säännön \eqref{exp(z):n laskukaava} perusteella on
\begin{align*}
e^z &= [1+x+r_1(x)]\,[1+iy+r_2(y)+ir_3(y)] \\
    &= 1 + x +iy + r(z) \\
    & = 1+z+r(z), \quad z=x+iy,
\end{align*}
missä jäännöstermille pätee $\,\lim_{z \kohti 0} z^{-1}r(z)=0$. Näin ollen
\[
\lim_{z \kohti 0} \frac{e^z-e^0}{z} = \lim_{z \kohti 0} \frac{e^z-1}{z} 
                                    = \lim_{z \kohti 0} \left(1+\frac{r(z)}{z}\right) = 1.
\]
Siis $e^z$ on origossa derivoituva ja derivaatan arvo $=1$.

(C2) ja (C1) \ Ilmeisiä. \loppu

Funktio $e^z$ on derivoituva, ei ainoastaan origossa (aksiooma C3), vaan koko
kompleksitasossa, sillä ominaisuudet C3--C4 yhdistämällä seuraa
\[
\lim_{\Delta z \kohti 0} \frac{e^{z+\Delta z}-e^z}{\Delta z} 
           = e^z \lim_{\Delta z \kohti 0} \frac{e^{\Delta z}-1}{\Delta z} = e^z, \quad z\in\C.
\]
Derivoimissääntö on siis sama kuin reaaliselle (perus)eksponenttifunktiolle:
\index{derivoimissäännöt!g@eksponenttifunktio}%
\begin{equation} \label{exp(z):n derivaatta}
\boxed{\kehys\quad \dif\,e^z = e^z.\quad}
\end{equation}
Tämän mukaan kompleksifunktio $e^z$ on koko kompleksitasossa analyyttinen eli kokonainen funktio 
(vrt.\ Luku \ref{analyyttiset funktiot}). Tähän ja aksioomaan C2 viitaten sanotaankin,
\index{analyyttinen jatko}%
että $e^z$ on reaalisen eksponenttifunktion \kor{analyyttinen jatko} koko kompleksitasoon.

Derivoimissäännöstä \eqref{exp(z):n derivaatta} (joka siis seuraa suoraan aksioomista 
C3--C4) voidaan myös päätellä, että aksioomien C1--C4 mukainen funktio on yksikäsitteinen.
Nimittäin jos funktio $E(z)$ toteuttaa nämä aksioomat, niin on $DE(z)=E(z)$, jolloin
derivoimalla funktiota $u(z)=E(z)e^{-z}$ todetaan, että $Du(z)=0$ 
(vrt.\ Lauseen \ref{exp-dy} todistus). Koska aksiooman C2 mukaan on $u(0)=1$, on
pääteltävissä (sivuutetaan yksityiskohdat), että $u(z)=1\ \ekv\ E(z)=e^z\ \forall z\in\C$.

Määritelmästä \eqref{exp(z):n laskukaava} nähdään, että
\[
\abs{e^z} = e^x > 0 \quad \forall\ z=x+iy \in \C,
\]
joten kompleksisellakaan eksponenttifunktiolla ei ole nollakohtia. Kun määritelmässä valitaan
$x=\text{Re}\,z=0$, saadaan 
\kor{Eulerin kaava} \index{Eulerin!a@kaava}
\begin{equation} \label{eulerin kaava}
\boxed{\kehys\quad e^{iy}=\cos y + i\sin y,\quad y\in\R. \quad}
\end{equation}
Eulerin kaavaa \eqref{eulerin kaava} käyttäen voidaan trigonometriset funktiot $\cos$ ja $\sin$
lausua kompleksisen eksponenttifunktion avulla:
\begin{equation} \label{sin ja cos exp(z):n avulla}
\boxed{\kehys\quad \cos x=\tfrac{1}{2}\,(e^{ix}+e^{-ix}),\quad 
                   \sin x=\tfrac{1}{2i}\,(e^{ix}-e^{-ix}),\quad x\in\R. \quad}
\end{equation}
Tällä perusteella voidaan $\cos$ ja $\sin$ myös laajentaa kompleksimuuttujan funktioiksi. 
Määritelmät ovat
\index{funktio C!a@$\sin$, $\cos$} \index{kompleksimuuttujan!d@sini ja kosini}%
\begin{equation} \label{sin z ja cos z}
\boxed{\kehys\quad \cos z=\tfrac{1}{2}\,(e^{iz}+e^{-iz})\,,\quad 
                   \sin z=\tfrac{1}{2i}\,(e^{iz}-e^{-iz})\,,\quad z\in\C. \quad}
\end{equation}
Kaikki trigonometrisisille funktioille ominaiset laskusäännöt 
(vrt. Luku \ref{trigonometriset funktiot}) ulottuvat myös kompleksialueelle. Esimerkiksi
säännöt
\begin{align*}
&\cos^2 z+\sin^2 z = 1, \\
&\cos 2z = \cos^2 z-\sin^2 z, \quad \sin 2z = 2\sin z \cos z, \\
&\dif\sin z = \cos z, \quad \dif\cos z = -\sin z
\end{align*}
ovat todennettavissa suoraan määritelmistä \eqref{sin z ja cos z} ja $e^z$:n
derivoimissäännöstä \eqref{exp(z):n derivaatta}. Nähdään myös, että $\sin z$ ja $\cos z$
ovat ($e^z$:n tavoin) analyyttisiä koko kompleksitasossa eli kokonaisia funktioita.
\begin{Exa} Etsi yhtälön $\,\sin z = 2\,$ kaikki ratkaisut kompleksitasosta. 
\end{Exa}
\ratk Kun merkitään $t=e^{iz}$, niin määritelmän \eqref{sin z ja cos z} mukaan
\begin{align*}
\sin z = 2\,\ \ekv\,\ \frac{1}{2i}\,(t-t^{-1}) = 2\,\ &\ekv\,\ t^2-4it-1=0 \\
                                                      &\ekv\,\ t = (2\pm\sqrt{3})i.
\end{align*}
Määritelmän \eqref{exp(z):n laskukaava} ja aksiooman C3 mukaan
\[
t=e^{iz} = e^{-y+ix} = e^{-y}e^{ix}, \quad z=x+iy,
\]
joten on oltava
\[
\begin{cases} \,e^{-y}=\abs{t}=2\pm\sqrt{3}, \\ \,e^{ix}=i  \end{cases} \ekv\quad
\begin{cases} \,y=-\ln(2\pm\sqrt{3}), \\ \,x=\frac{\pi}{2}+2k\pi,\,\ k\in\Z. \end{cases}
\]
Tässä on $\,\ln(2-\sqrt{3})=-\ln(2+\sqrt{3})$, joten ratkaisut ovat
\[ 
z= \frac{\pi}{2}+2k\pi\,\pm\,\ln(\sqrt{3}+2)i, \quad k\in\Z. \loppu
\]

\subsection*{Hyperboliset funktiot}
\index{hyperboliset funktiot|vahv}
\index{funktio C!h@$\sinh$, $\cosh$, $\tanh$|vahv}
\index{funktio C!i@$\arsinh$, $\arcosh$, $\artanh$|vahv}

Hyberboliset funktiot ovat eksponenttifunktion johdannaisia, joilla on samantyyppisiä 
ominaisuuksia kuin trigonometrisillä funktioilla. \kor{Hyberbolinen kosini}, symboli $\cosh$
(cosinus hyperbolicus), ja \kor{hyberbolinen sini}, symboli $\sinh$ (sinus hyperbolicus) 
määritellään (vrt.\ trigonometristen funktioiden määritelmä \eqref{sin z ja cos z})
\begin{equation} \label{sinh ja cosh}
\boxed{\kehys\quad \cosh z=\tfrac{1}{2}(e^{z}+e^{-z}),\quad 
                   \sinh z=\tfrac{1}{2}(e^{z}-e^{-z}),\quad z\in\C. \quad}
\end{equation}
Esimerkiksi seuraavat laskulait ovat määritelmästä todennettavissa (vrt.\ trigonometristen
funktioiden vastaavat):
\index{derivoimissäännöt!i@hyperboliset funktiot}
\begin{align*}
&\cosh^2 z - \sinh^2 z = 1, \\
&\cosh 2z = \cosh^2 z + \sinh^2 z, \quad \sinh 2z = 2\sinh z\cosh z, \\
&\dif\sinh z = \cosh z, \quad \dif\cosh z = \sinh z.
\end{align*}
Reaalisilla muuttujan arvoilla $\cosh x=\frac{1}{2}(e^x+e^{-x})$ on parillinen ja 
$\sinh x=\frac{1}{2}(e^x-e^{-x})$ on pariton funktio. Edellinen on aidosti kasvava välillä 
$[0,\infty)$ ja bijektio kuvauksena $\cosh: [0,\infty)\kohti[1,\infty)$. Jälkimmäinen on 
aidosti kasvava koko $\R$:ssä ja bijektio kuvauksena $\sinh:\R\kohti\R$.
\begin{figure}[H]
\setlength{\unitlength}{1cm}
\begin{picture}(10,5)(-4,-1)
\put(-2,0){\vector(1,0){4}} \put(1.8,-0.4){$x$}
\put(0,-1){\vector(0,1){5}} \put(0.2,3.8){$y$}
\put(1,0){\line(0,-1){0.1}}
\put(0,1){\line(-1,0){0.1}}
\put(0.93,-0.4){$\scriptstyle{1}$}
\put(-0.4,0.83){$\scriptstyle{1}$}
\curve(
   -2.0000,    3.7622,
   -1.5000,    2.3524,
   -1.0000,    1.5431,
   -0.5000,    1.1276,
         0,    1.0000,
    0.5000,    1.1276,
    1.0000,    1.5431,
    1.5000,    2.3524,
    2.0000,    3.7622)
\put(5,1.5){\vector(1,0){4}} \put(8.8,1.1){$x$}
\put(7,-1){\vector(0,1){5}} \put(7.2,3.8){$y$}
\put(8,1.5){\line(0,-1){0.1}}
\put(7,2.5){\line(-1,0){0.1}}
\put(7.93,1.1){$\scriptstyle{1}$}
\put(6.6,2.6){$\scriptstyle{1}$}
\curve(
    5.5000,   -0.6293,
    5.7000,   -0.1984,
    5.9000,    0.1644,
    6.1000,    0.4735,
    6.3000,    0.7414,
    6.5000,    0.9789,
    6.7000,    1.1955,
    6.9000,    1.3998,
    7.1000,    1.6002,
    7.3000,    1.8045,
    7.5000,    2.0211,
    7.7000,    2.2586,
    7.9000,    2.5265,
    8.1000,    2.8356,
    8.3000,    3.1984,
    8.5000,    3.6293)
\end{picture}
\end{figure}

Reaalifunktioiden $\cosh$ ja $\sinh$ käänteisfunktioita ($\cosh$ rajoitettu välille 
$[0,\infty)$ tai $(-\infty,0]$) sanotaan
\index{area-funktiot}%
\kor{area-funktioiksi} ja merkitään arcosh, arsinh. Laskusäännöt saadaan
ratkaisemalla
\begin{align*}
\frac{1}{2}(e^y+e^{-y})=x \ \ekv \ y=\arcosh x, \\
\frac{1}{2}(e^y-e^{-y})=x \ \ekv \ y=\arsinh x.
\end{align*}
Nämä ovat toisen asteen yhtälöitä tuntemattoman $t=e^y$ suhteen, joten $y$ on ilmaistavissa 
logaritmien avulla:
\[ \boxed{
\begin{aligned}
\ykehys\quad y = \arcosh x &= \pm \ln\left(x+\sqrt{x^2-1}\,\right), \quad x \ge 1, \quad \\
             y = \arsinh x &= \ln\left(x+\sqrt{x^2+1}\,\right), \quad x\in\R. \akehys
\end{aligned} } \]
Tässä arcosh:n päähaara $\Arcosh$ saadaan etumerkillä +. Toisen haaran voi 
kirjoittaa myös muotoon
\[
y\ =\ -\ln\left(x+\sqrt{x^2-1}\,\right)\ =\ \ln\frac{1}{x+\sqrt{x^2-1}}\ 
                                         =\ \ln\left(x-\sqrt{x^2-1}\,\right).
\]

\kor{Hyberbolinen tangentti} $\tanh$ (tangens hyperbolicus) määritellään (ol.\ reaalimuuttuja)
\[
\boxed{\quad \tanh x\ =\ \frac{\ykehys\sinh x}{\akehys\cosh x}\ 
                      =\ \frac{e^x-e^{-x}}{e^x+e^{-x}}\ 
                      =\ \frac{1-e^{-2x}}{1+e^{-2x}}\,. \quad}
\]
\begin{figure}[H]
\setlength{\unitlength}{1cm}
\begin{center}
\begin{picture}(8,4)(-4,-2)
\put(-4,0){\vector(1,0){8}} \put(3.8,-0.4){$x$}
\put(0,-2){\vector(0,1){4}} \put(0.2,1.8){$y$}
\put(1,0){\line(0,-1){0.1}}
\put(0,1){\line(-1,0){0.1}}
\put(0.93,-0.4){$\scriptstyle{1}$}
\put(-0.4,0.9){$\scriptstyle{1}$}
\curve(
   -4.0000,   -0.9993,
   -3.5000,   -0.9982,
   -3.0000,   -0.9951,
   -2.5000,   -0.9866,
   -2.0000,   -0.9640,
   -1.5000,   -0.9051,
   -1.0000,   -0.7616,
   -0.5000,   -0.4621,
         0,         0,
    0.5000,    0.4621,
    1.0000,    0.7616,
    1.5000,    0.9051,
    2.0000,    0.9640,
    2.5000,    0.9866,
    3.0000,    0.9951,
    3.5000,    0.9982,
    4.0000,    0.9993)
\end{picture}
%\caption{$y=\tanh x$}
\end{center}
\end{figure}
Hyperbolinen tangentti on bijektio: $\ \tanh:\ \R \kohti (-1,1)$. Käänteisfunktio on
\[
\boxed{\kehys\quad \artanh\,x 
             = \frac{1}{2}\,\ln \left(\frac{1+x}{1-x}\right),\quad \abs{x}<1. \quad}
\]

Myöhempää käyttöä varten todettakoon vielä hyperbolisten käänteisfunktioiden derivoimiskaavat
\index{derivoimissäännöt!i@hyperboliset funktiot}%
\[ \boxed{ \begin{aligned}
\quad &\dif\ln(x+\sqrt{x^2+1}\,)\,=\,\frac{\ygehys 1}{\sqrt{x^2+1}}\,, \quad x\in\R, \\
      &\dif\,\bigl|\ln(x+\sqrt{x^2-1}\,\bigr|\,=\,\frac{\ykehys 1}{\sqrt{x^2-1}}\,, \quad 
                                                             \abs{x}>1, \quad \\[3mm]
      &\dif\,\frac{1}{2}\ln\left|\frac{1+x}{1-x}\right|\,
                               =\,\frac{1}{\akehys 1-x^2}\,, \quad \abs{x}\neq 1.
           \end{aligned} } \]

\subsection*{Kompleksinen logaritmifunktio}
\index{kompleksimuuttujan!e@logaritmifunktio|vahv}
\index{logaritmifunktio|vahv}
\index{funktio C!g@$\log$, $\ln$|vahv}

Kompleksinen (perus)logaritmifunktio määritellään
\[
e^w=z \ \ekv \ w=\log z.
\]
Kun kirjoitetaan $w=u+iv$ ja otetaan $z$:lle polaariesitys $z=r(\cos\varphi+i\sin\varphi)$,
niin yhtälöstä
\[
e^w=z\ \ekv\ e^u(\cos v+i\sin v) = r(\cos\varphi+i\sin\varphi)
\]
nähdään, että on oltava
\[
\begin{cases}
\,e^u=r\ \ekv\ u = \ln r = \ln\abs{z}, \\
\,v=\varphi +2k\pi,\quad k\in\Z.
\end{cases}
\]
Kyseessä on siis äärettömän monihaarainen funktio. Logaritmifunktion $\log z$ määritelmäksi
sovitaan tämän funktion päähaara, jossa $k=0$\,:
\[
\boxed{\kehys\quad \log z = \ln\abs{z}+i\arg z, \quad 0 \le \arg z < 2\pi. \quad}
\]
Näin määritelty funktio ei ole jatkuva positiivisella reaaliakselilla, sillä jos $z_n=x+iy_n$,
$x>0$, niin $z_n\kohti x$, kun $y_n\kohti 0$, mutta
\begin{align*}
y_n\kohti 0^+ \ &\impl \ \log z_n\kohti \ln x, \\
y_n\kohti 0^- \ &\impl \ \log z_n\kohti \ln x+2\pi i.
\end{align*}
Muissa määrittelyjoukon pisteissä $\log z$ on paitsi jatkuva myös derivoituva
(ks.\ Harj.teht.\,\ref{H-exp-3: logaritmin derivaatta}), eli $\log z$ on analyyttinen nk.\
\index{aukileikattu kompleksitaso} \index{kompleksitaso!a@aukileikattu}%
\kor{aukileikatussa kompleksitasossa}, josta origo ja positiivinen reaaliakseli on poistettu.

Johtuen kompleksisen logaritmifunktion haaraisuudesta eivät reaalialueen laskusäännöt yleisty
sellaisenaan. Esimerkiksi tulon logaritmi on em.\ määritelmän perusteella
(vrt. Luku \ref{exp(x) ja ln(x)})
\[
\log (z_1z_2)=\begin{cases}
\,\log z_1+\log z_2\,,           &\text{jos}\,\ 0\leq\arg z_1+\arg z_2<2\pi, \\
\,\log z_1+\log z_2 -2\pi i,\  &\text{muulloin}.
\end{cases}
\]
\begin{Exa} Funktion $\log$ määritelmän mukaisesti
\begin{align*}
&\log(-1)    \,=\, \log(1\vkulma{\pi})           
             \,=\, \ln 1 + \pi i \,=\, \pi i, \\[2mm]
&\log (1+i)  \,=\, \log\left(\sqrt{2}\vkulma{\pi/4}\right)  
             \,=\, \frac{1}{2}\ln 2+\frac{\pi}{4}\,i, \\
&\log (-1-i) \,=\, \log\left(\sqrt{2}\vkulma{5\pi/4}\right) 
             \,=\, \frac{1}{2}\ln 2+\frac{5\pi}{4}\,i.\loppu
\end{align*}
\end{Exa}
Kompleksisen logaritmifunktion avulla voidaan määritellä (vrt.\ edellisen luvun kaava
\eqref{exp-kaava 5})
\[
\boxed{\kehys\quad z^y = e^{y\log z}, \quad z,y\in\C,\ z\neq 0. \quad}
\]
\begin{Exa} Edellisen esimerkin perusteella
\[
(-1)^\pi=e^{\pi\log(-1)}=e^{i\pi^2}=\cos\pi^2+i\sin\pi^2. \loppu
\]
\end{Exa}

\Harj
\begin{enumerate}

\item \label{H-VII-3: E(z) vaihtoehto}
Määritellään funktion $e^x$ laajennus kompleksitasoon kaavalla
\[
\exp(z)=\exp(x+iy)=e^x(\cos ay + i\sin ay),
\]
missä $a\in\R,\ a \neq 1$. Mitkä kompleksisen eksponenttifunktion aksioomista ovat voimassa 
tälle funktiolle?

\item
Tarkista seuraavien kaavojen pätevyys, kun $z\in\C$\,:
\begin{align*}
&\text{a)}\ \ \sin^2 z+\cos^2 z=1, \quad \sin 2z=2\sin z\cos z, \quad \cos 2z=2\cos^2 z-1 \\
&\text{b)}\ \ \dif\sin z=\cos z, \quad \dif\cos z=-\sin z
\end{align*}

\item
a) Lausu $\sinh\frac{x}{2}$ ja $\cosh\frac{x}{2}$ $\cosh x$:n avulla. \newline
b) Millaisilla luvuilla $n$ pätee $\,(\cosh x+\sinh x)^n=\cosh nx+\sinh nx\,$?

\item
a) Määritellään $\coth x=1/\tanh x\ (x\in\R,\ x \neq 0)$. Mikä on funktion 
$f(x)=4\tanh x+\coth x$ arvojoukko? \\
b) Missä pisteessä funktio
$\D f(x)=\frac{\sinh x}{1-a\cosh x}\,, \quad x\in\R$ \vspace{1mm}\newline
saavuttaa suurimman arvonsa, kun $a\in\R,\ a>1$\,?

\item
Näytä, että yhtälöllä $\,\cos x\cosh x+1=0\,$ on äärettömän monta reaalista ratkaisua.

\item
a) Näytä, että funktioilla $\Arcosh$ ja $\arsinh$ on asymptoottina funktio 
$\ln (2x)=\ln x + \ln 2$, kun $x\kohti\infty$. \newline
b) Laske funktioiden $\arsinh x$ ja $\Arcosh x$ derivaatat sekä implisiittisesti
derivoimalla että suoraan ko.\ funktioiden lausekkeista.

\item
Sievennä:
\[
\text{a)}\ \ \Arcosh(\cosh x) \qquad
\text{b)}\ \ \tanh(\Arcosh x) \qquad 
\text{c)}\ \ \artanh\frac{x}{\sqrt{x^2+1}}
\]

\item
Laske seuraavat kompleksiluvut perusmuodossa $x+iy$ (tarkat arvot!): \vspace{1mm} \newline
a) \ $e^z$, $\sinh z$, $\cosh z$ ja $\tanh z$, kun $z=2+3i$ \newline
b) \ $e^z$, $\sin z$, $\cos z$ ja $\tan z$, kun $z=-1-i$ \newline
c) \ $e^z$, $\sin z$, $\cos z$ ja $\tan z$, kun $z=3-2i$ \newline
d) \ $\cosh(\arsinh\frac{4}{3}),\ \tanh(\arsinh\pi),\ \Arcosh\sqrt{2},\ \artanh 1$ \newline 
e) \ $e^i,\ i^e,\ \log i,\ 5^{-i},\ i^{-\pi},\ i^i$

\item
Tietokoneohjelma laskee luvun $(-\pi)^\pi$ numeroarvoksi $-32.9139-15.6897i$. Tarkista lasku! 

\item
Määritä (tarkasti perusmuodossa $x+iy$) seuraavien yhtälöiden kaikki ratkaisut 
kompleksitasossa: \newline
a) \ $e^z=e,\quad$ b) \ $\cos z=-2,\quad$ c) $\cosh z=-1,\quad$ d) \ $\sin z=i$.  

\item
Määritä seuraaville funktioille suurin kompleksitason avoin osajoukko $G$, jossa funktio on 
analyyttinen:
\[
\text{a)}\ \ f(z)=\frac{1}{e^z+1} \qquad \text{b)}\ \ f(z)=\frac{z}{\cos z} \qquad
\text{c)}\ \ f(z)=\frac{1}{z\cosh z+2z}
\]

\item (*) \label{H-exp-3: logaritmin derivaatta}
Johda suoraan derivaatan määritelmästä derivoimssääntö
\[
\dif\log z = \frac{1}{z}\,, \quad z=r\vkulma\varphi,\,\ r>0,\,\ 0<\varphi<2\pi.
\]
\kor{Vihje}: Kirjoita $\Delta z=\Delta r\vkulma(\varphi+\psi)$.

\end{enumerate}