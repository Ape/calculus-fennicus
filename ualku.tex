\section{Jordan-mitta ja Riemann-integraali} \label{jordanin mitta}
\sectionmark{Jordan-mitta}
\alku

Jos $A=[a,b]$ on suljettu väli ($a<b$), niin sanotaan, että luku
\[
\mu(A) = b-a
\]
$A$:n \kor{mitta}, tarkemmin sanoen \kor{Jordan-mitta}. Mitta (engl.\ measure) on yleisemmin 
\kor{joukkofunktio}, joka liittää \kor{mitalliseen} (engl. measurable) joukkoon 
(tässä reaalilukujoukkoon) reaaliluvun:
\[
\mu:\mathcal{M}\Kohti\R,\quad \mathcal{M}=\{\text{mitalliset joukot $A\subset\R$}\}.
\]
Toistaiseksi on siis sovittu, että ainakin suljetut välit ovat mitallisia ja mitta on välin 
pituus. Tämä sopimus on itse asiassa mittaa koskeva \pain{aksiooma}, joka kertoo, että kyseessä
on \kor{pituusmitta} (muitakin mittoja siis on). Jordanin pituusmitan aksioomat ovat:
\begin{itemize}
\item[(a)] \kor{Suljetun välin mitta}: \ $A=[a,b]\ \impl\ \mu(A)=b-a$.
\item[(b)] \kor{Positiivisuus}: \ $\mu(A)\geq 0\quad\forall A\in\mathcal{M}$.
\item[(c)] \kor{Additiivisuus}: \ $A,B\in\mathcal{M} \ \ja\ A \cap B=\emptyset$ \\[0.25cm]
           $\impl \ A \cup B \in \mathcal{M} \ \ja \ \mu(A \cup B)=\mu(A)+\mu(B)$.
\end{itemize}
Näistä kaksi jälkimmäistä ovat itse asiassa yleisempien \kor{positiivisten} mittojen 
ominaisuuksia, eli tällaisten mittojen \kor{mitta-aksioomia}.
\begin{Exa} Jos $A=[-1,1]$, $B=[2,5]$ ja $C=[-7,-3]$, niin aksioomien (a),(c) perusteella 
$\,\mu(A \cup B \cup C)=2+3+4=9$. \loppu
\end{Exa}
Tyhjä joukko $\emptyset$ katsotaan aina mitalliseksi, jolloin aksiooman (c) mukaan on oltava
\[
\mu(\emptyset)=0.
\]
Yleisemmin jos $A \in \mathcal{M}$ ja $\mu(A)=0$, sanotaan että $A$ on \kor{nollamittainen}. 
Nollamittaiset joukot ovat yleisessä mittateoriassa tärkeällä sijalla, ja niihin viitataan 
jatkossakin useasti. Käsitteeseen liittyy (matemaattisessa tekstissä hieman oudolta
kuullostava) termi \kor{melkein kaikkialla} (engl.\ almost everywhere). Tämä tarkoittaa
'muualla kuin nollamittaisessa (osa)joukossa'. --- Pituusmittaan liittyen voi yhden 
reaaliluvun muodostaman joukon $\{a\}$ sopia (suljetun välin rajatapauksena $[a,a]$) 
mitalliseksi ja nollamittaiseksi. Aksioomasta (c) seuraa silloin, että jokainen 
\pain{äärellinen} \pain{reaaliluku}j\pain{oukko} \pain{on} \pain{nollamittainen}.

Palautetaan mieliin määrätyn integraalin käsite Luvusta \ref{riemannin integraali}\,: Jos $f$ on
määritelty ja rajoitettu välillä $[a,b]$, niin $f$:n Riemann-integraali välillä $[a,b]$ on
\[
\int_a^b f(x)\, dx=\Lim_{h \kohti 0} \sum_{k=1}^n f(\xi_k)(x_k-x_{k-1}),
\]
sikäli kuin raja-arvo on olemassa (ks. Määritelmä \ref{raja-arvo Lim}). Tässä on aiempaan tapaan
\[
a=x_0<x_1<\ldots <x_n=b, \quad \xi_k\in [x_{k-1},x_k], \quad h=\max_{h=1\ldots n} (x_k-x_{k-1}).
\]
Merkitään
\[
A=[a,b],\quad T_k=[x_{k-1},x_k], \quad \mathcal{T}_h=\{T_k, \ k=1\ldots n\}.
\]
Joukkoa $\mathcal{T}_h\,$, joka siis koostuu jaon $\{x_k,\,k=0 \ldots n\}$ määrittelemistä välin
$A$ vierekkäisistä osaväleistä, sanotaan jatkossa välin $A$ \kor{ositukseksi} (engl. partition).
Koska $\mu(T_k)=x_k-x_{k-1}$, niin Riemann-integraalin määritelmä voidaan kirjoittaa muotoon
\[
\int_a^b f(x)\, dx=\Lim_{h \kohti 0} \sum_{T_k\in\mathcal{T}_h} f(\xi_k)\mu(T_k).
\]

Palautettakoon vielä mieleen Luvusta \ref{ylä- ja alaintegraalit}, että Riemannin integraali
voidaan määritellä raja-arvon 'Lim' sijasta vaihtoehtoisesti lähtien Riemannin y\pain{lä}- ja 
\pain{alasummista}
\[
\Sigma_h (f,\mathcal{T}_h) = \sum_{T_k\in\mathcal{T}_h} M_k\,\mu(T_k)\,, \quad
\sigma_h (f,\mathcal{T}_h) = \sum_{T_k\in\mathcal{T}_h} m_k\,\mu(T_k)\,,
\]
missä
\[
M_k=\sup_{x\in T_k} f(x),\quad m_k=\inf_{x\in T_k} f(x).
\]
Näiden avulla määriteltyjä lukuja
\[
\overline{I}(f,A)=\inf_{\mathcal{T}_h} \Sigma_h(f,\mathcal{T}_h)\,,\quad 
\underline{I}(f,A)=\sup_{\mathcal{T}_h} \sigma_h(f,\mathcal{T}_h)
\]
sanotaan $f$:n y\pain{läinte}g\pain{raaliksi} ja \pain{alainte}g\pain{raaliksi} välillä $[a,b]$.
Sikäli kuin $f$ on määritelty ja rajoitettu välillä $[a,b]$, ovat yläintegraali ja alaintegraali
molemmat olemassa. Tällöin $f$ on välillä $[a,b]$ Riemann-integroituva täsmälleen, kun ylä- ja
alaintegraalit ovat samat, jolloin $f$:n integraali = ylä- ja alaintegraalien yhteinen arvo 
(Lause \ref{Riemannin integraali - vaihtoehto})). 

Olettaen $f$ Riemann-integroituvaksi välillä $A=[a,b]$ kirjoitetaan nyt
\[
\int_a^b f(x)\, dx=\int_A f\, d\mu, \quad A=[a,b],
\]
ja luetaan oikea puoli: '$f$:n integraali yli $A$:n mitan $\mu$ suhteen'. Näin ajatellen 
määrätyssä integraalissa on siis \pain{kolme} komponenttia\,:
\[
\int_A f\, d\mu=I(f,A,\mu).
\]
Laajennetaan nyt em. integraalin määritelmä koskemaan yleisempää \pain{ra}j\pain{oitettua} 
joukkoa $A\subset\R$ seuraavasti: Jos $f$ on $A$:ssa määritelty ja rajoitettu, niin määritellään
ensin $f$:n \kor{nollajatko}
\[
f_0(x)=\begin{cases} 
       f(x), &\text{kun}\ x\in A \\ 0, &\text{kun}\ x\in\R,\ x\notin A 
       \end{cases}
\]
ja tämän avulla
\[
\boxed{\kehys\quad \int_A f \, d\mu=\int_T f_0 \, d\mu,\quad T=[a,b]\supset A \quad}
\]
Koska $f_0(x)=0$ kun $x\not\in A$, niin Riemannin integraalin ominaisuuksien perusteella on 
helppo päätellä, että sikäli kuin $f_0$ on Riemann-integroituva jollakin välillä 
$[a,b]\supset A$, niin $f_0$ on integroituva kaikilla väleillä $[a,b]\supset A$ ja integraalin
arvon on välistä $[a,b]$ riippumaton, kunhan $[a,b]\supset A$. Voidaan siis sopia, että $f$ on
Riemann-integroituva yli rajoitetun joukon $A\subset\R$ täsmälleen kun $f_0$ on
Riemann-integroituva jollakin välillä $[a,b]\supset A$.

Em. määritelmässä on integraalin riippuvuus mitasta varsin yksinkertainen, sillä määritelmässä
tarvitaan vain suljettujen välien mittoja. Mutta kun integraali on kerran määritelty käyttäen
hyväksi mitan yksinkertaisimpia ominaisuuksia, voidaankin määritellä yleisemmän joukon $A$ mitta
käyttäen hyväksi integraalia (!). Tällaisen 'vuorovedon' mahdollisuus (joka kuuluu yleisemmänkin 
mittateorian ominaispiirteisiin), kertoo että mitta ja integraali muodostavat toisiinsa
läheisesti liittyvän käsiteparin. Menettely on seuraava: Tarkastellaan funktiota
\[
f(x)=1,\quad x\in A=D_f.
\]
Tämän nollajatkoa sanotaan $A$:n \kor{karakteristiseksi funktioksi} ja merkitään
\[
\chi_A(x)=\begin{cases} 
          1, &\text{kun}\ x\in A \\ 0, &\text{kun}\ x\in\R, \ x\notin A 
          \end{cases}
\]
Sikäli kuin $\chi_A$ on Riemann-integroituva jollakin välillä $[a,b]\supset A$, on em.\ 
määritelmän mukaan
\[
\int_A d\mu=\int_T \chi_A\, d\mu,\quad T=[a,b]\supset A.
\]
\begin{Def} \label{Jordanin pituusmitta}
Rajoitettu joukko $A\subset\R$ on \kor{Jordan-mitallinen} täsmälleen kun $A$:n 
karakteristinen funktio $\chi_A$ on Riemann-integroituva väleillä $[a,b]\supset A$,
ja $A$:n \kor{Jordan-mitta} on tällöin
\[
\boxed{\kehys\quad \mu(A)=\int_A d\mu \quad}
\]
\end{Def}
Mitta-aksiooman (b) toteutuminen on määritelmän perusteella ilmeistä. Myös 
additiiviuusaksiooma (c) toteutuu Jordan-mitalle, sillä karakteristisen funktion
määritelmästä seuraa helposti, että pätee 
(Harj.teht. \ref{H-Uint-1: karakteristinen funktio})
\[ 
A \cap B = \emptyset \qimpl \chi_{A \cup B} = \chi_A + \chi_B.
\]
Tällöin jos $T=[a,b] \supset A \cup B$, niin integraalin lineaarisuuden nojalla 
\[
\int_T \chi_{A \cup B}\,d\mu = \int_T \chi_A\,d\mu + \int_T \chi_B\,d\mu \\
                               \qekv \mu(A \cup B) = \mu(A)+\mu(B).
\]
Jordan-mitta, jota sanotaan myös \kor{Peanon--Jordanin mitaksi}\footnote[1]{Termit viittaavat
matemaatikkoihin \hist{Camille Jordan} (ransk. 1838-1922) ja \hist{Giuseppe Peano} 
(ital. 1858-1932).}, ei ole ainoa ainoa suljetun välin pituusmitan yleistys. Matemaattisen
analyysin kannalta monessa mielessä parempi on (Jordan-mittaa hieman myöhemmin kehitetty) 
\kor{Lebesquen mitta}, ks.\ kommentit luvun lopussa. Käytännön integraalilaskentaan Jordan-mitta
tarjoaa kuitenkin riittävän ja Lebesguen mittaa jonkin verran havainnollisemman perustan.

\subsection*{Ulkomitta ja sisämitta}

Jos $f=\chi_A$ ja $\mathcal{T}_h$ on välin $T=[a,b]\supset A$ ositus osaväleihin
$T_k=[x_{k-1},x_k],\ k=1 \ldots n$, niin Riemannin ylä- ja alasummien määritelmän perusteella
\begin{align*}
\Sigma_h(\chi_A,\mathcal{T}_h) &= \sum_{T_k\in\mathcal{T}_h:\,T_k\cap A\neq\emptyset}\mu(T_k) \\
\sigma_h(\chi_A,\mathcal{T}_h) &= \sum_{T_k\in\mathcal{T}_h:\,T_k\subset A}\mu(T_k)
\end{align*}
Näihin summiin liitettäviä lukuja
\begin{alignat*}{2}
\overline{\mu}(A)  &= \inf_{\mathcal{T}_h} \Sigma_h(\chi_A,\mathcal{T}_h) & 
                   &= \overline{I}(\chi_A,T) \\
\underline{\mu}(A) &= \sup_{\mathcal{T}_h} \sigma_h(\chi_A,\mathcal{T}_h) & 
                   &= \underline{I}(\chi_A,T)
\end{alignat*}
sanotaan $A$:n (Jordanin) \kor{ulkomitaksi} ja \kor{sisämitaksi}. Molemmat ovat määriteltyjä
jokaiselle rajoitetulle joukolle, ja $\underline{\mu}(A)\leq\overline{\mu}(A)$. 
(Jos $T_k\not\subset A \ \forall T_k\in\mathcal{T}_h$, niin asetetaan
$\sigma_h(\chi_A,\mathcal{T}_h)=0$). Mitallisuuden ja Riemann-integroituvuuden määritelmien
perusteella pätee
\[
\boxed{\kehys\quad A\ \text{Jordan-mitallinen} 
            \qekv \overline{\mu}(A)=\underline{\mu}(A) \quad (=\mu(A)) \quad}
\]
Hiukan havainnollisempi mitallisuuden kriteeri saadaan tarkastelemalla $A$:n \pain{reunaa}
$\partial A$ (vrt.\ Luku \ref{kompaktit joukot}) ja vertaamalla erotusta 
$\overline{\mu}(A)-\underline{\mu}(A)$ reunan ulkomittaan $\overline{\mu}(\partial A)$. 
Nimittäin jos jako $\mathcal{T}_h$ on \pain{tasavälinen} 
(eli osavälit $T_k\subset\mathcal{T}_h$ ovat samanpituiset) ja $a,b\not\in\partial A$, niin voidaan
päätellä, että pätee (ks.\ Harj.teht. \ref{H-Uint-1: mitallisuuskriteeri})
\[
\frac{1}{2}\sum_{T_k\in\mathcal{T}_h:\,T_k\cap \partial A\neq\emptyset} \mu(T_k)\
  \le\ \Sigma_h(\chi_{A},\mathcal{T}_h) - \sigma_h(\chi_{A},\mathcal{T}_h)\
  \le \sum_{T_k\in\mathcal{T}_h:\,T_k\cap \partial A\neq\emptyset} \mu(T_k).
\]
Ottamalla tässä infimum yli ositusten $\mathcal{T}_h$ ja huomioimalla, että ulko- ja
sisämittojen ym.\ määritelmissä voidaan ositukset $\mathcal{T}_h$ rajoittaa tasavälisiksi
määritelmien muuttumatta (ks.\ Lause \ref{ylä- ja alaintegraalit raja-arvoina}), seuraa 
\[
\frac{1}{2}\,\overline{\mu}(\partial A)\ \le\ \overline{\mu}(A)-\underline{\mu}(A)\
                                         \le\ \overline{\mu}(\partial A). 
\]
Tämän mukaan $A$ on mitallinen täsmälleen kun $\overline{\mu}(\partial A)=0$, eli kun 
$\partial A$ on (mitallinen ja) nollamittainen:
\begin{Lause} \vahv{(Mitallisuuskriteeri)} \label{mitallisuuskriteeri-R} Joukoille
$A\subset\R$ pätee
\[ 
\boxed{\kehys\quad A\ \text{Jordan-mitallinen} \qekv \mu(\partial A)=0 \quad} \Akehys
\]
\end{Lause}

\begin{Exa} \label{mitallisuusesim-R}
Tutki seuraavien joukkojen Jordan-mitallisuutta: \vspace{1mm}\newline
a) \ $A=(0,1)\qquad$  b) \ $A=\{\frac{1}{k}, \ k\in\N\} \qquad$ c) \ $A=\Q\cap (0,1)$
\end{Exa}
\ratk \ a) \ Kun valitaan $[a,b]=[0,1]$, nähdään että kaikille osituksille pätee
\[
\sigma_h(\chi_A,\mathcal{T}_h)=\Sigma_h(\chi_A,\mathcal{T}_h)=1,
\]
joten $\overline{\mu}(A)=\underline{\mu}(A)=1$. Siis $A$ on Jordan-mitallinen ja 
$\underline{\underline{\mu(A)=1}}$.

b) \ Valitaan $[a,b]=[0,1]$ ja ositus $\mathcal{T}_h=\{[x_{k-1},x_k],\ k=1\ldots n\}$
siten, että
\[
[x_0,x_1]=[0,h]\ \ja\ x_k-x_{k-1} \le h^2,\quad k=2\ldots n.
\]
Tällöin jos $m-1 < h^{-1} \le m,\ m \in \N$, niin $[h,1] \subset [1/m,1]$, jolloin välillä 
$[h,1]$ on enintään $m$ joukon $A$ alkiota. Näistä kukin voi kuulua enintään kahteen osaväliin
$[x_{k-1},x_k]$, joten $A$:n ulkomitalle saadaan arvio
\[
\overline{\mu}(A) \le \Sigma_h(\chi_A,\mathcal{T}_h) 
                  \le h + 2m \cdot h^2 \le h + 2h^{-1} \cdot h^2 = 3h.
\]
Tämä pätee jokaisella $h>0$, joten on oltava $\overline{\mu}(A)=0$. Siis $A$ on (mitallinen ja)
nollamittainen: $\underline{\underline{\mu(A)=0}}$.

c)\ \, Kun valitaan $[a,b]=[0,1]$, niin nähdään, että $\overline{\mu}(A)=1$, 
$\underline{\mu}(A)=0$, joten $A$ ei ole Jordan-mitallinen. Samaan tulokseen johtaa päättely:
$\,\partial A = [0,1]\ \impl\ \mu(\partial A) \neq 0$. \loppu

Seuraavan tuloksen mukaan joukon (mitallisuus ja) mitta säilyy, jos joukosta poistetaan 
nollamittainen osajoukko. --- Huomattakoon, että yleisesti \pain{ei} päde, että mitallisen 
joukon osajoukko on mitallinen, vrt.\ esimerkki edellä.
\begin{Prop} \label{nollamittaisen osajoukon poisto} Jos $A \cup B$ on mitallinen ja $B$ on
nollamittainen, niin $A$ on mitallinen ja $\mu(A)=\mu(A \cup B)$.
\end{Prop}
\tod Ulko- ja sisämittojen määritelmistä voi helposti päätellä, että rajoitetuille joukoille 
$A,B \subset \R$ pätee (Harj.teht. \ref{H-Uint-1: mitta-arvioita})
\begin{align*}
\overline{\mu}(A)\  &\le\ \overline{\mu}(A \cup B)\ \le\ \overline{\mu}(A)+\overline{\mu}(B), \\
\underline{\mu}(A)\ &\le\ \underline{\mu}(A \cup B)\ \le\ \underline{\mu}(A)+\overline{\mu}(B).
\end{align*}
Koska $A \cup B$ on mitallinen ja $B$ nollamittainen, niin 
$\overline{\mu}(A \cup B) = \underline{\mu}(A \cup B)$ ja $\overline{\mu}(B)=0$, joten seuraa
$\overline{\mu}(A)\le\mu(A \cup B)\le\underline{\mu}(A)$. Siis
$\overline{\mu}(A)=\underline{\mu}(A)=\mu(A \cup B)$. \loppu

Jos $\mathcal{M}=\{\text{Jordan-mitalliset, rajoitetut joukot } A\subset\R\}$, niin 
Propositiosta \ref{nollamittaisen osajoukon poisto} on johdettavissa seuraava mitan 
additiivisuusaksiooman (c) vahvennettu muoto:
\[
\boxed{ \begin{aligned}
        \quad &A,B\in\mathcal{M} \ \ja \ \mu(A\cap B)=0 \ykehys\\
              &\impl\quad A\cup B\in\mathcal{M}\ \ja\ \mu(A\cup B)=\mu(A)+\mu(B) \quad\akehys
        \end{aligned} }
\]

\subsection*{Integroituvuus yli mitallisen joukon}

Kun joukon $A\subset\R$ mitallisuus on em.\ tavalla määritelty karakteristisen funktion 
Riemann-integroituvuutena, voidaan puolestaan tutkia yleisempien funktioiden integroituvuutta
mitallisten joukkojen yli. Perustulos on seuraava.
\begin{Lause} \label{integroituvuus yli joukon - R} Jos $f$ on rajoitettu ja 
Riemann-integroituva välillä $[a,b]$ ja $A\subset [a,b]$ on Jordan-mitallinen, niin $f$ on 
Riemann-integroituva yli $A$:n.
\end{Lause}
\tod Olkoon $\abs{f(x)} \le M,\ x\in[a,b]$ ja olkoon
$\mathcal{T}_h=\{[x_{k-1},x_k],\ k=1\ldots n\}$ välin $[a,b]$ ositus, missä
$h=\max_k (x_k-x_{k-1})$. Merkitään
\begin{align*}
&M_k=\sup_{x\in T_k} f(x), \quad\   m_k=\inf_{x\in T_k} f(x), \\
&M_k^0=\sup_{x\in T_k} f_0(x),\quad m_k^0=\inf_{x\in T_k} f_0(x).
\end{align*}
Olkoon edelleen
\begin{align*}
\Lambda_1    &= \{k\in\{1,\ldots,n\} \ | \ T_k\subset A\}, \\
\Lambda_2    &= \{k\in\{1,\ldots,n\} \ | \ k\notin\Lambda_1\ \ja\ T_k\cap A\neq\emptyset\},
                \quad \partial A_h = \bigcup_{k\in\Lambda_2} T_k\,.
\end{align*}
Tällöin
\begin{align*}
\Sigma_h(f_0,\mathcal{T}_h)-\sigma_h(f_0,\mathcal{T}_h)
    &= \sum_{k\in\Lambda_1}(M_k^0-m_k^0)\,\mu(T_k)
     + \sum_{k\in\Lambda_2}(M_k^0-m_k^0)\,\mu(T_k) \\
    &= \sum_{k\in\Lambda_1}(M_k-m_k)\,\mu(T_k)
     + \sum_{k\in\Lambda_2}(M_k^0-m_k^0)\,\mu(T_k).
\end{align*}
Koska $f$ on integroituva välillä $[a,b]$, niin tässä
\[
\sum_{k\in\Lambda_1}(M_k-m_k)\,\mu(T_k) \le \sum_{k=1}^n(M_k-m_k)\,\mu(T_k)\ 
                                        \kohti\ 0, \quad \text{kun}\ h \kohti 0.
\]
Toisaalta koska $A$ on mitallinen ja siis $\,\overline{\mu}(\partial A)=0$ 
(Lause \ref{mitallisuuskriteeri-R}), niin \newline
$\overline{\mu}(\partial A_h) \kohti 0$ kun $ h \kohti 0$, joten voidaan arvioida
\[
\Bigl|\sum_{k\in\Lambda_2}(M_k^0-m_k^0)\mu(T_k)\Bigr| \le \sum_{k\in\Lambda_2} M\mu(T_k) 
                             = M\mu(\partial A_h) \kohti 0, \quad \text{kun}\ h \kohti 0.
\]
On siis päätelty:
\[
\Sigma_h(f_0,\mathcal{T}_h)-\sigma_h(f_0,\mathcal{T}_h) \kohti 0, \quad \text{kun}\ h \kohti 0.
\]
Näin ollen $f_0$:n ylä- ja ala-integraalit välillä $[a,b]$ ovat samat, siis $f_0$ on 
Riemann-integroituva yli välin $[a,b]$ eli $f$ on Riemann-integroituva yli $A$:n. \loppu

\subsection*{Riemann-integraalin laajennukset}

Riemann-integraalin käsite voidaan laajentaa koskemaan myös ei-rajoitettuja funktioita ja 
ei-rajoitettuja joukkoja. Seuraavat laajennussäännöt vastaavat Luvussa 
\ref{integraalin laajennuksia} tehtyjä sopimuksia. Sikäli kuin integaali on näiden sääntöjen
mukaisesti määritelty, sanotaan, että se \kor{suppenee}.
\begin{itemize}
\item[1.] Jos $A\subset[a,b]$ ja $f$ on rajoitettu ja Riemann-integroituva yli joukon 
          $A\cap[c,d]$ aina kun $[c,d]\subset(a,b)$, niin
          \[
          \int_A f\, d\mu
          =\lim_{\substack{\ c\kohti a^+ \\ d\kohti b^-}}\int_{A\cap [c,d]} f\,d\mu,
          \]
          sikäli kuin raja-arvo on olemassa.
\item[2.] \kor{Additiivisuussääntö}: Jos $A_1,A_2\subset\R$ ovat rajoitettuja joukkoja, jos
          $f$ on integroituva yli joukkojen $A_1$ ja $A_2$, joko tavanomaisessa mielessä tai
          säännön 1 mukaisesti, ja jos $\mu(A_1 \cap A_2)=0$, niin $f$ on integroituva yli 
          joukon $A=A_1 \cup A_2$ ja
          \[
          \int_A f\,d\mu = \int_{A_1} f\,d\mu + \int_{A_2} f\,d\mu.
          \]
\item[3.] Jos $A$ ei ole rajoitettu, mutta jokaisella $a,b \in \R,\ a<b$, joukko $A\cap [a,b]$ 
          on mitallinen ja $f$ on integroituva yli joukon $A\cap [a,b]$, joko tavanomaisessa
          mielessä tai sääntöjen 1--2 mukaisesti, niin
          \[
          \int_A f\, d\mu
          =\lim_{\substack{\ a\kohti -\infty \\ b\kohti +\infty}}\int_{A\cap [a,b]} f\,d\mu,
          \]
          sikäli kuin raja-arvo on olemassa.
\end{itemize}
\begin{Exa} Olkoon
\[
A=\bigcup_{k=0}^\infty\,[k,\,k+a_k],
\]
missä $0 \le a_k \le 1\ \forall k$. Täsmälleen millä ehdolla integraali $\int_A x\,d\mu$
suppenee?
\end{Exa}
\ratk Koska jokaisella $n\in\N$ pätee
\[
\int_{[0,n] \cap A} f\,d\mu = \sum_{k=0}^{n-1} \int_k^{k+a_k} x\,dx
                            = \sum_{k=0}^{n-1}\sijoitus{k}{k+a_k} \frac{1}{2}\,x^2
                            = \sum_{k=0}^{n-1}(ka_k+a_k^2/2)
\]
ja koska $ka_k \le ka_k+a_k^2/2 \le 2ka_k$, kun $a_k\in[0,1]$ ja $k \ge 1$, niin vastaus on:
Täsmälleen kun sarja $\sum_{k=1}^\infty ka_k$ suppenee. \loppu 

\subsection*{Lebesguen mitta ja integraali}

Kun joukon $A\subset\R$ pituusmitta halutaan määrätä mahdollisimman yleispätevästi ja 
elegantisti, on tuloksena \kor{Lebesguen mitta} (\hist{H. Lebesgue}, 1906). Todettakoon tässä
ainoastaan Lebesguen mittateorian keskeisin ero verrattuna Peano-Jordanin teoriaan: Lebesguen 
mitta on (siis toisin kuin Jordan-mitta) \kor{numeroituvasti additiivinen}: Jos $A_1,A_2,\ldots$
ovat Lebesgue-mitallisia ja pistevieraita, niin $A=A_1\cup A_2\cup\ldots$ on myös 
Lebesgue-mitallinen ja
\[
\mu(\bigcup_{i=1}^\infty A_i)=\sum_{i=1}^\infty \mu(A_i),\quad A_i\cap A_j=\emptyset, \ i\neq j.
\]
Jos oikealla oleva sarja ei suppene, annetaan $A$:n mitalle arvo $\mu(A)=\infty$. Tässäkin 
tapauksessa siis joukkoa $A$ pidetään mitallisena, eli lukuasteikko (mahdollisten mittalukujen
joukko) laajennetaan käsittämään myös luku $\infty$. (Tällä luvulla laskemisen säännöt on 
erikseen sovittava). Esimerkiksi $A=\R$ on numeroituvan additiivisuuden perusteella mitallinen
ja $\mu(\R)=\infty$. Numeroituvasta additiivisuudesta seuraa myös, että jokainen numeroituva 
joukko $A\subset\R$ on Lebesguen mielessä nollamittainen --- vrt. Esimerkki 
\ref{mitallisuusesim-R}c edellä, jossa joukko $A$ on numeroituva, muttei Jordan-mitallinen.

Sikäli kuin $A$ on Jordan-mitallinen, se on myös Lebesgue-mitallinen, ja mitat ovat samat. Sama
ei siis päde kääntäen, eli Lebesguen mitta on Jordan-mitan aito laajennus.

Kuten Jordan-mitta, myös Lebesguen mitta voidaan liittää määrättyyn integraaliin, mutta tällöin
kyseessä ei ole Riemann-integraali vaan \kor{Lebesguen integraali}. Todettakoon ainoastaan, että
sikäli kuin joukolla $A\subset\R$ on äärellinen Lebesguen mitta, voidaan kirjoittaa
\[
\mu(A)=\int_A d\mu,
\]
missä siis integraali on ymmärrettävä Lebesguen mielessä. Lebesguen mitta ja integraali 
muodostavat näin ollen vastaavanlaisen käsiteparin kuin Jordan-mitta ja Riemannin integraali.

Lebesguen mitalla ja integraalilla on nykyisin hyvin keskeinen merkitys matemaattisessa 
analyysissä, etenkin sellaisissa syvällisemmissä tarkasteluissa, joissa pyritään mahdollisimman
eheään ja yleispätevään teoriaan. Käytännön integraalilaskennassa tullaan sen sijaan yleensä 
hyvin toimeen yksinkertaisemmilla Jordan-mitan ja Riemann-integraalin käsitteillä. Jatkossa 
siirrytäänkin useampaan ulottuvuuteen tältä yksinkertaisemmalta perustalta.

\Harj
\begin{enumerate}

\item
Laske seuraavien joukkojen Jordan-mitat (pituusmitat) suoraan mitan määritelmästä, tai totea,
ettei joukko ole mitallinen. 
\begin{align*}
&\text{a)}\ \ [-1,1] \cup (2,3) \cup \{7\} \qquad
 \text{b)}\ \ \{2^{-n} \mid n\in\N\} \qquad
 \text{c)}\ \ \bigcup_{k=1}^{\infty} \,[3 \cdot 2^{-k},\,4 \cdot 2^{-k}] \\
&\text{d)}\ \ \{x\in[0,1] \mid x \neq e^{-n}\ \forall n\in\N\} \qquad
 \text{e)}\ \ \{x\in[0,1] \mid x \neq e^{-t}\ \forall t\in\Q\} \qquad
\end{align*}

\item \label{H-Uint-1: karakteristinen funktio}
Näytä, että reaalilukujoukon karakteristiselle funktiolle pätee
\[
A \cap B = \emptyset \qimpl \chi_{A \cup B} = \chi_A + \chi_B\,.
\]

\item \label{H-Uint-1: mitta-arvioita}
Näytä, että rajoitetuille joukoille $A,B\subset\R$ pätee
\begin{align*}
\overline{\mu}(A)\  &\le\ \overline{\mu}(A \cup B)\  \le\ \overline{\mu}(A)+\overline{\mu}(B), \\
\underline{\mu}(A)\ &\le\ \underline{\mu}(A \cup B)\ \le\ \underline{\mu}(A)+\overline{\mu}(B).
\end{align*}

\item
Laske $f$:n integraali yli annetun joukon $A$ (tavallinen tai laajennettu Riemannin integraali)
sikäli kuin integraali on määritelty:
\begin{align*}
&\text{a)}\ \ f(x)=x^2,\,\ A=(-1,0]\cup\{1\}\cup[5,7) \qquad
 \text{b)}\ \ f(x)=e^{x^2},\,\ A=\Z \\[3mm]
&\text{c)}\ \ f(x)=x^{-2},\,\ 
              A=\bigcup_{n=1}^\infty\,\left[\frac{2n-1}{2n}\,,\,\frac{2n}{2n+1}\right] \\
&\text{d)}\ \ f(x)=x^{-1},\,\ 
              A=\bigcup_{n=1}^\infty\,\left[\frac{1}{3n+1}\,,\,\frac{1}{3n-1}\right] 
\end{align*}

\item (*) \label{H-Uint-1: mitallisuuskriteeri}
Olkoon $A\subset[a,b]$ reaalilukujoukko ja $\mathcal{T}_h$ välin $[a,b]$ ositus osaväleihin
$T_k=[x_{k-1},x_k],\ k=1 \ldots n$. \ a) \ Näytä, että jos 
$x_k\not\in\partial A,\ k=0 \ldots n$, niin
\begin{align*}
&\Sigma_h(\chi_{A},\mathcal{T}_h) - \sigma_h(\chi_{A},\mathcal{T}_h)\
  = \sum_{T_k\in\mathcal{T}_h:\,T_k\cap \partial A\neq\emptyset} \mu(T_k).
\intertext{b) \ Näytä, että yleisemmin pätee}
&\Sigma_h(\chi_{A},\mathcal{T}_h) - \sigma_h(\chi_{A},\mathcal{T}_h)\
  \le \sum_{T_k\in\mathcal{T}_h:\,T_k\cap \partial A\neq\emptyset} \mu(T_k).
\intertext{c) \ Näytä, että jos ositus $\mathcal{T}_h$ on tasavälinen ja 
                $a,b\not\in\partial A$, niin pätee}
&\Sigma_h(\chi_{A},\mathcal{T}_h) - \sigma_h(\chi_{A},\mathcal{T}_h)\
  \ge\ \frac{1}{2}\sum_{T_k\in\mathcal{T}_h:\,T_k\cap \partial A\neq\emptyset} \mu(T_k).
\end{align*}

\end{enumerate}