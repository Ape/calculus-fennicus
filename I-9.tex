\section{Reaaliluvut} \label{reaaliluvut}
\alku

Aloitetaan määritelmästä.
\begin{Def} \label{reaaliluvut desimaalilukuina} \index{reaaliluvut!a@desimaalilukuina|emph}
(\vahv{Reaaliluvut}) \kor{Reaalilukujen} joukko $\R$ koostuu äärettömistä desimaaliluvuista,
ts.\ $\R=\DD$.
\end{Def}
Määritelmä \ref{reaaliluvut desimaalilukuina} reaaliluvuille ei ole ainoa mahdollinen, sillä on
ilmeistä, että esimerkiksi äärettömät binaariluvut voitaisiin yhtä hyvin ottaa määrittelyn
lähtökohdaksi. Monia muitakin lähestymistapoja on, kuten havaitaan tuonnempana. Määritelmä 
\ref{reaaliluvut desimaalilukuina} antaa joka tapauksessa reaaliluvuille erään mahdollisen 
tulkinnan, joka jatkossa otetaan lähtökohdaksi. 

Yhdessä aiemmin sovitun samastusrelaation (Määritelmä \ref{samastus DD}) kanssa Määritelmä 
\ref{reaaliluvut desimaalilukuina} kertoo vasta, miltä reaaliluvut 'näyttävät' ja miten niitä 
erotellaan. Jotta reaaliluvuilla päästäisiin myös laskemaan, on määriteltävä lukujen väliset
laskuoperaatiot. 

\subsection*{$\R$:n laskuoperaatiot}

Viitaten Määritelmään \ref{jonon raja - desim} asetetaan
\begin{Def} \label{reaalilukujen laskutoimitukset}
\index{laskuoperaatiot!b@reaalilukujen|emph}
\index{reaalilux@reaalilukujen!a@laskuoperaatiot|emph}
(\vahv{Reaalilukujen laskutoimitukset}) Jos $\x = \seq{x_n}$ ja $\y = \seq{y_n}$ ovat
reaalilukuja, niin määritellään
\begin{align*}
\text{yhteenlasku:} \quad\quad       \x+\y &= \lim_n\,(x_n+y_n) = \z\in\R. \\
\text{kertolasku:} \ \quad\quad\quad  \x\y &= \lim_n\,x_n y_n = \z\in\R.\\
\text{jakolasku:} \,\ \ \quad\quad   \x/\y &= \lim_n\,x_n / y_n = \z\in\R.
\end{align*} \end{Def}
Tässä on jakolaskussa oletettava tavalliseen tapaan $\y \neq 0$, eli $y_n \not\kohti 0$
(Määritelmä \ref{samastus QD}). Ehto merkitsee, että jollakin $m$ on  $y_m \neq 0$, jolloin
tästä indeksistä alkaen on $\abs{y_n} \ge \abs{y_m} > 0,\ n \ge m$ (koska $\seq{\abs{y_n}}$ on
kasvava jono). Jakolaskun määritelmä on siis muodollisesti kunnossa, kunhan $\y \neq 0$ ja
jonossa $\seq{x_n/y_n}$ rajoitutaan indekseihin $n \ge m$. Yhteenlaskun määritelmästä nähdään,
että luvun $\x = \seq{x_n}$ vastaluku on $-\x = \seq{-x_n}$, eli operaatio $\x \map -\x$ vastaa
desimaaliluvun etumerkin vaihtoa, kuten jo Luvussa \ref{desimaaliluvut} sovittiin. Vähennyslasku
määritellään tämän jälkeen normaaliin tapaan, eli $\x-\y=\x+(-\y)=\lim_n(x_n-y_n)\in\R$.

Määritelmä \ref{reaalilukujen laskutoimitukset} on käytännön läheinen sikäli, että se
perustuu suoraan lukujen $\breve{x}$ ja $\breve{y}$ esitysmuotoihin annetussa
lukujärjestelmässä, tässä kymmenjärjestelmässä. Lukujonojen $\seq{x_n+y_n}$, $\seq{x_ny_n}$ ja
$\seq{x_n/y_n}$ termien laskemisen voi ajatella vastaavan likimääräisiä laskuoperaatioita
liukuluvuilla (vrt.\ Luku \ref{desimaaliluvut}). Määritelmään sisältyy kuitenkin toistaiseksi
ratkaisematon teoreettinen ongelma: Lukujonojen $\seq{x_n+y_n}$, $\seq{x_ny_n}$ ja
$\seq{x_n/y_n}$ oletetetaan suppenevan kohti reaalilukua (eli ääretöntä desimaalilukua), mutta
toistaiseksi tämä on varmistettu vain monotonisen ja rajoitetun lukujonon osalta
(Lause \ref{monotoninen ja rajoitettu jono}). Määritelmän
\ref{reaalilukujen laskutoimitukset} lukujonot ovat kyllä rajoitettuja (mainituin lisäehdoin
koskien jonoa $\seq{x_n/y_n}$) mutta eivät välttämättä monotonisia. Määritelmän tueksi
tarvitaankin seuraava lause.
\begin{Lause} \label{monotoninen ja rajoitettu jono - yleistys} Jos $\seq{a_n}$ ja $\seq{b_n}$
ovat monotonisia ja rajoitettuja lukujonoja, niin lukujonot $\seq{a_n+b_n}$ ja $\seq{a_nb_n}$
suppenevat Määritelmän \ref{jonon raja - desim} mukaisesti kohti reaalilukua. Jos edelleen
$|b_n| \ge b>0\ \forall n$, niin myös jono $\seq{a_n/b_n}$ suppenee kohti reaalilukua.
\end{Lause}
\tod \ \underline{Yhteenlasku}. Jos $\seq{a_n}$ ja $\seq{b_n}$ ovat molemmat kasvavia tai
molemmat väheneviä, niin $\seq{a_n+b_n}$ on vastaavasti kasvava/vähenevä
(Harj.teht.\,\ref{desimaaliluvut}:\ref{H-I-5: monotonisten jonojen yhdistely}), jolloin
$\lim_n(a_n+b_n)\in\R$ on olemassa Lauseen \ref{monotoninen ja rajoitettu jono} perusteella.
Olkoon toinen jonoista kasvava ja toinen vähenevä, esim.\ $\seq{a_n}$ kasvava. Valitaan
$b\in\Q$ siten, että $b_n+b \ge 0\ \forall n$ (mahdollista, koska $\seq{b_n}$ on rajoitettu) ja
kirjoitetaan
\[
a_n+b_n = (a_n-b)+(b_n+b).
\]
Tässä $\seq{b_n+b}$ on rajoitettu ja vähenevä jono, joten Lauseen
\ref{monotoninen ja rajoitettu jono} mukaan se suppenee:
$\lim_n(b_n+b)=\breve{x}=\seq{x_n}\in\R$. Koska $b_n+b \ge 0\ \forall n$, on $\breve{x}$:n
etumerkki $+$ (mahdollisesti $\breve{x}=0$), joten lukujono $\seq{x_n}$ on kasvava.
Koska myös $\seq{a_n-b}$ on kasvava jono, niin samoin on $\seq{a_n-b+x_n}$, joten
Lauseen \ref{monotoninen ja rajoitettu jono} mukaan 
$\lim_n(a_n-b+x_n)=\breve{y}=\seq{y_n}\in\R$. Näin määrätty $\breve{y}$ on lukujonon
$\seq{a_n+b_n}$ raja-arvo Määritelmän \ref{jonon raja - desim} mukaisesti, sillä ko.\
määritelmän ja Lauseen \ref{raja-arvojen yhdistelysäännöt} perusteella
\[
(a_n+b_n)-y_n = [(a_n-b+x_n)-y_n]+(b_n+b-x_n) \kohti 0+0 = 0.
\]
\underline{Kertolasku}. Olkoon $\lim_na_n=\breve{x}=\seq{x_n}\in\R$ ja
$\lim_nb_n=\breve{y}=\seq{y_n}\in\R$. Jos $\breve{x}$:llä ja $\breve{y}$:llä on sama etumerkki,
niin jono $\seq{x_ny_n}=\seq{|x_n||y_n|}$ on kasvava (koska $\seq{|x_n|}$ ja $\seq{|y_n|}$, ovat
kasvavia, vrt.\ Harj.teht.\,\ref{desimaaliluvut}:\ref{H-I-5: monotonisten jonojen yhdistely}),
muussa tapauksessa $\seq{x_ny_n}=\seq{-|x_n||y_n|}$ on vähenevä jono. Lauseen
\ref{monotoninen ja rajoitettu jono} mukaan on siis olemassa raja-arvo
$\lim_nx_ny_n=\breve{z}=\seq{z_n}\in\R$. Tämä on myös lukujonon $\seq{a_nb_n}$ raja-arvo
Määritelmän \ref{jonon raja - desim} mukaisesti, sillä ko.\ määritelmän ja Lauseen
\ref{jonotuloksia} (V3) mukaan
\[
a_nb_n-z_n = (a_n-x_n)b_n+(b_n-y_n)x_n+(x_ny_n-z_n) \kohti 0+0+0=0.
\]
\underline{Jakolasku}. Koska $\seq{b_n}$ on monotoninen lukujono, niin tehdyin lisäoletuksin
on jostakin indeksistä $n=m$ alkaen oltava joko $b_n \ge b$ tai $b_n \le -b$. Tällöin
lukujono $\seq{1/b_n}$ on monotoninen indeksistä $n=m$ alkaen. Kirjoitetaan tästä indeksistä
eteenpäin $a_n/b_n=a_n(1/b_n)$ ja vedotaan jo käsiteltyyn kertolaskuun. \loppu

Koska Määritelmän \ref{reaalilukujen laskutoimitukset} lukujonot $\seq{x_n}$ ja $\seq{y_n}$ ovat
monotonisia ja rajoitettuja, niin Lause \ref{monotoninen ja rajoitettu jono - yleistys} takaa
määritelmän toimivuuden. Havainnollistettakoon lauseen todistuksessa käytettyä päättelyä vielä
laskuesimerkillä.
\begin{Exa} \label{laskuesimerkki desimaaliluvuilla pi ja e} Olkoon
\begin{align*}
\breve{\pi} &= 3.141592653\ldots = \seq{3,3.1,3.14,3.141\ldots} = \seq{p_n}, \\
\breve{e}   &= 2.718281828\ldots = \seq{2,2.7,2.71,2.718\ldots} = \seq{e_n}.
\end{align*}
Määritä a) $\breve{\pi}+\breve{e}$ ja $\breve{\pi}\breve{e}$, b) $\breve{\pi}-\breve{e}$,
c) $\breve{\pi}/\breve{e}$ Määritelmän \ref{reaalilukujen laskutoimitukset} mukaisesti
viiden desimaalin tarkkuudella käyttämällä Lauseen \ref{monotoninen ja rajoitettu jono}
todistusta algoritmina.
\end{Exa}
\ratk a) Lukujonot
\[
\seq{p_n+e_n}=\seq{5,5.8,5.85,5.859\ldots}, \quad 
\seq{p_ne_n}=\seq{6,8.37,8.5094,8.537238\ldots}
\]
ovat molemmat kasvavia ja rajoitettuja, joten ne suppenevat kohti reaalilukua
(Lause \ref{monotoninen ja rajoitettu jono}). Lauseen \ref{monotoninen ja rajoitettu jono} 
todistuskonstruktiota algoritmina käyttäen saadaan
\[
\breve{\pi}+\breve{e} = 5.85987\,..\,, \quad \breve{\pi}\breve{e}  = 8.53974\,..\,.
\]
b) Lukujono $\seq{p_n-e_n}=\seq{0,0.4,0.43,0.423,\ldots}$ ei ole monotoninen, joten Lause
\ref{monotoninen ja rajoitettu jono} ei sovellu suoraan. Kirjoitetaan sen vuoksi ensin
\[
p_n-e_n = (p_n-3) + (3-e_n).
\]
Tässä $\seq{3-e_n}$ on rajoitettu ja vähenevä lukujono. Sen raja-arvoksi saadaan Lauseen
\ref{monotoninen ja rajoitettu jono} todistuskonstruktiolla
\[
\lim_n(3-e_n) = 3-\breve{e} = 0.28171\,..
\]
Kun tämä välitulos tulkitaan lukujonoksi $\seq{0,0.2,0.28,0.281,\ldots}=\seq{x_n}$, niin
$\seq{p_n-3}$ ja $\seq{x_n}$ ovat molemmat kasvavia lukujonoja. Siis myös $\seq{(p_n-3)+x_n}$
on kasvava, ja voidaan laskea (algoritmi sama kuin dellä)
\[
\breve{\pi}-\breve{e} = \lim_n[(p_n-3)+x_n] = 0.42331\,.. 
\]
c) Lukujonon $\seq{p_n/e_n}$ ei ole monotoninen, joten Lause
\ref{monotoninen ja rajoitettu jono} ei sovellu tähänkään suoraan. Lasketaan ensin
välituloksena vähenevän lukujonon $\seq{1/e_n}$ raja-arvo:
\[
\lim_n(1/e_n) = 1/\breve{e} = 0.36787\,..
\]
Kun välitulos tulkitaan lukujonoksi $\seq{x_n}=\seq{0,0.3,0.36,0.367,\ldots}$, niin
$\seq{p_nx_n}$ on kasvava lukujono. Tähän Lauseen \ref{monotoninen ja rajoitettu jono}
todistuskonstruktio soveltuu ja antaa
\[
\breve{\pi}/\breve{e} = \breve{\pi}(1/\breve{e}) = \lim_n p_nx_n = 1.15572\,.. \loppu
\]

\subsection*{$\R$:n järjestysrelaatio}

Myös järjestysrelaation määritelmä reaaliluvuille on suoraviivainen. 
\begin{Def} \label{reaalilukujen järjestys} 
\index{jzy@järjestysrelaatio!c@$\R$:n|emph}
\index{reaalilux@reaalilukujen!b@järjestysrelaatio|emph}
(\vahv{$\R$:n järjestysrelaatio}) Jos $\x=\seq{x_n}\in\R,\ \y=\seq{y_n}\in\R$, niin
$\,\x<\y\,$ täsmälleen kun $\,x_n<y_n\,$ jollakin $n\,$ ja $\,\x\neq\y$.
\end{Def}
Määritelmän \ref{reaalilukujen järjestys} mukaisesti voidaan reaalilukujen
$\x=\seq{x_n}$ ja $\y=\seq{y_n}$ suuruusjärjestys ratkaista laskennallisesti vertailemalla
lukuja $x_n,y_n,\ n=0,1,\ldots$ Vertailua jatketaan, kunnes tavataan ensimmäinen indeksi, 
jolla $x_n \neq y_n$. (Jollei tällaista indeksiä tavata, on $\x=\y$.) Jos 
$\abs{x_n-y_n} \ge 2 \cdot 10^{-n}$, on suuruusjärjestys ratkennut. Muussa tapauksessa,
eli jos $x_n-y_n = \pm 10^{-n}$, on vielä mahdollisuus, että $\x$ ja $\y$ samastuvat
kumpikin lukuun $x_n$ tai $y_n$, jolloin on $\x=\y$ (ks.\ Luvun \ref{suppenevat lukujonot}
tarkastelut liittyen Lauseeseen \ref{samastuslause DD}).

Määritelmistä \ref{reaaliluvut desimaalilukuina}--\ref{reaalilukujen järjestys} on
johdettavissa seuraava merkittävä tulos. Todistus esitetään luvun lopussa
(harjoitustehtävillä tuettuna).
\begin{*Lause} \label{R on kunta} (\vahv{$(\R,+,\cdot,<)$ on järjestetty kunta})
Reaalilukujen joukko varustettuna Määritelmän \ref{reaalilukujen laskutoimitukset}
mukaisilla laskuoperaatioilla ja Määritelmän \ref{reaalilukujen järjestys} mukaisella
järjestysrelaatiolla on järjestetty kunta. 
\end{*Lause}

Määritelmästä \ref{reaalilukujen järjestys} seuraa välittömästi, että jos
$\x = \seq{x_n} \in \R$, niin
\begin{align*}
\x = 0 \quad &\ekv \quad x_n = 0\ \ \forall n \\
\x < 0 \quad &\ekv \quad x_n < 0\ \ \text{jollakin}\ n \quad (\text{etumerkki}\ e = - ) \\
\x > 0 \quad &\ekv \quad x_n > 0\ \ \text{jollakin}\ n \quad (\text{etumerkki}\ e = + )
\end{align*}
Tämä vertailu siis ratkeaa pelkästään $\x$:n etumerkin perusteella, ellei ole $\x=0$. $\R$:n
järjestysrelaation määrittely voidaan vaihtoehtoisesti perustaa tähän vertailuun yhdistettynä
vähennyslaskuun (vrt.\ $\Q$:n järjestysrelaatio, Luku \ref{ratluvut})\,:
\[
\x<\y \,\ \ekv \,\ \x-\y<0.
\]
On ilmeistä, että järjestysrelaation aksioomista ainakin (J1) (vrt.\ Luku \ref{ratluvut})
on voimassa kummalla tahansa määritelmällä. 

\subsection*{Reaaliluvut abstrakteina lukuina}
\index{reaaliluvut!b@abstrakteina lukuina|vahv}

Kun reaaliluvuille on määritelty sekä laskuoperaatiot että järjestysrelaatio, voidaan 
reaalilukujen olemus jonoina haluttaesssa 'unohtaa' ja käsitellä lukuja vain abstrakteina 
lukuina, joilla voi laskea ja joita voi vertailla. Tämän ajattelutavan mukaisesti pidetään
jatkossa myös reaalilukuja 'oikeina' lukuina, joita ei symbolisissa merkinnöissä enää
erotella rationaaliluvuista. 
\begin{Exa} \label{kertausesimerkkejä} Luvun \ref{monotoniset jonot} tulosten ja 
Määritelmien \ref{reaaliluvut desimaalilukuina}-- \ref{reaalilukujen laskutoimitukset}
perusteella voidaan nyt kirjoittaa
\begin{align*}
\lim_n \left(1 + \dfrac{1}{n}\right)^n\ =\ \sum_{k=0}^\infty \frac{1}{k!}\    
                                           &= \ 2.71828182845905\,..\ = \ e\in\R, \\ 
\lim_n \left(1 - \dfrac{1}{n}\right)^n\    &= \ 0.36787944117144\,..  = \ \frac{1}{e}\,.
\end{align*}
Ottamalla käyttöön toinen hyvin tunnettu reaaliluku $\,\pi\ =\ 3.1415926535897\,..$ voidaan
myös osoittaa, että (vrt.\ Luku \ref{monotoniset jonot})
\[
\sum_{k=1}^\infty \dfrac{1}{k^2}\ = \ 1.64493406684822\,..\ = \ \frac{\pi^2}{6}\,.
\]
\end{Exa}
Reaalilukujen tultua määritellyksi järjestettynä kuntana tulee myös lukujonon raja-arvon 
alkuperäisestä määritelmästä (Määritelmä \ref{jonon raja}) pätevä, kun jonon termit ja/tai 
raja-arvo ovat reaalilukuja. Koska lukujonon suppeneminen kohti reaalista raja-arvoa on
aiemmin määritelty erikseen (Määritelmä \ref{jonon raja - desim}), tarvitaan ympyrän
sulkemiseksi seuraava tulos (todistus luvun lopussa).
\begin{*Lause} \label{suppeneminen kohti reaalilukua} Määritelmät \ref{jonon raja} ja 
\ref{jonon raja - desim} lukujonon suppenemiselle kohti reaalilukua ovat yhtäpitävät. 
\end{*Lause}

\subsection*{Kymmenjako- ja puolituskonstruktiot}
\index{kymmenjakokonstruktio|vahv} \index{puolituskonstruktio|vahv}%

Määritelmän \ref{reaaliluvut desimaalilukuina} mukaan reaalilukua voi laskennallisesti 
'vain lähestyä, ei saavuttaa', paitsi jos luku on rationaalinen. Algoritmisesti voi 
'lähestyminen' tapahtua esim.\ suoraan Määritelmään \ref{reaaliluvut desimaalilukuina} 
perustuen, jolloin lasketaan ensin luvun kokonaislukuosa ja sen jälkeen desimaaleja yksi
kerrallaan. Tarkastellaan nyt hieman yleisemmin tätä laskemisen ongelmaa.

Olkoon reaaliluku $a$ määritelty yksikäsitteisesti, esimerkiksi asettamalla luvulle jokin 
algebrallinen ehto, määrittelemällä luku jonkin tunnetun jonon raja-arvona, tai jollakin
muulla tavalla. Halutaan konstruoida $\,a\,$ Määritelmän \ref{reaaliluvut desimaalilukuina}
mukaisena desimaalilukuna, eli muodossa $a = x_0.d_1 d_2\ldots$  Oletetaan jatkossa, että $a$:n
etumerkki on $+$ (muuten konstruoidaan luku $-a\,$). Tällöin $a$:n kokonaislukosa ja desimaalit
luvuissa $x_n=x_0.d_1 \ldots d_n$ määrytyvät seuraamalla toimintaohjetta:
\[
\text{Etsi $\,x_n\in\Q_n\,$ siten, että pätee}\ \ x_n \le a\ \ 
                            \text{ja}\ \ x_n + 10^{-n} > a, \quad n = 0,1, \ldots
\]
Jos $a$ on rationaaliluku, niin toimintaohje on sama kuin jakokulma-algoritmissa (vrt.\ Luku 
\ref{desimaaliluvut}). Yleisemmän reaaliluvun $a>0$ tapauksessa toimintaohjetta voi pitää
ajatuskonstruktiona, joka \pain{määrittelee} luvun $a$. Annetaan tälle nimi
\kor{kymmenjako}konstruktio. --- Kymmenjakokonstruktiota on jo aiemmin käytetty Lauseen
\ref{monotoninen ja rajoitettu jono} todistuksessa. Reaaliluvun määritelmänä 
konstruktiota voi pitää järjestysrelaatioon nojaavana Määritelmän
\ref{reaaliluvut desimaalilukuina} toisintona tai täsmennyksenä.

Kymmenjakokonstruktion on jo aiemmin todettu toimivan myös algoritmina (vrt.\ edellisen
luvun esimerkit ja harjoitustehtävät). Yleisemmin algoritmi toimii, jos oletetaan, että
konstruktioon sisältyvät vertailut ovat laskennallisesti toteutettavissa. Näin on ainakin,
jos oletetaan laskettava luku $a \in \R$ sellaiseksi, että minkä tahansa \pain{rationaaliluvun}
$x \in \Q$ kohdalla voidaan selvittää \pain{äärellisellä} määrällä \pain{rationaalisia}
operaatioita (laskuoperaatioita ja vertailuja), mikä vaihtoehdoista $a < x$, $a = x$, $a > x$
on voimassa. Tällöin saadaan ensin $a$:n etumerkki selville vertaamalla lukuun $x=0$, minkä
jälkeen $a$:n tai $-a$:n kokonaislukuosa ja desimaalit $d_n$ voidaan laskea ym.\ toimintaohjetta
seuraamalla, periaatteessa mihin tahansa haluttuun indeksiin asti. Tällainen
\kor{kymmenjakoalgoritmi} on siis jakokulma-algoritmin yleistys.
\begin{Exa} \label{neliöjuuri 2} Konstruoi reaaliluku $a=\sqrt{2}$ kymmenjakoalgoritmilla.
\end{Exa}
\ratk Luvun $a$ ja positiivisen rationaaliluvun $x$ vertailu palutuu rationaaliseksi
ekvivalenssilla
\[
a<x\ \ekv\ 2<x^2.
\]
Koska $1^2<2$ ja $2^2>2$, on ensinnäkin oltava $\ a = 1.d_1d_2d_3\,..\ $ Edelleen koska
$\,1.4^2 = 1.96 < 2$, $\,1.5^2 = 2.25 > 2$, $\,1.41^2 = 1.9881 < 2$, $\,1.42^2 = 2.0164 > 2$,
$\,1.414^2 = 1.999396 < 2$ ja $1.415^2 = 2.002225 > 2$, on $d_1 = 4$, $d_2=1$ ja $d_3=4$.
Jatkamalla tällä tavoin on tuloksena (jaksoton) desimaaliluku
$a = 1.4142135623730950488016887\,..$\footnote[2]{Ennen laskimia on neliöjuurien käsinlaskua
esimerkin tapaan harjoiteltu kouluissakin. -- Nykyisin laskimet ja tietokoneet laskevat
neliöjuuria kymmenjakoa paljon tehokkaammilla, palautuviin lukujonoihin perustuvilla
algoritmeilla, vrt.\ edellisen luvun Esimerkki \ref{sqrt 2 algoritmina}.} \loppu
\begin{Exa} Laske luku $a = \sqrt{\sqrt{2}+1}\ $ $20$ merkitsevän numeron tarkkuudella käyttäen
kymmenjakoalgoritmia. 
\end{Exa}
\ratk Verrattaessa lukua $a$ rationaalilukuun $x \ge 1$ pätee
\[
a\,<\,x \qekv \sqrt{2} + 1\,<\,x^2 \qekv \sqrt{2}\,<\,x^2 - 1 \qekv 2\,<\,(x^2 - 1)^2,
\]
joten vertailu palautuu rationaalioperaatioksi. Algoritmia seuraten saadaan
\[
a\ =\ 1.5537739740300373073\,.. \quad \loppu
\]
\index{symbolinen laskenta}%
Esimerkeissä vertailun $a<x$ palauttaminen rationaaliseksi perustuu \kor{symboliseen} 
(ei-numeeriseen) \kor{laskentaan}, tässä algebraan, joka nojaa viime kädessä reaalilukujen
kunta-aksioomiin, järjestysrelaation aksioomiin ja neliöjuuren symboliseen määritelmään
$(\sqrt{a})^2 = a$. Kuten näissä esimerkeissä, symbolisen laskennan tehtävänä on yleensä
yksinkertaistaa tai selkeyttää laskentatehtävää ennen varsinaisia numeerisia laskuja.
 
Kymmenjaon vastine voidaan luonnollisesti konstruoida myös muihin lukujärjestelmiin perustuvana. 
Binaarijärjestelmän tapauksessa käytetään nimitystä \kor{puolituskonstruktio} (tai 
puolitusmenetelmä, engl.\ bisection). Puolituskonstruktiossa reaaliluku $a \ge 0$ konstruoidaan
binaarimuodossa $a = x_0.b_1b_2b_3\,..$, missä kokonaislukuosa $x_0$ ja bitit $b_n \in \{0,1\}$
valitaan siten, että luvut $x_0$ ja $\,x_n = x_0 + \sum_{k=1}^n b_k \cdot 2^{-k},\ n \in \N,$
toteuttavat  
\[
x_n \le\,a \quad \text{ja} \quad x_n + 2^{-n} >\,a, \quad n = 0,1, \ldots
\]
Yksinkertaisen logiikkansa vuoksi puolitusmenetelmää käytetään matemaattisten todistusten 
ajatuskonstruktioissa hyvin usein. (Puolituskonstruktio olisi ollut vaihtoehto myös Lauseen
\ref{monotoninen ja rajoitettu jono} todistuksessa.) Ym.\ lisäoletusten voimassa ollessa voi
konstruktiota käyttää myös toimivana algoritmina.

\subsection*{Reaalilukujen jonot}
\index{lukujono|vahv}

Reaalilukujen jono-olemuksen 'unohtaminen' on erityisen suositeltavaa silloin, kun tutkitaan 
\kor{reaalilukujonoja} ja niiden suppenemista. Järjestysrelaatioon sekä (vähäisessä määrin)
kunta-algebraan vetoava Määritelmä \ref{jonon raja} on sellaisenaan pätevä myös reaalilukujonon
suppenemisen määritelmänä. Jos reaalilukujonolla on tämän määritelmän mukainen raja-arvo
(reaalilukuna), niin sanotaan, että jono
\index{suppeneminen!a@lukujonon} \index{hajaantuminen!a@lukujonon}%
\kor{suppenee}, muussa tapauksessa \kor{hajaantuu} (vrt.\ Määritelmä \ref{jonon raja - desim}). 
\begin{Exa} \label{reaalinen geometrinen sarja} Määritä perusmuotoisen geometrisen sarjan summa,
kun $q = 1/\sqrt{2}$. 
\end{Exa}
\ratk Luvussa \ref{jonon raja-arvo} johdettu geometrisen sarjan summakaava on pätevä, kun 
$q \in \R$ ja $\abs{q}<1$. Tässä on $0<q<1$, joten
\[
\sum_{n=0}^{\infty} \bigl( \frac{1}{\sqrt{2}} \bigr)^n\ 
                      =\ \frac{1}{1-\frac{1}{\sqrt{2}}}\ =\ \frac{\sqrt{2}}{\sqrt{2}-1}\ 
                      =\ 2 + \sqrt{2}\ =\ 3.4142135623730950488 \ldots  \loppu
\]
Esimerkin lasku on jälleen esimerkki myös symbolisesta laskennasta, jossa lukua $\sqrt{2}$ 
käsitellään abstraktina lukuna pelkästään kunnan $(\R,+,\cdot)$ yleisiä ominaisuuksia 
(kunta-aksioomia) ja symbolista määritelmää $\ (\sqrt{2})^2 = 2\ $ hyväksi käyttäen.

Todettakoon, että kaikki Luvuissa \ref{jonon raja-arvo}--\ref{monotoniset jonot} esitetyt
lukujonoja koskevat määritelmät ja väittämät ovat päteviä reaalilukujen kunnassa
$(\R,+,\cdot,<)$ --- syystä, että nämä perustuvat vain järjestetyn kunnan algebraan ja
oletukseen, että kunta sisältää rationaaliluvut. 

Esimerkkinä reaalilukujonon suppenemistarkastelusta todistettakoon
\begin{Prop} \label{juurilemma} $\quad \displaystyle{\boxed{\kehys\quad 
\lim_n \sqrt[n]{a}=1 \quad \forall a\in\R,\ a>0. \quad}}$ 
\end{Prop}
\tod Merkitään $b_n=\sqrt[n]{a}$. Tapauksessa $a=1$ on väittämä ilmeisen tosi ja tapauksessa 
$0<a<1$ pätee: $\,b_n=1/(1/a)^{1/n} \kohti 1$, jos $(1/a)^{1/n} \kohti 1$ 
(Lause \ref{raja-arvojen yhdistelysäännöt} reaalilukujonoille). Riittää siis tarkastella
tapausta $a>1$. Tällöin on oltava $b_n>1\ \forall n$ (koska $b_n^n=a>1$), jolloin seuraa
\[
\left(\frac{b_{n+1}}{b_n}\right)^n =\, \frac{1}{b_{n+1}}\frac{b_{n+1}^{n+1}}{b_n^n}
\,=\, \frac{1}{b_{n+1}}\frac{a}{a} \,=\, \frac{1}{b_{n+1}} \,<\, 1. 
\]  
Siis $b_{n+1}/b_n<1\ \forall n$, joten $\seq{b_n}$ on aidosti vähenevä lukujono. Koska 
$\seq{b_n}$ on myös ilmeisen rajoitettu lukujono, niin $b_n \kohti b\in\R$
(Lause \ref{monotoninen ja rajoitettu jono}), ja koska $\,b_n>1\ \forall n$, niin $\,b \ge 1$
(Propositio \ref{jonotuloksia} (V1) reaalilukujonoille). Tässä vaihtoehto $b>1$ johtaisi
loogiseen ristiriitaan: $\,a=b_n^n \ge b^n\kohti\infty$. Siis $\,\lim_n b_n=1$. \loppu 

\subsection*{*Lauseiden \ref{R on kunta} ja \ref{suppeneminen kohti reaalilukua} todistukset}

\vahv{Lause \ref{R on kunta}}. \ Kunta-aksioomista todistetaan esimerkkinä ainoastaan 
aksioomien (K4) (kertolaskun liitäntälaki) ja (K9) (käänteisalkio) voimassaolo, muut
jätetään harjoitustehtäväksi (Harj.teht.\,\ref{H-I-9: R:n kunta-aksioomat}). 

\fbox{K4} \ Olkoon $\x = \seq{x_n},\ \y = \seq{y_n},\ \z = \seq{z_n}$ reaalilukuja ja olkoon
edelleen $\x\y = \breve{a} = \seq{a_n} \in \R\ $ ja $\ \y\z = \breve{b} = \seq{b_n} \in \R$. 
Tällöin aksiooman (K4) sisältö on Määritelmien \ref{reaalilukujen laskutoimitukset}
ja \ref{jonon raja - desim} perusteella
\[
\breve{a}\z\ =\ \x\breve{b} \quad \ekv \quad \lim_n\,(a_n z_n - x_n b_n)\ =\ 0.
\]
Kirjoitetaan tässä
\[
a_n z_n - x_n b_n\ =\ (a_n - x_n y_n) z_n + x_n (y_n z_n - b_n).
\]
Koska tässä $a_n - x_n y_n \kohti 0\ $ ja $\ y_n z_n - b_n \kohti 0$ (Määritelmät 
\ref{reaalilukujen laskutoimitukset} ja \ref{jonon raja - desim}) ja jonot $\seq{x_n}$
ja $\seq{z_n}$ ovat rajoitettuja, niin väite $a_n z_n - x_n b_n \kohti 0$ seuraa Lauseen
\ref{raja-arvojen yhdistelysäännöt} ja Proposition \ref{jonotuloksia} (V3) perusteella.

\fbox{K9} \ Luvun $\x = \seq{x_n} \in \R$ käänteisluvun $\x^{-1}$ konstruoimiseksi olkoon 
\mbox{$\x \neq 0$}, jolloin jollakin $m \in \N$ on oltava
$\,\abs{x_n} \ge \abs{x_m} > 0\ \ \forall n \ge m$.
Tällöin jono \seq{\,x_m^{-1}, x_{m+1}^{-1}, \ldots\,}\ on monotoninen ja rajoitettu, jolloin
on olemassa reaaliluku $\,\y = \seq{y_n}\,$ siten, että $\,x_n^{-1}-y_n \kohti 0$ 
(Lause \ref{monotoninen ja rajoitettu jono}). Proposition \ref{jonotuloksia} (V3) ja
Määritelmän \ref{reaalilukujen laskutoimitukset} perusteella päätellään tällöin
\[
x_n^{-1} - y_n \kohti 0 \quad \ekv \quad 1 - x_n y_n \kohti 0 \quad \ekv \quad \x\y = 1.
\]
Siis $\y = \x^{-1}$ ja aksiooma (K9) on näin ollen voimassa. \loppu

Järjestyrelaation aksiooman (J1) voimassaolo on jo todettu. Aksioomien (J2) ja (J4)
todentaminen jätetään harjoitustehtäväksi
(Harj.teht.\,\ref{H-I-9: R:n järjestysominaisuuksia}cd). Todistetaan siis ainoastaan aksiooman
(J3) voimassaolo.
\begin{Prop} \label{R:n aksiooma (J3)} Reaaliluvuille pätee: Jos $\x<\y$, niin
$\x+\z<\y+\z\ \forall \z\in\R$.
\end{Prop}
\tod Olkoon $\x=\seq{x_n}\in\R$, $\y=\seq{y_n}\in\R$, $\z=\seq{z_n}\in\R$ ja merkitään
$\x+\z=\breve{a}=\seq{a_n}\in\R$ ja $\y+\z=\breve{b}=\seq{b_n}\in\R$, jolloin
$x_n+z_n-a_n \kohti 0$ ja $y_n+z_n-b_n \kohti 0$, kun $n\kohti\infty$
(Määritelmä \ref{reaalilukujen laskutoimitukset}). Väitetään: Jos $\x<\y$, niin
$\breve{a}<\breve{b}$. Tehdään vastaoletus: $\breve{a}=\breve{b}$ tai 
$\breve{a}>\breve{b}$. Ensimmäisessä vaihtoehdossa on $\lim_n(a_n-b_n)=0$
(Määritelmä \ref{samastus DD}), jolloin oletusten ja Lauseen 
\ref{raja-arvojen yhdistelysäännöt} perusteella seuraa
\[
x_n-y_n \,=\, (x_n+z_n-a_n)-(y_n+z_n-b_n)+(a_n-b_n) \,\kohti\, 0.
\]
Siis $\x=\y$ (Määritelmä \ref{samastus DD}), mutta tämä on looginen ristiriita,
koska (J1) on voimassa ja oletettiin $\x<\y$. Toisessa vaihtoehdossa
($\breve{a}>\breve{b}$) on $a_n \ge b_n\ \forall n$
(Harj.teht.\,\ref{H-I-9: R:n järjestysominaisuuksia}a) ja samoin
$x_n \le y_n\ \forall n$ (koska $\x<\y$), joten seuraa
\begin{align*}
c_n \,&=\, (x_n+z_n-a_n)-(y_n+z_n-b_n) \\ 
      &=\, (x_n-y_n)+(b_n-a_n) \,\le\, x_n-y_n \,\le\, 0\,\ \forall n.
\end{align*}
Koska tässä $c_n \kohti 0$, niin $x_n-y_n \kohti 0$ (Propositio \ref{jonotuloksia} (V2)),
joten on jälleen päädytty loogiseen ristiriitaan: $\x<\y$ ja $\x=\y$. Vaihtoehdot
$\breve{a}=\breve{b}$ ja $\breve{a}>\breve{b}$ on näin muodoin pois suljettu, joten on
oltava $\breve{a}<\breve{b}$. \loppu

\vahv{Lause \ref{suppeneminen kohti reaalilukua}}. \ Perustetaan todistus seuraavaan
väittämään, joka seuraa helposti Määritelmästä \ref{reaalilukujen järjestys}
(Harj.teht.\,\ref{H-I-9: R:n järjestysominaisuuksia}b).
\begin{Lem} \label{R:n järjestyslemma} Jos $\,\x=\seq{x_n}\in\R$, niin 
$\,\abs{\x-x_n} \le 10^{-n}\ \forall n$.
\end{Lem}
Oletetaan, että $a_n \kohti \x,\ \x=\seq{x_n}\in\R$ Määritelmän \ref{jonon raja} 
mukaisesti ja olkoon $\eps > 0$. Tällöin on olemassa indeksit $N_1,N_2 \in \N$ siten, että 
$\abs{a_n-\x} < \eps/2\,$ kun $n > N_1$ ja $10^{-n} < \eps/2\,$ kun $n > N_2$, jolloin 
(järjestetyn kunnan) kolmioepäyhtälön ja Lemman \ref{R:n järjestyslemma} perusteella
\[
\abs{a_n-x_n} \le \abs{a_n-\x} + \abs{\x-x_n} < \abs{a_n-\x} + 10^{-n} 
              < \eps, \quad \text{kun}\ n > \max\,\{N_1,N_2\} = N.
\]
Tässä $\eps > 0$ oli mielivaltainen, joten Määritelmän \ref{jonon raja} mukaan 
$a_n-x_n \kohti 0$, eli $\,a_n \kohti \x$ Määritelmän \ref{jonon raja - desim} mukaisesti.
Tämä todistaa väittämän ensimmäisen osan. Toinen osa todistetaan samankaltaisella 
päättelyllä. \loppu

\Harj
\begin{enumerate}

\item \label{H-I-9: R:n kunta-aksioomat} 
Näytä Määritelmään \ref{reaalilukujen laskutoimitukset} perustuen, että reaaliluvuille
ovat voimassa: \newline
a) yhteen- ja kertolaskun vaihdantalait (K1,K2) \newline 
b) yhteenlaskun liitäntälaki (K3) \newline
c) yhteen- ja kertolaskun osittelulaki (K5)

\item \label{H-I-9: R:n järjestysominaisuuksia}
Olkoon $\x=\seq{x_n}\in\R$, $\y=\seq{y_n}\in\R$ ja $z=\seq{z_n}\in\R$. Näytä, että Määritelmien
\ref{reaalilukujen järjestys} ja \ref{reaalilukujen laskutoimitukset} perusteella pätee \newline 
a) \ $\x<\y\ \ekv\ \x \neq \y\ \ja\ x_n \le y_n\ \forall n$ \hspace{5mm}
b) \ $\abs{\x-x_n} \le 10^{-n}\ \forall n$ \newline
c) \ $\x<\y\ \ja\ \y<\z\ \impl\ \x<\z$ \hspace{15mm}\,
d) \ $\x>0\ \ja\ \y>0\ \impl\ \x\y>0$

\item
Tutki, kuinka suuri virhe tehdään, kun laskettaessa \ a) $e+\pi$, \ b) $e^2/\pi$, \ 
c) $\pi^2-e^2$ \ c) $e^5\pi^6$ katkaistaan $e$ ja $\pi$ ensin $9$ merkitsevään numeroon
ja laskuoperaatioissa tulos samoin $9$ merkitsevään numeroon.

\item
Olkoon lukujärjestelmän kantaluku $=k$. Käyttäen kymmenjakoalgoritmia vastaavaa 
$k$-jakoalgoritmia konstruoi neljän merkitsevän numeron tarkkuudella \ a) $\sqrt[3]{11}$
binaarijärjestelmässä, \ b) $\sqrt{7}$ $3$-kantaisessa järjestelmässä.

\item (*)
Tiedetään, että $\,\sum_{k=1}^\infty k^{-2}=\pi^2/6$. \,Näytä tämän tiedon perusteella:
\vspace{1mm}\newline
$\D
\text{a)}\,\ \sum_{k=0}^\infty \frac{1}{(2k+1)^2}=\frac{\pi^2}{8} \qquad
\text{b)}\,\ \sum_{k=0}^\infty \frac{(-1)^k}{(k+1)^2}=\frac{\pi^2}{12}$

%\item (*)
%Näytä, että $\,\lim_n \sqrt[n]{n^k}=1\ \forall k\in\N$.

\item (*)
Olkoon $a_k \kohti a\ (k \kohti \infty)$, missä on desimaalilukumerkinnöin 
\[
a_k=x_0^{(k)}.\,d_1^{(k)}d_2^{(k)}\,\ldots =\seq{x_n^{(k)}} \in\R, \quad a
                                           =x_0.d_1d_2\ldots = \seq{x_n} \in\R.
\]
Todista seuraavat väittämät: \newline
a) \ Jos $a$ ei ole äärellinen desimaaliluku, niin jokaisella $n\in\N$ on olemassa indeksi 
$N_n\in\N$ siten, että \ $a_k = x_0.d_1 \ldots d_n \ldots\ \forall k>N_n$. \newline
b) \ a-kohdan väittämä ei ole tosi jokaisella $a\in\R$ (vastaesimerkki\,!). \newline
c) \ Jos $a$ on äärellinen desimaaliluku, niin on olemassa $m\in\N$ ja jokaisella $n \ge m$ 
indeksi $N_n\in\N$ siten, että $d_n^{(k)}\in\{0,9\}\ \forall k>N_n$. \newline
d) \ Jokaisella $a\in\R$ pätee: $\,\lim_k x_k^{(k)}=a$.

\end{enumerate}