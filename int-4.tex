\section{Integraalifunktion numeerinen laskeminen. \\ Määrätty integraali}
\label{määrätty integraali}
\sectionmark{Määrätty integraali}
\alku

Annetun funktion integraalifunktion määräämiseksi on toistaiseksi tarkasteltu erilaisia
funktioalgebran keinoja. Näissä funktio on tunnettava laskusääntönä eli jonakin
algebrallisena lausekkeena tarkasteltavalla välillä tai sen osaväleillä. Sovelluksia
ajatellen tällainen oletus on melko rajoittava, sillä funktio voi yhtä hyvin olla pelkkä
pisteittäin määritelty 'luettelo'. Näin on esimerkiksi, jos funktion arvoja voidaan
käytännössä vain mitata tai jos funktion määritelmä on epäsuora niin, että funktioevaluaatio
$x \map f(x)$ onnistuu vain numeerisesti. Jos tällaisella funktiolla on integraalifunktio,
niin luonnollisesti integraalifunktionkaan arvoja ei voida käytännössä laskea muuten kuin
numeerisesti. Mutta miten? --- Jatkossa ratkaistaan tämä ongelma johtamalla yleispätevä
laskukaava numeeriselle funktioevaluaatiolle $x \map F(x)$, missä $F$ on annetun funktion
integraalifunktio. Laskukaava johdattaa samalla integraalifunktiosta riippumattomaan
\kor{määrätyn integraalin} käsitteeseen.

Oletetaan, että $f$ on määritelty (reaalifunktiona) välillä $[a,b]$, eli jokaiseen
$x\in[a,b]$ liittyy yksikäsitteinen luku $f(x)\in\R$. Täsmennetään $f$:n
integraalifunktion määrittämisen ongelma välillä $(a,b)$ seuraavasti:
\begin{itemize}
\item[(P)] Etsi funktio $y(x)$, joka on jatkuva välillä $[a,b]$, derivoituva välillä $(a,b)$
ja toteuttaa
\[
\begin{cases}
\,y'(x)=f(x),\quad x\in (a,b), \\
\,y(a)=0.
\end{cases}
\]
\end{itemize}
Tämän mukaan etsitään siis funktiota, joka on $f$:n integraalifunktio välillä $(a,b)$ ja
toteuttaa lisäksi asetetun jatkuvuusehdon ja alkuehdon. Koska derivoituvuus jo takaa
jatkuvuuden välillä $(a,b)$, niin jatkuvuusehto merkitsee ainoastaan lisäehtoja:
$y(x)$ on oikealta jatkuva $a$:ssa ja vasemmalta jatkuva $b$:sä. 

Sikäli kuin probleemalla (P) on ratkaisu, takaa asetettu alkuehto ratkaisun
yksikäsitteisyyden (vrt.\ Korollaari \ref{toiseksi yksinkertaisin dy}). Vaikeampaan
kysymykseen, joka koskee ratkaisun olemassoloa, antaa erään vastauksen seuraava huomattava
lause.
\begin{*Lause} \label{P:n ratkeavuus} Jos $f$ on jatkuva välillä $[a,b]$, niin probleemalla
(P) on ratkaisu.
\end{*Lause}
Lause \ref{P:n ratkeavuus} on keskeinen osaväittämä yleisemmässä lauseessa, joka tunnetaan
\kor{Analyysin peruslauseen} nimellä. Tämä muotoillaan ja (osittain) todistetaan jäljempänä
Luvussa \ref{analyysin peruslause}. Tässä luvussa asetetaan kysymys toisin: Jos yksinkertaisesti
\pain{oletetaan}, että (P) ratkeaa, niin miten luku $y(x)\in\R$ voidaan käytännössä
\pain{laskea}, kun $x\in(a,b]$ on annettu? Jatkossa tarkastellaan tätä kysymystä ensin
tapauksessa $x=b$, jolloin tehtävänä on siis laskea luku $y(b)$, kun $y(x)$ on määritelty
epäsuorasti (P):n ratkaisuna. Koska $y(b)$ on reaaliluku, niin ko.\ lukua ei yleisesti voi
laskea 'tarkasti', vaan käytännössä on tyydyttävä konstruoimaan $y(b)$ jonkin (viime kädessä
rationaalisen) lukujonon $\seq{Y_n}$ raja-arvona. Tehtävänä on siis määrittää algoritmi, jolla
on mahdollista laskea jokaisella $n\in\N$ luku $Y_n$ siten, että $\,\lim_nY_n=y(b)$.

Olkoon $n\in\N$ annettu. Lukua $Y_n$ määrättäessä otetaan lähtökohdaksi välin $[a,b]$
\index{jako (osaväleihin)}%
\kor{jako osaväleihin} $[x_{k-1},x_k]$, $k=1 \ldots n$, missä 
$a = x_0 < x_1 < \ldots < x_n = b$. Jako voi olla esim.\ 
\index{tasavälinen jako}%
\kor{tasavälinen}, jolloin on
$x_k=a+kh_n$, missä $h_n=(b-a)/n$. Yleisemmin osavälijakoja rajoitetaan ainoastaan ehdolla
\begin{equation} \label{int-4: tiheysehto}
\lim_nh_n=0, \quad \text{missä}\,\ h_n = \max_{k=1 \ldots n} (x_k-x_{k-1}),\,\ n\in\N.
                                                                       \tag{$\star$}
\end{equation}
Tässä $h_n$ on osavälijaon nk.\ 
\index{tiheysparametri}%
\kor{tiheysparametri}. Tasavälisille jaoille raja-arvoehto
\eqref{int-4: tiheysehto} luonnollisesti toteutuu.

Kun pisteet $x_k,\ k = 0 \ldots n$ on valittu em.\ tavalla, niin (P):n ratkaisulle $y(x)$
voidaan ensinnäkin kirjoittaa
\[
y(b) \,=\, [y(x_n)-y(x_{n-1})]+ \ldots +[y(x_1)-y(x_0)]+y(x_0)
\]
ja alkuehdon $y(x_0)=y(a)=0$ perusteella siis
\[
y(b)=\sum_{k=0}^n [y(x_k)-y(x_{k-1})].
\]
Funktio $y(x)$ toteuttaa Differentiaalilaskun väliarvolauseen ehdot osaväleillä $\quad$
$[x_{k-1},x_k]$ ja lisäksi $y'(x)=f(x),\ x\in(x_{k-1},x_k)$. Väliarvolauseen mukaan on tällöin
olemassa pisteet $\xi_k\in(x_{k-1},x_k),\ k=1 \ldots n$ siten, että
\[
y(x_k)-y(x_{k-1})=f(\xi_k)(x_k-x_{k-1}), \quad k=1\ldots n.
\]
Yhdistämällä kaksi viimeistä tulosta on päätelty, että (P):n ratkaisulle pätee
\[
y(b)=\sum_{k=1}^nf(\xi_k)(x_k-x_{k-1}).
\]
Tämä ei toimi suoraan laskukaavana, koska pisteitä $\xi_k$ ei tunneta, mutta kaava on
luonteva lähtökohta approksimatiolle: Korvataan tuntemattomat pisteet $\xi_k$ joillakin
\pain{valituilla} pisteillä $\xi_k\in[x_{k-1},x_k]$, esim.\ $\xi_k=x_{k-1}$ tai
$\xi_k=(x_{k-1}+x_k)/2$. Kun valinta on jokaisella osavälillä tehty, niin lasketaan
\[
Y_n = \sum_{k=1}^nf(\xi_k)(x_k-x_{k-1}).
\]
Näin muodostuu lukujono $\seq{Y_n}$, jonka $n$:s termi siis lasketaan seuraavasti:
\begin{itemize}
\item[A1.] Valitaan \kor{jakopisteet} $x_k,\ k=0 \ldots n$ niin, että
           $a=x_0 < x_1 < \ldots < x_n=b$. \index{jakopiste}%
\item[A2.] Valitaan \kor{välipisteet} $\xi_k\in[x_{k-1},x_k],\ k=1 \ldots n$.
           \index{vzy@välipiste(istö)}%
\item[A3.] Lasketaan $Y_n=\sum_{k=1}^n f(\xi_k)(x_k-x_{k-1})$.
\end{itemize}
Tämän mukaisesti $Y_n$:n laskemiseksi riittää suorittaa äärellinen määrä laskuoperaatioita:
$n$ kpl (käytännössä yleensä likimääräisiä) funktioevaluaatioita $x \map f(x)$ ja lisäksi
$n$ kertolaskua ja $n-1$ yhteenlaskua.

Verrattaessa algoritmilla A1--A3 laskettua lukua $Y_n$ lukuun $y(b)$ on em.\ tarkkaa
$y(b)$:n lauseketta verrattava $Y_n$:n vastaavaan lausekkeeseen (A3). Näissä välipisteet
$\xi_k$ eivät ole samat, joten kirjoitettakoon lausekkeessa (A3) $\xi_k$:n tilalle $\eta_k$.
Approksimaatiolle $\,Y_n \approx y(b)$ saadaan tällöin virhekaava
\[
y(b)-Y_n \,=\, \sum_{k=1}^n[f(\xi_k)-f(\eta_k)](x_k-x_{k-1}).
\]
Virheen arvioimiseksi on siis pystyttävä arvioimaan erotuksia $f(\xi_k)-f(\eta_k)$, kun
pisteistä $\xi_k$ ja $\eta_k$ tiedetään ainoastaan, että ne ovat samalla osavälillä
$[x_{k-1},x_k]$. Arvion onnistumiseksi toivotulla tavalla on ilmeisesti tehtävä jokin
säännöllisyys\-oletus funktiosta $f$. --- Yleisesti jos $f$ on riittävän säännöllinen, on
$\abs{f(\xi)-f(\eta)}$ enintään verrannollinen lukuun $\abs{\xi-\eta}$, ts.\
$f(\xi)-f(\eta)=\Ord{\abs{\xi-\eta}}$. Tämän takaava minimioletus on, että $f$ on välillä
$[a,b]$ Lipschitz-jatkuva, ts.\ $\exists\,L\in\R_+$ (= $f$:n Lipschitz-vakio) siten, että
\[
\abs{f(\xi)-f(\eta)} \le L \abs{\xi-\eta} \quad \forall\ \xi,\eta\in[a,b]
\]
(ks.\ Määritelmä \ref{funktion l-jatkuvuus} ja myös Lause \ref{Lipschitz-kriteeri}).

Kun oletetaan $f$:n Lipschitz-jatkuvuus, niin pätee
\[
\abs{f(\xi_k)-f(\eta_k)} \le L\abs{\xi_k-\eta_k}, \quad k=1 \ldots n.
\]
Tässä on $\,\abs{\xi_k-\eta_k} \le x_k-x_{k-1}$, koska $\xi_k,\eta_k\in[x_{k-1},x_k]$ ja
edelleen $x_k-x_{k-1} \le h_n$, joten $\,\abs{f(\xi_k)-f(\eta_k)} \le Lh_n,\ k=1 \ldots n$.
Kun käytetään näitä arvioita yhdessä kolmioepäyhtälön kanssa em.\ virhekaavassa, niin seuraa
\begin{align*}
\abs{y(b)-Y_n)} \,&\le\, \sum_{k=1}^n \abs{f(\xi_k)-f(\eta_k)}(x_k-x_{k-1}) \\
                  &\le\, Lh_n\sum_{k=1}^n (x_k-x_{k-1}) \,=\, L(b-a)h_n.
\end{align*}

On päädytty seuraavaan tulokseen.
\begin{Lause} \label{summakaavalause} Olkoon $f$ Lipschitz-jatkuva välillä $[a,b]$ vakiolla
$L$ ja olkoon $y(x)$ probleeman (P) ratkaisu. Tällöin jos $\seq{Y_n}$ on mikä tahansa
algoritmilla A1--A3 laskettu lukujono raja-arvoehdolla \eqref{int-4: tiheysehto}, niin
$\,\lim_nY_n=y(b)$, tarkemmin 
$\abs{y(b)-Y_n} \le L(b-a)h_n\ \forall n\in\N$.\footnote[2]{Kuten nähdään jäljempänä
Luvussa \ref{analyysin peruslause}, Lauseen \ref{summakaavalause} raja-arvoväittämä
(ilman tarkennusta) on tosi myös, jos $f$ oletetetaan ainoastaan jatkuvaksi välillä $[a,b]$.
Mitään kvantitatiivista virhearviota approksimaatiolle $Y_n \approx y(b)$ ei pelkän
jatkuvuusoletuksen perusteella saada.}
\end{Lause}

Edellä laskettiin toistaiseksi (P):n ratkaisu vain pisteessä $x=b$. Tulos on kuitenkin
helposti yleistettävissä. Nimittäin jos $y(x)$ on (P):n ratkaisu ja kiinnitetään $x\in(a,b]$,
niin funktio $y(t),\ t\in[a,x]$ on jatkuva välillä $[a,x]$ (koska on jatkuva välillä
$[a,b]$) ja ratkaisu alkuarvotehtävälle
\[
\begin{cases}
\,y'(t)=f(t),\,\ t\in(a,x), \\ \,y(a)=0.
\end{cases}
\]
Kyseessä on siis probleema (P), missä muuttujana on $x$:n sijasta $t$ ja $b$:n tilalla on $x$.
Näin muodoin kun tehdään samat vaihdokset algoritmissa A1--A3, niin saadaan lasketuksi $y(x)$
missä tahansa halutussa pisteessä $x\in(a,b]$.
\begin{Exa} Ratkaise alkuarvotehtävä $\,y'(x)=x,\ x>0,\ y(0)=0$ algoritmilla A1--A3.
\end{Exa}
\ratk  Kiinnitetään $x>0$ ja tarkastellaan väliä $[0,x]$. Asetetaan jakopisteet $t_k$
tasavälisesti, eli $t_k=kx/n,\ k=0 \ldots n$, jolloin laskukaavan (A3) mukaan on
\[
Y_n \,=\, \sum_{k=1}^n f(\xi_k)(t_k-t_{k-1}) 
    \,=\, \sum_{k=1}^n \xi_k\,\frac{x}{n}\,.
\]
Välipisteiden valinnassa kokeiltakoon vaihtoehtoja \ a) $\xi_k=t_k$, \ b) $\xi_k=t_{k-1}$ ja
\ c) $\xi_k=(t_{k-1}+t_k)/2$, \ $k=1 \ldots n$, jolloin saadaan:
\begin{align*}
\text{a)}\ \ Y_n &\,=\, \sum_{k=1}^{n}k\,\frac{x}{n}\cdot\frac{x}{n}
                  \,=\, \frac{x^2}{n^2}\sum_{k=1}^{n} k
                  \,=\, \frac{x^2}{n^2}\cdot\frac{1}{2}\,n(n+1)
                  \,=\, \frac{x^2}{2}+\frac{x^2}{2n}\,. \\
\text{b)}\ \ Y_n &\,=\, \sum_{k=1}^{n}(k-1)\,\frac{x}{n}\cdot\frac{x}{n}
                  \,=\, \frac{x^2}{n^2}\sum_{k=1}^{n} (k-1)
                  \,=\, \frac{x^2}{n^2}\cdot\frac{1}{2}\,n(n-1)
                  \,=\, \frac{x^2}{2}-\frac{x^2}{2n}\,. \\
\text{c)}\ \ Y_n &\,=\, \sum_{k=1}^{n}\frac{1}{2}\left[(k-1)\,\frac{x}{n}+k\,\frac{x}{n}\right]
                                                                            \cdot\frac{x}{n}
                  \,=\, \frac{x^2}{n^2}\sum_{k=1}^{n} \left(k-\frac{1}{2}\right)
                  \,=\, \frac{x²}{n^2}\cdot\frac{1}{2}\,n^2
                  \,=\, \frac{x^2}{2}\,.
\end{align*}
Havaitaan, että kaikissa tapauksissa on $\,\lim_nY_n=\tfrac{1}{2}x^2$. Lause
\ref{summakaavalause} soveltuu ($L=1$), joten $\,y(x)=\tfrac{1}{2}x^2,\ x>0$. \loppu

Esimerkissä pätee approksimaatiolle $Y_n \approx y(x)$ Lauseen \ref{summakaavalause} mukaan 
virhearvio $\,\abs{y(x)-Y_n} \le x^2/n\,$ ($L=1,\ b-a=x,\ h_n=x/n$). Havaitut virheet ovat
\ a) $y(x)-Y_n=-\tfrac{x^2}{2n}$, \ b) $y(x)-Y_n=\tfrac{x^2}{2n}$, \ $y(x)-Y_n=0$, joten
nämä ovat arvion kanssa sopusoinnussa.

Jos (P):n ratkaisu $y(x)$ halutaan laskea useammassa pisteessä, voidaan algoritmi A1--A3
käynnistää jokaista laskentapistettä varten erikseen, kuten esimerkissä. Numeerisesti
laskettaessa algoritmi kannattaa kuitenkin ottaa tehokkaampaan käyttöön huomioimalla, että
kaavan (A3) mukaan luku $Y_n$ tullaan laskeneeksi käytännössä palautuvasti muodossa
\[
Y_0=0, \quad Y_k=Y_{k-1}+f(\xi_k)(x_k-x_{k-1}), \quad k=1 \ldots n.
\]
Luvut $Y_k=\sum_{i=1}^k f(\xi_i)(x_i-x_{i-1}),\ k=1 \ldots n-1$ saadaan tällöin summauksen
välituloksina. Nämä ovat algoritmin A1--A3 mukaisia approksimaatioita luvuille $y(x_k)$, kun
jakopisteinä ovat $x_i,\ i=0 \ldots k$ välillä $[a,x_k]$. Samoin oletuksin kuin edellä myös
näiden virhe on $\Ord{h_n}$, tarkemmin $|y(x_k)-Y_k| \le L(x_k-a)h_n,\ k=1 \ldots n-1$.
Luvut $Y_k$ kannattaa siis kaikki huomioida, jolloin saadaan käsitys funktiosta $y(x)$ koko
tarkasteltavalla välillä ilman lisätyötä.
\jatko \begin{Exa} (jatko) Kuviossa on valittu $x=2,\ n=4$ ja yhdistetty esimerkin
valinnoilla lasketut pisteet $(x_k,Y_k)$ murtoviivaksi. Tämä esittää funktion $y(x)$
kuvaajaa likimäärin välillä $[0,2]$. Tulos on tarkin tapauksessa (c), mutta muissakin
tapauksissa tarkkuus kasvaa kiinteällä $x$, kun $n\kohti\infty$.
\end{Exa} 
\begin{figure}[H]
\setlength{\unitlength}{0.5cm}
\begin{center}
\begin{picture}(12,11)(1,2.5)
\put(1,0){\vector(1,0){12}} \put(13.5,-0.2){$x$}
\put(1,0){\vector(0,1){12.5}} \put(0.9,13){$y$}
\multiput(1,0)(2.5,0){5}{\line(0,-1){0.1}}
\put(1,5){\line(-1,0){0.1}} \put(1,10){\line(-1,0){0.1}}
\put(0.2,4.7){$1$} \put(0.2,9.7){$2$}
\put(0.8,-1){$0$} \put(5.8,-1){$1$} \put(10.8,-1){$2$}
\path(1,0)(3.5,0.625)(6,2.5)(8.5,5.625)(11,10) \put(11.5,9.9){(c)}
\path(1,0)(3.5,0.781)(6,3.125)(8.5,7.031)(11,12.5) \put(11.5,12.4){(a)}
\path(1,0)(3.5,0.469)(6,1.875)(8.5,4.219)(11,7.5) \put(11.5,7.4){(b)}
\end{picture}
\end{center}
\end{figure}


\subsection*{Määrätty integraali}
\index{mzyzyrzy@määrätty integraali|vahv}

Algoritmin A1--A3 mukaan lasketulla luvulla $y(b)=\lim_nY_n$, missä $y(x)$ on probleeman (P)
ratkaisu, on matematiikassa oma nimensä ja merkintänsä: Lukua kutsutaan funktion $f$
\kor{määrätyksi} \kor{integraaliksi yli välin} $[a,b]$, merkitään
\[
y(b)=\int_a^b f(x)\,dx
\]
ja luetaan 'integraali $a$:sta $b$:hen $f(x)\,dx$'. Sanotaan edelleen, että $a$ on integraalin
\kor{alaraja}, $b$ on \kor{yläraja} ja $[a,b]$ on \kor{integroimisväli}. Muuttuja $x$, eli
\kor{integroimismuuttuja}, on 'dummy' (kuten summausindeksi), ts.\ muuttuja voidaan vaihtaa
integraalin merkityksen muuttumatta.

Määrätty integraali on siis integroitavasta funktiosta $f$ ja integroimisvälistä riippuva
reaaliluku, joka on käytännösä laskettavissa lukujonon raja-arvona algoritmin (A1)--(A3)
mukaisesti. Kun $Y_n$:n laskukaavassa (A3) käytetään erotuksille $x_k-x_{k-1}$
lyhenysmerkintää $\Delta x_k$, niin määrätyn integraalin laskeminen edellä esitetyllä tavalla
voidaan tiivistää \kor{summakaavaksi} \index{summakaava (määr.\ integraalin)}
\begin{equation} \label{int-4: summakaava}
\boxed{\quad \int_a^b f(x)\,dx = \Lim_n \sum_{k=1}^n f(\xi_k)\Delta x_k. \quad}
\end{equation}
\index{raja-arvo!f@määrätyn integraalin}%
Tässä raja-arvomerkintään '$\Lim_n$' on sisällytetty ensinnäkin algoritmissa A1--A3 asetetut
rajoitukset koskien pisteiden $x_k$ ja $\xi_k$ valintaa, toiseksi ehto
\eqref{int-4: tiheysehto} ja kolmanneksi vaatimus, että \pain{kaikki} algoritmin (A1)--(A3)
mukaiset lukujonot $\seq{Y_n}$ suppenevat kohti \pain{samaa} raja-arvoa (= määrätyn integraalin
arvo). Edellä esitetyn perusteella viimeksi mainittu vaatimus toteutuu (ja summakaava on siis
pätevä) ainakin oletuksin, että probleema (P) on ratkeava (perusoletus toistaiseksi!) ja $f$ on
Lipschitz-jatkuva välillä $[a,b]$.

Em.\ oletuksilla probleeman (P) ratkaisulle johdettiin edellä myös yleisempi laskukaava, joka
voidaan nyt kirjoittaa määrättynä integraalina:
\begin{equation} \label{int-4: P:n ratkaisukaava}
y(x)=\int_a^x f(t)\,dt, \quad a < x \le b.
\end{equation}
Koska (P):n ratkaisu on (eräs) $f$:n integraalifunktio välillä $(a,b)$, niin kaava
\eqref{int-4: P:n ratkaisukaava} yhdessä summakaavan \eqref{int-4: summakaava} kanssa antaa
tähänastisesta integroimistekniikasta (myös derivoimissäännöistä!) riippumattoman menetelmän
integraalifunktion määräämiseksi.\footnote[2]{Integraalifunktion merkinnän '$\int f(x)\,dx$'
taustalla on määrätyn integraalin summakaava. Merkinnän otti käyttöön \hist{G.W. Leibniz}
ajatellen summakaavan 'rajankäyntejä' $\sum\hookrightarrow\int$ ja
$\Delta x_k\hookrightarrow dx$, kun $n\kohti\infty$. Leibnizin omien päiväkirjamerkintöjen
mukaan integraalimerkinnän tarkka keksimispäivä oli 29.10.1675. \index{Leibniz, G. W.|av}}

Toisaalta jos välillä $(a,b)$ tunnetaan $f$:n integraalifunktio $F$ 
(esim.\ alkeisfunktiona tai sarjana) ja $F$ on jatkuva välillä $[a,b]$
(välttämäntön lisäoletus!), niin luvun $\int_a^b f(x)\,dx\,$ laskeminen käy päinsä suoraan
$F$:n avulla. Nimittäin näillä oletuksilla probleeman (P) ratkaisu on $y(x)=F(x)-F(a)$,
jolloin
\[
\int_a^b f(x)\, dx \,=\, y(b) \,=\,F(b)-F(a).
\]
Tämä tunnettuun integraalifunktioon $F$ perustuva määrätyn integraalin laskukaava esitetään
yleensä \kor{sijoituskaavana} \index{sijoituskaava (määr.\ integraalin)}
\begin{equation} \label{int-4: sijoituskaava}
\boxed{\kehys\quad \int_a^b f(x)\,dx = \sijoitus{x=a}{x=b} F(t) 
                                     = \sijoitus{a}{b} F(x). \quad}
\end{equation}
Tässä oikea puoli luetaan 'sijoitus $a$:sta $b$:hen $F(x)$'. Kaava \eqref{int-4: sijoituskaava}
saadaan päteväksi myös tapauksessa $a>b$, kun \pain{sovitaan}, että määrätylle integraalille
on voimassa \kor{vaihtosääntö} \index{vaihtoszyzy@vaihtosääntö!a@määrätyn integraalin}
\begin{equation} \label{int-4: vaihtosääntö}
\boxed{\quad \int_{a}^b f(x)\, dx=-\int_{b}^a f(x)\, dx. \quad}
\end{equation}
Mukavuussyistä oletetaan tämä päteväksi myös kun $a=b$, jolloin tulee sovituksi, että
\[
\int_a^a f(x)\, dx=0.
\]

Huomautettakoon määrätyn integraalin käsitteestä, että toistaiseksi kyse ei ole muusta kuin
erikoisesta merkinnästä luvulle $y(b)$ siinä tapauksessa, että probleemalla (P) on ratkaisu
$y(x)$. Samaa ajatusta kertaavat myös sijoituskaava \eqref{int-4: sijoituskaava} ja
vaihtosääntö \eqref{int-4: vaihtosääntö}. Seuraavassa luvussa nähdään, että
määrätystä integraalista tulee (P):n ratkeavuudesta (ja yleisemminkin integraalifunktion
olemassaolosta) riippumaton käsite, kun määritelmäksi otetaan suoraan summakaava
\eqref{int-4: summakaava}, johon edellä päädyttiin numeerisena laskukaavana luvulle $y(b)$.
Myöhemmissä luvuissa määrätyn integraalin käsite saa edelleen lisää 'eloa' erilaisista
sovelluksista. Kuten tullaan näkemään, sovelluksissa määrättyyn integraaliin päädytään
suoraan summakaavan kautta. Tällöin sijoituskaava \eqref{int-4: sijoituskaava} näyttäytyy
oikotienä integraalin arvon 'tarkkaan' laskemiseen silloin, kun funktiolle $f$ on löydettävissä
(edellisten lukujen menetelmin) välillä $[a,b]$ jatkuva integraalifunktio $F$.  

Seuraavassa esitetään määrätyn integraalin kolme keskeistä ominaisuutta. Nämä pysyvät voimassa,
kun määrittelyn perustaksi jatkossa otetaan summakaava \eqref{int-4: summakaava}, mutta
toistaiseksi ominaisuudet perustellaan vedoten oletettuun probleeman (P) ratkeavuuteen. 


\subsection*{Additiivisuus. Lineaarisuus. Vertailuperiaate}

Jos probleema (P) on ratkeava, niin jokaisella $c\in(a,b)$ voidaan kirjoittaa
$y(b)=y(c)+[y(b)-y(c)]$ eli määrätyn integraalin avulla ilmaistuna
\index{additiivisuus!a@integraalin}%
\begin{equation} \label{int-4: additiivisuussääntö}
\boxed{\quad \int_a^b f(x)\,dx = \int_a^c f(x)\,dx + \int_c^b f(x)\,dx. \quad}
\end{equation}
Tämän säännön perusteella sanotaan, että määrätty integraali on \kor{additiivinen}
\kor{integroimisvälin suhteen}. Kun huomiodaan myös vaihtosääntö \eqref{int-4: vaihtosääntö},
niin todetaan, että additiivisuussääntö \eqref{int-4: additiivisuussääntö} on pätevä lukujen
$a,b,c$ suuruusjärjestyksestä riippumatta edellyttäen, että probleema (P) on ratkaistavissa,
kun $a$:n tilalla on $\min\{a,b,c\}$ ja $b$:n tilalla $\max\{a,b,c\}$.

Olkoon (P):n ratkaisu $F(x)$ ja lisäksi olkoon (P):llä ratkaisu $G(x)$, kun $f$:n tilalla on
funktio $g$. Tällöin jos $\alpha,\beta\in\R$, niin $y(x)=\alpha F(x)+\beta G(x)\,$ on
(P):n ratkaisu, kun $f$:n tilalla on $\alpha f +\beta g$. Erityisesti on siis
$y(b)=\alpha F(b)+\beta G(b)$, eli määrätylle integraalille pätee \kor{lineaarisuussääntö}
(vrt.\ määräämättömän integraalin vastaava sääntö \ref{integraalifunktio}:\,(I-1))
\index{lineaarisuus!b@integroinnin}%
\begin{equation} \label{int-4: lineaarisuussääntö}
\boxed{\quad \int_a^b [\alpha f(x)+\beta g(x)]\, dx 
        = \alpha\int_a^b f(x)\, dx + \beta\int_a^b g(x)\, dx, \quad \alpha,\beta\in\R. \quad}
\end{equation}
Jos mainittujen oletusten lisäksi oletetetaan, että että $f(x) \le g(x)\ \forall x\in[a,b]$,
niin $F'(x)-G'(x) = f(x)-g(x) \le 0,\ x\in(a,b)$, jolloin $F(x)-G(x)$ on vähenevä välillä
$[a,b]$ (Lause \ref{monotonisuuskriteeri}). Erityisesti on $F(b)-G(b) \le F(0)-G(0) = 0$ eli
$F(b) \le G(b)$. Näin ollen määrätyille integraaleille pätee \kor{vertailuperiaate}
\index{vertailuperiaate!a@integraalien}%
\begin{equation} \label{int-4: vertailuperiaate}
\boxed{\quad f(x)\leq g(x)\quad\forall x\in [a,b] 
             \ \impl \ \int_a^b f(x)\,dx \le \int_a^b g(x)\,dx. \quad}
\end{equation}
Koska $\pm f(x) \le \abs{f(x)}$ ja $\int_a^b [\pm f(x)]\,dx = \pm\int_a^b f(x)\,dx$
(lineaarisuussääntö!), niin vertailuperiaattetta soveltaen seuraa erityisesti
\begin{equation} \label{int-4: kolmioepäyhtälö}
\boxed{\quad \left|\int_a^b f(x)\, dx\right|\leq \int_a^b\abs{f(x)}\,dx. \quad}
\end{equation}
Tätä sanotaan 
\index{kolmioepäyhtälö!f@integraalien}%
\kor{integraalien kolmioepäyhtälöksi}.\footnote[2]{Epäyhtälö
\eqref{int-4: kolmioepäyhtälö} on sukua järjestetyn kunnan kolmioepäyhtälölle, joka pätee
summakaavan \eqref{int-4: summakaava} äärellisille summille.}
\begin{Exa} Sijoituskaavan \eqref{int-4: sijoituskaava} perusteella on
\begin{align*}
&\int_0^2 \sin x\, dx=\sijoitus{0}{2}(-\cos x)=\sijoitus{2}{0}\cos x=1-\cos 2, \\
&\int_x^{x^2} t\, dt =\sijoitus{x}{x^2}\frac{1}{2}t^2=\frac{1}{2}(x^4-x^2), \quad x\in\R.
\end{align*}
\end{Exa}
\begin{Exa} Laske $\int_0^2 f(x)\,dx$, kun $f(x)=\max\{\sqrt{x},x^2\}$. \end{Exa}
\ratk Todetaan ensin, että $f(x)=\sqrt{x}$ välillä $[0,1]$ ja $f(x)=x^2$ välillä $[1,2]$.
Tällöin käyttämällä ensin additiivisuussääntöä \eqref{int-4: additiivisuussääntö} ja sitten
sijoituskaavaa \eqref{int-4: sijoituskaava} saadaan
\[
\int_0^2 f(x)\,dx \,=\, \int_0^1 \sqrt{x}\,dx + \int_1^2 x^2\,dx
                  \,=\, \sijoitus{0}{1}\frac{2}{3}\,x^{3/2}+\sijoitus{1}{2}\frac{1}{3}\,x^3
                  \,=\, 3. \ \loppu
\]
\begin{Exa} Määritä funktion $f(x)=\min\{3x,4-x^2\}$ integraalifunktio $\R$:ssä käyttäen
hyväksi määrättyä integraalia. \end{Exa}
\ratk Funktio $\,y(x)=\int_0^x f(t)\,dt\,$ on kysytty integraalifunktio lisäehdolla $y(0)=0$.
Koska $f(x)=3x$, kun $x\in[-4,1]$, ja $f(x)=4-x^2$, kun $x \ge 1$ tai $x \le -4$, niin pätee
\begin{align*}
x\in[-4,1]       \qimpl y(x) &= \int_0^x 3t\,dt = \sijoitus{0}{x}\frac{3}{2}\,t^2 
                              = \frac{3}{2}\,x^2, \\[1mm]
x\in[1,\infty)   \qimpl y(x) &= y(1)+[y(x)-y(1)] \\[3mm]
                             &= \frac{3}{2}+\int_1^x (4-t^2)\,dt \\
                             &= \frac{3}{2}+\sijoitus{1}{x}\left(4t-\frac{1}{3}\,t^3\right)
                              = -\frac{1}{3}\,x^3 + 4x -\frac{13}{6}\,, \\[2mm]
x\in(-\infty,-4] \qimpl y(x) &= y(-4)+[y(x)-y(-4)] \\[3mm]
                             &= 24+\int_{-4}^x (4-t^2)\,dt \\
                             &= 24+\sijoitus{-4}{x}\left(4t-\frac{1}{3}\,t^3\right)
                              = -\frac{1}{3}\,x^3 + 4x +\frac{56}{3}\,.
\end{align*}
Tässä $y(1)=3/2$ ja $y(-4)=24$ saatiin ensimmäisestä lausekkeesta. Yleinen integraalifunktio
on $F(x)=y(x)+C,\ C\in\R$, joten
\[
\int f(x)\,dx\,= \begin{cases}
                  -\frac{1}{3}\,x^3 + 4x +\frac{56}{3}+C,  &\text{kun}\,\ x < -4, \\
                 \,\frac{3}{2}\,x^2+C,                     &\text{kun}\ -4 \le x \le 1, \\
                  -\frac{1}{3}\,x^3 + 4x -\frac{13}{6}+C,  &\text{kun}\,\ x>1.
                \end{cases} \quad\loppu
\]
\begin{Exa} Sievennä lauseke $\D \,\frac{d}{dx}\int_{\sqrt{x}}^{1/\sqrt{x}}e^{-t^2},\ x>0$.
\end{Exa}
\ratk Funktiolla $e^{-t^2}$ on $\R$:ssä sarjamuotoinen integraalifunktio $F(t)$, mutta
sievennyksessä riittää tieto, että $F(t)$ on olemassa:
\begin{align*}
\frac{d}{dx}\int_{\sqrt{x}}^{1/\sqrt{x}}e^{-t^2}\, dt 
           &\,=\,\frac{d}{dx} [F(\frac{1}{\sqrt{x}})-F(\sqrt{x})] \\
           &\,=\,-\frac{1}{2x\sqrt{x}}f(\frac{1}{\sqrt{x}})-\frac{1}{2\sqrt{x}}f(\sqrt{x}) \\
           &\,=\, -\frac{1}{2\sqrt{x}}\left(\frac{e^{-1/x}}{x}+e^{-x}\right). \loppu
\end{align*}
\begin{Exa} Arvioi virhe approksimaatiossa
$\ \D\int_{10}^{20}\frac{x^4}{x^5+1}\,dx \approx \ln 2$.
\end{Exa}
\ratk Integroimisvälillä on
\[
\frac{x^4}{x^5+1}=\frac{1}{x}-\frac{1}{x(x^5+1)}\,,
\]
joten lineaarisuussäännön \eqref{int-4: lineaarisuussääntö} ja sijoituskaavan
\eqref{int-4: sijoituskaava} nojalla on
\[
\int_{10}^{20}\frac{x^4}{x^5+1}\,dx \,=\, \int_{10}^{20}\frac{1}{x}\,dx-\delta
                                   \,=\, \sijoitus{10}{20}\ln x -\delta
                                   \,=\, \ln 2 - \delta,
\]
missä on edelleen vertailuperiaatteen \eqref{int-4: vertailuperiaate} ja sijoituskaavan nojalla
\[
0 \,\le\, \delta = \int_{10}^{20}\frac{1}{x(x^5+1)}\,dx
                 \,\le\, \int_{10}^{20}\frac{1}{x^6}\,dx
                 \,=\, \sijoitus{10}{20}-\frac{1}{5x^5}
                 \,<\, 2 \cdot 10^{-6}.
\]
Siis approksimaatio on ylälikiarvo, jonka virhe on alle $2 \cdot 10^{-6}$. \loppu

\Harj
\begin{enumerate}

\item
Ratkaise probleema (P), tai osoita ratkeamattomuus, kun $[a,b]=[0,1]$ ja
\begin{align*}
&\text{a)}\,\ f(x)=\begin{cases} 
                   \,\ln x, &\text{kun}\ x>0, \\ \,0, &\text{kun}\ x=0,
                   \end{cases} \qquad
 \text{b)}\,\ f(x)=\begin{cases} 
                   \,\frac{1}{x}, &\text{kun}\ x>0, \\ \,0, &\text{kun}\ x=0,
                   \end{cases} \\
&\text{c)}\,\ f(x)=\begin{cases} 
                   \,\frac{1}{\sqrt{x}}, &\text{kun}\ x>0, \\ \,1, &\text{kun}\ x=0,
                   \end{cases} \qquad\
 \text{d)}\,\ f(x)=\begin{cases} 
                   \,\frac{1}{\sqrt{|2x-1|}}, &\text{kun}\ x \neq \frac{1}{2}, \\ 
                   \,0,                       &\text{kun}\ x=\frac{1}{2}.
                   \end{cases}
\end{align*} 

\item
Ratkaise alkuarvotehtävä $\,y'(x)=f(x),\ x>0,\ y(0)=0$ käyttämällä algoritmia A1--A3 ja
tasavälisiä jakoja välillä $[0,x]$, kun \ a) $f(x)=x^2$, \newline
b) $f(x)=x^3$, \ c) $f(x)=e^{-x}$, \ d) $f(x)=2^x$. \newline
Lisätietoja: $\,\sum_{k=1}^n k^2 = \frac{1}{6}\,n(n+1)(2n+1),\,\
\sum_{k=1}^n k^3 = \frac {1}{4}\,n^2(n+1)^2$.

\item
Laske seuraavat raja-arvot tulkitsemalla ne määrätyiksi integraaleiksi.
\begin{align*}
&\text{a)}\ \ \lim_{n\kohti\infty} \frac{1}{n^4}\left[1^3+2^3+ \ldots +(4n-1)^3\right] \qquad
 \text{b)}\ \ \lim_{n\kohti\infty} \sum_{k=1}^n \frac{1}{n+k} \\
&\text{c)}\ \ \lim_{n\kohti\infty} \frac{\pi}{n} \sum_{k=1}^{n-1} \sin\frac{k\pi}{n} \qquad
 \text{d)}\ \ \lim_{n\kohti\infty} \frac{1}{n^2} \sum_{k=0}^{n-1} \sqrt{n^2-k^2} \\
&\text{e)}\ \ \lim_{n\kohti\infty} \frac{1}{n^2} \sum_{k=0}^{n} \sqrt{n^2+k^2} \qquad
 \text{f)}\ \ \lim_{n\kohti\infty} \sum_{k=0}^n \frac{n}{n^2+k^2}
\end{align*}

\item
Laske sääntöjen \eqref{int-4: sijoituskaava}--\eqref{int-4: lineaarisuussääntö} avulla tarkasti
(jos mahdollista) tai numeerisena likiarvona:
\begin{align*}
&\text{a)}\,\ \int_{-2}^2 (x^2+3)^2\,dx \qquad
 \text{b)}\,\ \int_4^9 \left(\sqrt{x}-\frac{1}{\sqrt{x}}\right)\,dx \qquad
 \text{c)}\,\ \int_{1}^{10} \frac{1}{x^3+x}\,dx \\
&\text{d)}\,\ \int_{-1}^1 2^x\,dx \qquad
 \text{e)}\,\ \int_0^4 \abs{\sin\theta}\,d\theta \qquad
 \text{f)}\,\ \int_0^\pi \max\{\cos x,\,\sin 2x\}\,dx \\
&\text{g)}\,\ \int_1^2|x^3+x^2-3|\,dx \qquad
 \text{h)}\,\ \int_0^\pi|x-\cos x|\,dx \qquad
 \text{i)}\,\ \int_0^4 \min\{4x,\,e^x\}\,dx
\end{align*}

\item
Laske seuraavien funktioiden derivaatat vapaan muuttujan suhteen.
\begin{align*}
&\text{a)}\ \ \int_\pi^x \sin^3 t\,dt \qquad
 \text{b)}\ \ \int_x^{2\pi}(\sin^2u-u^4 e^{-u})\,du \qquad
 \text{c)}\ \ \int_{x}^{5x} \frac{e^s}{s^2+1}\,ds \\
&\text{d)}\ \ \int_0^{3\sinh 2x} \sqrt{9+t^2}\,dt \qquad
 \text{e)}\ \ \int_{-\pi}^t \frac{\cos y}{1+y^2}\,dy \qquad
 \text{f)}\ \ \int_{\sin\theta}^{\cos\theta} \sqrt{1-x^2}\,dx
\end{align*}

\item
Määritä seuraavien funktioiden integraalifunktiot $\R$:ssä käyttäen hyväksi määrättyä
integraalia:
\begin{align*}
&\text{a)}\ \ f(x)=|x|-|x-2| \qquad
 \text{b)}\ \ f(x)=|x^2-7x+10| \\
&\text{c)}\,\ f(x)=\max\,\{\,x^2+2x+3,\,9-2x-x^2\}
\end{align*}

\item
Määritä seuraavien funktioiden pienimmät arvot ja piirrä funktioiden \newline kuvaajat.
\[
\text{a)}\ \ f(x)=\int_0^1 \abs{x-t}\,dt \qquad
\text{b)}\ \ f(x)=\int_0^\pi (x\cos t-t\cos x)^2\,dt
\]

\item
Todista:
\begin{align*}
&\text{a)}\ \ 1 \le \int_0^1 \frac{1+x^{20}}{1+x^{21}} \le \frac{22}{21} \qquad
 \text{b)}\ \ \frac{2}{\sqrt[4]{e}} \le \int_0^2 e^{x^2-x}\,dx \le 2e^2 \\
&\text{c)}\ \ \int_3^5 \frac{x}{e\ln x}\,dx > 2 \qquad
 \text{d)}\ \ 0 < \int_{50}^{100} \frac{x^3}{x^6+8x+9}\,dx < 1.5 \cdot 10^{-4} \\
&\text{e)}\ \ \int_{100}^{300} \frac{x^5}{x^6+x-1}\,dx 
                               = \ln 3-\delta, \quad 0<\delta<2 \cdot 10^{-11}
\end{align*}

\item (*)
Perustele likimääräinen laskukaava
\[
\int_1^2 \frac{e^{-x}}{x}\,dx \,\approx\, \ln 2+\sum_{k=1}^n \frac{(-1)^k}{k \cdot k!}(2^k-1),
                                          \quad n\in\N,\,\ n \gg 1
\]
ja arvioi tämän virhe, kun $n=10$.

\item (*) 
Olkoon $\D F(x)=\int_0^{2x-x^2} \cos\left(\frac{1}{1+t^2}\right)\,dt, \quad
           G(x)=\int_4^{x^2} e^{t^2}\,dt$. \vspace{2mm}\newline
a) Tutki, saavuttaako $F$ jollakin $x$ pienimmän tai suurimman arvonsa. \newline
b) Laske raja-arvo $\ \lim_{x \kohti 2} G(x)/(x^3-8)$.

\item (*)
Probleemassa (P) olkoon $[a,b]=[-1,1]$, $f(0)=0$ ja $f$:n määritelmä muualla kuin origossa
\[
\text{a)}\,\ f(x)=\frac{1}{x}\left(\sin\frac{1}{x}+x\cos\frac{1}{x}\right), \quad
\text{b)}\,\ f(x)=\frac{1}{x}\left(\sin\frac{1}{x^2}+x^2\cos\frac{1}{x^2}\right).
\]
Näytä, että probleema (P) \ a) ei ratkea, \ b) ratkeaa. 

\end{enumerate}