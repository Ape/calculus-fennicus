\chapter{Yhden muuttujan differentiaalilaskenta} 
\label{yhden muuttujan differentiaalilaskenta}
\chaptermark{Differentiaalilaskenta}

Kun derivaattaa k�ytet��n laskennassa v�lineen�, puhutaan \kor{differentiaalilaskennasta}
(kirjaimellisesti 'pienten erotusten laskennasta', engl.\ differential calculus). Derivaatan
toivat matematiikkaan toisistaan riippumatta englantilainen fyysikko-matemaatikko
\index{Newton, I.} \index{Leibniz, G. W.}%
\hist{Isaac Newton} (1642-1727) ja saksalainen filosofi-matemaatikko 
\hist{Gottfried Wilhelm Leibniz} (1646-1716) 1600-luvun lopulla. Leibniz on vaikuttanut
huomattavasti viel� nykyisinkin k�yt�ss� oleviin differentiaalilaskun
(my�s my�hemmin tarkasteltavan integraalilaskun) merkint�ihin. 

Koska derivaatta oli alunperinkin fysiikan motivoima k�site (etenkin Newtonin tutkimuksissa),
ei derivaatan soveltuvuudessa fysiikkaan ole ihmettelemist�. Pitk�lle 1800-luvulle 
differentiaalilaskennan ja sen pohjalta nousevien matematiikan alojen kehitys olikin 
voimakkaasti sidoksissa fysiikkaan, ja viel� nykyisinkin on yhteys s�ilynyt vahvana monissa 
matematiikan lajeissa.

T�ss� luvussa tarkastellaan yhden muuttujan differentiaalilaskennan soveltamista
k�yr�teoriassa, fysiikassa (mm.\ liikeopissa) ja funktioiden approksimoinnissa. Funktion
approksimoinnin keskeinen tulos on Luvussa \ref{taylorin lause} esitett�v� ja todistettava
\kor{Taylorin lause}. T�ss� on kyse linearisoivan approksimaation yleist�misest�
\kor{Taylorin polynomeihin} perustuvaksi yleisemm�ksi polynomiapproksimaatioksi. Suotuisissa
oloissa funktio voidaan esitt�� my�s tarkasti \kor{Taylorin sarjana}. Taylorin sarjat ovat
potenssisarjoja --- ja potenssisarjojen teorian kauniiksi lopuksi osoittautuukin, ett�
jokainen potenssisarja, jonka suppenemiss�de on positiivinen, on itse asiassa sarjan summana
m��ritellyn funktion Taylorin sarja. Viimeisess� osaluvussa esitell��n viel� numeerisissa
laskentamenetelmiss� yleisesti k�ytettyjen \kor{interpolaatiopolynomien} teoriaa ja
k�ytt�tapoja.