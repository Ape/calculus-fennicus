\chapter{Usean muuttujan integraalilaskenta}

Tämän luvun sisällön muodostavat erilaiset \pain{määrät}y\pain{n} \pain{inte}g\pain{raalin}
käsitteen laajennukset useampiin ulottuvuuksiin ja erilaisiin geometrisiin tilanteisiin.
Tarkasteltavia integraalien lajeja ovat kahden tai useamman muuttujan funktioihin 
liitettävät \kor{taso-} ja \kor{avasuusintegraalit}, \kor{viiva-} eli 
\kor{käyräintegraalit} ja \kor{pintaintegraalit}. Sovelluksina käsitellään mm.\ pinta-alojen,
tilavuuksien, kaarenpituuksien, \kor{paino\-pisteiden}, \kor{hitausmomenttien}, ym.\ laskemista.

Luvussa \ref{tasointegraalit} tarkastellaan aluksi tasointegraalien \kor{mittateoreettisia}
perusteita. Osoittautuu, että tasointegraalissa, samoin muissakin integraalien lajeissa, on
kyse integroimisesta jonkin \kor{mitan} suhteen. Tasointegraaliin liittyvä mitta on $\R^2$:n
osajoukkoihin liitettävä \kor{pinta-alamitta}, joka Luvussa \ref{tasointegraalit} määritellään
tarkemmin \kor{Jordanin} mittana. Paitsi tämän mitan ominaisuuksia ja suhdetta integraaliin,
Luvussa \ref{tasointegraalit} tarkastellaan myös tasointegraalin laskemista numeerisesti 
'suoraan määritelmästä'.

Usean muuttujan integraalilaskun keskeistä sisältöä on väistämättä integraalien klassinen
laskutekniikka, jonka perusta on \kor{Fubinin lause}
(Luku \ref{tasointegraalien laskutekniikka}). Lause palauttaa taso- ja avaruusintegraalit
peräkkäisiksi eli \kor{iteroiduiksi} 1-ulotteisiksi integraaleiksi, jotka voidaan suotuisissa
oloissa laskea suljetussa muodossa. Fubinin lauseen käyttöä tarkastellaan Luvuissa
\ref{tasointegraalien laskutekniikka}--\ref{avaruusintegraalit}. 

Integraalien klassisessa laskutekniikkassa auttaa usein myös \kor{muuttujan vaihto}, esimerkiksi
siirtyminen tason tai avaruuden käyräviivaisiin koordinaatistoihin. Muuttujan vaihdon
laskutekniikkaa tarkastellaan Luvussa \ref{muuttujan vaihto integraaleissa}.

Taso- ja avaruusintegraalien moninaisia sovelluksia käydään läpi Luvussa
\ref{pinta- ja tilavuusintegraalit}. Lopuksi Luvuissa 
\ref{viivaintegraalit}--\ref{pintaintegraalit} tarkastellaan taso- ja avaruuskäyriin sekä
$\R^3$:n kaareviin pintoihin liittyviä viiva- ja pintaintegraaleja sovelluksineen. 

