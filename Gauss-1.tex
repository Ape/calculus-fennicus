\section{Vektorikentät ja polkuintegraalit} \label{polkuintegraalit}
\alku
\index{polkuintegraali|vahv}

Luvussa \ref{viivaintegraalit} tarkasteltiin viivaintegraaleja, joissa tason tai avaruuden
käyrän yli integroidaan kaarenpituusmitan suhteen. Tässä luvussa tarkastelun kohteena on toinen
viivaintegraalien luokka, jolle käytetään jatkossa nimeä \kor{polkuintegraalit}
(engl.\ path integral, suom.\ myös \kor{tieintegraali}). Polkuintegraaleille on ominaista, että
integrointi käyrää pitkin tapahtuu tiettyyn \pain{suuntaan}, siksi nimitys \kor{polku}, jonka
voi tulkita suunnatuksi käyräksi. Suunnan ohella polkuintegraaleille on tyypillistä, että
integrointiin liittyvä mitta \pain{ei} ole kaarenpituusmitta vaan muu yksi\-ulotteinen mitta,
joka on tapauskohtainen. Polun suunta vaikuttaa polkuintegraaliin niin, että jos vain suunta
vaihtuu, eli polku pysyy muuten (käyränä) samana, niin polkuintegraalin arvo vaihtuu
vastaluvukseen. Polun parametrisoinnin kautta tämä vastaa määrätyn integraalin vaihtosääntöä,
ks.\ esimerkit jäljempänä.

Jatkossa rajoitutaan sellaisiin polkuintegraaleihin, jotka sovelluksissa liitetään yleensä
fysikaalisiin vektorikenttiin (kuten voima-, sähkö- ja magneettikenttiin). Vektorikenttiin
liittyvillä polkuintegraaleilla on jatkossa käyttöä myös Gaussin ja Stokesin lauseiden
yhteydessä.\footnote[2]{Matemaattisissa teksteissä erilaisten viivaintegraalien nimet eivät ole
täysin vakiintuneet. Esim.\ saatetaan puhua 'vektorikenttien viivantegraaleista', kun
tarkoitetaan polkuintegraaleja tämän tekstin merkityksessä.}

Erotukseksi käyrästä polku merkitään jatkossa symbolilla $p$ tai tarkemmin
\[
p: A \kohti B,
\]
jolloin merkintä kertoo sekä polun (käyrän) päätepisteet että polun suunnan. Luontevasti
polun suunnan määrittää parametrisointi: suunta on joko parametrin kasvusuunta tai vastakkainen
suunta. Jos polku on parametrisoitu välillä $t\in[a,b]$, niin polku voidaan merkitä tarkemmin
kuten parametrinen käyrä:   
\[ 
p:\ t \in [a,b]\ \map\ \vec r\,(t).
\]
Parametrisoinnin ei tarvitse olla 1--1, joten polku (kuten parametrinen käyrä) voi leikata
itsensä tai kiertyä itsensä päälle.
\begin{figure}[H]
\begin{center}
\import{kuvat/}{kuvaUint-34.pstex_t}
\end{center}
\end{figure}

Jatkossa tarkastelun kohteena ovat tason tai avaruuden polut muotoa $t\in[a,b]\map(x(t),y(t))$
tai $t\in[a,b]\map(x(t),y(t),z(t))$. Funktioiden $x(t)$, $y(t)$, $z(t)$ oletetaan olevan joko
jatkuvasti derivoituvia välillä  $[a,b]$ tai toteuttavan vastaavat, hieman heikommat
säännöllisyysehdot, vrt. Luku \ref{viivaintegraalit}. 

Kuten aiemmin, voidaan laskea \kor{polun pituus} integraalina
\[
\mu(p)=\int_a^b \abs{\vec r\,'(t)}\,dt.
\]
Tässä on kuitenkin kyse jo ennestään tutusta viivaintegraalista, jossa mitta on
kaarenpituusmitta eikä polun suunnalla ole väliä.

Polkuintegraaleja (jatkon kannalta myös merkittävimpiä) ovat
\begin{equation} \label{polkuintegraaleja}
\int_p f\,dx, \quad \int_p f\,dy, \quad \int_p f\,dz, \tag{$\star$}
\end{equation}
missä $f(x,y,z)$ (tasossa $f(x,y)$) on tunnettu funktio. Näihin integraaleihin liittyvä
mitta on tavallinen ($1$-ulotteinen) Jordan-mitta. Jos tunnetaan polun parametrisointi
välillä $t\in[a,b]$, niin esimerkiksi $\int_p f\,dx$ lasketaan yksinkertaisesti kirjoittamalla
$dx=x'(t)dt$ (kuten muuttujan vaihdossa). Jos vielä oletetaan, että $t=a$ vastaa polun
alkupistettä ja $t=b$ loppupistettä, niin saadaan laskukaava
\[
\int_p f\,dx = \int_a^b f(x(t),y(t),z(t)\,x'(t)\,dt.
\]
Parametrisoinnin vaihto vastaa tässä kaavassa (toista) muuttujan vaihtoa, joten parametrisointi
ei vaikuta integraalin arvoon.
\begin{Exa} Tason polku $p$ kulkee pisteestä $(0,0)$ pisteeseen $(1,1)$ pitkin käyrää $y=x^2$
ja polku $-p$ pitkin samaa käyrää pisteesät $(1,1)$ pisteeseen $(0,0)$. Laske
$\int_p xy\,dy$ ja $\int_{-p} xy\,dy$.
\end{Exa}
\ratk Valitaan molemmissa integraaleissa parametriksi $t=x$\,:
\begin{align*}
&\int_p xy\,dy     = \int_0^1 x \cdot x^2 \cdot 2x\,dx = \int_0^1 2x^4\,dx
                  = \underline{\underline{\frac{2}{5}}}\,, \\
&\int_{-p} xy\,dy  = \int_1^0 2x^4\,dx = -\int_0^1 2x^4\,dx 
                  = \underline{\underline{-\frac{2}{5}}}\,. \loppu
\end{align*}

Esimerkin jälkimmäisessä integraalissa käytettiin määrätyn integraalin vaihtosääntöä. --- Itse
asiassa kun vaihtosääntö huomioidaan, niin määrätty integraali on itsekin tulkittavissa
polkuintegraaliksi: $\int_a^b f(x)\,dx = \int_p f\,dx$, missä $p$ on $\R$:n polku, jonka
alkupiste on $a$ ja loppupiste $b$ (!).

\subsection*{Polkuintegraali $\int_p \vec F \cdot d\vec r$}
\index{polkuintegraali!a@työintegraali|vahv}
\index{tyzz@työintegraali|vahv}

Fysiikan sovelluksissa polkuintegraalit liittyvät usein vektorikenttiin. Tyypillinen esimerkki
on \pain{voimakentässä} $\vec F(x,y,z)$ liikkuva (pistemäinen) kappale, jonka liikerata
tunnetaan jollakin aikavälillä $t\in [a,b]$. Tällöin voimakentän kappaleeseen tekemä
\pain{t}y\pain{ö} lasketaan polkuintegraalina pitkin kappaleen kulkemaa liikerataa. Liikerata
tulkitaan siis poluksi $p:t \map \vec r\,(t)$, $t\in [a,b]$. Jos $\vec F=\text{vakio}$, niin
fysiikan lakien mukaan työ $=|\vec F| \cdot s$, missä $s=$ polulla kuljettu matka voiman
vaikutussuunnassa, eli
\[
W=\vec F\cdot[\vec r\,(b)-\vec r\,(a)]\quad (\vec F\text{ vakio}).
\]
Vakiovoimakentän tekemän työn kannalta polkua siis 'mittaa' vektori
\[
\vec\mu(p)=\vec r\,(b)-\vec r\,(a).
\]
Jos tämä tulkitaan polun (vektoriarvoiseksi) mitaksi, niin nähdään, että tämäkin mitta on
additiivinen: Jos $p_1$ ja $p_2$ ovat $p$:n osapolkuja vastaten parametrin arvoja väleillä
$[a,t_0]$ ja $[t_0,b]$, $t_0\in (a,b)$, niin $\vec \mu(p)=\vec \mu(p_1)+\vec\mu(p_2)$,
vrt.\ kuvio.
\begin{figure}[H]
\begin{center}
\import{kuvat/}{kuvaUint-35.pstex_t}
\end{center}
\end{figure}
Entä jos voimakenttä $\vec F$ ei ole vakio, mutta on jatkuva? Tällöin menetellään niinkuin
integraaleissa yleensä: Otetaan käyttöön välin $[a,b]$ jako $\{t_k, \ k=0,\ldots,n\}$,
$a=t_0<t_1<\ldots t_n=b$, jolloin polku $p$ jakautuu peräkkäisiksi osapoluiksi $\Delta p_k$,
$k=1\ldots n\,$ vastaten parametrin arvoja väleillä $[t_{k-1},t_k]$. Koska funktio
$t \map \vec r\,(t)$ on jatkuva, niin osapolkujen $\Delta p_k$ päätepisteet 
$\vec r\,(t_{k-1})=\vec r_{k-1}$ ja $\vec r\,(t_k)=\vec r_k$ tulevat yhä lähemmäksi toisiaan
jaon tihetessä. Tällöin voima $\vec F$ on osapolulla $\Delta p_k$ likimain vakio 
(koska $\vec F$ oli jatkuva), joten voiman tekemä työ tällä osapolulla on likimain
\[
\Delta W_k\approx\vec F(\vec r_{k-1})\cdot\vec\mu(\Delta p_k),\quad 
\vec\mu(\Delta p_k)=\vec r_k-\vec r_{k-1}\,.
\]
Kun jaon tiheysparametri $h=\max_k|\vec r_k-\vec r_{k-1}|\kohti 0$, saadaan voimakentän
tekemälle kokonaistyölle integraalilauseke
\[
W=\Lim_{h\kohti 0}\sum_{k=1}^n\Delta W_k=\int_p \vec F\cdot d\vec\mu.
\]
Tämä on siis tulkittava polkuintegraaliksi vektorimitan $\vec\mu$ suhteen (!). Integraali saa
hieman konkreettisen muodon, kun käytetään merkintää $d\vec\mu=d\vec r$, jolloin työn
integraalikaava siis on
\[
\boxed{\kehys\quad W=\int_p \vec F\cdot d\vec r \quad (\text{työintegraali}). \quad}
\]
Määritelmän mukaisesti $W$:lle saadaan likiarvoja summien avulla. Esimerkiksi
\[
W\approx\sum_{k=1}^n \vec F(\vec r_{k-1})\cdot (\vec r_k-\vec r_{k-1}).
\]
Näin laskettaessa polusta $p$ ei tarvitse tehdä voimakkaita säännöllisyysoletuksia. Esimerkiksi
työ $W$ voidaan laskea, vaikka $p$ ei olisi suoristuva (ts.\ pituusmitta ei määritelty). Tämä
johtuu siitä, että työintegraali mittaa vain siirtymää voiman vaikutussuunnassa, ei
kaarenpituutta.

Jos oletetaan parametrisointi $p: t\in[a,b] \map \vec r\,(t)$, niin työintegraalissa voidaan
kirjoitta $d\vec r=\dvr(t)dt$, jolloin saadaan laskukaava
\[
W=\int_a^b \vec F(\vec r(t))\cdot \dvr(t)\,dt.
\]
Jos taas voimakenttä esitetään koordinaattimuodossa
\[
\vec F(x,y,z)=F_1(x,y,z)\vec i + F_2(x,y,z)\vec j+F_3(x,y,z)\vec k,
\]
niin kirjoittamall $d\vec r=dx\,\vec i+dy\,\vec j+dz\,\vec k$ työintegraali purkautuu
polkuintegraalien \eqref{polkuintegraaleja} summaksi:
\[
W=\int_p (F_1\,dx+F_2\,dy+F_3\,dz).
\]
Riittävän säännöllisellä polulla työintegraalin voi ilmaista kolmannellakin tavalla, sillä
\[
d\vec r = \frac{\dvr(t)}{|\dvr(t)|}|\dvr(t)|dt = \vec t\,ds,
\]
missä $\vec t$ on polun suuntainen yksikkötangenttivektori. Tämän mukaan siis työintegraali
voidaan haluttaessa liittää myös kaarenpituusmittaan laskukaavalla
\[
W = \int_p \vec F\cdot\vec t\,ds.
\]
Kuten kaavan johdosta ilmenee, tässä on $\vec t\,ds=\dvr(t)\,dt$, joten riippuvuus
kaarenpituusmitasta on näennäinen.
\begin{Exa}
Määritä voimakentän
\[
\vec F=(x^2-y)\vec i-2xy\vec j
\]
tekemä työ kappaleen liikkuessa polulla
\[
p: \ x(t)=2\cos t, \ y(t)=\sin t, \ t\in [0,2\pi].
\]
\end{Exa}
\ratk $\quad dx=-2\sin t\,dt,\quad dy=\cos t\,dt, \quad t\in [0,2\pi]$
\begin{align*}
\impl \quad W &= \int_0^{2\pi} [\,(4\cos^2 t-\sin t)(-2\sin t)-(4\cos t\sin t)\cos t\,]\,dt \\
              &= \int_0^{2\pi} (-12\cos^2 t\sin t+2\sin^2 t)\,dt \\
              &= \sijoitus{0}{2\pi} (4\cos^3 t-\cos t\sin t+t) 
               = \underline{\underline{2\pi}}. \loppu
\end{align*}

\subsection*{Gradienttikenttä ja työintegraali}
\index{polkuintegraali!a@työintegraali|vahv}
\index{tyzz@työintegraali|vahv}
\index{gradienttikenttä|vahv}

Jos voimakenttä $\vec F$ on g\pain{radienttikenttä}, eli lausuttavissa skalaaripotentiaalin $u$
avulla muodossa
\[
\vec F=-\nabla u,
\]
niin työintegraali voidaan laskea hyvin yksinkertaisesti. Nimittäin tässä tapauksessa on
derivoinnin ketjusäännön (Luku \ref{osittaisderivaatat}) perusteella
\[
\vec F(\vec r(t))\cdot \dvr(t) = -\nabla u(\vec r(t))\cdot \dvr(t)
                               = -\frac{d}{dt} u(\vec r(t)),
\]
joten
\[
W = \int_a^b \vec F(\vec r(t))\cdot \dvr(t)\,dt 
  = -\sijoitus{a}{b} u(\vec r(t))
  = u(\vec r(a))-u(\vec r(b)).
\]
Siis gradienttikentän työintegraali määräytyy pelkästään polun päätepisteistä:
\[ 
\boxed{ \begin{aligned} \quad\ygehys 
                 &\text{Gradienttikentän tekemä työ} \\
                 &= \text{potentiaaliero polun alku- ja loppupisteiden välillä}. \quad\agehys
           \end{aligned} } 
\]

\jatko \begin{Exa} (jatko). Esimerkissä polun alku- ja loppupisteet ovat samat. Koska
$W\neq 0$, ei esimerkin kenttä $\vec F$ ole gradienttikenttä. Sen sijaan jos esimerkiksi
\[
\vec F=(x^2-y^2)\vec i-2xy\vec j,
\]
niin ilman enempää laskemista selviää, että $W=0$, sillä
\[
\vec F=-\nabla(-\frac{1}{3}x^3+xy^2). \loppu
\]
\end{Exa}

\subsection*{Vektoriarvoiset polkuintegraalit}
\index{polkuintegraali!b@vektoriarvoinen p.-integraali|vahv}

Fysiikan sovelluksissa (esimerkiksi sähkömagnetiikassa) esiintyy myös vektoriarvoisia 
polkuintegraaleja muotoa
\[
\int_p f\,d\vec r\quad\text{tai}\quad\int_p \vec F\times d\vec r.
\]
Nämä voidaan laskea parametrisoinnin avulla samalla periaatteella kuin työintegraalikin.
\jatko \begin{Exa} (jatko) Laske $\int_p x\,d\vec r\,$ ja $\int_p \vec r \times d\vec r\,$,
kun $p$ on esimerkin polku.
\end{Exa}
\ratk
\begin{align*}
\int_p x\,d\vec r\,
&= \int_0^{2\pi} x(t)[x'(t)\vec i+y'(t)\vec j\,]\,dt \\
&= \int_0^{2\pi} 2\cos t\,(-2\sin t\,\vec i+\cos t\,\vec j\,)\,dt \\
&= -\vec i\int_0^{2\pi} 4\cos t\sin t\,dt + \vec j\int_0^{2\pi} 2\cos^2 t\,dt \\
&= -\vec i\sijoitus{0}{2\pi} 2\sin^2 t 
   +\vec j\sijoitus{0}{2\pi}(t+\cos t\sin t)
 = \underline{\underline{2\pi\vec j}}, \\
\int_p \vec r\times d\vec r\, 
&= \int_0^{2\pi} [x(t)\vec i+y(t)\vec j\,]\times[x'(t)\vec i+y'(t)\vec j\,]\,dt \\
&= \int_0^{2\pi}(2\cos t\,\vec i+\sin t\,\vec j\,)\times
               (-2\sin t\,\vec i+\cos t\,\vec j\,)\,dt \\
&= \int_0^{2\pi} (2\cos^2 t+2\sin^2 t)\vec k\, dt
 = \vec k \int_0^{2\pi} 2\,dt
 =\underline{\underline{4\pi\vec k}}. \loppu
\end{align*}

\Harj
\begin{enumerate}

\item
Laske polkuintegraali $\int_p (9x^2y\,dx-11xy^2\,dy)$, kun polku $p$ kulkee origosta 
pisteeseen $(1,1)$ \ a) pitkin käyrää $x(t)=t^2,\ y(t)=t^3$, \ b) pitkin suoraa, \ 
c) pitkin käyrää $\vec r\,(t)=t\vec i+t^\alpha\vec j,\ \alpha>0$.

\item
Laske polkuintegraali $\int_p (xdy-ydx)$, kun polku $p$ kulkee pisteestä $(1,0)$ pitkin
logaritmista spiraalia $r=e^{-\varphi}$ origoon.

\item
Laske polkuintegraali $\int_p [(y-x)\,dx+xy\,dy]$, kun polku polku $p$ on määritelty 
seuraavasti:

a) Pisteestä $(1,0)$ pisteeseen $(-1,0)$ yksikköympyrää pitkin vastapäivään

b) Pisteestä $(1,0)$ pisteeseen $(-1,0)$ yksikköympyrää pitkin myötäpäivään

c) Murtoviiva $ABCD$, missä $A=(1,0)$, $B=(1,1)$, $C=(-1,1)$ ja $D=(-1,0)$

d) Pisteestä $(1,0)$ yksikköympyrää pitkin takaisin lähtöpisteeseen vastapäivään kiertäen

e) Origosta pitkin $x$-akselia pisteeseen $(\pi,0)$ ja takaisin origoon pitkin käyrää $y=\sin x$

f) Pisteestä $(a,0)$ takaisin lähtöpisteeseen kiertäen vastapäivään ellipsiä
   $x^2/a^2+y^2/b^2=1$ ($a,b>0$).

\item 
Laske seuraavat polkuintegraalit.

a) $\int_p \vec F \cdot d\vec r$, kun $\vec F(x,y,z)=\sqrt y\,\vec i+2x\vec j+3y\vec k$ ja
polku kulkee origosta pisteeseen $(3,9,27)$ pitkin käyrää
$\vec r\,(t)=t\vec i+t^2\vec j+t^3\vec k$.

b) $\int_p \vec F\,\cdot\,d\vec r$, kun $\vec F(x,y,z)=x^3\vec i+y^2\vec j+z\vec k$ ja polku
kulkee origosta pisteeseen $(1,1,2)$ pitkin käyrää $S:\,x=y,\ z=x^2+y^2$.

c) $\int_p \vec F \times d\vec r$, kun $\vec F(x,y,z)=xyz\vec i+y^2\vec k$ ja polku $p$ seuraa
tason $x=y$ ja pinnan $z=x^2$ leikkauskäyrää origosta pisteeseen $(2,2,4)$.

\item 
Polun $p$ alkupiste on $(-1,1,-1)$ ja loppupiste $(1,2,3)$. Laske näillä tiedoilla seuraavat
polkuintegraalit (työintegraalit) kirjoittamalla integraalit ensin muotoon 
$\int_p \nabla u \cdot d\vec r$. \vspace{1mm}\newline
a) \ $\int_p (yz\,dx + zx\,dy + xy\,dz) \qquad\qquad\ \ $
b) \ $\int_p (yz^2\,dx+xz^2\,dy+2xyz\,dz)$ \newline
c) \ $\int_p [e^x y\,dx+(e^x+z^2)\,dy+2yz\,dz] \quad\ $
d) \ $\int_p \sin\frac{\pi(x+y+z)}{6}\,(dx+dy+dz)$

\item
Määritellään $g(u,v)=\int_p (y\,dx+2x\,dy)$, missä $p$ kulkee origosta pisteeseen $(u,v)$
suoraa pitkin. Mikä on $g$:n maksimiarvo yksikköympyrällä?

\item (*)
Kappale, jonka massa $=m$, liikkuu aikavälillä $[t_1,t_2]$ pitkin polkua $p:\,t\map\vec r\,(t)$
pisteestä $P_1$ pisteeseen $P_2$. Liikkeen aikana kappaleeseen vaikuttaa voimakenttä $\vec F$.
Johda liikeyhtälöstä $m\vec r\,''=\vec F$ energiaperiaate
\[
\frac{1}{2}\,m(v_2^2-v_1^2)=\int_p \vec F \cdot d\vec r, \quad 
           \text{missä}\ v_i=\abs{\dvr(t_i)},\ i=1,2.
\]
\end{enumerate}