Aiemmin Luvussa \ref{pinta-ala ja kaarenpituus} johdettiin käyrän $y=f(x)$ ja $x$-akselin
rajaaman alueen pinta-alalle välillä $[a,b]$ laskukaava määrätyn integraalin avulla.
Varmistetaan nyt, että tämä kaava on edelleen pätevä.
\begin{Lause} Jos $f$ on määritelty, rajoitettu, ja ei-negatiivinen välillä $[a,b]$, niin
$\,A = \{(x,y) \in \R^2 \mid x \in [a,b]\ \ja\ 0 \le y \le f(x)\}$
on Jordan-mitallinen täsmälleen kun $f$ on Riemann-integroituva välillä $[a,b]$, jolloin
$\,\mu(A)=\int_a^b f(x)\,dx$.
\end{Lause}
\tod Olkoon $T = [a,b] \times [0,c]$, missä $c \ge f(x)\ \forall x \in [a,b]$, jolloin 
$T \supset A$, ja olkoon $\mathcal{T}_h$ jokin $T$:n jako suorakulmioihin 
$T_{kl} = [x_{k-1},x_k] \times [y_{l-1},y_l]$, $k = 1 \ldots m,\ l = 1 \ldots n$. Olkoon tässä 
edelleen jakopisteistö $X = \{x_k\}$ kiinnitetty, ja tutkitaan, miten pisteistön $Y = \{y_l\}$ 
valinta vaikuttaa ala- ja yläsummiin 
$\underline{\sigma}(\chi_A,\mathcal{T}_h),\ \overline{\sigma}(\chi_A,\mathcal{T}_h)$. Nähdään
helposti, että
\[ 
\sup_Y \underline{\sigma}(\chi_A,\mathcal{T}_h) = \underline{\sigma}(f,X), \quad 
\inf_Y \overline{\sigma}(\chi_A,\mathcal{T}_h) = \overline{\sigma}(f,X), 
\]
missä $\underline{\sigma}(f,X),\ \overline{\sigma}(f,X)$ ovat funktion $f$ ja pisteistön $X$
määräämään välin $[a,b]$ jakoon liittyvät Riemannin ala- ja yläsummat
(vrt.\ Luku \ref{riemannin integraali}). Tästä voidaan päätellä, että
\[ 
\underline{\mu}(A) = \underline{\int_a^b} f(x)\,dx, \quad 
\overline{\mu}(A) = \overline{\int_a^b} f(x)\,dx \qimpl \text{väite.} \loppu 
\]