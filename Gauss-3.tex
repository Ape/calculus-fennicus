\section{Gaussin lauseen sovelluksia} \label{Gaussin lauseen sovelluksia}
\alku

Fysiikassa Gaussin avaruuskaava (Lause \ref{Gaussin lause avaruudessa}) esitetään usein
muodossa
\[
\int_V \nabla\cdot\vec F\,dV=\int_{\partial V} \vec F\cdot d\vec a,\quad V\subset\R^3,
\]
missä on merkitty $dV=dxdydz$. Jos $A \subset S$, missä $S$ on avaruuden pinta, niin
pintaintegraali
\begin{equation} \label{vuokaava}
\phi=\int_A \vec F\cdot d\vec a = \int_A \vec F\cdot\vec n\,dS \tag{$\star$}
\end{equation}
\index{vuo, vuontiheys}%
on vektorikentän $\vec F$ \kor{vuo} (engl. flux) $A$:n läpi. Ajatellen tätä yhteyttä
sanotaan itse vektorikenttää fysikaalisissa sovelluksissa usein \pain{vuontihe}y\pain{deksi}.
Seuraavassa kahdessa sovellusesimerkissä johdetaan Gaussin kaavan avulla fysikaalista ilmiötä
kuvaava \pain{säil}y\pain{mislaki} osittaisdifferentiaaliyhtälön muodossa. Ensimmäinen esimerkki
on virtausmekaniikasta ja toinen lämpöopista.

\subsection*{Sovellusesimerkki: Massan säilymislaki virtauksessa}
\index{massan säilymislaki|vahv}
\index{szy@säilymislaki|vahv}
\index{zza@\sov!Massan säilymislaki virtauksessa|vahv}

Tarkastellaan virtaavaa nestettä tai kaasua, jonka nopeus hetkellä $t$ pisteessä $(x,y,z)$ on
$\vec v = \vec v(t,x,y,z)$ (vektorikenttä, yksikkö m/s) ja tiheys on $\rho=\rho(t,x,y,z)$
(skalaarikenttä, yksikkö kg/m$^3$). Olkoon $A \subset S$, missä $S$ on avaruuden säännöllinen
(tai ainakin paloittain säännöllinen) pinta. Tarkastellaan $A$:n pientä, lähes tasomaista palaa
$\Delta A$, jonka pinta-ala $=\Delta S$. Jos oletetaan, että $\rho$ ja $\vec v$ ovat
$\Delta A$:n ympäristössä lähes vakioita (jatkuvuusoletus!), niin voidaan päätellä, että
(lyhyellä) aikavälillä $[t,t+\Delta t]$ pinnanpalan $\Delta A$ läpi menevät ne hiukkaset, jotka
ovat hetkellä $t$ joukossa
\[
\Delta V 
= \{\,(x,y,z)\vastaa\vec r \mid \vec r=\vec r_0-\tau\vec v,\ r_0\in A,\ \tau\in[0,\Delta t]\,\}.
\]
\begin{figure}[H]
\setlength{\unitlength}{1cm}
\begin{center}
\begin{picture}(11,2)(0,1)
\thicklines
\put(2,1){\line(1,0){4}} \put(1,2){\line(1,0){4}} \put(2,1){\line(-1,1){1}} 
\put(6,1){\line(-1,1){1}}
\put(1,2){\line(2,1){1}} \put(6,1){\line(2,1){1}} \put(5,2){\line(2,1){1}} 
\put(7,1.5){\line(-1,1){1}}
\put(2,2.5){\line(1,0){4}} 
\thinlines
\put(6,1.7){\vector(2,1){1.8}} \put(6,1.7){\vector(1,0){3}}
\put(8.05,2.6){$\vec n$} \put(9.2,1.6){$\vec v$}
\put(3,1.5){\line(1,-1){1}} \put(6.4,1.4){\line(1,-1){0.6}}
\put(4.2,0.4){$\Delta V$} \put(7.2,0.7){$\Delta A$}
\end{picture}
\end{center}
\end{figure}
Olkoon $\vec n$ virtaussuuntaan osoittava $S$:n yksikkönormaalivektori, ts.\
$\vec n\cdot\vec v \ge 0$. Jos oletetaan $\vec v$ vakioksi ja samoin $\vec n$ vakioksi
$\Delta A$:ssa (eli pinta tasoksi), niin $\Delta V$ on näillä oletuksilla suuntaissärmiö, jonka
poikkipinta-ala virtausta vastaan kohtisuorassa tasossa on
$\Delta S\,\vec v\cdot\vec n/|\vec v\,|$. Jos edellen myös $\rho$
oletetaan vakioksi, niin $\Delta V$:ssä olevan nesteen/kaasun massa on tehdyin oletuksin
\begin{align*}
\Delta m &=\rho\mu(\Delta V)=\rho\,[\Delta S\,\vec v\cdot\vec n/|\vec v\,|]\,|\vec v\,|\Delta t
                           = \rho\vec v\cdot\vec n\,\Delta S\Delta t \\
         &\impl\quad \frac{\Delta m}{\Delta S\Delta t} = \rho\vec v\cdot\vec n.
\end{align*}
Voidaan olettaa, että tehdyt likmääräistykset ovat voimassa lähes kaikissa $A$:n pisteissä,
lukuun ottamatta mahdollista (pinta-alamitan suhteen) nollamittaista osajoukkoa. Muissa kuin
näissä poikkeuspisteissä voidaan mainittujen oletusten myös olettaa toteutuvan tarkasti
raja-arvoina, kun $\Delta t \kohti 0$ ja $\Delta A$ kutistuu pisteeksi. Näin olettaen voidaan
\pain{massavuo} pinnan $A$ läpi (yksikkö kg/s) laskea integraalikaavalla \eqref{vuokaava},
missä \pain{massavuon} \pain{tihe}y\pain{s} (yksikkö kg/m$^2$/s) määritellään
\[
\vec F = \rho\vec v.
\]
Oletetaan nyt, että $S$ on suljettu pinta, joka sulkee sisäänsä perusalueen $V\subset\R^3$ 
($S=\partial V$). Tällöin nesteen/kaasun kokonaismassa $V$:ssä hetkellä $t$ on
\[
m(t) = \int_V \rho(t,x,y,z)\,dV.
\]
Oletetaan, että $m(t)$ voi muuttua vain $V$:n reunapinnan läpi tapahtuvan virtauksen vuoksi
(eli muilla tavoilla massaa ei 'häviä' tai 'synny'). Tällöin on voimassa tasapainolaki
\[
m'(t) = -\int_{\partial V} \rho\vec v\cdot d\vec a,
\]
missä $d\vec a=\vec n dS$ osoittaa $V$:stä poispäin, eli $\vec n=V$:n ulkonormaali. Kun tässä
oikealla käyttetään Gaussin kaavaa ja vasemmalla kirjoitetaan (vrt.\ Luku 
\ref{osittaisderivaatat})
\[
m'(t) = \frac{d}{dt}\int_V \rho(t,x,y,z)\,dV 
      = \int_V \frac{\partial \rho}{\partial t}\,(t,x,y,z)\,dV,
\]
niin tasapainolaki saa muodon
\begin{equation} \label{säilymislaki a}
\int_V \bigl[\rho_t+\nabla\cdot(\rho\vec v)\bigr]\,dV=0. \tag{a}
\end{equation}
Olkoon nyt $t\in\R$ kiinteä ajanhetki, $(x,y,z)\in\R^3$ kiinteä piste, ja oletetaan, että $\rho$
ja $\vec v$ ovat jatkuvasti derivoituvia (osittaisderivaatat jatkuvia) pisteen
$(t,x,y,z)\in\R^4$ ympäristössä. Tällöin jos tasapainolaissa \eqref{säilymislaki a} valitaan
$V=B_\eps=\eps$-säteinen pallo (kuula), jonka keskipiste on $(x,y,z)$, niin
\begin{align*}
0\ &=\ \int_{B_\eps}\bigl[\rho_t+\nabla\cdot(\rho\vec v)\bigr]\,dV \\
   &=\ \bigl[\rho_t+\nabla\cdot(\rho\vec v)\bigr](t,x,y,z)\,\mu(B_\eps)
         + o(1)\mu(B_\eps), \quad \text{kun}\ \eps \kohti 0.
\end{align*}
Tämän mukaan on tarkasteltavassa pisteessä $(t,x,y,z)$ oltava voimassa
\begin{equation} \label{säilymislaki b}
\boxed{\kehys\quad \rho_t + \nabla\cdot(\rho\vec v)=0. \quad} \tag{b}
\end{equation}
Tämä tunnetaan virtausmekaniikan (differentiaalisena) \pain{massan}
\pain{säil}y\pain{mislakina}. --- Huomattakoon, että koska tätä johdettaessa tehdyt
jatkuvuusoletukset eivät ole fysiikan sanelemia, niin säilymislain integraalimuoto
\eqref{säilymislaki a} (joka pätee heikommin säännöllisyysehdoin jokaiselle perusalueelle $V$)
on luonnonlakina alkuperäisempi kuin differentiaalinen muoto \eqref{säilymislaki b}.
Vrt.\ vastaava asetelma Esimerkissä \ref{analyysin peruslause}:\,\ref{liikelaki}.


\subsection*{Sovellusesimerkki: Lämmön johtuminen}
\index{szy@säilymislaki|vahv}
\index{zza@\sov!Lzy@Lämmön johtuminen|vahv}

Tarkastellaan lämmön johtumisen ongelmaa, kun avaruuden $\R^3$ täyttää fysikaalisilta 
ominaisuuksiltaan tunnettu materiaali. Ongelmaan liittyvät perussuureet ovat
\pain{läm}p\pain{ötila} $u=u(t,x,y,z)$, \pain{läm}p\pain{övuon} \pain{tihe}y\pain{s} 
$\vec J=\vec J(t,x,y,z)$ (yksikkö W/m$^2$), ja \pain{läm}p\pain{ölähteen} \pain{tihe}y\pain{s} 
$\rho(t,x,y,z)$ (yksikkö W/m$^3$). Materiaalikertoimina oletetaan vielä tunnetuiksi materiaalin 
\pain{ominaisläm}p\pain{ö} $c=c(x,y,z,u)$ ja \pain{lämmön}j\pain{ohtavuus} 
$\lambda=\lambda(x,y,z,u)$. (Tässä oletetaan, että materiaalikertoimet voivat riippua 
paikkamuuttujien lisäksi lämpötilasta.)

Jos $V\subset\R^3$, niin $V$:n sisältämän \pain{läm}p\pain{öener}g\pain{ian} kokonaismäärä
hetkellä $t$ on (ominaislämmön määritelmä)
\[
E(t)=\int_V cu\,dV.
\]
Jos oletetaan, että $V$:n energiataseeseen vaikuttavat vain lämpölähde $V$:ssä ja lämmön
johtuminen $V$:n reunapinnan läpi, niin energian tasapainoyhtälö $V$:ssä on
\[
E'(t)= \int_V \rho\,dV - \int_{\partial V} \vec J\cdot d\vec a.
\]
Kun tässä kirjoitetaan
\[
E'(t)=\frac{d}{dt}\int_V cu\,dV = \int_V cu_t\,dV, \qquad
      \int_{\partial V} \vec J\cdot d\vec a = \int_V \nabla\cdot\vec J\,dV,
\]
niin tasapainoyhtälö saa muodon
\[
\int_V (cu_t+\nabla\cdot\vec J-\rho)\,dV = 0.
\]
Samanlaisella päättelyllä (ja oletuksilla) kuin edellisessä esimerkissä seuraa tästä 
differentiaalinen \pain{ener}g\pain{ian} \pain{säil}y\pain{mislaki}
\[
cu_t+\nabla\cdot\vec J = \rho.
\]
Jos vielä oletetaan lämmönjohtumisen \pain{Fourier'n} \pain{laki}
\index{Fourierb@Fourier'n laki}%
\[
\vec J = -\lambda \nabla u,
\]
\index{lzy@lämmönjohtumisyhtälö}%
niin energian säilymislaki saa \pain{lämmön}j\pain{ohtumis}y\pain{htälönä} tunnetun muodon
\[
\boxed{\kehys\quad cu_t-\nabla\cdot(\lambda\nabla u)=\rho. \quad}
\]
Jos $\rho=0$ ja $c$ ja $\lambda$ ovat vakioita (homogeeninen materiaali, $c$:llä ja
$\lambda$:lla ei lämpötilariippuvuutta), niin lämmönjohtumisyhtälö yksinkertaistuu muotoon
\[
u_t=k\Delta u, \quad k=\lambda/c.
\]
 
\subsection*{*Vektorikenttien epäjatkuvuudet}
\index{vektorikenttä!d@epäjatkuva|vahv}
\index{epzy@epäjatkuvuus (vektorikentän)|vahv}

Olkoon vektorikenttä $\vec F$ määritelty $\R^3$:ssa ja jatkuvasti derivoituva. 
Tällöin kentän lähde $\rho=\nabla\cdot\vec F$ on $\R^3$:ssa jatkuva, ja Gaussin lauseen mukaan
pätee
\begin{equation} \label{kenttä ja lähde}
\int_V \rho\,dV = \int_{\partial V}\vec F\cdot d\vec a \quad (V\subset\R^3\,\ \text{perusalue}).
\end{equation}
Toisaalta, jos mainittujen säännöllisyysoletusten lisäksi oletetaan ainoastaan
\eqref{kenttä ja lähde}, niin aiemman päättelyn mukaisesti seuraa, että on oltava
$\nabla\cdot\vec F=\rho$. Siis jos $\vec F$ on $\R^3$:ssa jatkuvasti derivoituva ja $\rho$
jatkuva, niin pätee
\[
\text{Ehto \eqref{kenttä ja lähde} voimassa} \qekv \nabla\cdot\vec F = \rho\ \ \R^3:\text{ssa}.
\]

Entä jos $\vec F$ ei ole jatkuvasti derivoituva, tai $\rho$ ei ole jatkuva, mutta $\vec F$:n ja
$\rho$:n välillä on ehdon \eqref{kenttä ja lähde} mukainen yhteys? --- Tällöin on luontevaa
sopia, että ehto \eqref{kenttä ja lähde} \pain{määrittelee} $\rho$:n kentän $\vec F$ lähteeksi.
Fysikaalisissa sovelluksissa ajatellaan tällöin, että ehto \eqref{kenttä ja lähde} itse asiassa
ilmaisee luonnonlain sen alkuperäisessä muodossa (vrt.\ massan säilymislain johto edellä).

Vektorikentän ja lähteen välisen riippuvuuden esittäminen integraalimuotoisena 'Gaussin lakina'
\eqref{kenttä ja lähde} on erityisen hyödyllistä silloin, kun halutaan johtaa fysikaalisten
vektorikenttien j\pain{atkuvuusehto}j\pain{a} \pain{materiaalira}j\pain{a}p\pain{innoilla}.
Tarkastellaan esimerkkinä tällaisten ehtojen asettamista tasolla (tasomaisella 
materiaalirajapinnalla). Olkoon $T$ avaruustaso, joka jakaa $\R^3$:n kahteen avoimeen osaan 
$V_1$ ja $V_2$ (eli $\R^3$ jakautuu pistevieraisiin osiin $V_1$, $V_2$ ja $T$), ja olkoon
$\vec F$ paloittain jatkuvasti derivoituva vektorikenttä muotoa
\[
\vec F(x,y,z) = \begin{cases} \,\vec F_1(x,y,z), \quad \text{kun}\ (x,y,z) \in V_1, \\
                              \,\vec F_2(x,y,z), \quad \text{kun}\ (x,y,z) \in V_2,
                \end{cases}
\]
missä $\vec F_1$ ja $\vec F_2$ ovat molemmat koko $\R^3$:ssa määriteltyjä ja jatkuvasti
derivoituvia. Olkoon vastaavasti $\,\rho\,$ paloittain jatkuva funktio muotoa
\[
\rho(x,y,z) = \begin{cases} \,\rho_1(x,y,z), \quad \text{kun}\ (x,y,z) \in V_1, \\
                            \,\rho_2(x,y,z), \quad \text{kun}\ (x,y,z) \in V_2,
                \end{cases}
\]
missä $\,\rho_1$ ja $\,\rho_2$ ovat molemmat koko $\R^3$:ssa määriteltyjä ja jatkuvia.
\begin{figure}[H]
\setlength{\unitlength}{1cm}
\begin{center}
\begin{picture}(11,4)(0,0.5)
\thicklines
\put(2.2,4){$T$} \put(3,1){$V_1$} \put(6,1){$V_2$} \put(1,2){$\vec F_1,\ \ \rho_1$}
\put(4.5,3.5){$\vec F_2,\ \ \rho_2$}
\put(2,4){\line(1,-1){4}} \put(4.5,1.5){\vector(1,1){1}} \put(5.5,2){$\vec n$} 
\end{picture}
\end{center}
\end{figure}

Em.\ säännöllisyysoletuksien lisäksi oletetaan vielä, että kenttää $\,\vec F\,$ ja funktiota
$\,\rho\,$ sitoo ehto \eqref{kenttä ja lähde}, ts.\ $\,\rho\,$ on kentän $\,\vec F\,$ lähde.
Tällöin jos ko.\ ehdossa valitaan $V \subset V_1$ tai $V \subset V_2$ niin seuraa, että on 
oltava $\nabla\cdot\vec F_i=\rho_i$ $\,V_i$:ssä, $i=1,2$, eli pätee
\begin{equation} \label{lähdey-1}
 \nabla\cdot\vec F(x,y,z) = \rho(x,y,z), \quad (x,y,z) \in V_1 \cup V_2.
\end{equation}
Tutkitaan seuraavaksi, mitä ehdosta \eqref{kenttä ja lähde} seuraa, kun joukko $V$ valitaan
siten, että $V \cap\,T \neq \emptyset$. Tätä silmällä pitäen tarkastellaan pistettä 
$(x,y,z) \in T$ ja määritellään tätä pistettä ympäröivä 'pillerirasia' $B_{h,d}$ ehdoilla:
(i) $B_{h,d}$ on suorakulmainen särmiö, jonka keskipiste $\,=(x,y,z)$ ja kaksi sivutahkoa ovat
tason $T$ suuntaiset. (ii) $B_{h,d}$:n tasoa $T$ vastaan kohtisuoran särmän pituus $\,=d$. 
(iii) Suorakulmio $T_h = B_{h,d} \cap T$ on $d$:stä riippumaton ja $T_h$:n suurimman sivun
pituus $=h$. 
\begin{figure}[H]
\setlength{\unitlength}{1cm}
\begin{center}
\begin{picture}(11,4)(0,1)
\thicklines
\put(6,1){\line(-4,3){5}} \put(6,1){\line(4,1){4}}
\put(4.5,1.625){\line(0,1){1}} \put(4.5,1.625){\line(-4,3){1.5}} \put(3,2.75){\line(0,1){1}}
\put(4.5,2.625){\line(4,1){2}} \put(4.5,2.625){\line(-4,3){1.5}} \put(3,3.75){\line(4,1){2}}
\put(6.5,3.125){\line(-4,3){1.5}}
\put(4.5,2.125){\line(4,1){2}} \put(6.5,2.625){\line(0,1){0.5}} \put(4.5,1.625){\line(4,1){0.5}}
\put(4.75,3.2){\vector(0,1){1.8}} \put(4.9,5){$\vec n$} \put(9,2){$T$}
\end{picture}
\end{center}
\end{figure}
Valitaan ehdossa \eqref{kenttä ja lähde} $V=B_{h,d}$, missä $T_h = B_{h,d} \cap T$ on kiinteä ja
$d \kohti 0$. Tällöin seuraa
\[
\lim_{d \kohti 0} \int_{\partial B_{h,d}} \vec F\cdot d\vec a
        = \int_{T_h} \vec n\cdot(\vec F_2-\vec F_1)\,d\mu 
        = \lim_{d \kohti 0} \int_{B_{h,d}} \rho\,dV = 0,
\]
missä $\mu$ on tason $T$ pinta-alamitta ja $\vec n$ on $T$:n yksikkönormaalivektori, joka
osoittaa osajoukon $V_2$ suuntaan (ks.\ kuviot). Kun tässä tuloksessa annetaan edelleen 
$h$:n lähestyä $0$:aa ja huomioidaan kenttien $\vec F_1$ ja $\vec F_2$ jatkuvuus pisteessä
$(x,y,z)$, niin seuraa, että on oltava $\vec n\cdot(\vec F_1-\vec F_2)(x,y,z)=0$. Tämä on
voimassa jokaisessa $T$:n pisteessä, joten on päätelty, että tasolla $T$ on voimassa
jatkuvuusehto 
\begin{equation} \label{lähdey-2}
\vec n\cdot\vec F_1(x,y,z) = \vec n\cdot\vec F_2(x,y,z), \quad (x,y,z) \in T.
\end{equation}
Sekä tämä että \eqref{lähdey-1} ovat siis ehdon \eqref{kenttä ja lähde} seurauksia 
(tehtyjen säännöllisyysoletusten puitteissa).

Toisaalta jos lähdetään samoista säännöllisyysoletuksista ja oletetaan lisäksi \eqref{lähdey-1}
ja \eqref{lähdey-2}, niin näistä yhdessä seuraa \eqref{kenttä ja lähde}. Nimittäin jos 
$V$ on perusalue ja $A_i = V \cap V_i \neq \emptyset$, $i=1,2$, niin oletuksen \eqref{lähdey-1}
ja Gaussin lauseen perusteella voidaan päätellä
\begin{align*}
\text{\eqref{lähdey-1}} &\qimpl \sum_{i=1}^2 \int_{A_i} \nabla\cdot\vec F_i\,dV 
                                     = \sum_{i=1}^2 \int_{A_i} \rho_i\,dV = \int_V \rho\,dV \\
                        &\qimpl \sum_{i=1}^2 \int_{\partial A_i} \vec F_i\cdot d\vec a 
                                     = \int_V \rho\,dV.
\end{align*}
Koska tässä on
\[ 
\partial A_1 \cup \partial A_2 = \partial V \cup (\partial A_1 \cap \partial A_2), \quad
\partial A_1 \cap \partial A_2 \subset T, 
\]
niin oletuksen \eqref{lähdey-2} perusteella päätellään edelleen
\[
\sum_{i=1}^2 \int_{\partial A_i} \vec F_i\cdot d\vec a 
     = \int_{\partial V} \vec F\cdot d\vec a 
           + \int_{\partial A_1 \cap \partial A_2} \vec n\cdot(\vec F_1-\vec F_2)\,d\mu
     = \int_{\partial V} \vec F\cdot d\vec a.
\]
Siis ehto \eqref{kenttä ja lähde} on voimassa. Näin on päätelty, että tehtyjen
säännöllisyysoletusten puitteissa pätee
\[ \boxed{
\quad \text{Ehto \eqref{kenttä ja lähde}} \qekv 
      \begin{cases} 
      \ \nabla\cdot\vec F = \rho \quad          &\text{joukossa}\ V_1 \cup V_2, \quad\ykehys \\
      \ \vec n\cdot(\vec F_1-\vec F_2)= 0 \quad &\text{rajapinnalla}\ T. \quad\akehys
      \end{cases} }
\]
Johtopäätös on sama myös, jos rajapinta $\partial V_1 \cap \partial V_2$ on osa kaarevaa, 
riittävän säännöllistä pintaa $S$. Koskien fysikaalisen vektorikentän jatkuvuusehtoja
materiaalirajapinnalla on siis päätelty, että jos vektorikentällä on materiaalirajapinnan
läheisyydessä paloittain jatkuva lähde, niin \pain{kentän} \pain{normaalikom}p\pain{onentti}
\pain{on} j\pain{atkuva} materiaalirajapinnalla. Sen sijaan kentän tangentiaalikomponentin 
\pain{ei} tarvitse olla jatkuva.

Samaan tapaan kuin ehdossa \eqref{kenttä ja lähde} voidaan myös vektorikentän ja sen 
pyörrekentän välinen yhteys esittää integraalimuotoisena määritelmänä. Jos $\vec F$ on 
jatkuvasti derivoituva $\R^3$:ssa, niin $\vec F$:n pyörrekenttä on 
$\vec\omega = \nabla\times\vec F$ (vrt.\ Luku \ref{divergenssi ja roottori}). Tällöin pätee
yleistetyn Gaussin lauseen perusteella
\begin{equation} \label{kenttä ja pyörre}
\int_V \vec\omega\,dV = \int_{\partial V} d\vec a\times\vec F \quad
                        (V\subset\R^3\,\ \text{perusalue}).
\end{equation}
Jos $\vec F$ ei ole koko $\R^3$:ssa jatkuvasti derivoituva, tai $\vec\omega$ ei ole jatkuva,
niin katsotaan ehto \eqref{kenttä ja pyörre} pyörrekentän $\vec\omega$ määritelmäksi. Jos nyt
$\vec F$  toteuttaa samat ehdot kuin edellä ja oletetaan, että määritelmän
\eqref{kenttä ja pyörre} mukainen pyörrekenttä on muotoa 
\[
\vec\omega(x,y,z) = \begin{cases} \,\vec\omega_1(x,y,z), \quad \text{kun}\ (x,y,z) \in V_1, \\
                                  \,\vec\omega_2(x,y,z), \quad \text{kun}\ (x,y,z) \in V_2,
                    \end{cases}
\]
missä $\,\vec\omega_1$ ja $\,\vec\omega_2$ ovat molemmat koko $\R^3$:ssa määriteltyjä ja 
jatkuvia, niin samalla tavoin kuin edellä päätellään, että pätee
\[ \boxed{
\quad \text{Ehto \eqref{kenttä ja pyörre}} \qekv 
\begin{cases} 
\ \nabla\times\vec F = \vec\omega \quad         &\text{joukossa}\ V_1 \cup V_2, \quad\ykehys \\
\ \vec n\times(\vec F_1-\vec F_2)= \vec 0 \quad &\text{rajapinnalla}\ T. \quad\akehys
\end{cases} }
\]
Tässä tapauksessa \pain{kentän} \pain{tan}g\pain{entiaalikom}p\pain{onentti} \pain{on} 
j\pain{atkuva} materiaalirajapinnalla, sillä tangentiaalikomponentti on (vrt.\ 
Luku \ref{ristitulo})
\[ 
\vec F_t = \vec F - (\vec n\cdot\vec F)\,\vec n = - \vec n\times(\vec n\times\vec F). 
\]
Kentän normaalikomponentin ei tarvitse olla jatkuva.

\begin{Exa} Staattinen sähkökenttä $\vec E$ ja sähkövuon tiheys $\vec D$ toteuttavat Maxwellin
yhtälöt (vrt.\ Luku \ref{divergenssi ja roottori})
\[ 
\nabla\times\vec E = \vec 0, \quad \nabla\cdot\vec D = \rho, 
\]
missä $\,\rho\,$ on varaustiheys. Oletetaan, että edellä $V_1$ ja $V_2$ edustavat kahta eri 
materiaalia, joissa vallitsevat materiaalilait ovat
\[ 
\vec D = \epsilon_i\vec E \quad V_i\,\text{:ssä}, \quad i=1,2, 
\]
missä $\epsilon_i$ (= materiaalin $i$ sähköinen permittiivisyys) oletetaan $V_i$:ssa vakioksi.
Jos oletetaan $\rho$ paloittain jatkuvaksi kuten edellä, niin jatkuvuusehdot 
materiaalirajapinnalla ovat
\[
\begin{cases} 
\ \vec n\times(\vec E_1-\vec E_2) = \vec 0, \\ \ \vec n\cdot(\vec D_1-\vec D_2) = 0 
\end{cases}
\qekv \begin{cases} 
      \ \vec n\times\vec E_1 = \vec n\times\vec E_2, \\ 
      \ \epsilon_1\,\vec n\cdot\vec E_1 = \epsilon_2\,\vec n\cdot\vec E_2.
      \end{cases}
\]
Sähkökentän tangentiaalikomponentti on siis materiaalirajapinnalla jatkuva. Normaalikomponentti
sen sijaan on epäjatkuva, ellei ole joko $\epsilon_1 = \epsilon_2$ tai $\vec n\cdot\vec E_i=0$.
\loppu
\end{Exa}
\jatko \begin{Exa} (jatko) Esimerkissä olkoon 
\[ 
V_1 = \{(x,y,z) \in \R^3 \mid x+y+z<0\}, \quad V_2 = \{(x,y,z) \in \R^3 \mid x+y+z>0\}, 
\]
ja $\,\epsilon_1/\epsilon_2=2$. Laske sähkökenttä $\vec E_2(0,0,0)$, kun tiedetään, että 
$\vec E_1(0,0,0)$ $=$ $E(\vec i+2\vec j-\vec k)$.
\end{Exa}
\ratk Materiaalirajapinta on taso $T:\ x+y+z=0$. Tämän materiaalia $2$ kohti osoittava 
yksikkönormaalivektori on $\vec n = (\vec i+\vec j+\vec k)/\sqrt{3}$, joten
\[
\vec n\cdot\vec E_1(0,0,0) = \frac{2E}{\sqrt{3}}, \quad \vec n\times\vec E_1(0,0,0) 
                           = \frac{E}{\sqrt{3}}(-3\vec i+2\vec j+\vec k).
\]
Jatkuvuusehtojen perusteella on
\[ 
\vec n\cdot\vec E_2(0,0,0) = \frac{\epsilon_1}{\epsilon_2}\,\vec n\cdot\vec E_1(0,0,0), \quad
\vec n\times\vec E_2(0,0,0)=\vec n\times\vec E_1(0,0,0), 
\]
joten
\begin{align*}
\vec E_2(0,0,0) &=[\vec n\cdot\vec E_2(0,0,0)]\,\vec n
                            -\vec n\times[\vec n\times\vec E_2(0,0,0)] \\[3mm]
                &= 2\,[\vec n\cdot\vec E_1(0,0,0)]\,\vec n 
                            - \vec n\times[\vec n\times\vec E_1(0,0,0)] \\[2mm]
                &= \frac{4E}{3}(\vec i+\vec j+\vec k) + \frac{E}{3}\,(\vec i+4\vec j-5\vec k) \\
                &= \underline{\underline{\frac{E}{3}\,(5\vec i + 8\vec j - \vec k\,)}}. \loppu
\end{align*}

\pagebreak
\Harj
\begin{enumerate}

\item
Laske vektorikentän $\vec F$ vuo origokeskisen, $R$-säteisen pallopinnan
läpi: \vspace{1mm}\newline
a) \ $\vec F=x^2y^4z\vec k \qquad $
b) \ $\vec F=x\vec i-2y\vec j+4z\vec k \qquad$
c) \ $\vec F=ye^z\vec i+x^2e^z\vec j+xy\vec k$ \newline
d) \ $\vec F=(x^2+y^2)\vec i+(y^2-z^2)\vec j+z\vec k \qquad$
e) \ $\vec F=x^2\vec i+3yz^2\vec j+(3y^2z+x^2)\vec k$

\item
Laske vektorikentän $\vec F=x^2\vec i+y^2\vec j+z^2\vec k$ vuo annetun alueen $V\subset\R^3$
reunapinnan $\partial V$ läpi: \vspace{1mm}\newline
a) \ $V:\ (x-2)^2+y^2+(z-3)^2 \le 9 \quad\,\ $
b) \ $V:\ x^2+y^2+4(z-1)^2 \le 4$ \newline
c) \ $V:\,\ x,y,z \ge 0\,\ja\,x+y+z \le 3 \qquad$
d) \ $V:\ x^2+y^2 \le 2y\,\ja\,0 \le z \le 4$

\item
Tetraedrin muotoista aluetta $V\subset\R^3$ rajoittavat tasot $z=0$, $x=2y$, $x=-y$ ja
$y+z=a$ ($a>0$). Laske vektorikentän
\[
\vec F=(3xyz+e^{yz})\vec i+(x^2+y^2z)\vec j+(y^2-2yz^2)\vec k
\]
vuo $V$:n reunapinnan $\partial V$ läpi sisältä ulospäin.

\item
Olkoon $S:\,x^2+y^2+z^2=1$, $B \subset S$ ja
\[
V= \{(x,y,z)\in\R^3 \mid (x,y,z)=t(u,v,w),\ t\in[0,R]\,\ja\,(u,v,w) \in B\}.
\]
Laske $V$:n tilavuus $\mu(V)$ \ a) pallokoordinaattien avulla, \ b) soveltamalla Gaussin 
lausetta vektorikenttään $\vec F=x\vec i+y\vec j+z\vec k$.

\item
Pyörivässä virtauskentässä massavuon tiheys pisteessä $\,(x,y,z) \vastaa \vec r\,$ on 
$\vec F=Q\,\vec r\times(2\vec i-\vec j+3\vec k)$, missä $Q=24$ kg/m$^3$/s ja $\vec r$:n
yksikkö = m. Olkoon $V$ kuutio, jonka yksi kärki on origossa, origosta lähtevät särmät ovat 
positiivisilla koordinaattiakseleilla ja sivun pituus on $a=2$ m. Laske massavuo (yksikkö kg/s)
$V$:n kunkin sivutahkon läpi laskettuna positiivisena kuution sisältä ulospäin.

\item 
Vektorikenttä $\vec F$ ja sen lähde $\rho$ ovat a) pallokoordinaatistossa, 
b) lieriökoordinaatistossa muotoa $\vec F= F(r)\vec e_r$, $\rho = \rho(r)$. Laske $F(r)$ 
lähteen $\rho(r)$ avulla käyttämällä Gaussin kaavaa sopivasti valitussa alueessa $V$.

\item
Olkoot $u$ ja $v$ säännöllisiä funktioita perusalueessa $A\subset\R^2$ tai $V\subset\R^3$.
Soveltamalla Gaussin lausetta vektorikenttään $\vec F=u\nabla v-v\nabla u$ näytä, että
\begin{align*}
&\int_A \left(u\Delta v-v\Delta u\right)\,dxdy
                 = \int_{\partial V} \left(u\pder{v}{n}-v\pder{u}{n}\right)ds, \\
&\int_V \left(u\Delta v-v\Delta u\right)\,dxdydz 
                 = \int_{\partial V} \left(u\pder{v}{n}-v\pder{u}{n}\right)dS,
\end{align*}
missä $\partial/\partial n$ tarkoittaa suunnattua derivaattaa $\partial A$:n tai $\partial V$:n
normaalin suuntaan.

\item \index{zzb@\nim!Autojen säilyminen}
(Autojen säilyminen) Yksiulotteisessa liikennevirrassa on $\rho(t,x)$ autotiheys tiellä
(yksikkö autoa/km) ja $v(t,x)$ nopeus (yksikkö km/h). \ a) Määrittele autovuo tiellä ja johda
autojen säilymislaki. \ b) Pisteessä $x=0$ tie 1 kapenee tieksi 2. Tiellä 1 ($x<0$)
liikennevirta on tasainen: $\rho=26$ autoa/km ja $v=120$ km/h. Myös tiellä 2 liikennevirta on
tasainen ($\rho$ ja $v$ vakioita) siten, että tien maksimikapasiteetti $3000$ autoa/h ei ylity.
Voiko tällaisessa liikennetilanteessa piste $x=0$ olla ruuhkaton? Jos ei, niin monellako
autolla vähintään ruuhka kasvaa tunnissa?

\item 
Lämmönjohtumisen perusyhtälöt ovat
\[
\nabla\cdot\vec J = \rho, \quad \vec J = - \lambda \nabla u,
\]
missä $\vec J$ on lämpövirran tiheys, $\rho$ lämpölähteen tiheys, $u$ lämpötila ja 
$\lambda$ materiaalin lämmönjohtavuus. Olkoon $\lambda = \lambda_i =$ vakio $V_i$:ssa, 
$i = 1,2$, missä $\lambda_2 = 3 \lambda_1$ ja
\[
V_{1, 2} = \{\,(x, y, z) \in \R^3 \mid x+2y+2z \lessgtr 0 \,\}.
\]
Määritä lämpövirtavektorin $\vec J$ raja-arvo $\vec J_2 (0, 0, 0)$ materiaalin 2 ($V_2$) 
puolelta origoa lähestyttäessä, kun tiedetään, että ko. raja-arvo materiaalin 1 puolelta on 
$\vec J_1(0, 0, 0) = J(-\vec i+\vec j+3\vec k)$ ($J$ vakio). Lähde $\rho$ on kummassakin 
materiaalissa vakio ja lämpötila $u$ on materiaalirajapinnalla jatkuva.

\item (*)
Olkoon $A\subset\R^2$ tai $V\subset\R^3$ perusalue. Näytä, että jos $u_1$ ja $u_2$ ovat 
reuna-arvotehtävän
\[
\begin{cases} 
\,\Delta u=\rho &\text{$A$:ssa ($V$:ssä)}, \\ 
\,u=0 &\text{reunalla $\partial A$ (reunalla $\partial V$)}
\end{cases}
\] 
ratkaisuja ja lisäksi riittävän säännöllisiä, niin funktiolle $u=u_1-u_2$ pätee
\[
\int_A \abs{\nabla u}^2\,dxdy=0 \quad \left(\int_V \abs{\nabla u}^2\,dxdydz=0\right).
\]
Päättele, että reuna-arvotehtävän ratkaisu --- sikäli kuin olemassa ja riittävän säännöllinen
--- on yksikäsitteinen.

\end{enumerate}