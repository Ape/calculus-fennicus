\chapter*{Esipuhe}

Tämä kirja on ensimmäinen osa kaksiosaisesta oppikirjasta, joka on syntynyt kirjoittajan
TKK:ssa pitämien matematiikan luentojen pohjalta. Kirja on tarkoitettu TKK:n nk.\ laajan 
matematiikan oppisisällöksi ensimmäisenä opiskeluvuotena eli kattamaan kurssit 
Matematiikan laaja peruskurssi I-II (osat 1--2).

Yliopistotasoisen (ehkä muunkin tasoisen) matematiikan perusopetuksen uudistaja joutuu 
väistämättä monien vaikeiden kysymysten ja valintojen eteen. Toisaalta matematiikka tuntuu 
olevan kuin kirkko, jossa perustotuudet ovat pysyviä ja liturgisiakin muutoksia vastustetaan
kiivaasti. Toisaalta kuitenkin ympäröivä maailma muuttuu koko ajan ja muutospaineet
kohdistuvat ennen pitkää myös matematiikkaan. Kirjoittaja on erityisesti joutunut pohtimaan,
millainen pitkän aikavälin vaikutus tietokoneilla ja tehokkailla laskimilla mahdollisesti on
matematiikan opetukseen yliopistotasolla. --- Kun numeerisia ja symbolisia manipulaatioita
voi suorittaa kätevästi koneella, niin millainen on se matematiikan taito, jota pitäisi
opettaa ihmisille? Olisiko ehkä syytä korostaa aiempaa enemmän matematiikan käytännöllistä
puolta ja sovelluksia? Vai päinvastoin matematiikkaa puhtaana abstraktin ajattelun taitona? 

Jääköön lukijan pääteltäväksi, onko mainituilla mietteillä ollut lopputulokseen jokin 
vaikutus (ja jos, niin minkä suuntainen). Mainittakoon kuitenkin yksi tälle oppikirjalle 
ominainen, matematiikan vakiintuneesta opetusperinteestä poikkeava piirre: Kirjassa
esitellään lukujen muodostamat lukujonot ja sarjat heti aluksi, jolloin reaaliluvut voidaan
määritellä sellaisina kuin ne laskimien ja tietokoneiden maailmassa 'näkyvät' eli äärettöminä
desimaalilukuina.

Kansainvälisenä vertailukohtana tälle kirjalle voi mainita amerikkalaiset Calculus-kirjat. 
Näihin verrattuna tämä kirja on tavoitteiltaan kunnianhimoisempi. Pyrkimyksenä on esittää
aivan perusteista lähtevä, yhtenäinen ja loogisesti etenevä johdatus matematiikan
perusideoihin ja moderniin laskutekniikkaan. Loogista aukottomuutta silmällä pitäen
abstraktiotasoa on paikoin nostettu tyypillisestä, esim.\ mainituille Calculus-kirjoille
ominaisesta 'katutasosta'. Kirja onkin tarkoitettu melko vaativalle yleisölle, jolle
matematiikka on paitsi laskutekniikkaa ja 'kaavoja' myös älyllinen haaste ja aito
kiinnostuksen kohde.

Otaniemen kotikentällä tämän kirjan edeltäjiä ovat apulaisprofessori Harri Rikkosen
1960- ja 1970-lukujen vaihteessa ja lehtori Simo Kivelän 1980- ja 1990-luvuilla kirjoittamat 
suomenkieliset oppikirjat tai luentomonisteet. Perinteen vaikutus näkyy myös tässä kirjassa
--- perinteestä on todella vaikea irrottautua. Perimätiedon ohella hyvin hyödyllisiä ovat
olleet ne lukuisat elävät keskustelut, joita kirjoittaja on käynyt TKK:n matematiikan
laitoksen henkilökuntaan kuuluvien kanssa. Monista kirjan syntyvaiheissa aktiivisista
keskustelukumppaneista on syytä mainita erityisesti lehtori (nyk.\ emer.) Simo Kivelä ja
tutkija (nyk.\ professori) Juha Kinnunen. Simo Kivelä ansaitsee 'Latex-guruna' vielä
erilliskiitoksen lukuisista matemaattista tekstinkäsittelyä koskevista neuvoista ja
teknisestä avusta. Erikseen kiitoksen ansaitsee myös lehtori Pekka Alestalo kirjaamistaan
yli sadasta kriittisestä kommentista vuodelta 2006, jolloin hän toimi laajojen peruskurssien
I-II sijaisluennoitsijana.

Kirjan ensimmäinen, käsin kirjoitettu versio syntyi lukuvuonna 2000-2001. Ko.\ vuoden
yleisöön kuulunut tekn.\ yo.\ (nyk.\ TkT) Antti H. Niemi työsti tekstistä ensimmäisen
tietokoneistetun version kesällä 2002. Kirjan muodon teksti sai ensimmäisen kerran
lukuvuonna 2005-2006. Nyt ilmestyvässä uudistetussa painoksessa on kirjan jako lukuihin ja
osalukuihin monin paikoin jäsennelty uudelleen. Joiltakin osin tekstiä on myös supistettu ja
toisilta osin hieman laajennettu. Myös harjoitustehtäviä on melkoisesti muokattu.

Syksystä 2000 kevääseen 2009 on TKK:n matematiikan laajoille peruskursseille I--II
osallistunut jo yli 2000 opiskelijaa. Haluan kiittää koko tähänastista yleisöäni paitsi
myötämielisyydestä matemaattisia haasteita kohtaan myös kriittisestä ja usein
kannustavastakin palautteesta.

Otaniemessä 17.8.2009

Juhani Pitkäranta

  
