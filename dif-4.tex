\section{Taylorin polynomit ja Taylorin lause} \label{taylorin lause}
\sectionmark{Taylorin lause}
\alku

Tässä luvussa tarkastellaan funktioita, jotka ovat annetun pisteen $x_0$ ympäristössä
riittävän sileitä, eli riittävän monta kertaa (jatkuvasti) derivoituvia.
\begin{Def} \index{Taylorin polynomi|emph}
Funktion $f:\DF_f\kohti\R$, $\DF_f\subset\R$, joka on $n$ kertaa derivoituva pisteessä
$x_0\in\DF_f$, \kor{Taylorin polynomi astetta $n$ pisteessä $x_0$} on
\[
\boxed{\kehys\quad T_n(x,x_0)=\sum_{k=0}^n\frac{f^{(k)}(x_0)}{k!}(x-x_0)^k. \quad}
\]
\end{Def}
Määritelmän mukaiset kolme ensimmäistä Taylorin polynomia ovat
\begin{align*}
T_0(x,x_0)\ &=\ f(x_0), \\[2mm]
T_1(x,x_0)\ &=\ f(x_0) + f'(x_0)(x-x_0), \\
T_2(x,x_0)\ &=\ f(x_0) + f'(x_0)(x-x_0) + \frac{1}{2}f''(x_0)(x-x_0)^2.
\end{align*}
Erityisesti siis $\,T_1(x,x_0) = f$:n linearisoiva approksimaatio pisteessä $x_0$.
\begin{Prop}
Funktion $f$ Taylorin polynomi $T_n(x,x_0)$ määräytyy yksikäsitteisesti ehdoista
\[
\frac{d^k}{dx^k}T_n(x,x_0)_{|\,x=x_0}\ =\ f^{(k)}(x_0),\quad k=0\ldots n.
\]
\end{Prop}
\tod Helposti nähdään, että $T_n(x,x_0)$ toteuttaa mainitut ehdot. Jos jokin toinen polynomi 
$p(x)$ toteuttaa samat ehdot, eli
\[
p^{(k)}(x_0)=f^{(k)}(x_0),\quad k=0\ldots n,
\]
niin silloin polynomi
\[
q(x)=T_n(x,x_0)-p(x)=\sum_{k=0}^n a_kx^k
\]
toteuttaa
\[
q^{(k)}(x_0)=0,\quad k=0\ldots n.
\]
Tällöin koska $q^{(n)}(x_0)=n!\,a_n$, seuraa $a_n=0$, jolloin 
$q^{(n-1)}(x_0)=(n-1)!\,a_{n-1} \ \impl \ a_{n-1}=0$, jne. Siis $q=0$, ja näin ollen em.\ 
ehdoista määräytyvä polynomi on yksikäsitteinen. \loppu
\begin{Exa} Määritä seuraavat Taylorin polynomit pisteessä $x_0=0\,$:
\begin{align*} 
&\text{a)}\ \ f(x)=\sqrt[3]{1+x},\ \ n=2 \qquad \text{b)}\ \ f(x)=\tan x,\ \ n=5 \\ 
&\text{c)}\ \ f(x)=\ln (1+x),\ \ n \in \N 
\end{align*}
\begin{align*} 
\text{\ratk} \text{a)} 
&\quad f'(x)=\frac{1}{3}(1+x)^{-2/3}, \quad f''(x)=-\frac{2}{9}(1+x)^{-5/3} \\
&\quad\quad\impl\ f(0)=1,\quad f'(0)=\frac{1}{3},\quad f''(0)=-\frac{2}{9} \\
&\quad\quad\impl\ T_2(x,0)=\underline{\underline{1+\frac{x}{3}-\frac{1}{9}x^2.}}\\ \\
&\text{b)} \quad f'(x)=1/\cos^2 x, \quad f''(x)=2\sin x/\cos^3 x, \\
&\quad\quad f'''(x)= 2/\cos^2 x+6\sin^2 x/\cos^4 x=-4/\cos^2 x + 6/\cos^4 x \\
&\quad\quad f^{(4)}(x)=-8\sin x/\cos^3 x+24\sin x/\cos^5 x \\
&\quad\quad f^{(5)}(x)=-16/\cos^2 x+120\sin^2 x/\cos^6 x \\ 
&\quad\quad\quad\impl \begin{cases} 
                      \,f(0)=f''(0)=f^{(4)}=0, \\ 
                      \,f'(0)=1,\quad f'''(0)=2,\quad f^{(5)}(0)=16 
                      \end{cases} \\
&\quad\quad\quad\impl\ T_5(x,0)=\underline{\underline{x+\frac{1}{3}x^3+\frac{2}{15}x^5.}} \\ \\
&\text{c)} \quad f'(x)=(1+x)^{-1}, \quad f''(x)=-(1+x)^{-2},\quad f'''(x)=2(1+x)^{-3},\\
&\quad\quad\ldots,\quad f^{(k)}=(-1)^{k-1}(k-1)!(1+x)^{-k} \\
&\quad\quad\impl \ f(0)=0,\quad f^{(k)}(0)= (-1)^{k-1}(k-1)!,\quad k=1,2,\ldots, \\
&\quad\quad\impl \ T_n(x,0)
  =\underline{\underline{x-\frac{1}{2}x^2+\frac{1}{3}x^3+\cdots +(-1)^{n-1}\frac{x^n}{n}\,.}} 
                                                                                       \loppu
\end{align*}
\end{Exa}
Taylorin polynomin derivaatta on
\[
\frac{d}{dx} T_n(x,x_0) = f'(x_0) + f''(x_0)(x-x_0) + \ldots 
                                  + \frac{f^{(n-1)}(x_0)}{(n-1)!}(x-x_0)^{n-1}.
\]
Derivoinnin tulos = $f'$:n Taylorin polynomi astetta $n-1$ pisteessä $x_0$, eli lyhyesti:
Taylorin polynomin derivaatta = derivaatan Taylorin polynomi (astetta alempi).

\pagebreak

\jatko\begin{Exa}
(jatko). Esimerkin tuloksista saadaan derivoimalla seuraavat Taylorin polynomit: \vspace{0.5cm}
\newline
\begin{tabular}{rlcl}
a) & $f(x)=(1+x)^{-2/3}$ & : & $T_1(x,0)=1-\frac{2}{3}x$. \\
b) & $f(x)=1/\cos^2 x$ & : & $T_4(x,0)=1+x^2+\frac{2}{3}x^4$. \\
c) & $f(x)=1/(1+x)$ & : & $T_{n-1}(x,0)=1-x+\cdots +(-1)^{n-1}x^{n-1}$. \loppu
\end{tabular}
\end{Exa}

\subsection*{Taylorin lause}
\index{Taylorin lause|vahv}

Taylorin polynomeihin perustuu seuraava huomattava approksimaatiolause. Todistus esitetään
luvun lopussa.
\begin{Lause} (\vahv{Taylorin lause}) \label{Taylor}
Jos $f$ on jatkuva välillä $[a,b]$ ja $n+1$ kertaa derivoituva välillä $(a,b)$, niin jokaisella
$x_0\in(a,b)$ ja $x\in[a,b],\ x \neq x_0$ pätee
\[
f(x)=T_n(x,x_0)+R_n(x),
\]
missä
\[
R_n(x)=\frac{f^{(n+1)}(\xi)}{(n+1)!}\,(x-x_0)^{n+1} \quad 
                \text{jollakin}\,\ \xi\in (x_0,x)\ \text{ tai }\ \xi\in (x,x_0).
\]
\end{Lause}
Taylorin lauseen mukaan funktiota, joka on tietyn pisteen $x_0$ ympäristössä
säännöllinen, voi tässä ympäristössä approksimoida polynomilla --- nimittäin Taylorin
polynomilla --- ja approksimaatio on yleisesti ottaen sitä tarkempi, mitä korkeampi on
polynomin asteluku, ja mitä lähempänä ollaan pistettä $x_0$. Tuloksen voi esittää
kvalitatiivisesti muodossa:
\[
\boxed{\kehys\quad \text{\pain{Sileä} funktio}\ 
               \approx\ \text{polynomi \pain{l}y\pain{h}y\pain{ellä} välillä}. \quad}
\]

Taylorin lauseen tuloksella on perustavaa laatua oleva merkitys lähes kaikessa numeerisessa 
laskennassa, johon sisältyy funktioiden approksimointia. Virhetermille $R_n(x,x_0)$, eli 
\index{jzyzy@jäännöstermi (Lagrangen)}%
Taylorin polynomiapproksimaation nk.\ \kor{jäännöstermille} (engl.\ remainder), tunnetaan
monia muotoja. Lauseen \ref{Taylor} esittämää sanotaan jäännöstermin
\index{Lagrangen!a@jäännöstermi}% 
\kor{Lagrangen}\footnote[2]{Italialais-ranskalainen \hist{Joseph Louis}
(synt.\ Giovanni Luigi) \hist{Lagrange} (1736-1813) oli aikansa huomattavimpia
matemaatikkoja. Erityisesti differentiaalilaskennan (myös integraalilaskennan) kehittämisessä
Lagrangen panos oli merkittävä. Matematiikan ohella Lagrange tutki mekaniikkaa ja saavutti
silläkin alalla pysyvän nimen. \index{Lagrange, J. L.|av} \index{Taylor, B.|av} 

Taylorin polynomit, Taylorin lause, ja erityisesti jäljempämä esitettävät
\kor{Taylorin sarjat} viittaavat englantilaiseen matemaatikkoon \hist{Brook Taylor}iin 
(1685-1731). Nimeään kantavaa lausetta ei Taylor todellisuudessa tuntenut.} muodoksi.

\begin{Exa} \label{Taylor ja exp,cos,sin} Soveltamalla derivointisääntöjä
\[ \begin{cases}
\,D^ke^x=e^x,\quad k=0,1,2,\ldots \\
\,D^{2k}\cos x=(-1)^k\cos x,\quad D^{2k+1}\cos x=(-1)^{k+1}\sin x, \quad k=0,1,2,\ldots \\
\,D^{2k}\sin x=(-1)^k\sin x,\quad D^{2k+1}\sin x=(-1)^k\cos x, \quad k=0,1,2,\ldots
\end{cases} \]
ja Lausetta \ref{Taylor} nähdään, että jos $x\in\R,\ x \neq 0$ ja $n\in\N$, niin funktioille 
$e^x,\ \cos x,\ \sin x$ pätee jollakin $\xi\in(0,x)\ (x>0)$ tai $\xi\in(x,0)\ (x<0)$ 
\newline
\[ \boxed{ \begin{aligned}
         e^x\ &=\ \Bigl(\,1+x+\frac{x^2}{2!}+\cdots +\frac{\ykehys x^n}{n!}\,\Bigr)
                             \,+\,\frac{e^\xi}{(n+1)!}\,x^{n+1}, \\
      \cos x\ &=\ \left(\,1-\frac{x^2}{2!}+\cdots +(-1)^n\frac{x^{2n}}{(2n)!}\,\right) 
                             \,+\,(-1)^{n+1}\frac{\cos\xi}{\akehys (2n+2)!}\,x^{2n+2}, \quad \\ 
\quad \sin x\ &=\ \left(\,x-\frac{x^3}{3!}+\cdots +(-1)^n\frac{x^{2n+1}}{(2n+1)!}\,\right) 
                             \,+\,(-1)^{n+1}\frac{\cos\xi}{(2n+3)!}\,x^{2n+3}. \quad
           \end{aligned} } \]
\newline
Tässä on sulkeilla ympäröity Taylorin polynomit 
\begin{align*}
e^x\,:    \quad &T_n(x,0), \\
\cos x\,: \quad &T_{2n}(x,0)=T_{2n+1}(x,0), \\
\sin x\,: \quad &T_{2n+1}(x,0)=T_{2n+2}(x,0). \loppu
\end{align*}
\end{Exa}

Taylorin lause on muotoiltavissa myös niin, että funktion $(n+1)$-kertaisen derivoituvuuden
sijasta oletetaan ainoastaan $n$-kertainen derivoituvuus ja derivaatan $f^{(n)}$ jatkuvuus.
\begin{Lause} \label{Taylorin approksimaatiolause} Jos $f$ on $n$ kertaa derivoituva välillä
$(a,b)$ ja $f^{(n)}$ on jatkuva ko.\ välillä, niin jäännöstermille $R_n(x)=f(x)-T_n(x,x_0)$
pätee jokaisella $x_0\in(a,b)$
\[
\lim_{x \kohti x_0} \frac{R_n^{(k)}(x)}{(x-x_0)^{n-k}} = 0, \quad k = 0 \ldots n.
\]
\end{Lause}
\tod Tapauksessa $n=0$ väittämä on tosi jatkuvuuden määritelmän nojalla. Jos $n \ge 1$, niin 
oletusten ja Taylorin lauseen perusteella on
\begin{align*}
f(x)\ &=\ T_{n-1}(x,x_0) + \frac{f^{(n)}(\xi)}{n!}(x-x_0)^n \\
      &=\ \left[T_{n-1}(x,x_0) + \frac{f^{(n)}(x_0)}{n!}(x-x_0)^n\right] 
                               + \frac{1}{n!}\,[f^{(n)}(\xi)-f^{(n)}(x_0)]\,(x-x_0)^n \\
      &=\ T_n(x,x_0) + \frac{1}{n!}\,[f^{(n)}(\xi)-f^{(n)}(x_0)]\,(x-x_0)^n, \quad x\in(a,b),
\end{align*} 
missä $\xi=\xi(x)=x_0$, jos $x=x_0$, muulloin $\xi(x)\in(x_0,x)$ tai $\xi\in(x,x_0)$. Siis
\[
R_n(x)\ =\  \frac{1}{n!}\,[f^{(n)}(\xi(x))-f^{(n)}(x_0)]\,(x-x_0)^n,
\]
missä $\xi(x) \kohti x_0$ kun $x \kohti x_0$. Koska $f^{(n)}$ on jatkuva $x_0$:ssa, niin
seuraa
 \[
\lim_{x \kohti x_0} \frac{R_n(x)}{(x-x_0)^n} 
              = \lim_{x \kohti x_0} \frac{1}{n!}\,[f^{(n)}(\xi(x))-f^{(n)}(x_0)] = 0.
\]
Muut väitetyt raja-arvotulokset seuraavat tästä derivoimalla: Koska
\[
f^{(k)}(x) = \left(\frac{d}{dx}\right)^k T_n(x,x_0) + R_n^{(k)}(x), \quad 
                                  x \in (x_0,x_0+a),\ \ k=1 \ldots n,
\]
ja koska tässä $(d/dx)^k\,T_n(x,x_0) = f^{(k)}$:n Taylorin polynomi astetta $n-k$, niin jo 
todistetun perusteella
\[
\lim_{x \kohti x_0} \frac{R_n^{(k)}(x)}{(x-x_0)^{n-k}} = 0, \quad k=1 \ldots n. \loppu
\]

\subsection*{Taylorin polynomien nopea laskeminen}

Joskus $f$:n derivaatat ovat niin hankalia laskea, että Taylorin polynomin saa määrätyksi 
suoremmin muilla menetelmillä, jolloin polynomin avulla voi päinvastoin määrittää derivaatat 
$f^{(k)}(x_0)$, $k=0\ldots n$ (!). Polynomia muilla keinoin määrättäessä riittää, että 
jäännöstermi saadaan riittävän pieneksi, sillä tälläkin kriteerillä polynomi on
yksikäsitteinen:
\begin{Prop} \label{Taylor-prop}
Olkoon funktio $f$ $\,n$ kertaa derivoituva välillä $(a,b)$ ja olkoon $f^{(n)}$ jatkuva
välillä $(a,b)$. Tällöin jos $p$ on polynomi astetta $\le n$ siten, että jollakin
$x_0\in(a,b)$ pätee
\[
\lim_{x \kohti x_0} \frac{f(x)-p(x)}{(x-x_0)^n}=0,
\]
niin $p(x) = f$:n Taylorin polynomi $T_n(x,x_0)$.
\end{Prop}
\tod Kun merkitään $q(x)=p(x)-T_n(x,x_0)$, niin raja-arvojen yhdistelysääntöjen
(Lause \ref{funktion raja-arvojen yhdistelysäännöt}), Lauseen
\ref{Taylorin approksimaatiolause} ja oletuksen mukaan
\[
\lim_{x \kohti x_0} \frac{q(x)}{(x-x_0)^n} 
     \,=\, \lim_{x \kohti x_0} \left[\frac{f(x)-T_n(x,x_0)}{(x-x_0)^n}
                                 - \frac{f(x)-p(x)}{(x-x_0)^n}\right] = 0-0 = 0.
\]
Tämä on mahdollista vain kun $q(x)=0$, koska $q$ on polynomi astetta $\le n$. \loppu

Seuraavissa esimerkeissä käytetään lyhennysmerkintää $[x^m]$ funktiosta muotoa $x^m g(x)$, 
missä $g$ on rajoitettu pisteen $x=0$ ympäristössä.
\begin{Exa} \label{nopea Taylor 1}
$f(x)=(x+1)/\cos x,\quad$ $T_5(x,0)=\ ?$
\end{Exa}
\ratk Koska $\,\cos x=1-\frac{x^2}{2}+\frac{x^4}{24}+[x^6]\,$ ja
$\,1/(1-t)=1+t+t^2+[t^3]$, niin
\begin{align*}
f(x) &= (x+1)\left(1-\frac{x^2}{2}+\frac{x^4}{24}\right)^{-1}(1+[x^6])^{-1} \\
     &= (x+1)\left[1-\left(\frac{x^2}{2}-\frac{x^4}{24}\right)\right]^{-1}+[x^6] \\
     &= (x+1)\left[1+\left(\frac{x^2}{2}-\frac{x^4}{24}\right)
                    +\left(\frac{x^2}{2}-\frac{x^4}{24}\right)^2\right]+[x^6] \\ 
     &= (x+1)\left(1+\frac{1}{2}x^2+\frac{5}{24}x^4\right)+[x^6] \\
     &= \Bigl(1+x+\frac{1}{2}x^2+\frac{1}{2}x^3+\frac{5}{24}x^4+\frac{5}{24}x^5\Bigr)+[x^6].
\end{align*}
Proposition \ref{Taylor-prop} mukaan sulkeissa oleva polynomi $=f$:n Taylorin polynomi 
$T_5(x,0)$. \loppu
\begin{Exa} \label{nopea Taylor 2}
$f(x)=1/(1+x^4e^{x^2}),\quad$ $f^{(8)}(0)=\ ?$
\begin{align*}
\text{\ratk} \quad f(x) &= 1-(x^4e^{x^2})+(x^4e^{x^2})^2+[x^{12}] \hspace{6cm} \\
                        &= 1-x^4\Bigl(1+x^2+\frac{1}{2}x^4+[x^6]\Bigr)
                                           +x^8\left(1+[x^2]\right)^2+[x^{12}] \\
                        &= 1-x^4-x^6+\frac{1}{2}x^8+[x^{10}] = T_8(x,0)+[x^{10}] \\
                        &\impl \ f^{(8)}(0) = \frac{1}{2}\cdot8!=\underline{\underline{20160}}.
                                                                                       \loppu
\end{align*}
\end{Exa}

\subsection*{Taylorin sarjat}
\index{Taylorin sarja|vahv}

Kun jäännöstermi Taylorin lauseessa \ref{Taylor} arvioidaan funktioille $\cos x$ ja $\sin x$, 
niin nähdään, että (ks.\ Esimerkki \ref{Taylor ja exp,cos,sin} edellä)
\begin{alignat*}{2}
\abs{\cos x -T_{2n}(x,0)}   &\leq \frac{1}{(2n+2)!}\abs{x}^{2n+2},\quad &x\in\R, \\
\abs{\sin x -T_{2n+1}(x,0)} &\leq \frac{1}{(2n+3)!}\abs{x}^{2n+3},\quad &x\in\R.
\end{alignat*}
Koska $\abs{x}^n/n!\kohti 0 \ \forall x\in\R$, kun $n\kohti\infty$, niin jokaisella $x\in\R$
pätee
\[ 
\cos x = \lim_{n\kohti\infty}T_{2n}(x,0), \qquad \sin x = \lim_{n\kohti\infty} T_{2n+1}(x,0),
\] 
eli (vrt.\ Harj.teht.\,\ref{exp(x) ja ln(x)}:\ref{H-exp-2: kosini ja sini})
\[
\boxed{\begin{aligned}
\quad\cos x\ &=\ \sum_{k=0}^\infty (-1)^k\frac{x^{2k}}{(2k)!}, \quad x\in\R, \\
     \sin x\ &=\ \sum_{k=0}^\infty (-1)^k\frac{x^{2k+1}}{(2k+1)!}, \quad x\in\R. \quad
\end{aligned}}
\]
\begin{Def}
Jos $f$ on mielivaltaisen monta kertaa derivoituva $x_0$:ssa, niin sarja
\[
\sum_{k=0}^\infty \frac{f^{(k)}(x_0)}{k!}(x-x_0)^k 
\]
on $f$:n \kor{Taylorin sarja} $x_0$:ssa.\footnote[2]{Tapauksessa $x_0=0$ käytetään Taylorin
sarjasta myös nimitystä \kor{Maclaurinin sarja}. \index{Maclaurinin sarja|av}}
\end{Def}
Taylorin sarjojen teoriassa aivan ilmeisesti keskeisin kysymys on: Milloin sarja suppenee
kohti $f(x)$:ää, ts. milloin pätee
\[
f(x)=\lim_{n\kohti\infty} T_n(x,x_0)=\sum_{k=0}^\infty \frac{f^{(k)}(x_0)}{k!}(x-x_0)^k \ ?
\]
Funktioiden $\sin x$ ja $\cos x$ kohdalla vastaus on: Aina, eli jokaisella $x\in\R$ 
(myös jokaisella $x_0 \in \R$). Myös eksponenttifunktion $e^x$ kohdalla vastaus on sama; tälle
Taylorin lause vahvistaa ennestään tunnetun tuloksen (vrt. Luku \ref{exp(x) ja ln(x)})
\[
\boxed{\quad e^x=\sum_{k=0}^\infty \frac{x^k}{k!}, \quad x\in\R. \quad}
\]
Tarkasti ottaen ym. kysymys Taylorin sarjan suppnemisesta sisältää kaksi erillistä
kysymystä, kuten nähdään seuraavasta esimerkistä.
\begin{Exa} \label{outo Taylorin sarja} Funktio
\[
f(x)=\begin{cases}
e^{-1/x^2}, \quad   &\text{kun}\ x \neq 0, \\
0\ ,                &\text{kun}\ x = 0
\end{cases}
\]
on mielivaltaisen monta kertaa derivoituva pisteessä $x=0$ (myös muualla) ja
\[
f^{(k)}(0)=0,\quad k=0,1,2,\ldots\,,
\]
joten $T_n(x,0)=0 \ \forall n$. Tässä tapauksessa siis Taylorin sarja suppenee $\forall x\in\R$,
mutta $\lim_n T_n(x,0) = 0 \neq f(x)$, kun $x \neq 0 $. \loppu
\end{Exa}
Esimerkin mukaan kahdella eri funktiolla voi olla sama Taylorin sarja (esimerkissä funktioilla 
$f$ ja $g=0$), joten Taylorin sarjan kertoimista (tai sarjan summasta) ei voi päätellä 
funktiota, josta sarja on johdettu. Useille 'normaaleille' funktioille $f$ kuitenkin pätee,
että $f$:n Taylorin sarjan summa $=f(x)$ aina kun sarja suppenee. Tällaisia normaalitapauksia
ovat esim.\ rationaalifunktiot.
\begin{Exa}
Funktion $f(x)=1/(1+4x^2)$ Taylorin polynomit origossa ovat
\[
T_{2n}(x,0)=T_{2n+1}(x,0)=\sum_{k=0}^n (-4)^kx^{2k},\quad n=0,1,\ldots
\]
Taylorin sarja eli potenssisarja $\{T_n(x,0), \ n=0,1,2,\ldots\}$ suppenee tässä tapauksessa
täsmälleen kun $\abs{x}<1/2$ (vrt.\ Luku \ref{potenssisarja}), ja tällöin summa $=f(x)$\,:
\[
\sum_{k=0}^\infty (-4)^kx^{2k} = \frac{1}{1+4x^2} 
                               = f(x), \quad x \in (-\tfrac{1}{2},\,\tfrac{1}{2}\,). \loppu
\]
\end{Exa}

Taylorin sarjojen suppenemista tutkittaessa voidaan aina tehdä muuttujan vaihdos 
$x-x_0\hookrightarrow x$, jolloin riittää tarkastella yleistä potenssisarjaa muotoa
\[ 
f(x) = \sum_{k=0}^n a_k x^k. 
\]
Tällaisen sarjan suppenemiskysymys on ratkaistu Luvussa \ref{potenssisarja}: Lauseen 
\ref{suppenemissäde} mukaan sarja suppenee joko (a) vain kun $x=0$ tai (b) välillä 
$(-\rho,\rho)$ (mahdollisesti myös kun $x = \pm \rho$), missä $\rho$ on sarjan suppenemissäde
($\rho \in \R_+$ tai $\rho = \infty$). Luvussa \ref{derivaatta} osoitettiin, että 
potenssisarjan summana määritelty funktio on mielivaltaisen monta kertaa derivoituva välillä 
$(-\rho,\rho)$ ja että derivaatat voidaan laskea derivoimalla sarja termeittäin
(Lause \ref{potenssisarja on derivoituva}). Näin ollen jos potenssisarjan
$\sum_{k=0}^\infty a_k x^k$ suppenemissäde on $\rho>0$ ja $x_0 \in \R$, niin funktio
\[ 
f(x) = \sum_{k=0}^\infty a_k\,(x-x_0)^k 
\]
on määritelty ja mielivaltaisen monta kertaa derivoituva välillä $(x_0-\rho,x_0+\rho)$ 
(koko $\R$:ssä, jos $\rho=\infty$) ja $f$:n derivaatat voidaan laskea derivoimalla sarja 
termeittäin. Kun derivoimispisteeksi valitaan erityisesti $x_0$, saadaan tulos
\[
f^{(k)}(x_0) = k!\,a_k \qekv a_k = \frac{f^{(k)}(x_0)}{k!}, \quad k = 0,1,2, \ldots 
\]
Siis $f(x)$ on esitettävissä muodossa
\[
f(x) = \sum_{k=0}^\infty \frac{f^{(k)}(x_0)}{k!}\,(x-x_0)^k.
\]
On tultu seuraavaan huomionarvoiseen tulokseen:
\begin{Lause} Jos sarja $\,\sum_{k=0}^\infty a_k\,(x-x_0)^k\,$ suppenee välillä 
$(x_0-\rho,x_0+\rho)$, $\rho>0$, niin ko.\ sarja = sarjan summana määritellyn funktion
Taylorin sarja pisteessä $x_0$.
\end{Lause}
\begin{Exa} \label{sinx/x}
Funktion $\,\sin x\,$ Taylorin sarjasta nähdään, että
\[
\sum_{k=0}^\infty (-1)^k\frac{x^{2k}}{(2k+1)!} = f(x) = \begin{cases}
                                                        \ \sin x/x\ , \ \ &x\neq 0, \\
                                                        \ 1\ ,            &x=0.
                                                        \end{cases}
\]
Koska sarja suppenee $\forall x\in\R$, niin kyseessä on sarjan summana määritellyn funktion 
Taylorin sarja origossa. Funktio $f$ on siis mielivaltaisen monta kertaa derivoituva
jokaisessa pisteessä $x\in\R$, origo mukaan lukien (!). \loppu
\end{Exa}
\begin{Exa} Ratkaise Taylorin sarjoilla alkuarvotehtävä
\[ \begin{cases} 
    \,y' = e^{-x^2}, \quad x \in \R, \\
    \,y(0) = 0.
\end{cases} \]
\end{Exa}
\ratk Koska eksponenttifunktion $e^x$ Taylorin sarja suppenee kaikkialla, niin
\[ 
e^{-x^2}\ =\ \sum_{k=0}^\infty \frac{(-x^2)^k}{k!}\ 
          =\ \sum_{k=0}^\infty \frac{(-1)^k}{k!}\,x^{2k}\,, \quad x \in \R. 
\]
Kun valitaan
\[ 
y(x)\ =\ \sum_{k=0}^\infty \frac{(-1)^k}{(2k+1)\,k!}\,x^{2k+1}\ 
      =\ x - \frac{x^3}{18} + \frac{x^5}{600} - \ldots, 
\]
niin nähdään termeittäin derivoimalla, että $y'(x)=e^{-x^2},\ x\in\R$. Koska on myös $y(0)=0$,
niin ratkaisu on tässä. \loppu

\subsection*{Taylorin lauseen todistus}
\index{Taylorin lause|vahv}

Taylorin lauseelle on monia erilaisia todistustapoja. Seuraavassa
'lyhyen kaavan' mukaisessa todistuksessa päättelyn kulmakivi on Rollen lause
(Lause \ref{Rollen lause}).

Olkoon $x_0<x$ (tapaus $x_0>x$ käsitellään vastaavasti) ja tarkastellaan välillä $[x_0,x]$
funktiota $g(t)$, joka määritellään
\[
g(t) = f(t) - T_n(t,x_0) - H(t-x_0)^{n+1}, \quad H = \frac{f(x)-T_n(x,x_0)}{(x-x_0)^{n+1}}\,. 
\]
Tälle pätee $g^{(k)}(x_0)=0,\ k=0 \ldots n$ ja $g(x)=0$. Koska siis $g(x_0)=g(x)=0$, niin
Rollen lauseen mukaan on $g'(\xi_1)=0$ jollakin $\xi_1 \in (x_0,x)\subset(a,b)$. Jos $n \ge 1$,
on myös $g'(x_0)=0$, jolloin saman lauseen mukaan on $g''(\xi_2)=0$ jollakin
$\xi_2 \in (x_0,\xi_1)$. Jos edelleen $n \ge 2$, on myös $g''(x_0)=0$, joten saman lauseen
mukaan on $g'''(\xi_2)=0$ jollakin $\xi_2\in(x_0,\xi_1)$. Jatkamalla samalla tavoin päätellään,
että yleisesti on $g^{(n)}(x_0)=g^{(n)}(\xi_n)=0$ jollakin $\xi_n \in (x_0,\xi_{n-1})$, jolloin
Rollen lauseen mukaan on $g^{(n+1)}(\xi_{n+1})=0$ jollakin 
$\xi_{n+1} \in (x_0,\xi_n) \subset (x_0,x) \subset (a,b)$. Mutta 
$g^{(n+1)}(t) = f^{(n+1)}(t)-H(n+1)!$ --- Siis $f^{(n+1)}(\xi_{n+1})-H(n+1)!=0\ \impl$ väite.
\loppu

\Harj
\begin{enumerate}

\item
Laske funktion $\,f(x)=x^4+3x^3+x^2+2x+8,$ kaikki Taylorin polynomit $T_n(x,2),\ n=0,1,2,\ldots$
ja saata ne polynomin perusmuotoon ($x$:n potenssien mukaan). Piirrä ko.\ polynomien kuvaajat
ja vertaa funktioon $f$.

\item
Laske funktion $f(x)=(x-1)/(x-2)$ Taylorin polynomi $T_3(x,0)$ ja piirrä funktion, Taylorin
polynomin ja jäännöstermin kuvaajat.
 
\item
Määritä seuraaville funktioille Taylorin polynomi $T_n(x,0)$ annettua astetta $n$ ja arvioi 
jäännöstermin Lagrangen muodosta mahdollisimman tarkasti luku 
$\displaystyle{r_n=\max_{x\in[-1,1]}\abs{R_n(x)}}$\,: \newline
a) \ $f(x)=\cosh x,\,\ n=4\qquad\qquad\quad\,$ b) \ $f(x)=e^{-0.2x},\,\ n=3$ \newline
c) \ $f(x)=\sqrt{5+x},\,\ n=3\qquad\qquad\ \ $ d) \ $f(x)=\sqrt[5]{5-x},\,\ n=3$ \newline
e) \ $f(x)=2^x,\,\ n=5\qquad\qquad\qquad\quad$ f) \  $f(x)=\ln(e+x),\,\ n=6$ \newline
g) \ $f(x)=\cot(x+\pi/3),\,\ n=4\qquad$ \      h) \ $f(x)=\sin x-1/\cos x,\,\ n=3$

\item
a) Funktioiden $\cosh x$ ja $\sinh x$ Taylorin polynomit $T_n(x,0)$ voidaan laskea joko suoraan
määritelmästä tai funktioiden $e^x$ ja $e^{-x}$ Taylorin polynomien avulla. Varmista, että
kummallakin tavalla tulos on sama. \vspace{1mm}\newline
b) Näytä, että parillisen (vastaavasti parittoman) funktion Taylorin polynomissa $T_n(x,0)$ on
vain parillisia (parittomia) potensseja.

\item
Todista Taylorin lauseen avulla:
\[
1-\frac{1}{2}x^2\,\le\,\cos x\,\le\,1-\frac{1}{2}x^2+\frac{1}{24}x^4\,, \quad
                                 \text{kun}\ x\in\left[-\frac{\pi}{2}\,,\,\frac{\pi}{2}\right].
\]
Ovatko nämä epäyhtälöt tosia myös välin $[-\pi/2,\,\pi/2]$ ulkopuolella?

\item
Harjoitustehtävässä \ref{ääriarvot}:\ref{H-V-5: jatkaminen polynomeilla}b oletetaan, että
funktio $f$ on $m$ kertaa jatkuvasti derivoituva välillä $[a,b]$. Näytä, että tehtävän
ratkaisu on $p_1(x)=T_m^+(a,x)$, $p_2(x)=T_m^-(b,x)$, missä $T_m^+(a,x)$ ja
$T_m^-(b,x)$ ovat toispuolisten derivaattojen $\dif_+^kf(a)$ ja $\dif_-^kf(b)$ avulla
määritellyt $f$:n Taylorin polynomit.

\item
Laske seuraavien funktioiden Taylorin polynomi $T_n(x,0)$ annettua astetta käyttäen 
mahdollisimman nopeita oikoteitä:
\begin{align*}
&\text{a)}\ \ f(x)=2/(4+x^3), \quad n=12 \qquad
 \text{b)}\ \ f(x)=\cos x^4, \quad n=16 \\
&\text{c)}\ \ f(x)=\Arcsin x^3,\quad n=6 \qquad
 \text{d)}\ \ f(x)=(x^3-x^5)\Arctan x^2, \quad n=7
\end{align*}

\item
Laske seuraavat derivaatat käyttäen hyväksi Taylorin polynomeja:
\newline
a) \ $f(x)=\sin x^8,\,\ f^{(40)}(0)\qquad\qquad\quad$ 
b) \ $f(x)=e^{-x^4},\,\ f^{(20)}(0)$ \newline
c) \ $f(x)=x^2/(1+x^4),\,\ f^{(100)}(0)\qquad\,$
d) \ $f(x)=x^3\ln(2+x^2),\,\ f^{(87)}(0)$ \newline
e) \ $f(x)=e^{x^3}/(1+e^{x^3}),\,\ f^{(9)}(0)\qquad\ $
f) \ $f(x)=\cos(x^2\sin^2 x),\,\ f^{(12)}(0)$ 

\item
Seuraavilla käyrillä on kääntymispiste (vrt.\ edellisen luvun Esimerkki \ref{evoluutta})
annetussa pisteessä $P$. Mihin suuntaan käyrä lähtee pisteestä $P$? \vspace{1mm}\newline
a)\ $x(t)=4t+1/t,\ y(t)=t^2+16/t,\ \ P=(17/2,12)$ \newline
b)\ $\vec r(t) = t^4\vec i + (2-2\cos t - t^2)\vec j + (t^2-t\sinh t)\vec k,\ \ P=(0,0,0)$

\item
Mikä on sarjan
\[
\text{a)}\,\ 1+4+\frac{16}{2!}+\frac{64}{3!}+\frac{256}{4!}+ \ldots \quad\
\text{b)}\,\ 1+\frac{4}{3!}+\frac{16}{5!}+\frac{64}{7!}+ \ldots  
\]
summa?

\item
Esitä seuraavien funktioiden funktioiden Taylorin sarjat
pisteessä $x_0=0\,$: \vspace{1mm}\newline
a) \ $e^{3x+1}\quad$ 
b) \ $\cos(2x^3)\quad$ 
c) \ $\sin(x-\frac{\pi}{4})\quad$ 
d) \ $\cos(2x-\pi)\quad$ 
e) \ $x^2\sin 3x$\vspace{1mm}\newline
f) \ $\sin x\cos x\quad$ 
g) \ $(1+x^3)/(1+x^2)\quad$ 
h) \ $\ln(2+x^2)\quad$ 
i) \ $x^2\ln(1+x)$

\item
Määritä seuraavien funktioiden Taylorin sarja annetussa pisteessä sekä sarjan
suppenemisväli: \vspace{1mm}\newline
a) \ $e^{-2x},\,\ x_0=-1 \qquad$
b) \ $\sin x,\,\ x_0=\tfrac{\pi}{2} \qquad\,\ $
c) \ $\ln x,\,\ x_0=1$ \newline
d) \ $\cos x,\,\ x_0=\pi \qquad\,\ $
e) \ $\cos^2 x,\,\ x_0=\tfrac{\pi}{8} \qquad$ 
f) \ $x/(4+x),\,\ x_0=3$

\item
Olkoon
\[
f(x)= \begin{cases} 
      \dfrac{2\cos x-2}{x^2}\,, &\text{kun}\ x \neq 0, \\[2mm] -1, &\text{kun}\ x=0.
      \end{cases}
\]
Laske $f'(0)$ ja $f''(0)$ \ a) suoraan derivaatan määritelmästä, \ b) $f$:n Taylorin sarjan
avulla.

\item
Seuraavat funktiot $f$ määritellään kukin jatkuvaksi pisteessä $x=0$, jolloin funktiot ovat
tässä pisteessä (ja siis koko $\R$:ssä) mielivaltaisen monta kertaa derivoituvia. Määritä
funktioiden Taylorin sarjat pisteesä $x_0=0$ ja näiden avulla $f(0)$, $f'(0)$ ja $f''(0)$.
\begin{align*}
&\text{a)}\ \ \frac{e^x-1}{x} \qquad 
 \text{b)}\ \ \frac{e^{-2x}-1+2x-2x^2}{x^3} \qquad
 \text{c)}\ \ \frac{\sinh x-x}{x^3} \\
&\text{d)}\ \ \frac{\ln(1+x)-\sin x}{x^2} \qquad
 \text{e)}\ \ \frac{e^{2x}-4e^{-x}-6x+3}{x^2}
\end{align*}

\item
Tiedetään, että
\[
\sum_{k=0}^\infty a_k(x+\pi)^k\,=\,x\cos x, \quad x\in\R.
\]
Laske sarjan kertoimet $a_0$, $a_1$ ja $a_2$.

\item (*) \label{H-dif-4: Newtonin konvergenssi}
Todista Taylorin lauseen avulla Newtonin menetelmän konvergenssia koskeva Lause
\ref{Newtonin konvergenssi} .

\item (*) Potenssisarja $\sum_{k=0}^\infty a_k (x+1)^k$ suppenee pisteen $x=-1$ lähellä,
jolloin sarjan summa on
\[
\sum_{k=0}^\infty a_k (x+1)^k\,=\,\frac{x+2}{x^2+2x+3}\,.
\]
Mitkä ovat kertoimien $a_0$, $a_1$ ja $a_2$ arvot ja mikä on sarjan suppenemisväli?

\end{enumerate}