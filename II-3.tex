\section{Skalaaritulo} \label{skalaaritulo}
\alku
\index{laskuoperaatiot!c@tason vektoreiden|vahv}

Olkoot $\vec a \neq \vec 0$ ja $\vec b \neq \vec 0$ kaksi tason vektoria. Näiden kanssa
samansuuntaiset
\index{yksikkövektori}%
\kor{yksikkövektorit} (yksikön pituiset vektorit) ovat
\[
\vec a^{\,0} = \frac{1}{\abs{\vec a}}\,\vec a=\frac{\vec a}{\left|\vec a\right|}\,, \quad\ 
\vec b^{\,0} = \frac{1}{\abs{\vec b}}\,\vec b=\frac{\vec b}{|\vec b|}\,.
\]
Liitetään jatkossa vektoreihin $\vec a,\vec b$ luku $\kos(\vec a, \vec b) \in \R$, joka 
riippuu vain $\vec a$:n ja $\vec b$:n suunnista, eli vain yksikkövektoreista 
$\vec a^{\,0},\vec b^{\,0}$. Liittäminen tapahtuu geometrisella konstruktiolla seuraavasti:
Olkoon $\vec a^{\,0}=\Vect{OA}$, $\vec b^{\,0} = \Vect{OB}$ ja valitaan pisteiden $O$ ja $A$
kautta kulkevalta suoralta piste $B'$, joka on lähinnä pistettä $B$. Tällöin kulma
$\kulma OB'B$ (tai $\kulma AB'B$, jos $B'=O$) on suora.
\begin{figure}[H]
\setlength{\unitlength}{1cm}
\begin{center}
\begin{picture}(14,3.5)(0,0)
%\Thicklines
\put(0,0){\vector(1,1){2.1}} \put(0,0){\vector(1,0){3}} \dashline{0.2}(2.1,2.1)(2.1,0) 
\put(5,0){\vector(1,0){3}} \put(5,0){\vector(0,1){3}}
\put(11,0){\vector(1,0){3}} \put(11,0){\vector(-1,3){0.95}} \dashline{0.2}(10.05,2.85)(10.05,0) 
\dashline{0.2}(9.5,0)(11,0)
\put(2.8,-0.5){$A$} \put(7.8,-0.5){$A$} \put(13.8,-0.5){$A$}
\put(-0.1,-0.5){$O$} \put(4.9,-0.5){$O=B'$} \put(10.9,-0.5){$O$}
\put(2,-0.5){$B'$} \put(10,-0.5){$B'$}
\put(2.7,0.2){$\vec a^{\,0}$} \put(7.7,0.2){$\vec a^{\,0}$} \put(13.7,0.2){$\vec a^{\,0}$}
\put(1.6,2){$\vec b^{\,0}$} \put(5.2,2.8){$\vec b^{\,0}$} \put(10.25,2.65){$\vec b^{\,0}$}
\put(2.3,2.1){$B$} \put(4.5,2.8){$B$} \put(9.55,2.65){$B$}
\path(2.1,0.15)(1.95,0.15)(1.95,0) \path(5,0.15)(5.15,0.15)(5.15,0) 
\path(10.05,0.15)(10.2,0.15)(10.2,0)
\end{picture}
\end{center}
\end{figure}
Kuvioon liittyen asetetaan:
\[
\kos(\vec a,\vec b) = \kos(\vec a^{\,0}, \vec b^{\,0}) = \left\{ \begin{array}{rl} 
 | \Vect{OB'} |, & \text{jos } \  \Vect{OB'} \ \uparrow \uparrow \ \vec a^{\,0}, \\
              0, & \text{jos } \  \Vect{OB'} = \vec 0, \\
-| \Vect{OB'} |, & \text{jos } \  \Vect{OB'} \ \uparrow \downarrow \ \vec a^{\,0}. 
\end{array} \right.
\]
Luku $\kos(\vec a,\vec b)$ määräytyy siis geometrisesti janan pituutena tai sen vastalukuna,
kun tunnetaan vektorit $\vec a$ ja $\vec b$. Konstruktio on ilmeisen symmetrinen näiden
vektorien suhteen, ts.\ $\kos(\vec a, \vec b)=\kos(\vec b, \vec a)$. Ilmeistä on myös, että
$\kos(\vec a, \vec b)$ ei muutu, jos konstruktioon liittyvää geometrista kuviota
muutetaan euklidisella liikkeellä. Tästä voidaan päätellä, että $\kos(\vec a,\vec b)$
riippuu vain vektoreiden $\vec a,\vec b$ suuntien määräämän \pain{sisäkulman}
\pain{mittaluvusta}. Konstruktiosta nähdään, että sama pätee myös kääntäen: Luku 
$\kos(\vec a,\vec b)$ määrää minitun mittaluvun yksikäsitteisesti.

Jatkossa vektorien $\vec a,\vec b$ suuntien määräämää kulmaa merkitään symbolilla
$\kulma(\vec a,\vec b)$.\footnote[2]{Merkinnässä $\kulma(\vec a,\vec b)$ ei ole tarpeen
tehdä eroa sisä- ja ulkokulman välillä, ts.\ kulmaa ei tarvitse tulkita sektoriksi.
Vrt.\ Luku \ref{geomluvut}.} 
Lukua $\kos(\vec a,\vec b)$ sanotaan tästä lähtien \kor{kulman} $\kulma(\vec a,\vec b)$
\kor{kosiniksi} ja merkitään
\index{kulma!c@kulman kosini}%
\[
\boxed{\kehys\quad \kos(\vec a,\vec b) = \cos\kulma(\vec a,\vec b). \quad}
\]
Koska luku $\cos\kulma(\vec a, \vec b)$ määrää kulmaan $\kulma(\vec a,\vec b)$ liittyen
sisäkulman mittaluvun, niin se määrää myös kulman geometrisen 'ulkonäön'. Erityisesti pätee
\[
\begin{array}{lcll}
\cos\kulma(\vec a, \vec b) = 1 \quad  & \Leftrightarrow \quad & \vec a 
                                  \uparrow \uparrow \vec b \quad   & \text{(nollakulma),} \\
\cos\kulma(\vec a, \vec b) = -1 \quad & \Leftrightarrow \quad & \vec a 
                                  \uparrow \downarrow \vec b \quad & \text{(oikokulma),} \\
\cos\kulma(\vec a, \vec b) = 0 \quad  & \Leftrightarrow \quad & \vec a \perp \vec b \quad 
                                                                   & \text{(suora kulma).}
\end{array}
\]
Tässä on käytetty merkintää $\vec a \perp \vec b$ ilmaisemaan, että vektorit $\vec a$ ja 
\index{ortogonaalisuus!a@vektoreiden} \index{kohtisuoruus (vektoreiden)}%
$\vec b$ ovat \kor{kohtisuorat} eli \kor{ortogonaaliset} ($\kulma(\vec a,\vec b)$ 
= suora kulma). Luvun $\cos\kulma(\vec a,\vec b)$ määritelmän ja Pythagoraan lauseen
perusteella on kaikissa tapauksissa
\begin{equation} \label{skalaari1}
-1 \le \cos\kulma(\vec a, \vec b) \leq 1.
\end{equation}
Määritelmästä nähdään myös, että pätee
\begin{equation} \label{skalaari2}
\cos\kulma(\lambda\vec a,\vec b)=\cos\kulma(\vec a,\lambda\vec b)
                                =\begin{cases}
                                  \ \ \cos\kulma(\vec a,\vec b), \ \ \text{jos}\ \lambda>0, \\
                                     -\cos\kulma(\vec a,\vec b), \ \ \text{jos}\ \lambda<0.
                                 \end{cases}
\end{equation}
\begin{Def} (\vahv{Skalaaritulo}) \label{vektorien skalaaritulo}
\index{skalaaritulo!a@tason vektoreiden|emph} \index{pistetulo|emph}
Vektorien $\vec a, \vec b$ \kor{skalaaritulo} eli \kor{pistetulo} on reaaliluku, joka merkitään
$\vec a \cdot \vec b$, \,luetaan '$a$ piste $b$', ja määritellään
\[
\vec a\cdot\vec b \
                  = \begin{cases} 
                    \,\abs{\vec a} \abs{\vec b}\cos\kulma(\vec a, \vec b), 
                         &\text{jos}\,\ \vec a\neq\vec 0\,\,\ \text{ja}\ \ \vec b\neq\vec 0, \\
                    \,0, &\text{jos}\,\ \vec a = \vec 0\,\ \text{tai}\,\ \vec b = \vec 0.
                    \end{cases}
\]
\end{Def}
Määritelmästä seuraa ensinnäkin
\begin{equation} \label{skalaari3}
\boxed{\kehys\quad \vec a \cdot \vec a = \abs{\vec a}^{2}. \quad}
\end{equation}
Toiseksi on voimassa symmetrialaki
\begin{equation} \label{skalaari4}
\boxed{\kehys\quad \vec a \cdot \vec b = \vec b \cdot \vec a. \quad}
\end{equation}
Kolmanneksi seuraa määritelmästä ja ominaisuudesta \eqref{skalaari2} skalaarilla kertomisen
ja pistetulon välinen osittelulaki
\begin{equation} \label{skalaari5}
\boxed{\kehys\quad (\lambda \vec a) \cdot \vec b 
  = \vec a \cdot (\lambda \vec b) = \lambda ( \vec a \cdot \vec b), \quad \lambda\in\R. \quad}
\end{equation}
Viimeksi mainitun lain perusteella sulkeiden pois jättäminen merkinnässä 
$\lambda \vec a\cdot\vec b$ ei aiheuta sekaannusta. 

Hieman vähemmän ilmeinen skalaaritulon määritelmän seuraamus on pistetulon ja vektorien 
yhteenlaskun välinen osittelulaki
\begin{equation} \label {skalaari6}
\boxed{\kehys\quad \vec a \cdot (\vec b + \vec c\,) 
                        = \vec a \cdot \vec b + \vec a \cdot \vec c. \quad}
\end{equation}
Tämän todistamiseksi tarkastellaan alla olevia kuvioita. \vspace{5mm}\newline
Kuvio 1:
\begin{figure}[H]
\setlength{\unitlength}{1cm}
\begin{center}
\begin{picture}(10,5.2)(-1,-0.5)
%\Thicklines
\put(-1,0){\line(1,0){9.5}} \put(8.7,-0.13){$S$}
\put(0,0){\vector(3,2){6}}
\dashline{0.2}(6,4)(6,0)
\put(0,0){\vector(1,0){3}} \put(0,0){\vector(1,0){6}}
\put(-0.5,-0.5){$O$} \put(6.2,-0.5){$B'$} \put(6.2,3.9){$B$}
\put(2.7,0.2){$\vec a$} \put(5.5,3.9){$\vec b$}
\dashline{0.2}(2,1.33)(2,0)
\multiput(2,0)(4,0){2}{\path(0,0.15)(0.15,0.15)(0.15,0)}
\put(0,-0.1){$\underbrace{\hspace{2cm}}_{\displaystyle{\cos\kulma(\vec a,\vec b)}}$}
\put(0.8,0.9){$1$}
\put(1,1.1){\vector(3,2){0.9}}
\put(0.7,0.9){\vector(-3,-2){0.8}}
\end{picture}
\end{center}
\end{figure}
Kuvio 2: \vspace{2mm}\newline
\begin{figure}[H]
\setlength{\unitlength}{1cm}
\begin{center}
\begin{picture}(14,6)(-0.1,-1.7)
\path(0,0)(5,0) \path(6,0)(13,0)
\put(0,0){\vector(4,1){4}} \put(0,0){\vector(3,4){3}} \put(3,4){\vector(1,-3){1}} 
\put(0,0){\vector(1,0){2}}
\put(-0.1,-0.5){$O$} \put(2.9,-0.5){$B'$} \put(3.9,-0.5){$C'$} \put(5.1,-0.1){$S$}
\dashline{0.2}(3,0)(3,4) \dashline{0.2}(4,0)(4,1)
\put(1.7,-0.5){$\vec a$} \put(2.5,3.7){$\vec b$} \put(4,1.2){$\vec c$} 
\put(3.05,0.35){$\vec b+\vec c$}
\put(2.9,4.2){$B$} \put(4.1,0.9){$C$}
\put(6,0){\vector(1,1){4}} \put(6,0){\vector(3,1){6}} \put(12,2){\vector(-1,1){2}}
\dashline{0.2}(10,0)(10,4) \dashline{0.2}(12,0)(12,2)
\put(5.9,-0.5){$O$} \put(9.9,-0.5){$C'$} \put(11.9,-0.5){$B'$} \put(13.1,-0.1){$S$}
\put(7.7,-0.5){$\vec a$} \put(8.8,3.7){$\vec b+\vec c$} \put(10.5,3.7){$\vec c$} 
\put(11.4,1.9){$\vec b$}
\put(12.1,1.9){$B$} \put(9.9,4.1){$C$}
\put(0.5,-2){$\displaystyle{
\begin{array}{ll}
\abs{\vec a}\abs{\Vect{OC'}} = \abs{\vec a}\abs{\Vect{OB'}} + \abs{\vec a}\abs{\Vect{B'C'}} 
\qquad & \qquad \abs{\vec a}\abs{\Vect{OC'}} = \abs{\vec a}\abs{\Vect{OB'}} 
                                             - \abs{\vec a}\abs{\Vect{B'C'}} \\
         \impl \ \vec a \cdot (\vec b + \vec c\,) = \vec a \cdot \vec b + \vec a \cdot \vec c 
\qquad & \qquad \impl \ \vec a \cdot (\vec b + \vec c\,) 
         = \vec a \cdot \vec b + \vec a \cdot \vec c \hspace{1cm}
\end{array}}$}
\end{picture}
\end{center}
\end{figure}

Kuviossa 1 on $\abs{\Vect{OB'}} = \abs{\cos\kulma(\vec a, \vec b)} \abs{\vec b}$
kolmioiden yhdenmuotoisuuden perusteella, jolloin skalaaritulon määritelmästä seuraa
\[
\vec a \cdot \vec b = \left\{ \begin{array}{rcl}
\abs{\vec a}\abs{\Vect{OB'}},  & \text{jos} & \Vect{OB'} \ \uparrow\uparrow \ \vec a, \\ 
-\abs{\vec a}\abs{\Vect{OB'}}, & \text{jos} & \Vect{OB'} \ \uparrow\downarrow \ \vec a.
\end{array} \right.
\]
Perustuen tähän geometriseen tulkintaan on Kuviossa 2 johdettu osittelulaki \eqref{skalaari6}
tapauksessa $\cos\kulma(\vec a, \vec b)>0$ ja $\cos\kulma(\vec a,\vec b+\vec c)>0$. Muissakin
tapauksissa ($\cos\kulma(\vec a, \vec b) \le 0$ ja/tai $\cos\kulma(\vec a,\vec b+\vec c) \le 0$)
voidaan päättellä vastaavalla tavalla, että osittelulaki \eqref{skalaari6} on voimassa.

Skalaaritulon ominaisuuksia \eqref{skalaari4}, \eqref{skalaari5}, \eqref{skalaari6} 
yhdistelemällä saadaan nk.\
\index{bilineaarisuus}%
\kor{bilineaarisuusominaisuudet}
\begin{equation} \label{skalaari7}
\boxed{\begin{array}{ll}
\ykehys \quad (\lambda \vec a + \mu \vec b) \cdot \vec c 
                    &= \ \lambda\,\vec a \cdot \vec c + \mu\,\vec b \cdot \vec c, \\ 
\akehys \quad\vec a \cdot (\lambda \vec b + \mu \vec c\,)  
                    &= \ \lambda\,\vec a \cdot \vec b + \mu\,\vec a \cdot \vec c,
                                                    \quad \lambda,\mu \in \R. \quad
\end{array}}
\end{equation} 

Skalaaritulo, joka edellä on määritelty geometristen ideoiden pohjalta, on itse asiassa hyvin 
yleinen vektoriavaruuksiin liittyvä käsite. Yleisemmissä yhteyksissä ei skalaaritulolle oleteta
\index{symmetrisyys!b@skalaaritulon}%
lähtökohtaisesti muita ominaisuuksia kuin (i) \kor{symmetrisyys}, eli symmetrialakia 
\eqref{skalaari4} vastaava ominaisuus, (ii) \kor{bilineaarisuus}, eli laskusääntöjen 
\eqref{skalaari7} vastineet, ja
\index{positiividefiniittisyys!a@skalaaritulon}%
(iii) \kor{positiividefiniittisyys}, jonka muotoilu tason vektoreille on
\[
\vec a \cdot \vec a \geq 0 \,\ \forall \vec a \quad \text{ja} \quad
             \vec a \cdot \vec a = 0  \,\impl\, \vec a = \vec 0.
\]
Tämä on ilmeinen em.\ geometrisen määritelmän perusteella (ominaisuus \eqref{skalaari3}).
Määritelmästä ja ominaisuudesta \eqref{skalaari1} seuraa myös, että tason vektoreiden 
skalaaritulolle pätee epäyhtälö
\begin{equation} \label{Cauchy-Schwarz}
\boxed{\kehys\quad\abs{\vec a \cdot \vec b} \leq \abs{\vec a\,} \abs{\vec b\,}. \quad}
\end{equation}
Tällä epäyhtälöllä on yleisempiä --- vähemmän ilmeisiä --- algerallisia ulottuvuuksia, joita
tarkastellaan lähemmin seuraavassa luvussa.
\begin{Exa}
Tason vektoreista $\vec a,\vec b$ ja $\vec u$ tiedetään
\[
\abs{\vec a}=1, \quad \abs{\vec b} = 3, \quad \vec a\cdot\vec b = 2, \quad 
\vec a\cdot\vec u = -2, \quad \vec b\cdot\vec u = 1.
\]
Määritä $\vec u$:n koordinaatit kannassa $\{\vec a,\vec b\}$. \end{Exa}
\ratk Jos merkitään $\vec u = x\vec a + y\vec b$, niin kertomalla skalaarisesti $\vec a$:lla ja
$\vec b$:lla ja käyttämällä bilineaarisuussääntöjä \eqref{skalaari7} ja sääntöjä 
\eqref{skalaari3},\eqref{skalaari4} saadaan
\[
x\vec a + y\vec b = \vec u 
                      \qimpl \begin{cases}
                             \,\abs{\vec a}^2 x + (\vec a\cdot\vec b) y = \vec a\cdot\vec u \\
                             \,(\vec a\cdot\vec b) x + \abs{\vec b}^2 y = \vec b\cdot\vec u
                             \end{cases}
\]
Siis
\[
\begin{cases} \,x + 2y = -2 \\ \,2x + 9y = 1 \end{cases} \impl\quad 
\begin{cases} \,x=-4 \\ \,y=1 \end{cases} \loppu
\]
\jatko \begin{Exa} (jatko) Jaa esimerkin vektori $\vec u$ kahteen komponenttiin siten,
että toinen komponentti on $\vec a$:n suuntainen ja toinen on kohtisuora a) $\vec a$:ta, 
b) $\vec b$:ta  vastaan. 
\end{Exa}
\ratk Halutaan laskea $\vec u_1,\vec u_2$ siten, että $\vec u = \vec u_1 + \vec u_2$, 
$\,\vec u_1 = t\vec a,\ t\in\R$ ja
\[
\text{a)}\,\ \vec u_2\cdot\vec a = (\vec u - t\vec a)\cdot\vec a = 0, \qquad 
\text{b)}\,\ \vec u_2\cdot\vec b = (\vec u - t\vec a)\cdot\vec b = 0.
\]
Esimerkin tiedoin saadaan
\[
\text{a)}\,\ t = \frac{\vec a\cdot\vec u}{\abs{\vec a}^2} = -2, \qquad
\text{b)}\,\ t = \frac{\vec b\cdot\vec u}{\vec a\cdot\vec b} = \frac{1}{2}\,.
\]
Siis $\vec u = \vec u_1 + \vec u_2$, missä
\begin{align*}
&\text{a)}\,\ \vec u_1 = -2\vec a,\quad \vec u_2 
                       = \vec u - \vec u_1 = (-4\vec a + \vec b) + 2\vec a 
                       = -2\vec a + \vec b, \\
&\text{b)}\,\ \vec u_1 = \tfrac{1}{2}\vec a,\quad\ \ \vec u_2 
                       = \vec u - \vec u_1 = (-4\vec a + \vec b) - \tfrac{1}{2}\vec a
                       = -\tfrac{9}{2}\vec a + \vec b. \loppu
\end{align*}
Esimerkin a)-kohdassa on kyse perustehtävästä, jossa annettu vektori $\vec u\in V$
halutaan jakaa kahteen komponenttiin $\vec u_1,\,\vec u_2$ siten, että $\vec u_1$ on
annetun vektorin $\vec a$ suuntainen ja $\vec u_2 \perp \vec u_1$. Tällöin sanotaan,
\index{ortogonaaliprojektio}%
että $\vec u_1$ on $\vec u$:n \kor{ortogonaaliprojektio} $\vec a$:n \kor{suuntaan} eli
$\vec a$:n virittämään $V$:n yksiulotteiseen aliavaruuteen
\[
W=\{\vec v = \lambda \vec a \mid \lambda \in \R\}.
\]
Euklidisessa tasossa aliavaruutta $W$ vastaa origon kautta kulkeva suora.
\begin{figure}[H]
\setlength{\unitlength}{1cm}
\begin{center}
\begin{picture}(9,5)
\path(0,0)(8,2)
\put(2,0.5){\vector(4,1){4}}
\put(2,0.5){\vector(1,1){3.4}}
\put(6,1.5){\vector(-1,4){0.6}}
\put(1.9,0.1){$O$}
\put(5.7,0.9){$\vec u_1$} \put(4.9,3.7){$\vec u$} \put(5.6,3.7){$\vec u_2$}
%\put(6,1.5){\arc{0.6}{-1.8}{-0.3}} \put(6.02,1.58){$\scriptscriptstyle{\bullet}$}
\path(6.16,1.54)(6.12,1.7)(5.96,1.66)
\put(7,2.2){$W\leftrightarrow S\subset E^2$} \curve(6.9,2.3,6.7,2,6.5,1.63)
\end{picture}
\end{center}
\end{figure}

\subsection*{Ortonormeerattu kanta}
\index{kanta!a@ortonormeerattu|vahv}

Luvussa \ref{tasonvektorit} konstruoitiin avaruudelle $V$ kanta käyttäen kahta lineaarisesti 
riippumatonta vektoria. Skalaaritulon tultua määritellyksi saadaan käytännön laskut 
huomattavasti yksinkertaistumaan valitsemalla kantavektorit niin, että ne ovat ortogonaaliset,
ts.\ kantavektorien välinen skalaaritulo $=0$. Jos vielä kantavektorit
\index{normeeraus (vektorin)}  \index{ortogonaalisuus!b@kannan}% 
\kor{normeerataan} yksikkövektoreiksi, niin sanotaan, että näin saatu kanta on
\kor{ortonormeerattu} (muussa tapauksessa vain \kor{ortogonaalinen}). Ortonormeerattu kanta
$\{\vec i, \vec j\}$ on siis sellainen, että pätee
\[
\left\{ \begin{array}{ll}
\vec i \cdot \vec j = 0 & \text{(ortogonaalisuusehto),} \\
\abs{\vec i} = \abs{\vec j} = 1 & \text{(normeerausehto).}
\end{array} \right.
\]
Nämä ehdot toteuttava $\{\vec i, \vec j\}$ on lineaarisesti riippumaton, sillä
\[
x\vec i+y\vec j = \vec 0 \qimpl \begin{cases} 
                                \,x=\vec i\cdot(x\vec i+y\vec j)=\vec i\cdot\vec 0=0, \\ 
                                \,y=\vec j\cdot(x\vec i+y\vec j)=\vec j\cdot\vec 0=0.
                                \end{cases}
\]
Vastaavaa euklidisen tason koordinaatistoa sanotaan
\index{koordinaatisto!b@karteesinen}%
\kor{karteesiseksi}\footnote[2]{Termi on muotoutunut ranskalaisen \hist{Ren\'e Descartes}in
latinankielisestä nimestä \newline
Cartesius, vrt.\ alaviite edellisessä luvussa.} koordinaa\-tis\-toksi.
\begin{figure}[H]
\setlength{\unitlength}{1cm}
\begin{center}
\begin{picture}(7,4)(-1,-1)
\put(-1,0){\vector(1,0){5}} \put(3.8,-0.5){$x$}
\put(0,-1){\vector(0,1){4}} \put(0.2,2.8){$y$}
\put(-0.5,-0.5){$O$}
\put(0,0){\vector(1,0){1}} \put(0.8,-0.6){$\vec i$}
\put(0,0){\vector(0,1){1}} \put(-0.3,0.7){$\vec j$}
\put(2.9,1.4){$\bullet$}
\dashline{0.2}(0,1.5)(3,1.5) \dashline{0.2}(3,1.5)(3,0)
\put(2.9,-0.4){$x$} \put(-0.4,1.4){$y$}
\put(2.9,1.8){$P=(x,y)$}
\end{picture}
\end{center}
\end{figure}
Ortonormeeratussa kannassa $\{\vec i,\vec j\}$ annettujen vektorien skalaaritulo on helposti
laskettavissa: Jos
\[
\vec v_1 = x_1 \vec i + y_1 \vec j, \quad \vec v_2 = x_2 \vec i + y_2 \vec j,
\]
niin skalaaritulon bilineaarisuuden ja symmetrisyyden perusteella
\begin{align*}
\vec v_1 \cdot \vec v_2 &= (x_1 \vec i + y_1 \vec j) \cdot (x_2 \vec i + y_2 \vec j) \\
                        &= x_1x_2\,\vec i \cdot \vec i + (x_1y_2 + x_2y_1)\,\vec i \cdot \vec j 
                                                       + y_1y_2\,\vec j \cdot \vec j.
\end{align*}
Koska tässä on $\vec i\cdot\vec i=\vec j\cdot\vec j=1$ ja $\vec i\cdot\vec j=0$, niin 
laskukaavaksi tulee
\begin{equation} \label{skalaari9}
\boxed{\kehys\quad \vec v_1\cdot\vec v_2 = x_1x_2 + y_1y_2. \quad}
\end{equation}
Tämän mukaan skalaaritulo määräytyy ortonormeeratussa kannassa laskukaaviolla 
(vrt.\ vektorien yhteenlaskun vastaava kaavio edellisessä luvussa)
\[
\begin{array}{cccccc}
&\vec v_1   &  &\vec v_2   &     & \\
&\downarrow &  &\downarrow &     & \\ 
&(x_1,y_1)  &  &(x_2,y_2)  &\map &(x_1x_2+y_1y_2)\,=\,\vec v_1\cdot\vec v_2
\end{array}
\]
Myös vektorin itseisarvon laskeminen käy ortonormeeratussa kannassa helposti, sillä
laskukaavojen \eqref{skalaari3} ja \eqref{skalaari9} mukaan
\begin{equation} \label{skalaari10}
\boxed{\kehys\quad \abs{\vec v\,}^2= \vec v\cdot\vec v=x^2+y^2, \quad 
                                                \vec v=x\vec i+y\vec j. \quad }
\end{equation}
\begin{Exa} Laske $\cos\kulma(\vec a,\vec b)$, kun $\vec a=4\vec i+3\vec j$ ja
$\vec b=2\vec-3\vec j$.
\end{Exa}
\ratk Määritelmän \ref{vektorien skalaaritulo} ja kaavojen
\eqref{skalaari9}--\eqref{skalaari10} perusteella
\[
\cos\kulma(\vec a,\vec b)=\frac{\vec a\cdot\vec b}{\abs{\vec a\,}\abs{\vec b\,}}
                   =\frac{4\cdot 2+3\cdot(-3)}{\sqrt{4^2+3^2}\sqrt{2^2+3^2}}
                   =-\frac{1}{5\sqrt{13}}\,. \loppu
\]

\Harj
\begin{enumerate}

\item
Laske \newline
a) \ $\abs{4\vec a-5\vec b}$, kun $\abs{\vec a}=1$, $\abs{\vec b}=2$ ja 
     $\vec a\cdot\vec b=-\frac{1}{3}\abs{\vec a}\abs{\vec b}$ \newline
b) \ $\vec a\cdot\vec b$, kun $\abs{\vec a+3\vec b}=16$ ja $\abs{\vec a-3\vec b}=2\sqrt{58}$

\item
a) Kolmiossa $ABC$ ovat sivujen $AB$, $AC$ ja $BC$ pituudet $a$, $b$ ja $c$. Näytä skalaaritulon
avulla, että pätee $\,c^2 = a^2+b^2-2ab\cos\kulma BAC$. \newline
b) Nelikulmiossa $ABCD$ on $\cos\kulma BAD=\gamma$ ja sivujen $AB$, $AD$, $BC$ ja $CD$ pituudet
ovat $a$, $b$, $c$ ja $d$. Laske $\cos\kulma BCD$.

\item
Kun $\vec a$ ja $\vec b$ ovat tason vektoreita ja $|\vec a |=|\vec b |$, saa
skalaaritulo $(\vec a+\vec b)\cdot(\vec a-\vec b)$ yksinkertaisen muodon. Millaisen? Mitä 
tulos tarkoittaa geometrisesti, jos $\,\vec a=\Vect{OA}\,$ ja $\,\vec b=\Vect{OB}$, missä \ 
a) $\,OA\,$ ja $\,OB\,$ ovat suunnikkaan kaksi sivua, \ b) $\,O$ on ympyrän keskipiste ja
$A$, $B$ sen kehän pisteitä? Kuva! 

\item
Suorakulmaisessa kolmiossa $ABC$ on suoran kulman kärjestä lähtevä korkeusjana $AD$. Sivujen
$AB$ ja $AC$ pituudet ovat $5$ ja $12$. Laske skalaaritulot $\Vect{AB}\cdot\Vect{DC}$,
$\Vect{BD}\cdot\Vect{CA}$ ja $\Vect{AC}\cdot\Vect{CD}$.

\item
Tason vektoreista $\vec a$ ja $\vec b$ tiedetään, että $|\vec a|=|\vec b|=1$ ja
$\vec a \cdot \vec b =\frac{1}{7}$. Lisäksi tiedetään vektorista $\vec v$,
että $\vec a \cdot \vec v =3$ ja $\vec b \cdot \vec v =-1$. Määritä $\vec v$:n
koordinaatit kannassa $\{\vec a ,\vec b \}$.

\item
Tason koordinaatistossa $\{O,\vec a,\vec b\}$ ovat pisteen $P$ koordinaatit $(2,1)$ ja pisteen
$O'$ koordinaatit $(-1,3)$. Määritä pisteen $P$ koordinaatit koordinaatistossa
$\{O',\vec a+\vec b,\vec a-2\vec b\}$, kun tiedetään, että $\abs{\vec a}=\abs{\vec b}=1$ ja
$\vec a\cdot\vec b=-\frac{1}{3}\,$.

\item
Laske $\abs{\vec a}$, $\abs{\vec b}$ ja $\cos\kulma(\vec a,\vec b)$, kun \newline
a) \ $\vec a=\vec i-\vec j,\,\ \vec b=\vec i+2\vec j\qquad\qquad\quad$ \newline
b) \ $\vec a=-2\vec i+3\vec j,\,\  \vec b=3\vec i-\vec j$ \newline
c) \ $\vec a=-68\vec i+51\vec j,\,\ \vec b=3\vec i+4\vec j\qquad$ \newline
d) \ $\vec a=76\vec i-57\vec j,\,\ \vec b=92\vec i+69\vec j$ \newline
e) \ $\vec a=(3\sqrt{2}+5)\vec i+(5\sqrt{2}-3)\vec j,\ \ $
     $\vec b=(5\sqrt{2}+3)\vec i+(3\sqrt{2}-5)\vec j$

\item
Tason vektoreista $\vec a,\vec b \in V$ tiedetään, että $\abs{\vec a}=1$, $\abs{\vec b}=3$ ja
$\vec a\cdot\vec b=-2$. Millä kertoimien $\lambda,\mu$ arvoilla 
$\{\vec a,\lambda\vec a+\mu\vec b\}$ on $V$:n ortonormeerattu kanta\,?

\item (*)
Näytä skalaarituloa käyttäen, että kolmion korkeusjanat ovat kolmella saman pisteen kautta
kulkevalla suoralla, ts.\ korkeusjanat tai niiden jatkeet leikkaavat samassa pisteessä.

\item (*)
Joukko $S\subset\Ekaksi$ koostuu pisteistä $P$, joiden karteesiset koordinaatit toteuttavat
ehdon
\[
(x-1)^2-(y+2)^2=1.
\]
Olkoot $(\xi,\eta)$ pisteen $P$ koordinaatit toisessa ortonormeeratussa koordinaatistossa,
jonka origo on pisteessä $(x,y)=(1,-2)$ ja kantavektorit
\[
\vec e_1=\frac{1}{\sqrt{2}}(\vec i+\vec j),\ \ \vec e_2=\frac{1}{\sqrt{2}}(\vec i-\vec j).
\]
Jos pisteen $P=(x,y)$ koordinaatit tässä koordinaatistossa ovat $(x',y')$, niin millä
koordinaateille $x',y'$ asetettavalla ehdolla on $P=(x',y') \in S\,$? Piirrä kuva, jossa
näkyvät molemmat koordinaatistot, ja hahmottele kuvaan joukko $S$.

\end{enumerate}

